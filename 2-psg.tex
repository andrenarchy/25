%% -*- coding:utf-8 -*-

\chapter{Phrase structure grammar}
\label{Kapitel-PSG}
\is{phrase structure grammar|(}

This chapter deals with phase structure grammars, which play an important role in several of the theories we will encounter in later chapters.

\section{Symbols and rewrite rules}

Words can be assigned to a particular part of speech on the basis of their inflectional properties
and syntactic distribution. Thus, \emph{weil} `because' in (\mex{1})
is a conjunction\is{conjunction}, whereas \emph{das} `the' and \emph{dem} `the' are
articles\is{article} and therefore classed as determiners\is{determiner}. Furthermore, \emph{Buch} `book' and \emph{Mann} `man' are nouns\is{noun} 
and \emph{gibt} `gives' is a verb\is{verb}.
\ea\label{bsp-weil-er-das-buch-dem-mann-gibt}
\gll weil er das Buch dem Mann gibt\\
	 because he the book the man gives\\
\glt `because he gives the man the book'
\z
Using the constituency tests we introduced in Section~\ref{konstituententests}, we can show that
individual words as well as the strings \emph{das Buch} `the book' and \emph{dem Mann} `the man',
form constituents. These get then assigned certain symbols. Since nouns form an important part of
the phrases \emph{das Buch} and \emph{dem Mann}, these are referred to as \emph{noun phrases} or
NPs, for short. The pronoun \emph{er} `he' can occur in the same positions as full NPs and can
therefore also be assigned to the category NP.

Phrase structure grammars come with rules specifying which symbols are assigned to certain kinds of words and how these are combined to create more
complex units. A simple phrase structure grammar which can be used to analyze (\mex{0}) is given in (\mex{1}):\footnote{
	I ignore the conjunction \emph{weil} `because' for now. Since the exact analysis of
        German verb"=first and verb"=second clauses requires a number of additional assumptions, we will restrict ourselves to verb"=final clauses in this chapter.
}$^,$\footnote{\label{fn-np-pron-ps-rule}
	The rule NP $\to$ er may seem odd. We could assume the rule PersPron $\to$ er instead but then would have to posit a further rule which
	would specify that personal pronouns can replace full NPs: NP $\to$ PersPron. The rule in (\mex{1}) combines the two aforementioned rules and states
	that \emph{er} `he' can occur in positions where noun phrases can.
}
\ea
\label{bsp-grammatik-psg}
\begin{tabular}[t]{@{}l@{ }l}
{NP} & {$\to$ D N}\\          
{S}  & {$\to$ NP NP NP V}
\end{tabular}\hspace{2cm}%
\begin{tabular}[t]{@{}l@{ }l}
{NP} & {$\to$ er}\\
{D}  & {$\to$ das}\\
{D}  & {$\to$ dem}\\
\end{tabular}\hspace{8mm}
\begin{tabular}[t]{@{}l@{ }l}
{N} & {$\to$ Buch}\\
{N} & {$\to$ Mann}\\
{V} & {$\to$ gibt}\\
\end{tabular}
\z
We can therefore interpret a rule such as NP $\to$\is{$\to$} D N as meaning that a noun phrase, that is, something which is assigned the symbol NP, can consist
of a determiner (D) and a noun (N).

We can analyze the sentence in (\mex{-1}) using the grammar in (\mex{0}) in the following way:
first, we take the first word in the sentence and check if there is a rule in which this word occurs on the right"=hand
side of the rule. If this is the case, then we replace the word with the symbol on the left"=hand side of the rule. This happens
in lines 2--4, 6--7 and 9 of the derivation in (\mex{1})\vpageref*{bsp-anwendung-grammatik}. For
instance, in line~2 \emph{er} is replaced by NP.
If there are two or more symbols which occur together on the right"=hand side of a rule, then all
these words are replaced with the symbol on the left. This happens in lines 5, 8 and 10. For
instance, in line 5 and 8, D and N are rewritten as NP.
%\begin{figure}
\ea
\label{bsp-anwendung-grammatik}
\begin{tabular}[t]{@{}r|llllll@{\hspace{1.7cm}}l@{}}
 & \multicolumn{6}{l@{}}{words and symbols} & rules that are applied\\\hline
 1 & er            & das          & Buch          & dem          & Mann & gibt                \\
 2 & {NP}          & das          & Buch          & dem          & Mann & gibt & {NP $\to$ er}  \\
 3 & NP            & {D}          & Buch          & dem          & Mann & gibt & {D $\to$ das}  \\
 4 & NP            & D            & {N}           & dem          & Mann & gibt & {N $\to$ Buch} \\
 5 & NP            &              & {NP}          & dem          & Mann & gibt & {NP $\to$ D N}\\
 6 & NP            &              & NP            & {D}          & Mann & gibt & {D $\to$ dem}  \\
 7 & NP            &              & NP            & D            & {N}  & gibt & {N $\to$ Mann} \\
 8 & NP            &              & NP            &              & {NP} & gibt & {NP $\to$ D N}\\
 9 & NP            &              & NP            &              & NP   & {V} & {V $\to$ gibt}  \\
10 &               &              &               &              &      & {S} & {S $\to$ NP NP NP V}\\
\end{tabular}
\z
%\vspace{-\baselineskip}\end{figure}%
In (\mex{0}), we began with a string of words and it was shown that we can derive the structure of a sentence by applying the rules of
a given phrase structure grammar. We could have applied the same steps in reverse order: starting
with the sentence symbol S, we would have applied the steps~9--1 and arrived at the string of words.
Selecting different rules from the grammar for rewriting symbols, we could use the grammar in (\mex{-1}) to get
from S to the string \emph{er dem Mann das Buch gibt} `he the man the book gives'.
We can say that this grammar licenses (or generates)\label{Seite-generiert} a set of sentences.\is{Generative Grammar}

The derivation in (\mex{0}) can also be represented as a tree. This is shown by Figure~\vref{er-das-buch-dem-mann-gibt-flat}.
\begin{figure}
\centerline{
\begin{forest}
sn edges
[S
  [NP [er;he] ]
  [NP
    [Det [das;the] ]
    [N [Buch;book] ] 
  ]
  [NP
    [Det [dem;the] ]
    [N [Mann;man] ] 
  ]
  [V [gibt;gives] ]
]
\end{forest}
%
%% \begin{tikzpicture}
%% \tikzset{level 1+/.style={level distance=3\baselineskip}}
%% \tikzset{level 2+/.style={level distance=2\baselineskip}}
%% \tikzset{frontier/.style={distance from root=8\baselineskip}}
%% \tikzset{every leaf node/.append style={text depth=0pt}}
%% \Tree[.S
%%        [.NP er\\he ]
%%        [.NP
%%          [.Det das\\the ]
%%          [.N Buch\\book ] ]
%%        [.NP
%%          [.Det der\\the ]
%%          [.N Frau\\woman ] ]
%%        [.V gibt\\gives ] ]
%% \end{tikzpicture}
}
\caption{\label{er-das-buch-dem-mann-gibt-flat}Analysis of \emph{er das Buch dem Mann gibt} `he the
book the woman gives'}
\end{figure}%
The symbols in the tree are called \emph{nodes}\is{node}. We say that S immediately dominates the NP nodes and the V node\is{dominance}.
The other nodes in the tree are also dominated, but not immediately dominated, by S\is{dominance!immediate}. If we want to talk about the
relationship between nodes, it is common to use kinship terms. In Figure~\ref{er-das-buch-dem-mann-gibt-flat}, S is the mother node\is{node!mother}
of the three NP nodes and the V node.
The NP node and V are sisters\is{node!sister} since they have the same mother node.
If a node has two daughters\is{node!daughter}, then we have a binary\is{binary} branching structure\is{branching}\is{branching!binary}.
If there is exactly one daughter, then we have a unary\is{unary}\is{branching!unary} branching
structure. Two constituents are said to be \emph{adjacent}\is{adjacency}
if they are directly next to each other.

Phrase structure rules are often omitted in linguistic publications. Instead, authors opt for tree diagrams or the compact equivalent bracket notation
such as (\mex{1}).
\ea
\gll {}[\sub{S} [\sub{NP} er] [\sub{NP} [\sub{D} das] [\sub{N} Buch]]  [\sub{NP} [\sub{D} dem] [\sub{N} Mann]] [\sub{V} gibt]]\\
     {}         {}        he  {}        {}       the  {}       book    {}        {}       the  {}       man     {}      gives\\  
\z
Nevertheless, it is the grammatical rules which are actually important since these represent grammatical knowledge which is independent of specific structures.
In this way, we can use the grammar in (\ref{bsp-grammatik-psg}) to parse or generate the sentence
in (\mex{1}), which differs from (\ref{bsp-weil-er-das-buch-dem-mann-gibt}) in the order of objects: 
\ea
\gll {}[weil] er dem Mann das Buch gibt\\
	 {}\spacebr{}because he.\nom{} the.\dat{} man the.\acc{} book gives\\
\glt `because he gives the man the book'
\z
The rules for replacing determiners and nouns are simply applied in a different order than in (\ref{bsp-weil-er-das-buch-dem-mann-gibt}). Rather than replacing the first Det with \emph{das} `the' and the first noun with \emph{Buch} `book', the first Det is replaced with \emph{dem} `the' and the first noun with \emph{Mann}.

At this juncture, I should point out that the grammar in (\ref{bsp-grammatik-psg}) is not the only possible grammar for the example sentence in
(\ref{bsp-weil-er-das-buch-dem-mann-gibt}). There is an infinite\label{page-unendlich-viele-grammatiken} number of possible grammars which could
be used to analyze these kinds of sentences (see exercise \ref{ua-psg-eins}). Another possible grammar is given in (\mex{1}):

\ea\label{psg-binaer}
\begin{tabular}[t]{@{}l@{ }l@{}}
NP & $\to$ D N  \\
V  & $\to$ NP V\\
\end{tabular}\hspace{2cm}%
\begin{tabular}[t]{@{}l@{ }l}
{NP} & {$\to$ er}\\
{D}  & {$\to$ das}\\
{D}  & {$\to$ dem}\\
\end{tabular}\hspace{8mm}
\begin{tabular}[t]{@{}l@{ }l}
{N} & {$\to$ Buch}\\
{N} & {$\to$ Mann}\\
{V} & {$\to$ gibt}\\
\end{tabular}
\z
This grammar licenses binary branching structures as shown in Figure~\vref{er-das-buch-dem-mann-gibt-bin}.
\begin{figure}
\centerline{
\begin{forest}
sn edges
[V
  [NP [er;he] ]
  [V
    [NP
      [Det [das;the] ]
      [N [Buch;book] ] ]
    [V
      [NP
        [Det [dem;the] ]
        [N [Mann;man] ] ]
      [V [gibt;gives] ] ] ] ]
\end{forest}
}
\caption{\label{er-das-buch-dem-mann-gibt-bin}Analysis of \emph{er das Buch dem Mann gibt} with a
  binary branching structure}
\end{figure}%

Both the grammar in (\mex{0}) and (\ref{bsp-grammatik-psg}) are too imprecise.
If we adopt additional lexical entries for \emph{ich} `I' and \emph{den} `the' (accusative) in our grammar, then we would incorrectly
license the ungrammatical sentences in (\mex{1}b--d):\footnote{
	With the grammar in (\ref{psg-binaer}), we also have the additional problem that we cannot determine when an utterance is complete
	since the symbol V is used for all combinations of V and NP. Therefore, we can also analyze the sentence in (i) with this grammar:
  
\eal
\ex[*]{
\gll der Mann erwartet\\
     the man expects\\
}
\ex[*]{
\gll des Mannes er das Buch dem Mann gibt\\
	 the.\gen{} man.\gen{} he the.\dat{} book.\dat{} the man gives\\
}
\zl
The number of arguments required by a verb must be somehow represented in the grammar. In the following chapters, we will see exactly
how the selection of arguments by a verb (valence) can be captured in various grammatical theories.
}
\eal
\ex[]{
\gll er        das Buch        dem Mann       gibt\\
     he.\nom{} the.\acc{} book the.\dat{} man gives\\
\glt `He gives the book to the man.'
}
\ex[*]{
\gll ich      das Buch        dem Mann       gibt\\
     I.\nom{} the.\acc{} book the.\dat{} man gives\\
}
\ex[*]{
\gll er        das        Buch den        Mann gibt\\
     he.\nom{} the.\acc{} book the.\acc{} man gives\\
}
\ex[*]{
\gll er        den        Buch       dem Mann gibt\\
     he.\nom{} the.\mas{} book(\neu) the man  gives\\
}
\zl
In (\mex{0}b), subject"=verb agreement\is{agreement|(} has been violated, in other words: \emph{ich} `I' and \emph{gibt} `gives' do not fit together.
(\mex{0}c) is ungrammatical because the case requirements of the verb have not been satisfied: \emph{gibt} `gives' requires a dative object. Finally, (\mex{0}d) is ungrammatical
because there is a lack of agreement between the determiner and the noun. It is not possible to combine \emph{den} `the', which is masculine and bears accusative case, 
and \emph{Buch} `book' because \emph{Buch} is neuter gender. For this reason, the gender properties
of these two elements are not the same and the elements can therefore not be combined.

In the following, we will consider how we would have to change our grammar to stop it from licensing the sentences in (\mex{0}b--d).
If we want to capture subject"=verb agreement, then we have to cover the following six cases in German, as the verb has to agree with the
subject in both person\is{person} (1, 2, 3) and number\is{number} (sg, pl)\is{agreement}:
\eal\jamwidth=8cm\relax%\settowidth\jamwidth{(3, sg)}
\ex 
\gll Ich schlafe.\\
     I   sleep\\      \jam(1, sg)
\ex 
\gll Du schläfst.\\
     you sleep\\      \jam(2, sg)
\ex 
\gll Er schläft.\\
     he sleeps\\      \jam(3, sg)
\ex 
\gll Wir schlafen.\\
     we sleep\\       \jam(1, pl)
\ex 
\gll Ihr schlaft.\\
     you sleep\\       \jam(2, pl)
\ex 
\gll Sie schlafen.\\   
     they sleep\\      \jam(3, pl)
\zl
It is possible to capture these relations with grammatical rules by increasing 
the number of symbols we use. Instead of the rule S $\to$ NP NP NP V, we can use
the following:
\ea
\begin{tabular}[t]{@{}l@{ }l}
S  & $\to$ NP\_1\_sg NP NP V\_1\_sg\\
S  & $\to$ NP\_2\_sg NP NP V\_2\_sg\\
S  & $\to$ NP\_3\_sg NP NP V\_3\_sg\\
S  & $\to$ NP\_1\_pl NP NP V\_1\_pl\\
S  & $\to$ NP\_2\_pl NP NP V\_2\_pl\\
S  & $\to$ NP\_3\_pl NP NP V\_3\_pl\\
\end{tabular}
\z
This would mean that we need six different symbols for noun phrases and verbs respectively, as well as six rules rather than one.

In order to account for case assignment by the verb, we can incorporate case information into the symbols in an analogous way. We would then
get rules such as the following:
\ea
\label{ditrans-ps-regeln}
\begin{tabular}[t]{@{}l@{ }l}
S  & $\to$ NP\_1\_sg\_nom NP\_dat NP\_acc V\_1\_sg\_nom\_dat\_acc\\
S  & $\to$ NP\_2\_sg\_nom NP\_dat NP\_acc V\_2\_sg\_nom\_dat\_acc\\
S  & $\to$ NP\_3\_sg\_nom NP\_dat NP\_acc V\_3\_sg\_nom\_dat\_acc\\
S  & $\to$ NP\_1\_pl\_nom NP\_dat NP\_acc V\_1\_pl\_nom\_dat\_acc\\
S  & $\to$ NP\_2\_pl\_nom NP\_dat NP\_acc V\_2\_pl\_nom\_dat\_acc\\
S  & $\to$ NP\_3\_pl\_nom NP\_dat NP\_acc V\_3\_pl\_nom\_dat\_acc\\
\end{tabular}
\z
Since it is necessary to differentiate between noun phrases in four cases, we have a total of six symbols for NPs in the nominative and three symbols for NPs with
other cases. Since verbs have to match the NPs, that is, we have to differentiate between verbs which select three arguments and those selecting only one or two (\mex{1}),
we have to increase the number of symbols we assume for verbs.\is{valence}
\eal
\ex[]{
\gll Er schläft.\\
	 he sleeps\\
\glt `He is sleeping.'
}
\ex[*]{
\gll Er schläft das Buch.\\
	 he sleeps the book\\
}
\ex[]{
\gll Er kennt das Buch.\\
	 he knows the book\\
\glt `He knows the book.'
}
\ex[*]{
\gll Er kennt.\\
	 he knows\\
}
\zl
In the rules above, the information about the number of arguments required by a verb is included in the marking `nom\_dat\_acc'.

In order to capture the determiner"=noun agreement in (\mex{1}), we have to incorporate information about gender\is{gender} (fem, mas, neu),
number\is{number} (sg, pl), case\is{case} (nom, gen, dat, acc) and the inflectional classes (strong, weak\footnote{
These are inflectional classes for adjectives which are also relevant for some nouns such as \emph{Beamter} `civil servant', 
\emph{Verwandter} `relative', \emph{Gesandter} `envoy', \ldots).
For more on adjective classes see page~\pageref{page-Flexionsklasse-Wunderlich}.%
}).
\eal\settowidth\jamwidth{(Inflectional class)}
\ex 
\gll der Mann, die Frau, das Buch\\
	 the man the woman the book\\\jambox{(gender)}
\ex 
\gll das Buch, die Bücher\\
	 the book the books\\\jambox{(number)}
\ex 
\gll des Buches, dem Buch\\
	 the.\gen{} book the.\dat{} book\\\jambox{(case)}
\ex\is{inflectional class} 
\gll ein Beamter, der Beamte\\
	 a civil.servant the civil.servant\\\jambox{(inflectional class)}
\zl
Instead of the rule NP $\to$ D N, we will have to use rules such as those in (\mex{1}):
\ea
%\resizebox{\linewidth}{!}{
\begin{tabular}[t]{@{}l@{ }l@{\hspace{4mm}}l@{ }l}
NP\_3\_sg\_nom  & $\to$ D\_fem\_sg\_nom\_weak N\_fem\_sg\_nom\_weak \\
NP\_3\_sg\_nom  & $\to$ D\_mas\_sg\_nom\_weak N\_mas\_sg\_nom\_weak \\
NP\_3\_sg\_nom  & $\to$ D\_neu\_sg\_nom\_weak N\_neu\_sg\_nom\_weak \\
NP\_3\_pl\_nom  & $\to$ D\_fem\_pl\_nom\_weak N\_fem\_pl\_nom\_weak \\
NP\_3\_pl\_nom  & $\to$ D\_mas\_pl\_nom\_weak N\_mas\_pl\_nom\_weak \\
NP\_3\_pl\_nom  & $\to$ D\_neu\_pl\_nom\_weak N\_neu\_pl\_nom\_weak \\[2mm]
\end{tabular}

\begin{tabular}[t]{@{}l@{ }l@{\hspace{4mm}}l@{ }l}
NP\_3\_sg\_nom  & $\to$ D\_fem\_sg\_nom\_stark N\_fem\_sg\_nom\_stark \\
NP\_3\_sg\_nom  & $\to$ D\_mas\_sg\_nom\_stark N\_mas\_sg\_nom\_stark \\
NP\_3\_sg\_nom  & $\to$ D\_neu\_sg\_nom\_stark N\_neu\_sg\_nom\_stark \\
NP\_3\_pl\_nom  & $\to$ D\_fem\_pl\_nom\_stark N\_fem\_pl\_nom\_stark \\
NP\_3\_pl\_nom  & $\to$ D\_mas\_pl\_nom\_stark N\_mas\_pl\_nom\_stark \\
NP\_3\_pl\_nom  & $\to$ D\_neu\_pl\_nom\_stark N\_neu\_pl\_nom\_stark \\[2mm]
\end{tabular}
\z
(\mex{0}) shows the rules for nominative noun phrases. We would need analogous rules for genitive,
dative, and accusative. We would then require 48 symbols for determiners ($3*2*4*2$), 48 symbols for nouns and
48 rules rather than one.\is{agreement|)} 

\section{Expanding PSG with features}
\label{sec-PSG-Merkmale}


Phrase structure grammars which only use atomic symbols are problematic as they cannot capture certain generalizations.
We as linguists can recognize that NP\_3\_sg\_nom stands for a noun phrase because it contains the letters NP. 
However, in formal terms this symbol is just like any other symbol in the grammar and we cannot capture the commonalities
of all the symbols used for NPs. Furthermore, unstructured symbols do not capture the fact that the rules in (\mex{0}) 
all have something in common. In formal terms, the only thing that the rules have in common is that there is one symbol on the
left"=hand side of the rule and two on the right.

We can solve this problem by introducing features which are assigned to category symbols and therefore allow for the values of
such features to be included in our rules. For example, we can assume the features person, number and case for the category
symbol NP. For determiners and nouns, we would adopt an additional feature for gender and one for inflectional class.

\ea
\begin{tabular}[t]{@{}l@{ }l}
NP(3,sg,nom)  & $\to$ D(fem,sg,nom,strong) N(fem,sg,nom,strong)\\
NP(3,sg,nom)  & $\to$ D(mas,sg,nom,strong) N(mas,sg,nom,strong)\\
\end{tabular}
\z

If we were to use variables rather than the values in (\mex{0}), we would get the following rules as in (\mex{1}):
\ea
\label{Regel-mit-Variablen}
\begin{tabular}[t]{@{}l@{ }l@{ }l}
NP({3},{Num},{Case}) & $\to$ & D(Gen,{Num},{Case},Infl) N(Gen,{Num},{Case},Infl)\\
\end{tabular}
\z
The values of the variables here are not important. What is important is that they match. The value
of the person feature (the first position in the NP(3,Num,Case)) is fixed at `3' by the rule. These
kind of restrictions on the values can, of course, be determined in the lexicon: 
\ea
\begin{tabular}[t]{@{}l@{ }l}
NP(3,sg,nom)  & $\to$ es\\
D(mas,sg,nom,strong)  & $\to$ des\\
\end{tabular}
\z

\noindent
The rules in (\ref{ditrans-ps-regeln})  can be collapsed into a single schema as in (\mex{1}):
\ea
\label{ditrans-schema}
\begin{tabular}[t]{@{}l@{ }l@{ }l}
S  & $\to$ & NP({Per1},{Num1},{nom}) \\
   &       & NP(Per2,Num2,{dat})\\
   &       & NP(Per3,Num3,{acc})\\
   &       & V({Per1},{Num1},ditransitive)\\
\end{tabular}
\z
The identification of Per1 and Num1 on the verb and on the subject ensures that there is subject"=verb agreement.
For the other NPs, the values of these features are irrelevant. The case of these NPs is explicitly determined.
\is{phrase structure grammar|)}

\section{Semantics}
\label{sec-PSG-Semantik}

In the introductory chapter and the previous sections, we have been dealing with syntactic aspects
of language and the focus will remain very much on syntax for the remainder of this book. It is,
however, important to remember that we use language to communicate, that is, to transfer information
about certain situations, topics or opinions. If we want to accurately explain our capacity for
language, then we also have to explain the meanings that our utterances have. To this end, it is
necessary to understand their syntactic structure, but this alone is not enough. Furthermore,
theories of language acquisition that only concern themselves with the acquisition of syntactic
constructions are also inadequate. The syntax"=semantics interface\is{syntax"=semantics interface}
is therefore important and every grammatical theory has to say something about how syntax and
semantics interact. In the following, I will show how we can combine phrase structure rules with
semantic information. To represent meanings, I will use first"=order predicate logic and
$\lambda$"=calculus\is{$\lambda$"=calculus}. Unfortunately, it is not possible to provide a detailed
discussion of the basics of logic so that even readers without prior knowledge can follow all the
details, but the simple examples discussed here should be enough to provide some initial
insights into how syntax and semantics interact and furthermore, how we can develop a linguistic
theory to account for this.

To show how the meaning of a sentence is derived from the meaning of its parts, we will consider (\mex{1}a). We
assign the meaning in (\mex{1}b) to the sentence in (\mex{1}a). 
\eal
\ex\label{Bsp-Max-schlaeft}
\gll Max schläft.\\
     Max sleeps\\
\glt `Max is sleeping.'
\ex\label{Bsp-schlafen-max} 
\relation{schlafen}(\relation{max})
\zl
Here, we are assuming \relation{schlafen} to be the meaning of  \emph{schläft} `sleeps'. We use
prime symbols to indicate
that we are dealing with word meanings and not actual words. At first glance, it may not seem that we have really gained anything
by using \relation{schlafen} to represent the meaning of (\mex{0}a), since it is just another form of the verb \emph{schläft} `sleeps'.
It is, however, important to concentrate on a single verb form as inflection is irrelevant when it comes to meaning. We can see this by comparing the 
examples in (\mex{1}a) and (\mex{1}b):
\eal
\ex 
\gll Jeder Junge schläft.\\
     every boy sleeps\\
\glt `Every boy sleeps.'
\ex 
\gll Alle Jungen schlafen.\\
     all boys sleep\\
\glt `All boys sleep.'	 
\zl

\noindent
When looking at the meaning in (\mex{-1}b), we can consider which part of the meaning comes from each word.
It seems relatively intuitive that \relation{max} comes from \emph{Max}, but the trickier question is what exactly
\emph{schläft} `sleeps' contributes in terms of meaning. If we think about what characterizes a `sleeping' event, we
know that there is typically an individual who is sleeping. This information is part of the meaning of the verb \emph{schlafen}
`to sleep'. The verb meaning does not contain information about the sleeping individual, however, as this verb can be used
with various subjects:
 \eal
\ex 
\gll Paul schläft.\\
     Paul sleeps\\
\glt `Paul is sleeping.'
\ex 
\gll Mio schläft.\\
     Mio sleeps\\
\glt `Mio is sleeping.'
\ex 
\gll Xaver schläft.\\
     Xaver sleeps\\
\glt `Xaver is sleeping.'
\zl
We can therefore abstract away from any specific use of \relation{schlafen} and instead of, for example, \relation{max} in (\mex{-2}b), we
use a variable (\eg x). This x can then be replaced by \relation{paul}, \relation{mio} or \relation{xaver} in a given sentence. To allow us
to access these variables in a given meaning, we can write them with a $\lambda$ in front. Accordingly, \emph{schläft} `sleeps' will have
the following meaning:
\ea
$\lambda x$ \relation{sleep}(x)
\z

The step from (\ref{Bsp-schlafen-max}) to (\mex{0}) is referred to as \emph{lambda abstraction}\is{lambda"=abstraction@$\lambda$"=abstraction}.
The combination of the expression (\mex{0}) with the meaning of its arguments happens in the following way: we remove the $\lambda$ and the 
corresponding variable and then replace all instances of the variable with the meaning of the
argument. If we combine (\mex{0}) and \relation{max} as in (\mex{1}),
we arrive at the meaning in (\ref{Bsp-schlafen-max}). 
\ea
$\lambda x$ \relation{sleep}(x) \relation{max}
\z
The process is called $\beta$"=reduction\is{beta"=reduction@$\beta$"=reduction} or 
$\lambda$"=conversion\is{lambda"=conversion@$\lambda$"=conversion}. To show this further, let us consider an example with a transitive verb. The sentence
in (\mex{1}a) has the meaning given in (\mex{1}b):
\eal
\ex\label{Bsp-Max-mag-Lotte} 
\gll Max mag Lotte.\\
     Max likes Lotte\\
\glt `Max likes Lotte.'
\ex \relation{like}(\relation{max}, \relation{lotte})
\zl
The $\lambda$"=abstraction of \emph{mag} `likes' is shown in (\mex{1}):
\ea
$\lambda y \lambda x$ \relation{like}(x, y)
\z
Note that it is always the first $\lambda$ that has to be used first. The variable $y$ corresponds
to the object of \emph{mögen}. For languages like English it is assumed that the object forms a verb
phrase (VP) together with the verb and this VP is combined with the subject. German differs from
English in allowing more freedom in constituent order. The problems that result for form meaning
mappings are solved in different ways by different theories. The respective solutions will be
addressed in the following chapters.


If we combine the representation in (\mex{0}) with that of the object \emph{Lotte}, we arrive at (\mex{1}a), and following
$\beta$"=reduction, (\mex{1}b):
\eal
\label{lambbbda-moegen}
\ex $\lambda y \lambda x$ \relation{like}(x, y) \relation{lotte}
\ex $\lambda x$ \relation{like}(x, \relation{lotte})
\zl

\noindent
This meaning can in turn be combined with the subject and we then get (\mex{1}a) and (\mex{1}b) after $\beta$"=reduction:
\eal
\ex $\lambda x$ \relation{like}(x, \relation{lotte}) \relation{max}
\ex \relation{like}(\relation{max}, \relation{lotte})
\zl

\begin{sloppypar}
\noindent
After introducing lambda calculus, integrating the composition of meaning into our phrase structure rules is simple. A rule for the
combination of a verb with its subject has to be expanded to include positions for the semantic contribution of the verb, the semantic
contribution of the subject and then the meaning of the combination of these two (the entire sentence). The complete meaning is the
combination of the individual meanings in the correct order. We can therefore take the simple rule in (\mex{1}a) and turn it into 
(\mex{1}b):
\end{sloppypar}
\eal
\ex S $\to$ NP(nom) V
\ex S(V$'$ NP$'$) $\to$ NP(nom, NP$'$) V(V$'$)
\zl
V$'$ stands for the meaning of V and NP$'$ for the meaning of the NP(nom). V$'$ NP$'$ stands for the combination of V$'$ and NP$'$. When analyzing
(\ref{Bsp-Max-schlaeft}), the meaning of V$'$ is $\lambda x$ schlafen(x) and the meaning of NP$'$ is \relation{max}. The combination of V$'$ NP$'$
corresponds to (\mex{1}a) or after $\beta$"=reduction to (\ref{Bsp-schlafen-max}) -- repeated here as (\mex{1}b):
\eal
\ex $\lambda x$ \relation{sleep}(x) \relation{max}
\ex \relation{sleep}(\relation{max})
\zl

\noindent
For the example with a transitive verb in (\ref{Bsp-Max-mag-Lotte}), the rule in (\mex{1}) can be proposed:
\ea
S(V$'$ NP2$'$ NP1$'$) $\to$ NP(nom, NP1$'$) V(V$'$) NP(acc, NP2$'$)
\z
The meaning of the verb (V$'$) is first combined with the meaning of the object (NP2$'$) and then with the meaning of the subject (NP1$'$). 

At this point, we can see that there are several distinct semantic rules for the phrase structure rules above. The hypothesis that we should analyze language
in this way is called the \emph{rule"=by"=rule hypothesis}\is{rule"=by"=rule hypothesis} \citep{Bach76a}. A more general process for deriving the
meaning of linguistic expression will be presented in Section~\ref{Sec-GPSG-Sem}.

\section{Phrase structure rules for some aspects of German syntax}

Whereas determining the direct constituents of a sentence is relative easy, since we can very much rely on the movement test due to the
somewhat flexible order of constituents in German, it is more difficult to identify the parts of the noun phrase. This is the problem
we will focus on in this section. To help motivate assumptions about \xbar~syntax to be discussed in Section~\ref{sec-xbar},
we will also discuss prepositional phrases.


\subsection{Noun phrases}
\label{sec-psg-np}

Up to now, we have assumed a relatively simple structure for noun phrases: our rules state that a noun phrase consists of a determiner and a
noun. Noun phrases can have a distinctly more complex structure than (\mex{1}a). This is shown by the following examples in (\mex{1}):
\eal
\label{Beispiele-NP-Adjunkte}
\ex 
\gll eine Frau\\
	 a woman\\
\ex 
\gll eine Frau, die wir kennen\\
	 a woman who we know\\
\ex 
\gll eine Frau aus Stuttgart\\
	 a woman from Stuttgart\\
\ex 
\gll eine kluge Frau\\
	 a smart woman\\
\ex 
\gll eine Frau aus Stuttgart, die wir kennen\\
	 a woman from Stuttgart who we know\\
\ex 
\gll eine kluge Frau aus Stuttgart\\
	 a smart woman from Stuttgart\\
\ex 
\gll eine kluge Frau, die wir kennen\\
	 a smart woman who we know\\
\ex 
\gll eine kluge Frau aus Stuttgart, die wir kennen\\
	 a smart woman from Stuttgart who we know\\
\zl

As well as determiners and nouns, noun phrases can also contain adjectives, prepositional phrases and relative clauses. 
The additional elements in (\mex{0}) are adjuncts\is{adjunct|(}. They restrict the set of objects which the noun phrase 
refers to. Whereas (\mex{0}a) refers to a being which has the property of being a woman, the referent of (\mex{0}b) must
also have the property of being known to us.

Our previous rules for noun phrases simply combined a noun and a determiner and can therefore only be used to
analyze (\mex{0}a). The question we are facing now is how we can modify this rule or which additional rules we would
have to assume in order to analyze the other noun phrases in (\mex{0}). In addition to rule (\mex{1}a), one could propose 
a rule such as the one in (\mex{1}b).\footnote{
	See \citew[\page 238]{Eisenberg2004a} for the assumption of flat structures in noun phrases.
}$^,$\footnote{
	There are, of course, other features such as gender and number, which should be part of all the rules
	discussed in this section. I have omitted these in the following for ease of exposition.
}\todostefan{These footnottes have to be blocked from moving to the next page.}
\eal
\ex NP $\to$ Det N
\ex NP $\to$ Det A N
\zl
However, this rule would still not allow us to analyze noun phrases such as (\mex{1}):
\ea
\label{Beispiel-alle-weitern-schlagkraeftigen-Argumente}
\gll alle weiteren schlagkräftigen Argumente\\
	 all further strong arguments\\
\glt `all other strong arguments'
\z
In order to be able to analyze (\mex{0}), we require a rule such as (\mex{1}): 
\ea 
NP $\to$ Det A A N
\z
It is always possible to increase the number of adjectives in a noun phrase and setting an upper limit for
adjectives would be entirely arbitrary. Even if we opt for the following abbreviation, there are still problems:

\ea 
NP $\to$ Det A* N
\z
The asterisk\is{*} in (\mex{0}) stands for any number of iterations. Therefore, (\mex{0}) encompasses rules with no adjectives
as well as those with one, two or more.

The problem is that according to the rule in (\mex{0}) adjectives and nouns do not form a constituent and we can therefore not explain why coordination 
is still possible in (\mex{1}):
\ea
\gll alle [[geschickten Kinder] und [klugen Frauen]]\\
	 all  \spacebr{}\spacebr{}skillful children and  \spacebr{}smart women\\
\glt `all the skillful children and smart women'	 
\z
If we assume that coordination involves the combination of two or more word strings with the same syntactic properties, then we would have to assume
that the adjective and noun form a unit.

The noun phrases with adjectives discussed thus far can be explained by the following rules:

\eal
\label{NP-Regeln}
\ex NP $\to$ Det \nbar
\ex\label{NP-Regeln-Adj} \nbar $\to$ A \nbar
\ex\label{NP-Regeln-Nbar-N} \nbar $\to$ N
\zl

\noindent
These rules state the following: a noun phrase consists of a determiner and a nominal element (\nbar). This nominal element
can consist of an adjective and a nominal element (\mex{0}b), or just a noun (\mex{0}c). Since \nbar is also on the right"=hand side
of the rule in (\mex{0}b), we can apply this rule multiple times and therefore account for noun phrases with multiple adjectives such as
(\ref{Beispiel-alle-weitern-schlagkraeftigen-Argumente}). Figure~\vref{Abbildung-Adjektive-in-NP} shows the structure of a noun phrase
without an adjective and that of a noun phrase with one or two adjectives.
\begin{figure}
\hfill%
\begin{forest}
sn edges
[NP
   [Det [eine;a] ]
   [\nbar
      [N [Frau;woman] ] ] ]
\end{forest}
\hfill
\begin{forest}
sn edges
[NP
   [Det [eine;a] ]
   [\nbar
      [A [kluge;smart] ]
      [\nbar
        [N [Frau;woman] ] ] ] ]
\end{forest}
%
\hfill
\begin{forest}
sn edges
[NP
  [Det [eine;a] ]
    [\nbar
    [A [glückliche;happy] ]
       [\nbar
       [A [kluge;smart] ]
         [\nbar
         [N [Frau;woman] ] ] ] ] ]
\end{forest}
\hfill\mbox{}
%
\caption{\label{Abbildung-Adjektive-in-NP}Noun phrases with differing numbers of adjectives}
\end{figure}%
The adjective \emph{klug} `smart' restricts the set of referents for the noun phrase. If we assume an
additional adjective such as \emph{glücklich} `happy', then it only refers to those women who are happy
as well as smart. These kinds of noun phrases can be used in contexts such as the following:

\ea
\label{Beispiel-Iteration-Adjektive}
\gll A: Alle klugen Frauen sind unglücklich.\\
\spacebr{} all smart women are unhappy\\

\gll B: Nein, ich kenne eine glückliche kluge Frau.\\
	\spacebr{} no I know a happy smart woman\\
\z
We observe that this discourse can be continued with \emph{Aber alle glücklichen
  klugen Frauen sind schön} `but all happy, smart women are beautiful' and a corresponding answer. The possibility
  to have even more adjectives in noun phrases such as \emph{eine glückliche kluge Frau} `a happy, smart
  woman'\todoandrew{Komma zwischen den Adjektiven?} is accounted
  for in our rule system in (\mex{-1}). In the rule (\ref{NP-Regeln-Adj}), \nbar occurs on the left as well as the right"=hand
  side of the rule. This kind of rule is referred to as \emph{recursive}\is{recursion}.
\is{adjunct|)}

We have now developed a nifty little grammar that can be used to analyze noun phrases containing
adjectival modifiers. As a result, the combination of an adjective and noun is given constituent
status. One may wonder at this point if it would not make sense to also assume that determiners and
adjectives form a constituent, as we also have the following kind of noun phrases: 
\ea
\gll diese schlauen und diese neugierigen Frauen\\
	 these smart and these curious women\\
\z
Here, we are dealing with a different structure, however. Two full NPs have been
conjoined\is{coordination} and part of the first conjunct has been deleted.
\ea
\gll diese schlauen \st{Frauen} und diese neugierigen Frauen\\
	 these smart women and these curious women\\
\z
One can find similar phenomena at the sentence and even word level:
\eal
\ex 
\gll dass Peter dem Mann das Buch \st{gibt} und Maria der Frau die Schallplatte gibt\\
	 that Peter the man the book gives and Maria the woman the record gives\\
\glt `that Peters gives the book to the man and Maria the record to the woman'
\ex 
\gll be- und ent"=laden\\
	 \prt{} and \prt{}"=load\\
\glt `load and unload'
\zl
% Dass in (\mex{-1}) wirklich keine normale symmetrische Koordination vorliegt, sieht man, wenn man
% (\mex{-1}) mit (\mex{1}) vergleicht:
% \ea
% diese schlauen Frauen und klugen Männer
% \z
% Mit (\mex{0}) verweist man auf eine Gruppe, die aus schlauen Frauen und klugen Männern besteht,
% wohingegen man mit (\mex{-2}) auf zwei Gruppen verweist, nämlich

\noindent
Thus far, we have discussed how we can ideally integrate adjectives into our rules for the structure of noun phrases.
Other adjuncts such as prepositional phrases or relative clauses can be combined with \nbar in an analogous way to adjectives:
\eal
\ex\label{xbar-PP-Adjunkt-an-N} \nbar $\to$ \nbar PP
\ex \nbar $\to$ \nbar relative clause
\zl
With these rules and those in (\ref{NP-Regeln}), it is possible -- assuming the corresponding rules for PPs and
relative clauses -- to analyze all the examples in (\ref{Beispiele-NP-Adjunkte}).

(\ref{NP-Regeln}c) states that it is possible for \nbar to consist of a single noun. A further important rule has not yet been
discussed: we need another rule to combine nouns such as \emph{Vater} `father', \emph{Sohn} `son' or \emph{Bild} `picture', 
so"=called \emph{relational nouns}\is{noun!relational}, with their arguments. Examples of these can be found in (\mex{1}a--b).
(\mex{1}c) is an example of a nominalization of a verb with its argument:
\eal
\label{Beispiele-NP-relationale-Nomina}
\ex 
\gll der Vater von Peter\\
	 the father of Peter\\
\glt `Peter's father'
\ex 
\gll das Bild vom Gleimtunnel\\
	 the picture of.the Gleimtunnel\\
\glt `the picture of the Gleimtunnel'
\ex 
\gll das Kommen des Installateurs\\
	 the coming of.the plumber\\
\glt `the plumber's visit'
\zl
\noindent
The rule that we need to analyze (\ref{Beispiele-NP-relationale-Nomina}a,b) is given in
(\mex{1}):
%\todostefan{Martin: It is not said why relational nouns should behave in a special way.}
\ea
\nbar $\to$ N PP
\z


Figure~\ref{Abbildung-NP-mit-PP-Argument} shows two structures with PP"=arguments. The tree on the right also contains an additional PP"=adjunct, which is licensed
by the rule in (\ref{xbar-PP-Adjunkt-an-N}).
\begin{figure}
\centerfit{%
\begin{forest}
sn edges
[NP
 [Det [das;the] ]
 [\nbar
   [N [Bild;picture] ]
   [PP [vom Gleimtunnel;of.the Gleimtunnel,triangle ] ] ] ]
\end{forest}%
\hspace{2em}%
\begin{forest}
sn edges
[NP
  [Det [das;the] ]
  [\nbar
    [\nbar
      [N [Bild;picture] ]
      [PP [vom Gleimtunnel;of.the Gleimtunnel,triangle ] ] ] 
    [PP [im Gropiusbau;in.the Gropiusbau,triangle ] ] ] ]
\end{forest}}
\caption{\label{Abbildung-NP-mit-PP-Argument}Combination of a noun with PP complement
  \emph{vom Gleimtunnel} to the right with an adjunct PP}
\end{figure}%

\noindent
In addition to the previously discussed NP structures, there are other structures where the determiner or noun is missing.
Nouns can be omitted via ellipsis. (\mex{1}) gives an example of noun phrases, where a noun that does not require a complement
has been omitted. The examples in (\mex{2}) show NPs in which only one determiner and complement of the noun has been realized,
but not the noun itself:%\todoandrew{Ist das nicht \emph{a smart one}?}
\eal
\label{ex-nounless-np}
\ex 
\gll eine kluge \_\\
	 a smart\\
\glt `a smart one'
\ex 
\gll eine kluge große \_\\
     a    smart tall\\
\glt `a smart tall one'

\ex 
\gll eine kluge \_ aus Hamburg\\
	 a smart {} from Hamburg\\
\glt `a smart one from Hamburg'
\ex 
\gll eine kluge \_, die alle kennen\\
	 a smart {} who everyone knows\\
\glt `a smart one who everyone knows'
\zl

\eal
\label{ex-nounless-np-relational-noun}
\ex 
\gll (Nein, nicht der Vater von Klaus), der \_ von Peter war gemeint.\\
	\spacebr{}no not the father of Klaus the {} of Peter was meant\\
\glt `No, it wasn't the father of Klaus, but rather the one of Peter that was meant.'
\ex 
\gll (Nein, nicht das Bild von der Stadtautobahn), das \_ vom Gleimtunnel war beeindruckend.\\
	 \spacebr{}no not the picture of the motorway the {} of.the Gleimtunnel was impressive\\
\glt `No, it wasn't the picture of the motorway, but rather the one of the Gleimtunnel that was impressive.'
\ex 
\gll (Nein, nicht das Kommen des Tischlers), das \_ des Installateurs ist wichtig.\\
	 \spacebr{}no not the coming of.the carpenter the {} of.the plumber is important\\
\glt `No, it isn't the visit of the carpenter, but rather the visit of the plumber that is important.'
\zl
The underscore marks the position where the noun would normally occur. In English, the pronoun
\emph{one} must often be used in the corresponding position, but in German the noun\is{noun} is
simply omitted. (See \citet*[Section~4.12]{FLGR2012a} for English\il{English} examples without the
pronoun \emph{one}.)
%\todostefan{ich habe hier mit absicht ``often'' hinzugefügt, weil in den obengenannten beispielen
%man nicht in allen Fällen ``one'' benutzen kann, vgl. (\mex{0}c)}
In phrase structure grammars, this can be described by a so"=called \emph{epsilon production}\is{epsilon production}\is{empty element}.
These rules replace a symbol with nothing (\mex{1}a). The rule in (\mex{1}b) is an equivalent variant which is responsible for the term \emph{epsilon production}:
\eal
\label{np-epsilon}
\ex N $\to$
\ex N $\to$ $\epsilon$
\zl 

\noindent
The corresponding trees are shown in Figure~\vref{Abbildung-NP-ohne-Nomen}.
\begin{figure}
\hfill
\begin{forest}
sn edges
[NP
  [Det [eine;a] ]
  [\nbar
    [A [kluge;smart] ]
    [\nbar
      [N [\trace ] ] ] ] ]
\end{forest}
\hfill
\begin{forest}
sn edges
[NP
  [Det [das;the] ]
  [\nbar
    [N [\trace] ]
    [PP [vom Gleimtunnel;of.the Gleimtunnel, triangle] ] ] ]
\end{forest}
\hfill%
\mbox{}
\caption{\label{Abbildung-NP-ohne-Nomen}Noun phrases without an overt head}
\end{figure}%
Going back to boxes, the rules in (\mex{0}) correspond to empty boxes with the same labels as the boxes
of ordinary nouns. As we have considered previously, the actual content of the boxes is unimportant when
considering the question of where we can incorporate them. In this way, the noun phrases in (\ref{Beispiele-NP-Adjunkte})
can occur in the same sentences. The empty noun box also behaves like one with a genuine noun. If we
do not open the empty box, we will not be able to ascertain the difference to a filled box. 

It is not only possible to omit the noun from noun phrases, but the determiner can also remain unrealized in certain contexts.
(\mex{1}) shows noun phrases in plural\is{plural}:
\eal
\ex 
\gll Frauen\\
	 women\\
\ex 
\gll Frauen, die wir kennen\\
	 women who we know\\
\ex 
\gll kluge Frauen\\
	 smart women\\
\ex 
\gll kluge Frauen, die wir kennen\\
	 smart women who we know\\
\zl
The determiner can also be omitted in singular if the noun denotes a mass noun\is{noun!mass}: 
\eal
\ex 
\gll Getreide\\
	 grain\\
\ex 
\gll Getreide, das gerade gemahlen wurde\\
	 grain that just ground was\\
\glt `grain that has just been ground'
\ex 
\gll frisches Getreide\\
	 fresh grain\\
\ex 
\gll frisches Getreide, das gerade gemahlen wurde\\
	 fresh grain that just ground was\\
\glt `fresh grain that has just been ground'
\zl
Finally, both the determiner and the noun can be omitted: 
\eal
\ex 
\gll Ich helfe klugen.\\
	 I help smart\\
\glt `I help smart ones.'
\ex 
\gll Dort drüben steht frisches, das gerade gemahlen wurde.\\
	 there over stands fresh that just ground was\\
\glt `Over there is some fresh (grain) that has just been ground.'
\zl
Figure~\vref{Abbildung-NP-ohne-Det} shows the corresponding trees. 

\begin{figure}
\hfill
\begin{forest}
sn edges
[NP
  [Det [\trace] ]
  [\nbar
    [N [Frauen;women] ] ] ]
\end{forest}
\hfill
\begin{forest}
sn edges
[NP
  [Det [\trace] ]
  [\nbar
    [A [klugen;smart] ]
    [\nbar
      [N [\trace] ] ] ] ]
\end{forest}
\hfill
\mbox{}
\caption{\label{Abbildung-NP-ohne-Det}Noun phrases without overt determiner}
\end{figure}%

It is necessary to add two further comments to the rules we have developed up to this point: up to now, I have
always spoken of adjectives. However, it is possible to have very complex adjective phrases in pre"=nominal position.
These can be adjectives with complements (\mex{1}a,b) or adjectival participles (\mex{1}c,d):

\eal
\ex 
\gll der seiner Frau treue Mann\\
	 the his.\dat{} wife faithful man\\
\glt `the man faithful to his wife'
\ex 
\gll der auf seinen Sohn stolze Mann\\
	 the on his.\acc{} son proud man\\
\glt `the man proud of his son'
\ex 
\gll der seine Frau liebende Mann\\
	 the his.\acc{} woman loving man\\
\glt `the man who loves his wife'
\ex 
\gll der von seiner Frau geliebte Mann\\
     the by his.\dat{} wife loved man\\
\glt `the man loved by his wife'	 
\zl
Taking this into account, the rule (\ref{NP-Regeln-Adj}) has to be modified in the following way:
\ea
\label{NP-Regeln-AP} 
\nbar $\to$ AP \nbar
\z
An adjective phrase (AP) can consist of an NP and an adjective, a PP and an adjective or just an adjective:
\eal
\ex AP $\to$ NP A
\ex AP $\to$ PP A
\ex AP $\to$ A
\zl
There are two imperfections resulting from the rules we have developed thus far. These are the rules for adjectives
or nouns without complements in (\mex{0}c) as well as (\ref{NP-Regeln-Nbar-N}) -- repeated here as (\mex{1}):
\ea
\nbar $\to$ N
\z
If we apply these rules, then we will generate unary branching subtrees, that is trees with a mother that
only has one daughter. See Figure~\ref{Abbildung-NP-ohne-Det} for an example of this. If we maintain the
parallel to the boxes, this would mean that there is a box which contains another box which is the one with 
the relevant content.

In principle, nothing stops us from placing this information directly into the larger box. Instead of
the rules in (\mex{1}), we will simply use the rules in (\mex{2}):
\eal
\ex A $\to$ kluge
\ex N $\to$ Mann
\zl
\eal
\label{Lexikon-Projektion}
\ex AP $\to$ kluge
\ex \nbar $\to$ Mann
\zl
(\mex{0}a) states that \emph{kluge}  `smart' has the same properties as a full adjective phrase, in particular that it cannot be combined
with a complement. This is parallel to the categorization of the pronoun \emph{er} `he' as an NP in the grammars
(\ref{bsp-grammatik-psg}) and (\ref{psg-binaer}).

Assigning \nbar to nouns which do not require a complement has the advantage that we do not have to explain why the analysis in (\mex{1}b) is possible as well
as (\mex{1}a) despite there not being any difference in meaning.
\eal
\ex 
\gll {}[\sub{NP} einige [\sub{\nbar} kluge [\sub{\nbar} [\sub{\nbar} [\sub{N} Frauen ] und  [\sub{\nbar} [\sub{N} Männer ]]]]]]\\
	 {}      some   {}           smart {}          {}           {}       women  {} and {} {}          men\\
\ex 
\gll {}[\sub{NP} einige [\sub{\nbar} kluge [\sub{\nbar} [\sub{N} [\sub{N} Frauen ] und [\sub{N} Männer
]]]]]\\
	{}       some   {}           smart {}          {}       {}       women  {} and {} men\\
\zl
%
In (\mex{0}a), two nouns have projected to \nbar and have then been joined by coordination. The result of coordination
of two constituents of the same category is always a new constituent with that category. In the case of (\mex{0}a), this
is also \nbar. This constituent is then combined with the adjective and the determiner. In (\mex{0}b), the nouns themselves
have been coordinated. The result of this is always another constituent which has the same category as its parts. In this case,
this would be N. This N becomes \nbar and is then combined with the adjective. If nouns which do not require complements were
categorized as \nbar rather than N, we would not have the problem of spurious ambiguities. The
structure in (\mex{1}) shows the only possible analysis.
\ea
\gll {}[\sub{NP} einige [\sub{\nbar} kluge [\sub{\nbar} [\sub{\nbar} Frauen ] und [\sub{\nbar} Männer
]]]]\\
      {}	some    {}           smart {}          {}           women  {}  and {} men\\
\z

\subsection{Prepositional phrases}
\label{Abschnitt-PP-Syntax}

Compared to the syntax of noun phrases, the syntax of prepositional phrases (PPs) is relatively straightforward. PPs normally 
consist of a preposition and a noun phrase whose case is determined by that preposition. We can capture this with the following
rule:
\ea
\label{Regel-PP-einfach}
PP $\to$ P NP
\z
This rule must, of course, also contain information about the case of the NP. I have omitted this for ease of exposition as I did
with the NP"=rules and AP"=rules above.

The Duden grammar \citep[\S 1300]{Duden2005-Authors} offers examples such as those in (\mex{1}), which show that certain prepositional phrases
serve to further define the semantic contribution of the preposition by indicating some measurement, for example:
\eal
\ex\label{Beispiel-Schritt-vor-dem-Abgrund} 
\gll {}[[Einen Schritt] vor dem Abgrund] blieb er stehen.\\
	 {}\spacebr{}\spacebr{}one step before the abyss remained he stand\\
\glt `He stopped one step in front of the abyss.'
\ex 
\gll {}[[Kurz] nach dem Start] fiel die Klimaanlage aus.\\
	 {}\spacebr{}\spacebr{}shortly after the take.off fell the air.conditioning out\\
\glt `Shortly after take off, the air conditioning stopped working.'
\ex 
\gll {}[[Schräg] hinter der Scheune] ist ein Weiher.\\
	 {}\spacebr{}\spacebr{}diagonally behind the barn is a pond\\
\glt `There is a pond diagonally across from the barn.'
\ex 
\gll {}[[Mitten] im Urwald] stießen die Forscher auf einen alten Tempel.\\
	 {}\spacebr{}\spacebr{}middle in.the jungle stumbled the researchers on an old temple\\
\glt `In the middle of the jungle, the researches came across an old temple.'
\zl
To analyze the sentences in (\mex{0}a,b), one could propose the following rules in (\mex{1}):
\eal
\ex PP $\to$ NP PP
\ex PP $\to$ AP PP
\zl
These rules combine a PP with an indiciation of measurement. The resulting constituent is another PP. It is
possible to use these rules to analyze prepositional phrases in (\mex{-1}a,b), but it unfortunately also allows
us to analyze those in (\mex{1}):
\eal
\ex[*]{
\gll [\sub{PP} einen Schritt [\sub{PP} kurz [\sub{PP} vor dem Abgrund]]]\\
	 {} one step {} shortly {} before the abyss\\
}
\ex[*]{
\gll [\sub{PP} kurz [\sub{PP} einen Schritt [\sub{PP} vor dem Abgrund]]]\\
	 {} shortly {} one step {} before the abyss\\
}
\zl
Both rules in (\mex{-1}) were used to analyze the examples in (\mex{0}). Since the symbol PP occurs on both the left 
and right"=hand side of the rules, we can apply the rules in any order and as many times as we like.
% Fn: Semantik hilft nicht.

We can avoid this undesired side"=effect by reformulating the previously assumed rules:
\eal
\ex PP $\to$ NP \pbar
\ex PP $\to$ AP \pbar
\ex PP $\to$ \pbar\label{Regel-PP-P}
\ex \pbar $\to$ P NP
\zl
Rule (\ref{Regel-PP-einfach}) becomes (\mex{0}d). The rule in (\mex{0}c) states that a PP can consist of \pbar.
Figure~\vref{Abbildung-PP} shows the analysis of (\mex{1}) using (\mex{0}c) and (\mex{0}d) as well as the analysis
of an example with an adjective in the first position following the rules in (\mex{0}b) and (\mex{0}d):
\ea
\gll vor dem Abgrund\\
	 before the abyss\\
\glt `in front of the abyss'
\z
\begin{figure}
\hfill
\begin{forest}
sn edges
[PP
  [\pbar
    [P [vor;before] ]
    [NP [dem Abgrund;the abyss, triangle] ] ] ]
\end{forest}
\hfill
\begin{forest}
sn edges
[PP
  [AP [kurz;shortly,triangle] ]
  [\pbar
    [P [vor;before] ]
    [NP [dem Abgrund;the abyss,triangle] ] ] ]
\end{forest}
%% \begin{tikzpicture}
%% \tikzset{level 1+/.style={level distance=2\baselineskip}}
%% \tikzset{frontier/.style={distance from root=8\baselineskip}}
%% \Tree[.PP
%%        [.{\pbar}
%%          [.P vor;before ]
%%          [.NP \edge[roof]; {dem Abgrund;the abyss} ] ] ] 
%% \end{tikzpicture}
%% \hfill
%% \begin{tikzpicture}
%% \tikzset{level 1+/.style={level distance=2\baselineskip}}
%% \tikzset{frontier/.style={distance from root=8\baselineskip}}
%% \Tree[.PP
%%        [.AP \edge[roof]; {kurz;shortly} ]
%%        [.{\pbar}
%%          [.P vor;before ]
%%          [.NP \edge[roof]; {dem Abgrund;the abyss} ] ] ] 
%% \end{tikzpicture}
\hfill
\mbox{}
\caption{\label{Abbildung-PP}Prepositional phrases with and without measurement}
\end{figure}%

At this point, the attentive reader is probably wondering why there is no empty measurement phrase in
the left figure of Figure~\ref{Abbildung-PP}, which one might expect in analogy to the empty determiner in Figure~\ref{Abbildung-NP-ohne-Det}.
The reason for the empty determiner in Figure~\ref{Abbildung-NP-ohne-Det} is that the entire noun phrase
without the determiner has a meaning similar to those with a determiner. The meaning normally contributed
by the visible determiner has to somehow be incorporated in the structure of the noun phrase. If we
did not place this meaning in the empty determiner, this would lead to more complicated assumptions about semantic
combination: we only really require the mechanisms presented in Section~\ref{sec-PSG-Semantik} and these are
very general in nature. The meaning is contributed by the words themselves and not by any rules. If we were
to assume a unary branching rule such as that in the left tree in Figure~\ref{Abbildung-PP} instead of the
empty determiner, then this unary branching rule would have to provide the semantics of the determiner. This
kind of analysis has also been proposed by some researchers.\todostefan{Martin: obscure} See Chapter~\ref{Abschnitt-Diskussion-leere-Elemente} for
more on empty elements.

Unlike determiner"=less NPs, prepositional phrases without an indication of degree or measurement do
not lack any meaning component for composition. It is therefore not necessary to assume an empty indication of measurement, which
somehow contributes to the meaning of the entire PP. Hence, the rule in (\ref{Regel-PP-P}) states that a
prepositional phrase consists of \pbar, that is, a combination of P and NP.\is{preposition|)}

\section{\xbart}
\label{sec-xbar}

If\is{X theory@\xbar theory|(}  we look again at the rules that we have formulated in the previous section, we see that heads are always 
combined with their complements to form a new constituent (\mex{1}a,b), which can then be combined with further constituents (\mex{1}c,d):

\eal
\ex \nbar $\to$ N PP
\ex \pbar $\to$ P NP
\ex\label{Regel-NP-Xbar}
    NP $\to$ Det \nbar
\ex PP $\to$ NP \pbar
\zl
%
Grammarians working on English\il{English} noticed that parallel structures can be used for phrases which have adjectives or verbs as their head.
I discuss adjective phrases at this point and postpone the discussion of verb phrases to Chapter~\ref{Kapitel-GB}. As in German, certain adjectives 
in English can take complements with the important restriction that adjective phrases with complements cannot realize these pre"=nominally in English. 
(\mex{1}) gives some examples of adjective phrases:
\eal
\ex He is proud.
\ex He is very proud.
\ex He is proud of his son.
\ex He is very proud of his son.
\zl
Unlike prepositional phrases, complements of adjectives are normally optional. \emph{proud} can be used with or without a PP.
The degree expression \emph{very} is also optional.

The rules which we need for this analysis are given in (\mex{1}), with the corresponding structures in Figure~\vref{Abbildung-AP}.

\begin{samepage}
\eal
\ex AP $\to$ \abar
\ex AP $\to$ AdvP \abar
\ex \abar $\to$ A PP
\ex \abar $\to$ A
\zl
\end{samepage}

\begin{figure}
\hfill
\begin{forest}
sn edges
[AP
  [\abar
    [A [proud] ] ] ]
\end{forest}
%% \begin{tikzpicture}
%% \tikzset{level 1+/.style={level distance=2\baselineskip}}
%% \tikzset{frontier/.style={distance from root=6\baselineskip}}
%% \Tree[.AP
%%        [.{\abar}
%%          [.A proud ] ] ]
%% \end{tikzpicture}
\hfill
\begin{forest}
sn edges
[AP
  [AdvP [very] ]
  [\abar
    [A [proud] ] ] ]
\end{forest}
%% \begin{tikzpicture}
%% \tikzset{level 1+/.style={level distance=2\baselineskip}}
%% \tikzset{frontier/.style={distance from root=6\baselineskip}}
%% \Tree[.AP
%%        [.AdvP very ]
%%        [.{\abar}
%%          [.A proud ] ] ]
%% \end{tikzpicture}
\hfill
\begin{forest}
sn edges
[AP
  [\abar
    [A [proud] ]
    [PP [of his son,triangle] ] ] ]
\end{forest}
%% \begin{tikzpicture}
%% \tikzset{level 1+/.style={level distance=2\baselineskip}}
%% \tikzset{frontier/.style={distance from root=6\baselineskip}}
%% \Tree[.AP
%%        [.{\abar}
%%          [.A proud ]
%%          [.PP \edge[roof]; {of his son} ] ] ]
%% \end{tikzpicture}
\hfill
\begin{forest}
sn edges
[AP
  [AdvP [very] ]
  [\abar
    [A [proud] ]
    [PP [of his son,triangle] ] ] ]
\end{forest}
%% \begin{tikzpicture}
%% \tikzset{level 1+/.style={level distance=2\baselineskip}}
%% \tikzset{frontier/.style={distance from root=6\baselineskip}}
%% \Tree[.AP
%%        [.AdvP very ]
%%        [.{\abar}
%%          [.A proud ]
%%          [.PP \edge[roof]; {of his son} ] ] ]
%% \end{tikzpicture}
\hfill
\mbox{}
\caption{\label{Abbildung-AP}English adjective phrases}
\end{figure}%

As was shown in Section~\ref{sec-PSG-Merkmale}, it is possible to generalize over very specific
phrase structure rules and thereby arrive at more general rules. In this way, properties such as
person, number and gender are no longer encoded in the category symbols, but rather only simple
symbols such as NP, Det and N are used. It is only necessary to specify something about the values
of a feature if it is relevant in the context of a given rule. We can take this abstraction a step
further: instead of using explicit category symbols such as N, V, P and A for lexical categories and
NP, VP, PP and AP for phrasal categories, one can simply use a variable for the word class in question and speak of X and XP.

This form of abstraction can be found in so"=called \xbart (or X"=bar theory, the term \emph{bar}
refers to the line above the symbol.), which was developed by \citet{Chomsky70a} and refined by
\citet{Jackendoff77}. This form of abstract rules plays an important role in many different
theories. For example: Government \& Binding\indexgb (Chapter~\ref{Kapitel-GB}), Generalized Phrase
Structure Grammar\indexgpsg (Chapter~\ref{Kapitel-GPSG}) and Lexical Functional Grammar\indexlfg
(Chapter~\ref{Kapitel-LFG}). In HPSG\indexhpsg (Chapter~\ref{Kapitel-HPSG}), \xbar theory also plays
a role, but not all restrictions of the \xbar schema have been adopted.

(\mex{1}) shows a possible instantiation of \xbar rules, where the category X has been used in place of N, as well as examples of word strings
which can be derived by these rules:
%\todostefan{martin meint ich solle hier NP statt Strichnotation verwenden}
\eanoraggedright
\label{psg-xbar-schema}
\begin{tabular}[t]{@{}l@{\hspace{5mm}}l@{\hspace{5mm}}l@{}}
\xbar\mbox{ rule} & \mbox{with specific categories} & \mbox{example strings}\\[2mm]
$\overline{\overline{\mbox{X}}} \rightarrow \overline{\overline{\mbox{specifier}}}$~~\xbar &
$\overline{\overline{\mbox{N}}} \rightarrow \overline{\overline{\mbox{DET}}}$~~\nbar & \mbox{the [picture of Paris]} \\
$\xbar \rightarrow$ \xbar~~$\overline{\overline{\mbox{adjunct}}}$            & \nbar $\rightarrow$ \nbar~~$\overline{\overline{\mbox{REL\_CLAUSE}}}$ & \mbox{[picture of Paris]}\\
                            &                                              & \mbox{[that everybody knows]}\\
\xbar $\rightarrow \overline{\overline{\mbox{adjunct}}}$~~\xbar            & \nbar $\rightarrow \overline{\overline{\mbox{A}}}$~~\nbar & \mbox{beautiful [picture of Paris]}\\
\xbar $\rightarrow$ \mbox{X}~~$\overline{\overline{\mbox{complement}}}*$   & \nbar $\rightarrow$ \mbox{N}~~$\overline{\overline{\mbox{P}}}$ & \mbox{picture [of Paris]}\\
\end{tabular}
\z

Any word class can replace X (\eg V, A or P). The X without the bar stands for a lexical item in
the above rules. If one wants to make the bar level explicit, then it is possible to write \xnull. 
Just as with the rule in (\ref{Regel-mit-Variablen}), where we did not specify the case value of the
determiner or the noun but rather simply required that the values on the right"=hand side of the
rule match, the rules in (\mex{0}) require that the word class of an element on the right"=hand side
of the rule (X or \xbar) matches that of the element on the left"=hand side of the rule (\xbar or
$\overline{\overline{\mbox{X}}}$).

A lexical element can be combined with all its complements\is{complement}. The `*'\is{*} in the last rule stands for
an unlimited amount of repetitions of the symbol it follows. A special case is zero"fold occurrence of complements. There is no
PP complement of \emph{Bild} `picture' present in \emph{das Bild} `the picture' and thus N becomes \nbar. The result of the
combination of a lexical element with its complements is a new projection level of X: the projection level 1, which is marked by
a bar. \xbar can then be combined with adjuncts. These can occur to the left or right of \xbar. The result of this combination is
still \xbar, that is the projection level is not changed by combining it with an adjunct\is{adjunct}.
Maximal projections are marked by two
bars. One can also write XP\is{XP} for a projection of X with two bars. An XP consists of a specifier\is{specifier} and \xbar. Depending
on one's theoretical assumptions, subjects of sentences (\citealp{Haider95b-u,Haider97a}\NOTE{Haider noch mal lesen};
\citealp[Section~3.2.2]{Berman2003a}) and determiners in NPs \citep[\page
210]{Chomsky70a} are specifiers. Furthermore, degree modifiers \citep[\page
210]{Chomsky70a} in adjective phrases and measurement indicators in prepositional phrases are also counted as specifiers.

Non"=head positions can only host maximal projections and therefore complements, adjuncts and specifiers always have two bars. 
Figure~\vref{Abb-GB-Min-Max} gives an overview of the minimal and maximal structure of phrases.
\begin{figure}
\hfill
\begin{forest}
sn edges
[XP
  [\xbar [X] ] ]
\end{forest}
\hfill
\begin{forest}
%where n children=0{}{},
%sn edges
%for tree={parent anchor=south, child anchor=north,align=center,base=bottom}
[XP
  [specifier]
  [\xbar
    [adjunct]
    [\xbar
      [complement] [X] ] ] ]
\end{forest}
\hfill\mbox{}
\caption{\label{Abb-GB-Min-Max}Minimal and maximal structure of phrases}
\end{figure}%

Some categories do not have a specifier or have the option of having one. Adjuncts are optional and therefore
not all structures have to contain an \xbar with an adjunct daughter.
 In addition to the branching shown in the right"=hand figure, adjuncts to
XP and head"=adjuncts\is{adjunct!head} are sometimes possible. There is only a single rule in (\ref{psg-xbar-schema})
for cases in which a head precedes the complements, however an order in which the complement precedes the head is
of course also possible. 
This is shown in Figure~\ref{Abb-GB-Min-Max}.

Figure~\vref{Abb-das-schoene-Bild-von-Paris} shows the analysis of the NP structures \emph{das Bild} `the picture'
and \emph{das schöne Bild von Paris} `the beautiful picture of Paris'. The NP structures in Figure~\ref{Abb-das-schoene-Bild-von-Paris}
and the tree for \emph{proud} in Figure~\ref{Abbildung-AP} show examples of minimally populated structures.
The left tree in Figure~\ref{Abb-das-schoene-Bild-von-Paris} is also an example of a structure without an adjunct. The right"=hand structure
in Figure~\ref{Abb-das-schoene-Bild-von-Paris} is an example for the maximally populated structure:
specifier, adjunct, and complement are present.


\begin{figure}
\hfill
\begin{forest}
sn edges
[NP
  [DetP
    [\detbar
      [Det [das;the] ] ] ]
  [\nbar
    [N [Bild;picture] ] ] ]
\end{forest}
\hfill
\begin{forest}
sn edges
[NP
  [DetP
    [\detbar
      [Det [das;the] ] ] ]
  [\nbar
    [AP
      [\abar
        [A [schöne;beautiful] ] ] ]
    [\nbar
      [N [Bild;picture] ]
      [PP 
        [\pbar
          [P [von;of] ]
          [NP
            [\nbar
              [N [Paris;Paris] ] ] ] ] ] ] ] ]
\end{forest}
%
\hfill\mbox{}
\caption{\label{Abb-das-schoene-Bild-von-Paris}\xbar~analysis of \emph{das Bild} `the picture'
  and \emph{das schöne Bild von Paris} `the beautiful picture of Paris'}
\end{figure}%

The analysis given in Figure~\ref{Abb-das-schoene-Bild-von-Paris} assumes that all non"=heads in a rule are
phrases. One therefore has to assume that there is a determiner phrase even if the determiner is not combined with other elements.
The unary branching of determiners is not elegant but it is consistent.\footnote{
	For an alternative version of \xbar theory which does not assume elaborate structure for determiners see \citew{Muysken82a-andere-anfuehrungszeichen}.
}
The unary branchings for the NP \emph{Paris} in Figure~\ref{Abb-das-schoene-Bild-von-Paris} may also seem somewhat odd, but they actually become more
plausible when one considers more complex noun phrases:
\eal
\ex 
\gll das Paris der dreißiger Jahre\\
	 the Paris of.the thirty years\\
\glt `30's Paris'
\ex 
\gll die Maria aus Hamburg\\
	 the Maria from Hamburg\\
\glt `Maria from Hamburg'
\zl
Unary projections are somewhat inelegant but this should not concern us too much here, as we have
already seen in the discussion of the lexical entries in (\ref{Lexikon-Projektion})
that unary branching nodes can be avoided for the most part and that it is indeed desirable to avoid
such structures. Otherwise, one gets spurious ambiguities. In the following
chapters, we will discuss approaches such as Categorial Grammar and HPSG, which do not assume
unary rules for determiners, adjectives and nouns. 

Furthermore, other \xbar~theoretical assumptions will not be shared by several theories discussed in this book. In particular, the assumption that non"=heads always have
to be maximal projections\is{projection!maximal} will be disregarded. \citet{Pullum85a} and
\citet{KP90a} have shown that the respective theories are not necessarily less restrictive
than theories which adopt a strict version of the \xbar theory. See also the discussion in Section~\ref{sec-Diskussion-X-Bar}.
\is{X theory@\xbar theory|)}

\section*{Comprehension questions}

\begin{enumerate}
\item Why are phrase structure grammars that use only atomic categories inadequate for the description of natural languages?
\item Assuming the grammar in (\ref{psg-binaer}), state which steps (replacing symbols) one has to take to get to the symbol 
	 V in the sentence (\mex{1}).
\ea
\gll er das Buch dem Mann gibt\\
	 he the book the man gives\\
\glt `He gives the book to the man.'
\z
Your answer should resemble the analysis in (\ref{bsp-anwendung-grammatik}).
\item Give a representation of the meaning of (\mex{1}) using predicate logic:
\eal
\ex 
\gll Ulrike kennt Hans.\\
	 Ulrike knows Hans\\
\ex 
\gll Joshi freut sich.\\
	 Joshi is.happy \refl{}\\
\glt `Joshi is happy.'
\zl
\end{enumerate}


\section*{Exercises}

\begin{enumerate}
\item\label{ua-psg-eins}
	On page~\pageref{page-unendlich-viele-grammatiken}, I claimed that there is an infinite number of
	grammars we could use to analyze (\ref{bsp-weil-er-das-buch-dem-mann-gibt}).
	Why is this claim correct?
\item Try to come up with some ways in which we can tell which of these possible grammars is or are the best?

\item\label{uebung-np-empty} A fragment for noun phrase syntax was presented in Section~\ref{sec-psg-np}.
Why is the interaction of the rules in (\mex{1}) problematic?
\eal
\ex NP $\to$ Det \nbar
\ex \nbar $\to$ N
\ex Det $\to$ $\epsilon$
\ex N $\to$ $\epsilon$
\zl

\item Why is it not a good idea to mark \emph{books} as NP in the lexicon?

\item Can you think of some reasons why it is not desirable to assume the following rule for nouns such as \emph{books}:
\ea
NP $\to$ Modifier* books Modifier*
\z
The rule in (\mex{0}) combines an unlimited number of modifiers with the noun \emph{books} followed by an unlimited number
of modifiers. We can use this rule to derive phrases such as those in (\mex{1}):
\eal
\ex books
\ex interesting books
\ex interesting books from Stuttgart\\
\zl
Make reference to coordination data in your answer. Assume that symmetric coordination\is{coordination} requires that both
coordinated phrases or words have the same syntactic category.

\item \citet{FLGR2012a} suggested treating nounless structures like those in (\mex{1}) as involving
  a phrasal construction that combines the determiner \emph{the} with an adjective.
\eal
\ex Examine the plight of the very poor.
\ex Their outfits range from the flamboyant to the functional.
\ex The unimaginable happened.
\zl
(\mex{1}) shows a phrase structure rule that corresponds to their construction:
\ea
NP $\to$ the Adj
\z
Adj stands for something that can be a single word like \emph{poor} or complex like \emph{very
  poor}.

Revisit the German data in (\ref{ex-nounless-np}) and (\ref{ex-nounless-np-relational-noun})  and
explain why such an analysis and even a more general one as in (\mex{1}) would
not extend to German.
\ea
NP $\to$ Det Adj
\z

\item Why can \xbart not account for German adjective phrases without additional assumptions? (This
  task is for (native) speakers of German only.)

\item Come up with a phrase structure grammar that can be used to analyze the sentence in (\mex{1}), but also
rules out the sentences in (\mex{2}).

      \eal
      \ex[]{
      \gll Der Mann hilft der Frau.\\
           the.\nom{} man helps the.\dat{} woman\\
      \glt `The man helps the woman.'
      }
      \ex[]{
      \gll Er gibt ihr das Buch.\\
           he.\nom{} gives her.\dat{} the book\\
      \glt `He gives her the book.'
      }
      \ex[]{
      \gll Er wartet auf ein Wunder.\\
           he.\nom{} waits on a miracle\\
	  \glt `He is waiting for a miracle.'
      }
%      \ex[]{
%       Er wartet neben dem Bushäuschen auf ein Wunder.
%       }
      \zl%\enlargethispage{1\baselineskip}
      \eal
      \ex[*]{
      \gll  Der Mann hilft er.\\
		    the.\nom{} man helps he.\nom{}\\
      }
      \ex[*]{
      \gll  Er gibt ihr den Buch.\\
            he.\nom{} gives her.\dat{} the.\mas{} book.\neu{}\\
      }
      \zl
\item Consider which additional rules would have to be added to the grammar you developed in the previous exercise
	  in order to be able to analyze the sentences in (\mex{1}):

\eal
\ex 
\gll Der Mann hilft der Frau jetzt.\\
     the.\nom{} man helps the.\dat{} woman now\\
\glt `The man helps the woman now.'
\ex 
\gll Der Mann hilft der Frau neben dem Bushäuschen.\\
     the.\nom{} man helps the.\dat{} woman next to.the bus.shelter\\
\glt `The man helps the woman next to the bus shelter.'
\ex 
\gll Er gibt ihr das Buch jetzt.\\
     he.\nom{} gives her.\dat{} the.\acc{} book now\\
\glt 'He gives her the book now.'
\ex 
\gll Er gibt ihr das Buch neben dem Bushäuschen.\\
     he.\nom{} gives her.\dat{} the.\acc{} book next to.the bus.shelter\\
\glt `He gives her the book next to the bus shelter.'
\ex 
\gll Er wartet jetzt auf ein Wunder.\\
     he.\nom{} waits now on a miracle\\
\glt `He is waiting for a miracle now.'
\ex 
\gll Er wartet neben dem Bushäuschen auf ein Wunder.\\
     he.\nom{} waits next to.the.\dat{} bus.shelter on a miracle\\
\glt `He is waiting for a miracle next to the bus shelter.'
\zl
\item Install a Prolog system (\eg SWI"=Prolog\footnote{
\url{http://www.swi-prolog.org/} 
})
and try out your grammar. Details for the notation can be found in the corresponding handbook
under the key word Definite Clause Grammar (DCG)\is{Definite Clause Grammar (DCG)}.
\end{enumerate}


\section*{Further reading}

The expansion of phrase structure grammars to include features was proposed as early as 1963 by \citet{Harman63a}.

The phrase structure grammar for noun phrases discussed in this chapter covers a large part of the syntax
of noun phrases but cannot explain certain NP structures. Furthermore, it has the problem, which exercise~\ref{uebung-np-empty}
is designed to show. A discussion of these phenomena and a solution in the framework of HPSG can be found in \citew{Netter98a} and \citew{Kiss2005a}.

The discussion of the integration of semantic information into phrase structure grammars was very short. A detailed discussion of predicate logic and
its integration into phrase structure grammars -- as well as a discussion of quantifier scope -- can be found in \citew{BB2005a}. 


%      <!-- Local IspellDict: en_US-w_accents -->
