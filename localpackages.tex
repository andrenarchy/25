%\usepackage{bigfoot}


% http://tex.stackexchange.com/questions/38607/no-room-for-a-new-dimen
\usepackage{etex}\reserveinserts{28}

% http://tex.stackexchange.com/questions/229500/tikzmark-and-xelatex
% temporary fix, remove later
%\newcount\pdftexversion \pdftexversion140 \def\pgfsysdriver{pgfsys-dvipdfm.def} \usepackage{tikz} \usetikzlibrary{tikzmark}

%\usepackage[section]{placeins}

\usepackage{eurosym} % should go once Berthold fixes unicode


% http://tex.stackexchange.com/questions/284097/subscript-like-math-but-without-the-minus-sign?noredirect=1#comment685345_284097
% for subscripts
\usepackage{amsmath}
%\usepackage{unicode-math} breaks the CCG derivations, the horizontal lines are too high


% \justify to switch of \raggedright in translations
%\usepackage{ragged2e}

% Haitao Liu
\usepackage{xeCJK}
\setCJKmainfont{SimSun}



\hypersetup{bookmarksopenlevel=0}

%% now loaded by the langsci class
%% \iftoggle{draft}{
%% \usepackage{todonotes}
%% }{
%% \usepackage[disable]{todonotes}
%% }

\iftoggle{draft}{}{
\presetkeys{todonotes}{disable}{}
}


\usepackage{metalogo} % xelatex

\usepackage{multicol}

\usepackage{bookmark}

\usepackage{styles/my-ccg-ohne-colortbl}

\usepackage{langsci/styles/jambox}

% This has side effects on my-ccg commands do no know why
%\usepackage{langsci/styles/langsci-optional}

\usepackage{styles/oneline}

\usepackage{langsci/styles/langsci-lgr}

\usepackage{graphicx}



\usepackage{lastpage,float,soul,tabularx}




\usepackage{ogonek}        % For Ewa Dabrowska


\usepackage{mycommands}% \spacebr


% still needed
% for direction of government examples
% for CoreGram classes
% and one \Tree in mts.tex
\usepackage{tikz-qtree}

\usepackage{langsci/styles/langsci-gb4e}


\usepackage{subfig}

%\renewcommand{\xbar}{X̅\xspace}


% for reasons I do not understand this cannot be moved further down and 
% the loading of forest further down cannot be removed. St. Mü. 26.01.2017
% It breaks the dependency grammar trees in forest.
\usepackage{langsci/styles/langsci-forest-setup}


\forestset{
      terminus/.style={tier=word, for children={tier=tabular}, for tree={fit=band}, for descendants={no path, align=left, l sep=0pt}},
      sn edges original/.style={for tree={parent anchor=south, child anchor=north,align=center,base=top}},
      no path/.style={edge path={}},
      set me left/.style={calign with current edge, child anchor=north west, for parent={parent anchor=south west}},
}





% has to be loaded after forest-setup because of incompatibilities with the dg-style.
\usepackage{german}\selectlanguage{USenglish}

\usepackage{styles/makros.2e,styles/article-ex,styles/additional-langsci-index-shortcuts,
styles/eng-date,styles/my-theorems}

% loaded in macros.2e \usepackage[english]{varioref}
% do not stop and warn! This will be tested in the final version
%\vrefwarning


\setcounter{secnumdepth}{4}


\usepackage{dgmacros,pst-tree,trees,dalrymple} % Mary Dalrymples macros


%%% trick for using adjustbox
\let\pstricksclipbox\clipbox
\let\clipbox\relax

% http://tex.stackexchange.com/questions/206728/aligning-several-forest-trees-in-centered-way/206731#206731
% for aligning TAG trees
\usepackage[export]{adjustbox}

% draw a grid for getting the coordinates
\usepackage{tikz-grid}

% for offsets in trees
\newlength{\offset}
\newlength{\offsetup}

\ifxetex
\usepackage{styles/eng-hyp-utf8}
\else
\usepackage{styles/eng-hyp}
\fi

\usepackage{appendix}


% adds lines to both the odd and even page.
% bloddy hell! This is really an alpha package! Do not use the draft option! 07.03.2016
\usepackage{addlines}

%% \let\addlinesold=\addlines
%% % there is one optional argument. Second element in brackets is the default 
%% \renewcommand{\addlines}[1][1]{
%% \todosatz{addlines}
%% \addlinesold[#1]
%% }


% do nothing now
%\let\addlinesold=\addlines
%\renewcommand{\addlines}[1][1]{}

% for addlines to work
\strictpagecheck



% http://tex.stackexchange.com/questions/3223/subscripts-for-primed-variables
%
% to get 
% {}[ af   [~]\sub{V} ]\sub{V$'$}
%
% typeset properly. Thanks, Sebastian.
%
\usepackage{subdepth}


%\usepackage{caption}













