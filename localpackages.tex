

% http://tex.stackexchange.com/questions/38607/no-room-for-a-new-dimen
\usepackage{etex}\reserveinserts{28}

% http://tex.stackexchange.com/questions/229500/tikzmark-and-xelatex
% temporary fix, remove later
%\newcount\pdftexversion \pdftexversion140 \def\pgfsysdriver{pgfsys-dvipdfm.def} \usepackage{tikz} \usetikzlibrary{tikzmark}

%\usepackage[section]{placeins}

\usepackage{eurosym} % should go once Berthold fixes unicode


% http://tex.stackexchange.com/questions/284097/subscript-like-math-but-without-the-minus-sign?noredirect=1#comment685345_284097
% for subscripts
\usepackage{amsmath}
\usepackage{unicode-math} 


% \justify to switch of \raggedright in translations
\usepackage{ragged2e}

% Haitao Liu
%\usepackage{xeCJK}
%\setCJKmainfont{SimSun}



% http://tex.stackexchange.com/questions/38607/no-room-for-a-new-dimen
\usepackage{etex}\reserveinserts{28}

% http://tex.stackexchange.com/questions/229500/tikzmark-and-xelatex
% temporary fix, remove later
%\newcount\pdftexversion \pdftexversion140 \def\pgfsysdriver{pgfsys-dvipdfm.def} \usepackage{tikz} \usetikzlibrary{tikzmark}

%\usepackage[section]{placeins}

\usepackage{eurosym} % should go once Berthold fixes unicode

% http://tex.stackexchange.com/questions/284097/subscript-like-math-but-without-the-minus-sign?noredirect=1#comment685345_284097
% for subscripts
\usepackage{amsmath}
\usepackage{unicode-math} 


% \justify to switch of \raggedright in translations
\usepackage{ragged2e}

% Haitao Liu
%\usepackage{xeCJK}
%\setCJKmainfont{SimSun}


\hypersetup{bookmarksopenlevel=0}

\iftoggle{draft}{
\usepackage{todonotes}
}{
\usepackage[disable]{todonotes}
}



\usepackage{metalogo} % xelatex

\usepackage{multicol}

\usepackage{LSP/lsp-styles/lsp-forest-setup}

\usepackage{bookmark}

\forestset{
      terminus/.style={tier=word, for children={tier=tabular}, for tree={fit=band}, for descendants={no path, align=left, l sep=0pt}},
      sn edges original/.style={for tree={parent anchor=south, child anchor=north,align=center,base=top}},
      no path/.style={edge path={}},
      set me left/.style={calign with current edge, child anchor=north west, for parent={parent anchor=south west}},
}

% 
% \usepackage{my-ccg-ohne-colortbl}


\usepackage{LSP/lsp-styles/jambox}



\usepackage{german}\selectlanguage{USenglish}


\usepackage[final]{epsfig}
\usepackage{graphicx}


\usepackage{makros.2e,article-ex,additional-langsci-index-shortcuts,
eng-date,
lastpage,float,comment,my-theorems,soul,tabularx}

% loaded in macros.2e \usepackage[english]{varioref}
% do not stop and warn! This will be tested in the final version
\vrefwarning


\usepackage{ogonek}        % For Ewa Dabrowska


\usepackage{mycommands}% \dash


% still needed
\usepackage{tikz-qtree}

\usepackage{LSP/lsp-styles/lsp-gb4e}


\usepackage{subfig}

%\renewcommand{\xbar}{X̅\xspace}


% should be removed once that the \Tree figures are removed
\usepackage{forest}

\usepackage{pstricks,pst-node}

%\nodemargin5pt%\treelinewidth2pt\arrowwidth6pt\arrowlength10pt
\psset{nodesep=5pt} %,linewidth=0.8pt,arrowscale=2}
\psset{linewidth=0.5pt}
\setcounter{secnumdepth}{4}


\usepackage{dgmacros,pst-tree,trees,dalrymple} % Mary Dalrymples macros


%%% trick for using adjustbox
\let\pstricksclipbox\clipbox
\let\clipbox\relax

% http://tex.stackexchange.com/questions/206728/aligning-several-forest-trees-in-centered-way/206731#206731
% for aligning TAG trees
\usepackage[export]{adjustbox}

% draw a grid for getting the coordinates
\usepackage{tikz-grid}

% for offsets in trees
\newlength{\offset}
\newlength{\offsetup}

\ifxetex
\usepackage{eng-hyp-utf8}
\else
\usepackage{eng-hyp}
\fi

\usepackage{appendix}






















