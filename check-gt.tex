%%%%%%%%%%%%%%%%%%%%%%%%%%%%%%%%%%%%%%%%%%%%%%%%%%%%%%%%%
%%   $RCSfile: grammatiktheorie.tex,v $
%%  $Revision: 1.3 $
%%      $Date: 2010/01/18 14:55:27 $
%%     Author: Stefan Mueller (CL Uni-Bremen)
%%    Purpose: 
%%   Language: LaTeX
%%%%%%%%%%%%%%%%%%%%%%%%%%%%%%%%%%%%%%%%%%%%%%%%%%%%%%%%%



\documentclass[ ,output=long    % long|short|inprep              
	        %,blackandwhite
	        ,smallfont
                ,bibtex
%	        ,draftmode  
%                ,showindex
%                ,openreview
%                ,copyright=CC-BY-ND
		  ]{LSP/langsci}                          



\usepackage{ifthen}
\provideboolean{draft}
\setboolean{draft}{true}

\usepackage{etoolbox}
\newtoggle{draft}\toggletrue{draft}

%\newcommand{\NOTE}[1]{}
\newcommand{\NOTE}[1]{\marginpar{#1}}

\newcommand{\LATER}[1]{}

\includeonly{1-begriffe}
%\includeonly{1-begriffe,2-psg,3-tg,4-gpsg,5-merkmalstrukturen,6-lfg,7-cg,8-hpsg,9-cxg,tag}
%\includeonly{2-psg}
%\includeonly{3-gb}
%\includeonly{3-minimalism}
%\includeonly{3-gb,3-minimalism}
%\includeonly{4-gpsg}
%\includeonly{5-merkmalstrukturen}
%\includeonly{6-lfg}
%\includeonly{7-cg}
%\includeonly{8-hpsg}
%\includeonly{none}
%\includeonly{9-cxg}
%\includeonly{dg}
%\includeonly{tag}
%\includeonly{innateness}
%\includeonly{mts}
%\includeonly{competence}
%\includeonly{acquisition}
%\includeonly{complexity}
%\includeonly{branching-locality-recursion}
%\includeonly{empty}
%% \includeonly{movement}
%\includeonly{phrasal}

%\includeonly{coregram}
%\includeonly{conclusions}
%\includeonly{nothing}

%\includeonly{movement-scrambling-passive}

%\includeonly{loesungen}

\input grammatical-theory-include


DG:http://ufal.mff.cuni.cz/dg/dgmain.html

Mel'cuk, I. (1988): Dependency Syntax: Theory and Practice, New York: State University of New York Press.

Hudson, R. (1990): Recent Trends in Dependency Syntax. In: Handbücher zur Sprach- und Kommunikationswissenschaft, de Gruyter, pp.329--338.

Heringer, H.-J. (1990): Formalized Models. In: Handbücher zur Sprach- und Kommunikationswissenschaft, de Gruyter, pp.316--328.

Weber, H.J. (1997): Dependenzgrammatik. Ein interaktives Arbeitsbuch, Tübingen: Gunter Narr. 


