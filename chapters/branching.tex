%% -*- coding:utf-8 -*-

\section{Binary branching}
\label{sec-branching}

We\is{branching!binary|(} have seen that the question of the kind of branching structures assumed has received differing treatments in various theories.
Classical \xbart assumes that a verb is combined with all its complements. In later variants of GB, all structures are strictly binary branching.
Other frameworks do not treat the question of branching in a uniform way: there are proposals that assume binary branching structures and others
that opt for flat structures.

\citet[Section~2.5]{Haegeman94a-u} uses learnability arguments\is{language acquisition} (rate of acquisition, see Section~\ref{Abschnitt-Geschwindigkeit-Spracherwerb}
on this point).
She discusses the example in (\mex{1}) and claims that language learners have to choose one of eight structures if flat-branching structures can occur in natural
language. If, on the other hand, there are only binary-branching structures, then the sentence in (\mex{1}) cannot have the structures in
Figure~\vref{Abbildung-Haegmann-flach} to start with, and therefore a learner would not have to rule out the corresponding hypotheses.
\ea 
Mummy must leave now.
\z
\begin{figure}
\begin{forest}
sn edges, empty nodes
[{}
 [{} [Mummy]]
 [{} [must]]
 [{} [leave]]
 [{} [now]]]
\end{forest}
%
\hfill
\begin{forest}
sn edges, empty nodes
[{} 
 [{} [{} [Mummy]]
     [{} [must]]
     [{} [leave]]]
 [{} [now]]]
\end{forest}
\hfill
\begin{forest}
sn edges, empty nodes
[{} 
 [{} [Mummy]]
 [{} 
     [{} [must]]
     [{} [leave]]
     [{} [now]]]]
\end{forest}
\caption{\label{Abbildung-Haegmann-flach}Structures with partial flat-branching}
\end{figure}%

\noindent
However, \citet[\page 88]{Haegeman94a-u} provides evidence for the fact that (\mex{0}) has the structure in (\mex{1}):
\ea
{}[Mummy [must [leave now]]]
\z
The relevant tests showing this include elliptical constructions\is{ellipsis}, that is, the fact that it is possible to
refer to the constituents in (\mex{0}) with pronouns. This means that there is actually evidence for
the structure of (\mex{-1}) that is assumed by linguists and we therefore do not have to assume that
it is just hard-wired in our brains that only binary-branching structures are allowed. \citet[\page
  143]{Haegeman94a-u} mentions a consequence of the binary branching hypothesis: if all structures are
binary-branching, then it is not possible to straightforwardly account for sentences with
ditransitive verbs in \xbart. In \xbart, it is assumed that a head is combined with all its
complements at once (see Section~\ref{sec-xbar}). So in order to account for ditransitive verbs in
\xbart, an empty element\is{empty element} (\littlev) has to be assumed (see Section~\ref{sec-little-v}).

It should have become clear in the discussion of the arguments for the Poverty of the Stimulus in Section~\ref{Abschnitt-PSA} that
the assumption that only binary-branching structures are possible is part of our innate linguistic knowledge is nothing more than pure
speculation. Haegeman offers no kind of evidence for this assumption. As shown in the discussions of the various theories we have seen,  
it is possible to capture the data with flat structures. For example, it is possible to assume that, in English, the verb
is combined with its complements in a flat structure \citep[\page 39]{ps2}. There are sometimes theory-internal reasons for
deciding for one kind of branching or another, but these are not always applicable to other theories. For example, Binding Theory\is{Binding Theory}
in \gbt is formulated with reference to dominance relations in trees \citep[\page 188]{Chomsky81a}. If one assumes that syntactic structure plays
a crucial role for the binding of pronouns (see page~\pageref{Seite-Bindungstheorie}), then it is possible to make assumptions about syntactic
structure based on the observable binding relations. Binding data have, however, received a very different treatment in various theories.
In LFG\indexlfg, constraints on f"=structure\is{f"=structure} are used for Binding Theory \citep{Dalrymple93a}, whereas Binding Theory
in HPSG\indexhpsg operates on argument structure lists\isfeat{arg-st} (valence information that are ordered in a particular way,
see Section~\ref{Abschnitt-Arg-St}).
 
The opposite of Haegeman's position is the argumentation for flat structures put forward by Croft
\citeyearpar[Section~1.6.2]{Croft2001a}. In his\indexcxg Radical Construction Grammar FAQ, Croft observes that
a phrasal construction such as the one in (\mex{1}a) can be translated into a Categorial Grammar
lexical entry\indexcg like (\mex{1}b).
\eal
\ex {}[\sub{VP} V NP ]
\ex VP/NP
\zl
He claims that a disadvantage of Categorial Grammar is that it only allows for binary-branching structures and yet there exist constructions
with more than two parts (p.\,49). The exact reason why this is a problem is not explained, however. He even acknowledges himself that
it is possible to represent constructions with more than two arguments in Categorial Grammar. For a ditransitive verb, the entry in Categorial
Grammar of English would take the form of (\mex{1}):
\ea
((s\bs np)/np)/np
\z
If we consider the elementary trees for TAG in Figure~\vref{Abbildung-TAG-flach-binaer}, it becomes clear that it is equally possible
to incorporate semantic information into a flat tree and a binary-branching tree.
\begin{figure}
\hfill
\adjustbox{valign=c}{%
\begin{forest}
tag
[S
	[NP$\downarrow$]
	[VP
		[V
			[gives]]
		[NP$\downarrow$]
		[NP$\downarrow$]]]
\end{forest}
}
\hfill
\adjustbox{valign=c}{%
\begin{forest}
tag
[S
	[NP$\downarrow$]
	[VP
		[V$'$
			[V
				[gives]]
			[NP$\downarrow$]]
		[NP$\downarrow$]]]
\end{forest}}
\hfill\mbox{}
\caption{\label{Abbildung-TAG-flach-binaer}Flat and binary-branching elementary trees}
\end{figure}%
The binary-branching tree corresponds to a Categorial Grammar derivation. In both analyses in 
Figure~\ref{Abbildung-TAG-flach-binaer}, a meaning is assigned to a head that occurs with a certain
number of arguments. Ultimately, the exact structure required depends on the kinds of restrictions on structures
that one wishes to formulate.
In this book, such restrictions are not discussed, but we have seen some theories model binding relations\is{Binding Theory}
with reference to tree structures. Reflexive pronouns\is{pronoun!reflexive} must be bound within a particular local domain inside the
tree. In theories such as LFG\indexlfg and HPSG\indexhpsg, these binding restrictions are formulated
without any reference to trees.
%\todostefan{Das stand irgendwie schon oben. Vielleicht ist aber ein
%  bisschen Redundanz auch OK.}
This means that evidence from binding data for one of the structures in Figure~\ref{Abbildung-TAG-flach-binaer} (or for
other tree structures) constitutes nothing more than theory-internal evidence.

Another reason to assume trees with more structure is the possibility to insert adjuncts\is{adjunct} on any node.
In Chapter~\ref{Kapitel-HPSG}, an HPSG analysis for German that assumes binary-branching structures was proposed.
With this analysis, it is possible to attach an adjunct to any node and thereby explain the free ordering of adjuncts
in the middle field:
\eal
\ex 
\gll {}[weil] der Mann der Frau das Buch \emph{gestern} gab\\
	 {}\spacebr{}because the man the woman the book yesterday gave\\
\glt `because the man gave the woman the book yesterday'	 
\ex 
\gll {}[weil] der Mann der Frau \emph{gestern} das Buch gab\\
	 {}\spacebr{}because the man the woman yesterday the book gave\\
\ex 
\gll {}[weil] der Mann \emph{gestern} der Frau das Buch gab\\
	 {}\spacebr{}because the man yesterday the woman the book gave\\
\ex 
\gll {}[weil] \emph{gestern} der Mann der Frau das Buch gab\\
	 {}\spacebr{}because yesterday the man the woman the book gave\\
\zl
This analysis is not the only one possible, however. One could also assume an entirely flat structure where arguments
and adjuncts are dominated by one node. \citet{Kasper94a} suggests this kind of analysis in
\hpsg (see also Section~\ref{Abschnitt-Adjunkte-GPSG} for GPSG analyses that make use of metarules\is{metarule} for the introduction of adjuncts). Kasper requires complex relational constraints\is{relation} that
create syntactic relations between elements in the tree and also compute the semantic contribution of the entire constituent using the meaning
of both the verb and the adjuncts. The analysis with binary-branching structures is simpler than those with complex relational constraints and --
in the absence of theory-external evidence for flat structures -- should be preferred to the analysis with flat structures.
At this point, one could object that adjuncts in English cannot occur in all positions between arguments and therefore the binary-branching
Categorial Grammar analysis and the TAG analysis in Figure~\ref{Abbildung-TAG-flach-binaer} are wrong. This is not correct, however, as it is
the specification of adjuncts with regard to the adjunction site that is crucial in Categorial Grammar.
An adverb has the category (s\bs np)\bs (s\bs np) or (s\bs np)/(s\bs np) and can therefore only be combined with constituents that correspond to the VP node in
Figure~\ref{Abbildung-TAG-flach-binaer}. In the same way, an elementary tree for an adverb in TAG
can only attach to the VP node (see Figure~\ref{abb-Adjunktion} on
page~\pageref{abb-Adjunktion}). For the treatment of adjuncts in English, binary-branching
structures therefore do not make any incorrect predictions.
\is{branching!binary|)}


%      <!-- Local IspellDict: en_US-w_accents -->
