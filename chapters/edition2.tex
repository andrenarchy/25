%% -*- coding:utf-8 -*-

\section*{Forword of the second edition}

I want to thank Wang Lulu for pointing out some typos that she found while translating the book to
Chinese. Thanks for both the typos and the translation.

Fritz Hamm noticed that the definition of Intervention was incomplete and pointed out some
inconsistencies in translations of predicates in Section~\ref{sec-PSG-Semantik}.

% In GB-Kapitel what = roof
%
I turned some straight lines in Chapter~\ref{chap-GB} into triangles and added a discussion of
different ways to represent movement.

%Hamm: add Heim/Kratzer, Quantifier-Movement erklären

I extended the discussion of Pirahã in Section~\ref{sec-recursion-empirical-problems} and added
lexical items that show that Pirahã-like modification without recursion can be captured in a
straightforward way in Categorial Grammar. 

Sašo Živanović helped adapting version 2.0 of the \texttt{forest} package so that it could be used
with this large book. I am very graceful for this nice tree typesetting package and all the work
that went into it.

% SpecIP Begriff erklärt. Fußnote zu Spec als Label in Bäumen.

% Added Riemsdijk78:148 as first reference against the toolbox approach to UG.

% Bresnan94a zitiert: Dutch (Maling & Zaenen 1978, Perlmutter & Zaenen 1984), Icelandic and Faroese (Platzack 1987)
%Similarly, \citet[Section~4]{Safir85a-u} assumes that impersonal passives\is{passive!impersonal} in


~\medskip

\noindent
Berlin, \today\hfill Stefan Müller



%      <!-- Local IspellDict: en_US-w_accents -->
