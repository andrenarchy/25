%% -*- coding:utf-8 -*-

\section*{Forword of the second edition}

The first edition of this book was published almost exactly two years ago. The book has app.\,15,000
downloads and is used for teaching and in research all over the world. This is what every author and
every teacher dreams of: distribution of knowledge and accessibility for everybody. The foreword of
the first edition ends with a description of Language Science Press in 2016. This is the situation
now:\footnote{
  See \url{http://userblogs.fu-berlin.de/langsci-press/2018/01/18/achievements-2017/} for the
  details and graphics.
} We have 324 expressions of interest and 58 published books. Books are published in 20 book series with 263 members of editorial boards from 44
different countries from six continents. We have a total of 175,000 downloads. 138 linguists from
all over the world have participated in proofreading. There are currently 296 proofreaders
registered with Language Science Press. Language Science Press is a community"=based publisher, but
there is one person who manages everything: Sebastian Nordhoff\aimention{Sebastian Nordhoff}. His
position has to be payed. We were successful in acquiring financial support by almost 100 academic institutions including 
Harvard, the MIT, and Berkeley.\footnote{
  A full list of supporting institutions is available here:
  \url{http://langsci-press.org/knowledgeunlatched}.
}
If you want to support us by just signing the list of supporters, by publishing with us, by helping as proofreader or by
convincing your librarian/institution to support Language Science Press financially, please refer to \url{http://langsci-press.org/supportUs}.



After these more general remarks concerning Language Science Press I describe the changes I made for the second edition and I
thank those who pointed out mistakes and provided feedback.

I want to thank Wang Lulu for pointing out some typos that she found while translating the book to
Chinese. Thanks for both the typos and the translation.

Fritz Hamm noticed that the definition of Intervention (see p.\,\pageref{def-intervention}) was
incomplete and pointed out some inconsistencies in translations of predicates in
Section~\ref{sec-PSG-Semantik}. 
%
% Added 3, sg in the description of liest in Fig 1.4.
%
% In GB-Kapitel what = roof
%
I turned some straight lines in Chapter~\ref{chap-GB} into triangles and added a discussion of
different ways to represent movement (see Figure~\ref{fig-traces} on p.\,\pageref{fig-traces}).
%
% SpecIP Begriff erklärt. Fußnote zu Spec als Label in Bäumen.
I now explain what SpecIP stands for and I added footnote~\ref{fn-specxp-in-trees} on SpecIP as label in trees.
%
%Hamm: add Heim/Kratzer, Quantifier-Movement erklären (todo)
%
I extended the discussion of Pirahã in Section~\ref{sec-recursion-empirical-problems} and added
lexical items that show that Pirahã-like modification without recursion can be captured in a
straightforward way in Categorial Grammar. 

I reorganized the HPSG chapter to be in line with more recent approaches assuming the valence
featuers \spr and \comps \citep{Sag97a,MuellerGermanic} rather than a single valence feature.
% Removed section on LOCAL in SBCG since not the complete sign is shared anyway. It is only SYN that
% is shared.
I removed the section on the \localf in Sign-based Construction Grammar (Section~10.6.2.2 in the
first edition) since it was build on the wrong assumption that the filler would be identical to the
representation in the valence specification. In \citet[\page 536]{Sag2012a} only the information in
\textsc{syn} and \textsc{sem} is shared. 

I added the example (\ref{ex-ergert-aan-aergert-ueber}) on
page~\pageref{ex-ergert-aan-aergert-ueber} that shows a difference in choice of preposition in a
prepositionsl object in Dutch vs.\ German. Since the publication of the first English edition of the
Grammatical Theory textbook I worked extensively on the phrasal approach to benefactive
constructions in LFG \citep*{AGT2014a}. Section~\ref{sec-phrasal-LFG} was
revised and adapted to what will be published as \citew{MuellerLFGphrasal}.
There is now a brief chapter on complex predicates in TAG
and Categorial Grammar/HPSG (Chapter~\ref{chap-potenital-structure}), that shows that valence"=based
approaches allow for an underspecification of structure. Valence is potential structure, while
theories like TAG operate with actual structure.  

Apart from this I fixed several minor typos, added and updated some references and URLs. Thanks to
Philippa Cook and Timm Lichte for pointing out typos. Thanks to 
Leonel Figueiredo de Alencar\aimention{de Alencar, Leonel},
John Carroll\aimention{John Carroll},
Alexander Koller\aimention{Alexander Koller},
Emily M.\ Bender\aimention{Emily M. Bender},
and
Glenn C.\ Slayden\aimention{Slayden, Glenn C}
for pointers to literature.
%
Sašo Živanović\aimention{Sašo Živanović} helped adapting version 2.0 of
the \texttt{forest} package so that it could be used with this large book. I am very graceful for
this nice tree typesetting package and all the work that went into it.

The source code of the book and the version history is available on GitHub. Issues can be reported
there: \url{https://github.com/langsci/25}. The book is also available on paperhive, a platform for
collective reading and annotation: \url{https://paperhive.org/documents/tEbso9TaFoeS}. It would be
great if you would leave comments there. 

% Added Riemsdijk78:148 as first reference against the toolbox approach to UG.

% Bresnan94a zitiert: Dutch (Maling & Zaenen 1978, Perlmutter & Zaenen 1984), Icelandic and Faroese (Platzack 1987)
%Similarly, \citet[Section~4]{Safir85a-u} assumes that impersonal passives\is{passive!impersonal} in

% Philippa two typos

% Timm Lichte: Transformation kommt bei Chomsky so nicht vor: Übersetzungsfehler

% 18.06.2017
% removed [4] and [5] in schemata with MOTHER feature and translation to PSG
%
% updated URLs

% Mention Borsely on empty elements in Welsh and Arabic

% Changed reference to Dalrymple2006a to cite page number rather than sections since Elsevier does
% not provide section numbers.

% Added reference to Sag2018a paper on Auxiliaries in English

~\medskip

\noindent
Berlin, \today\hfill Stefan Müller



%      <!-- Local IspellDict: en_US-w_accents -->
