\section*{On the way this book is published}

I started to work on my dissertation in 1994 and defended it in 1997. During the whole time the
manuscript was available on my web page. After the defense, I had to look for a publisher. I was
quite happy to be accepted to the series \emph{Linguistische Arbeiten} by Niemeyer, but at the same time I
was shocked about the price, which was 186.00 DM for a paperback book that was written and typeset
by me without any help by the publisher (twenty times the price of a paperback novel).\footnote{
  As a side remark: in the meantime Niemeyer was bought by de Gruyter and closed down. The price of the book is now
  139.95 \euro / \$ 196.00. The price in Euro corresponds to 273.72 DM. 
%This is a price increase of 47\,\%.
} This
basically meant that my book was depublished: until 1998 it was available from my web page and after
this it was available in libraries only. My Habilitationsschrift was published by CSLI Publications
for a much more reasonable price. When I started writing textbooks, I was looking for alternative
distribution channels and started to negotiate with no-name print on demand publishers. Brigitte Narr,
who runs the Stauffenburg publishing house, convinced me to publish my HPSG textbook with her. The
copyrights for the German version of the book remained with me so that I could publish it on my web page. The collaboration was successful so that I also published my second textbook about
grammatical theory with Stauffenburg. I think that this book has a broader relevance and should be
accessible for non-German-speaking readers as well. I therefore decided to have it translated into
English. Since Stauffenburg is focused on books in German, I had to look for another publisher. Fortunately the situation in the publishing sector changed quite dramatically in comparison
to 1997: we now have high profile publishers with strict peer review that are entirely open access. I am very
glad about the fact that Brigitte Narr sold the rights of my book back to me and that I can now 
publish the English version with Language Science Press under a CC-BY license.

%      <!-- Local IspellDict: en_US-w_accents -->
