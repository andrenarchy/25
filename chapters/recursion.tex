%% -*- coding:utf-8 -*-
\section{Recursion}
\label{sec-recursion}

Every\is{recursion|(} theory in this book can deal with self"=embedding in language as it was
discussed on page~\pageref{ex-that-max-thinks-that-recursion}. The example
(\ref{ex-that-max-thinks-that-recursion}) is repeated here as (\mex{1}):
\ea
\label{ex-that-max-thinks-that-recursion-two}
that Max thinks [that Julia knows [that Otto claims [that Karl
suspects [that Richard confirms [that Friederike is laughing]]]]]
\z
Most theories
capture this directly with recursive phrase structure rules or dominance schemata. TAG\indextag is
special with regard to recursion since recursion is factored out of the trees. The corresponding
effects are created by an adjunction operation that allows any amount of material to be inserted
into trees.  It is sometimes claimed that Construction Grammar\indexcxg cannot capture the existence
of recursive structure in natural language (\eg, \citealp[\page 269]{Leiss2009a}).  This impression
is understandable since many analyses are extremely surface"=oriented. For example, one often talks
of a [Sbj TrVerb Obj] construction. However, the grammars in question also become recursive as soon
as they contain a sentence embedding or relative clause construction. A sentence embedding
construction could have the form [Sbj that-Verb that-S], where a that-Verb is one that can take
a sentential complement and that-S stands for the respective complement. A \emph{that}"=clause can then be inserted
into the that-S slot. Since this \emph{that}"=clause can also be the result of the application of
this construction, the grammar is able to produce recursive structures such as those in (\mex{1}):

\ea
Otto claims [\sub{that-S} that Karl suspects [\sub{that-S} that Richard sleeps]].
\z
In (\mex{0}), both \emph{Karl suspects that Richard sleeps} and the entire clause are instances of the [Sbj
that-Verb that-S] construction. The entire clause therefore contains an embedded subpart that is licensed by
the same construction as the clause itself. (\mex{0}) also contains a constituent of the category
\emph{that}-S that is embedded inside of \emph{that}-S. For more on recursion and self"=embedding\is{self"=embedding} in Construction Grammar, see \citew{Verhagen2010a}.

Similarly, every Construction Grammar that allows a noun to combine with a genitive\is{genitive} noun phrase also allows
for recursive structures. The construction in question could have the form [Det N
NP[gen] ] or [ N NP[gen] ]. The [Det N NP[gen] ] construction licenses structures such as (\mex{1}):
\ea
\gll [\sub{NP} des Kragens [\sub{NP} des Mantels [\sub{NP} der Vorsitzenden]]]\\
	{} the collar {} of.the coat {} of.the chairwoman\\
\glt `the collar of the coat of the chairwoman'
\z
\citet{Jurafsky96a} and \citet*{BLT2009a} use probabilistic context"=free grammars\is{context"=free grammar!probabilistic (PCFG)} (PCFG) for a Construction Grammar parser
with a focus on psycholinguistic plausibility and modeling of acquisition. Context"=free grammars
have no problems with self"=embedding\is{self"=embedding} structures like those in (\mex{0}) and thus this kind
of Construction Grammar itself does not encounter any problems with self"=embedding.

\citet[\page 192]{Goldberg95a} assumes that the resultative construction\is{construction!resultative} for English\il{English} has the following
form:
\ea
{}[SUBJ [V OBJ OBL]] 
\z
This corresponds to a complex structure as assumed for elementary trees in TAG. LTAG differs from Goldberg's approach in that every structure requires a lexical
anchor, that is, for example (\mex{0}), the verb would have to be fixed in LTAG. But in Goldberg's analysis, verbs can be inserted into independently
existing constructions (see Section~\ref{Abschnitt-Stoepselei}). In TAG publications, it is often emphasized that elementary trees do not contain any recursion.
The entire grammar is recursive however, since additional elements can be added to the tree using adjunction and -- as (\mex{-2}) and
(\mex{-1}) show -- insertion into substitution nodes can also create recursive structures.
\is{recursion|)}



%      <!-- Local IspellDict: en_US-w_accents -->
