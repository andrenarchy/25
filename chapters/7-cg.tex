%% -*- coding:utf-8 -*-

\chapter{Categorial Grammar}
\label{Kapitel-CG}\label{chap-CG}

Categorial Grammar\is{Categorial Grammar (CG)|(} is the second oldest of the approaches discussed in this book. It was developed in the 30s by the Polish logician
\href{http://en.wikipedia.org/wiki/Kazimierz_Ajdukiewicz}{Kazimierz Ajdukiewicz}
\citep{Ajdukiewicz35a-u}. Since
syntactic and semantic descriptions are tightly connected and all syntactic combinations correspond
to semantic ones, Categorial Grammar is popular amongst logicians and semanticists.
Some stellar works in the field of semantics are those of Richard Montague \citeyearpar{Montague74a-u}.

Other important works come from David Dowty in Columbus, Ohio \citeyearpar{Dowty79a}, Michael
Moortgat in Utrecht \citeyearpar{Moortgat89a-u}, Glyn Morrill in Barcelona
\citeyearpar{Morrill94a-u}, Bob Carpenter in New York \citeyearpar{Carpenter98a-u} and Mark Steedman
in Edinburgh \citeyearpar{Steedman91a,Steedman97a,Steedman2000a-u}. A large fragment for German using Montague Grammar has been developed by \citew{Stechow79}.
The 2569"=page grammar of the \emph{Institut für Deutsche Sprache} in Mannheim \citep*{IDS97} contains Categorial Grammar analyses in the relevant chapters.
\citet{Fanselow81a-u} worked on morphology in the framework of Montague Grammar.  \citet{Uszkoreit86d},
\citet{Karttunen86a,Karttunen89a-u} and \citet*{CKZ88a} developed combinations of unification"=based approaches and Categorial Grammar.

The basic operations for combining linguistic objects are rather simple and well"=understood so
that it is no surprise that there are many systems for the development and processing of Categorial Grammars
\citep*{YK90a-u,Carpenter1994a-u,BvN94a-u,Llore1995a-u,Koenig99a-u,Moot2002a-u,WB2003a-u,BCPW2007a,Morrill2012a}. An
important contribution has been made by Mark Steedman's group (see for instance \citealp*{CHS2002a-u,CC2007a-u}).

%% , aber auch in Deutschland gab und
%% gibt es computerlinguistische Gruppen, die in diesem theoretischen Rahmen arbeiten
%% \citep*{Uszkoreit86d,Koenig99a-u,VHE2003a}.
%% MCGTOOLS: Llore1995a-u
Implemented fragments exist for the following languages:
\begin{itemize}
\item German \citep*{Uszkoreit86d,Koenig99a-u,VHE2003a,VTBS2011a}
\item English\il{English} \citep{Villavicencio2002a,Baldridge2002a-u,Beavers2002a-u,Beavers2004a-u}
% CMZ86a ist ein Report, den gibt es aber nirgends
\item Finish\il{Finnish} \citep{Karttunen89a-u}
\item French\il{French} \citep*{BBCG87a-u}
\item Dutch\il{Dutch} \citep{BvN94a-u,Baldridge2002a-u}
\item Tagalog\il{Tagalog} \citep{Baldridge2002a-u}
\item Turkish\il{Turkish} \citep{Hoffmann95a-u,Baldridge2002a-u}
\end{itemize}
In addition, \citet*[\page 15]{BCPW2007a} mention an implementation for Classical Arabic\il{Arabic}.

Some of the systems for the processing of Categorial Grammars have been augmented by probabilistic\is{statistics}
components so that the processing is robust \citep*{OB97a,CHS2002a-u}. Some systems can derive lexical items from corpora, and \citet{Briscoe2000a} and \citet{Villavicencio2002a} use statistical
information in their UG"=based\is{Universal Grammar (UG)} language acquisition models.\is{language acquisition}


%      <!-- Local IspellDict: en_US-w_accents -->


\section{General remarks on the representational format}


\subsection{Representation of valence information}
\label{sec-forward-backward-application}

In\is{valence|(} Categorial Grammar, complex categories replace the \subcatf that is used in GPSG\indexgpsg to ensure that a head can only be used with
suitable grammatical rules. Simple phrase structure rules\is{phrase structure grammar} can be replaced with complex categories as follows:\is{/|(} 

\ea
\label{LE-CG}
\begin{tabular}[t]{@{}l@{\hspace{1cm}}l}
Rule                              & Category~in~the~lexicon\\
vp $\to$ v(ditrans) np~np         & (vp/np)/np  \\
vp $\to$ v(trans) np              & vp/np  \\
vp $\to$ v(np\_and\_pp) np~pp(to) & (vp/pp)/np  \\
\end{tabular}
\z
vp/np stands for something that needs an np in order for it to form a vp.

In Categorial Grammar, there are only a few very abstract rules. One of these is forward application, also referred to as the multiplication rule:
\ea
\label{vorwaertsapplikation}\label{forward-application}
forward application\is{forward application}:\\
X/Y $*$ Y = X
\z
This rule combines an X looking for a Y with a Y and requires that Y occurs to the right of X/Y.
The result of this combination is an X that no longer requires a Y. X/Y is called the \emph{functor}\is{functor}
and Y is the \emph{argument}\is{argument} of the functor.

As in \gbt, valence is encoded only once in Categorial Grammar, in the lexicon. In GPSG\indexgpsg, valence information was present in grammatical rules
and in the \subcatf of the lexical entry.

Figure~\vref{abb-cg-transitives-Verb} shows how a lexical entry for a transitive verb is combined
with its object.\todostefan{Die Linien sind zu hoch im vergleich zum $<$}
\begin{figure}
\centerline{%
\deriv{2}{
%\begin{tabular}{@{}cc@{}}
chased       & Mary\\
%\uline{1}    & \uline{1} \\
\hr & \hr\\
vp/np   & np\\
\multicolumn{2}{@{}c@{}}{\forwardapp} \\
\multicolumn{2}{@{}c@{}}{vp}\\
%\cgmc<2->{2}{vp}\\
%\end{tabular}
}
}
\caption{\label{abb-cg-transitives-Verb}Combination of a verb and its object (preliminary)}
\end{figure}%
A derivation in CG is basically a binary branching tree; it is, however, mostly represented as follows:
an arrow under a pair of categories indicates that these have been combined via a combinatorial rule. The direction of this arrow indicates the direction
of this combination. The result is given beneath the arrow. Figure~\vref{Abb-CG-als-Baum} shows the tree corresponding to Figure~\ref{abb-cg-transitives-Verb}.
\begin{figure}
\centerline{%
\begin{forest}
sn edges
[vp
	[vp/np
		[chased]]
	[np
		[Mary]]]
\end{forest}
}
\caption{\label{Abb-CG-als-Baum}Derivation in Figure~\ref{abb-cg-transitives-Verb} as a tree diagram}
\end{figure}%
One usually assumes left associativity\is{left associativity}
for `/'; that is, (vp/pp)/np = vp/pp/np.\is{/|)}

If we look at the lexical entries in (\ref{LE-CG}), it becomes apparent that the category v does not appear. The lexicon only determines what the product of
combination of a lexical entry with its arguments is. The symbol for vp can also be eliminated: an (English) vp is something that requires an NP to its left in
order to form a complete sentence. This\is{$\backslash$|(} can be represented as s$\backslash$np.
Using the rule for backward application, it is possible to compute derivations such as the one in Figure~\vref{abb-the-cat-chased-Mary}.\is{valence|)}
\ea
Backward application\is{backward application}:\\
Y $*$ X$\backslash$Y = X 
\z

\begin{figure}
\centerline{%
\deriv{4}{
the  & cat & chased         & Mary\\
\hr  & \hr & \hr            & \hr \\
np/n & n   & (s\bs np)/np   & np\\
\multicolumn{2}{@{}c}{\forwardapp} & \multicolumn{2}{c@{}}{\forwardapp}\\
\multicolumn{2}{c}{{np}}             & \multicolumn{2}{c@{}}{{s\bs np}}\\
\multicolumn{4}{@{}c@{}}{\backwardapp}\\
\multicolumn{4}{c@{}}{{s}}\\
}}
\caption{\label{abb-the-cat-chased-Mary}Analysis of a sentence with a transitive verb}
\end{figure}%

\noindent
In Categorial Grammar, there is no explicit difference made between phrases and words: an intransitive verb is described in the same way as a verb phrase with an
object:  s$\backslash$np. In the same way, proper nouns are complete noun phrases, which are assigned the symbol np.\is{$\backslash$|)}

\subsection{Semantics}

As already mentioned, Categorial Grammar is particularly popular among semanticists as syntactic
combinations always result in parallel semantic combinations and even for complex combinations such
as those we will discuss in more detail in the following sections, there is a precise definition of
meaning composition.
In the following, we will take a closer look at the representational format discussed in \citet[Section~2.1.2]{Steedman97a}.

Steedman proposes the following lexical entry for the verb \emph{eats}:\footnote{
 I have adapted his notation to correspond to the one used in this book.
}
\ea
eats := (s: \relation{eat}(x, y)\bs np\sub{3S}:x)/np:y
\z
In (\mex{0}), the meaning of each category is given after the colon. Since nothing is known about
the meaning of the arguments in the lexical entry of \emph{eat},
the meaning is represented by the variables $x$ and $y$. When the verb combines with an NP, the denotation of the NP is inserted. An example is given in (\mex{1}):\footnote{
The assumption that \emph{apples} means \relation{apples} and not \relation{apples}(z) minus the quantifier contribution is a simplification here.
}
\ea
\deriv{2}{
(s: eat'(x, y)\bs np_{3S}:x)/np:y & np: apples'\\
\multicolumn{2}{@{}c@{}}{\forwardapp}\\
s: eat'(x, apples')\bs np_{3S}:x\\
}
\z
When combining a functor with an argument, it must be ensured that the argument fits the functor, that is, it must be unifiable\is{unification} with it
(for more on unification see Section~\ref{sec-unification}). The unification of np:y with np: \relation{apples} results in np: \relation{apples} since \relation{apples}
is more specific than the variable y. Apart from its occurrence in the term np:y, y occurs in the description of the verb in another position (s:
\relation{eat}(x, y)\bs np\sub{3S}:x) and therefore also receives the value \relation{apples}
there. Thus, the result of this combination is s: \relation{eat}(x, \relation{apples})\bs np$_{3S}$:x 
as shown in (\mex{0}).

Steedman notes that this notation becomes less readable with more complex derivations and instead uses the more standard $\lambda$"=notation:
\ea
eats := (s\bs np\sub{3S})/np: $\lambda y.\lambda x.\relation{eat}(x, y)$
\z
Lambdas are used to allow access to open positions in complex semantic representations (see Section~\ref{sec-PSG-Semantik}). A semantic representation such as
$\lambda
y.\lambda x.\relation{eat}(x, y)$ can be combined with the representation of \emph{apples} by
removing the first lambda expression and inserting the denotation of \emph{apples}
in all the positions where the corresponding variable (in this case, y) appears (see Section~\ref{sec-PSG-Semantik} for more on this point):
\ea
$\lambda y.\lambda x.eat'(x, y)$ \relation{apples}\\
$\lambda x.eat'(x, apples')$
\z
This removal of lambda expressions is called $\beta$"=reduction\is{beta"=reduction@$\beta$"=reduction}\label{Seite-beta-Reduktion}.

If we use the notation in (\mex{-1}), the combinatorial rules must be modified as follows:
\ea
\begin{tabular}[t]{@{}l@{ * }l@{ = }c}
X/Y:f & Y:a & X: f a\\
Y:a & X\bs Y:f & X: f a\\ 
\end{tabular}
\z
In such rules, the semantic contribution of the argument (a) is written after the semantic denotation of the functor (f). The open positions in the denotation of
the functor are represented using lambdas. The argument can be combined with the first lambda expression using $\beta$-reduction.

Figure~\vref{Abb-Semantik-CG} shows the derivation of a simple sentence with a transitive verb.
\begin{figure}
\centerline{%
\deriv{3}{
Jacob & eats         & apples\\
\hr & \hr          & \hr \\
np:\relation{jacob}  & (s\bs np)/np: \lambda y.\lambda x.eat'(x, y) & np:apples'\\
         & \multicolumn{2}{c@{}}{\forwardapp}\\
         & \multicolumn{2}{c@{}}{\begin{tabular}[t]{@{}l@{~}l@{}}
                              $s\bs np$ &: $\lambda y.\lambda x.eat'(x, y)\; apples'$\\[4pt]
                                      &= $\lambda x.eat'(x, apples')$\\
                              \end{tabular}}\\
\multicolumn{3}{@{}c@{}}{\backwardapp}\\
\multicolumn{3}{c@{}}{\begin{tabular}[t]{@{}l@{~}l@{}}
                    $s$ &: $\lambda x.eat'(x, apples')\; jacob'$\\[4pt]
                      &= $eat'(jacob', apples')$\\
                              \end{tabular}}\\
}}
\caption{\label{Abb-Semantik-CG}Meaning composition in Categorial Grammar}
\end{figure}%
After forward and backward application, $\beta$"=reduction is immediately applied.

\subsection{Adjuncts}

As\is{adjunct|(} noted in Section~\ref{sec-intro-arg-adj}, adjuncts are optional. In phrase
structure grammars, this can be captured, for example, by rules that have a certain element (for
instance a VP) on the left"=hand side of the rule and the same element and an adjunct on the
right"=hand side of the rule. Since the symbol on the left is the same as the one on the right, this rule can be
applied arbitrarily many times. (\mex{1}) shows some examples of this:
\eal
\ex VP $\to$ VP~PP
\ex Noun $\to$ Noun~PP
\zl
One can analyze an arbitrary amount of PPs following a VP or noun using these rules.

In Categorial Grammar, adjuncts have the following general form: X\bs X or X/X.
Adjectives are modifiers, which must occur before the noun. They have the category n/n.
Modifiers occurring after nouns (prepositional phrases and relative clauses) have the category
n\bs n instead.\footnote{
	In Categorial Grammar, there is no category symbol like \xbar for intermediate projections of
	\xbart. So rather than assuming \nbar/\nbar, CG uses n/n. See Exercise~\ref{ue-Xbar-CG}.
} For VP-modifiers, X is replaced by the symbol for the VP (s\bs np) and this yields the relatively complex expression (s\bs np)\bs (s\bs np).
Adverbials in English are VP"=modifiers and have this category. Prepositions that can be used in a PP modifying a verb require an NP in order to
form a complete PP and therefore have the category ((s\bs np)\bs (s\bs
np))/np. Figure~\vref{abb-CG-Adjunktion} gives an example of an adverb (\emph{quickly}) and a preposition (\emph{round}). 
%
\begin{figure}
\oneline{%
\deriv{9}{
The  & small & cat & chased       & Mary & quickly                & round                     & the & garden\\
\hr  & \hr   & \hr & \hr          & \hr  & \hr                    & \hr                       & \hr & \hr\\
np/n & n/n   & n   & (s\bs np)/np & np   & (s\bs np)\bs (s\bs np) & (s\bs np)\bs (s\bs np)/np & np/n & n\\
     & \multicolumn{2}{c}{\forwardapp} & \multicolumn{2}{c}{\forwardapp}\\
     & \multicolumn{2}{c}{n}           & \multicolumn{2}{c}{s\bs np}\\
\multicolumn{3}{@{}c}{\forwardapp}        & \multicolumn{3}{c}{\backwardapp}\\
\multicolumn{3}{@{}c}{np}                 & \multicolumn{3}{c}{s\bs np}\\
&&&&&&&\multicolumn{2}{c@{}}{\forwardapp}\\
&&&&&&&\multicolumn{2}{c@{}}{np}\\
&&&&&&\multicolumn{3}{c@{}}{\forwardapp}\\
&&&&&&\multicolumn{3}{c@{}}{(s\bs np)\bs (s\bs np)}\\
&&&\multicolumn{6}{c@{}}{\backwardapp}\\
&&&\multicolumn{6}{c@{}}{(s\bs np)}\\
\multicolumn{9}{@{}c@{}}{\backwardapp}\\
\multicolumn{9}{@{}c@{}}{s}\\
}}
\caption{\label{abb-CG-Adjunktion}Example of an analysis with adjuncts in Categorial Grammar}
\end{figure}%
Note that the result of the combination of \emph{round} and \emph{the garden} corresponds to the category of the adverb ((s\bs np)\bs (s\bs np)).
In \gbt, adverbs and prepositions were also placed into a single class (see page~\pageref{Seite-Adverbien-PP}). This overarching class was then divided into subclasses
based on the valence of the elements in question.
\is{adjunct|)}

\section{Passive}

In Categorial Grammar, the passive\is{passive|(} is analyzed by means of lexical rule (\citealp[\page412]{Dowty78a};
\citealp[Section~3.4]{Dowty2003a}). (\mex{1}) shows the rule in \citew[\page 49]{Dowty2003a}.
\ea
\label{Lexikonregel-Passiv-CG}
\begin{tabular}[t]{@{}ll@{~$\to$~}l@{}}
Syntax:   & $\alpha \in$ (s\bs np)/np & PST-PART($\alpha$) $\in$ PstP/np$_{by}$\\
Semantics: & $\alpha'$                 & $\lambda y\lambda x \alpha'(y) (x)$
\end{tabular}
\z
Here, PstP stands for past participle and np$_{by}$ is an abbreviation for a verb phrase modifier of
the form vp\bs vp or rather (s\bs np)\bs (s\bs np).
The rule says the following: if a word belongs to the set of words with the category (s\bs
np)/np, then the word with past participle morphology also belongs in the set of words with the category PstP/np$_{by}$.

(\mex{1}a) shows the lexical entry for the transitive verb \emph{touch} and (\mex{1}b) the result of rule application:
\eal
\ex touch:   (s\bs np)/np
\ex touched: PstP/np$_{by}$ 
\zl
The auxiliary \emph{was} has the category (s\bs np)/PstP and the preposition \emph{by} has the category  np$_{by}$/np, or its unabbreviated form ((s\bs np)\bs (s\bs np))/np.
In this way, (\mex{1}) can be analyzed as in Figure~\vref{abb-CG-Passiv}.
\ea
John was touched by Mary.
\z
\begin{figure}
\centerline{%
\deriv{5}{%
John & was            & touched      & by         & Mary.\\
\hr  & \hr            & \hr_{\mathrm{LR}}  & \hr        & \hr\\
np   & (s\bs np)/\mathit{PstP} & \mathit{PstP}/np_{by} & np_{by}/np & np\\
     &                &              & \multicolumn{2}{c@{}}{\forwardapp}\\
     &                &              & \multicolumn{2}{c@{}}{np_{by}}\\
     &                & \multicolumn{3}{c@{}}{\forwardapp}\\
     &                & \multicolumn{3}{c@{}}{\mathit{PstP}}\\
     & \multicolumn{4}{c@{}}{\forwardapp}\\
     & \multicolumn{4}{c@{}}{s\bs np}\\
\multicolumn{5}{@{}c@{}}{\backwardapp}\\
\multicolumn{5}{@{}c@{}}{s}\\
}}
\caption{\label{abb-CG-Passiv}Analysis of the passive using a lexical rule}
\end{figure}%

\noindent
The question as to how to analyze the pair of sentences in (\mex{1}) still remains unanswered.\footnote{
	Thanks to Roland Sch\"{a}fer\aimention{Roland Sch{\"a}fer} (p.\,m., 2009) for pointing out these data to me.
}
\eal
\ex He gave the book to Mary.
\ex The book was given to Mary.
\zl
\emph{gave} has the category ((s\bs np)/pp)/np, that is, the verb must first combine with an NP (\emph{the
  book}) and a PP (\emph{to Mary}) before it can be combined with the subject. The problem is that the rule in (\ref{Lexikonregel-Passiv-CG}) cannot be applied to \emph{gave}
  with a \emph{to}-PP since the pp argument is sandwiched between both np arguments in ((s\bs np)/pp)/np. One would have to generalize the rule in (\ref{Lexikonregel-Passiv-CG}) somehow
  by introducing new technical means\footnote{
  Baldridge\aimention{Jason Baldridge} (p.\,M.\ 2010) suggests using regular expressions in a general lexical rule for passive.%
 }
  or assume additional rules for cases such as (\mex{0}b).\is{passive|)}

\section{Verb position}
\label{sec-Verbstellung-CG-Steedman}

\mbox{}\citet[\page 159]{Steedman2000a-u}\is{verb position|(} proposed an analysis with variable branching for Dutch\il{Dutch}, that is, there are two lexical entries for 
 \emph{at} `eat': an initial one with its arguments to the right, and another occupying final position with its arguments to its left.

\eal
\ex \emph{at} `eat' in verb-final position: (s\sub{+SUB}$\backslash$np)$\backslash$np
\ex \emph{at} `eat' in verb-initial position: (s\sub{$-$SUB}/np)/np
\zl
Steedman uses the feature \textsc{sub} to differentiate between subordinate and non-subordinate sentences. Both lexical items are related via lexical rules.\is{lexical rule}

One should note here that the NPs are combined with the verb in different orders. The normal order is:

\eal
\label{CG-Verbbewegung}
\ex in verb-final position: (s\sub{+SUB}$\backslash$np[nom])$\backslash$np[acc]
\ex in verb-initial position: (s\sub{$-$SUB}/np[acc])/np[nom]
\zl
The corresponding derivations for German sentences with a bivalent verb are shown in Figures~\ref{Abbildung-CG-der-Mann-der-Frau-das-Buch-gibt}
and~\ref{Abbildung-CG-gibt-der-Mann-der-Frau-das-Buch}.

\begin{figure}
\centerline{%
\deriv{3}{%
er      & ihn   & isst\\
\hr     & \hr   & \hr \\
np[nom] & np[acc]     & (s_{+\mathrm{SUB}}\bs np[nom]) \bs np[acc]\\
&\mc{2}{c@{}}{\backwardapp}\\
&\mc{2}{c@{}}{s_{+\mathrm{SUB}}\bs np[nom]}\\
\mc{3}{@{}c@{}}{\backwardapp}\\
\mc{3}{@{}c@{}}{s_{+\mathrm{SUB}}}\\
}}
\caption{\label{Abbildung-CG-der-Mann-der-Frau-das-Buch-gibt}Analysis of verb-final sentences following Steedman}
\end{figure}%
\begin{figure}
\centerline{%
\deriv{3}{%
isst & er  & ihn \\
\hr  & \hr & \hr    \\
((s_{-\mathrm{SUB}}/np[acc])/np[nom]& np[nom]    & np[acc]      \\
\mc{2}{@{}c}{\forwardapp}\\
\mc{2}{@{}c}{s_{-\mathrm{SUB}}/np[acc]}\\
\mc{3}{@{}c@{}}{\forwardapp}\\
\mc{3}{@{}c@{}}{s_{-\mathrm{SUB}}}\\
}}
\caption{\label{Abbildung-CG-gibt-der-Mann-der-Frau-das-Buch}Analysis of verb-initial sentences following Steedman}
\end{figure}%
In Figure~\ref{Abbildung-CG-der-Mann-der-Frau-das-Buch-gibt}, the verb is first combined with an
accusative object, whereas in Figure~\ref{Abbildung-CG-gibt-der-Mann-der-Frau-das-Buch}, the verb is first combined with the subject. For criticism of these kinds of analyses
with variable branching, see \citew{Netter92} and \citew{Mueller2005c,MuellerGS}.


\citet{Jacobs91a} developed an analysis which corresponds to the verb movement analysis in GB\indexgb. He assumes verb-final structures, that is, there is a lexical entry for verbs
where arguments are selected to the left of the verb. A transitive verb would therefore have the
entry in (\mex{1}a). Additionally, there is a trace in verb-final position that requires the
arguments of the verb and the verb itself in initial position. (\mex{1}b) shows what the verb trace
looks like for a transitive verb in initial position: 
\eal
\ex Verb in final position:\\
    (s\bs np[nom])\bs np[acc]
\ex Verb trace for the analysis of verb-first:\\
    ((s\bs ((s\bs np[nom])\bs np[acc]))\bs np[nom])\bs np[acc]
\zl
The entry for the verb trace is very complex. It is probably simpler to examine the analysis in Figure~\vref{Abbildung-CG-isst-der-junge-den-kuchen-jacobs}.

\begin{figure}
\oneline{%
%\begin{sideways}%
\deriv{4}{%
isst & er  & ihn & \_\\
\hr & \hr  & \hr & \hr \\
(s\bs np[nom]) \bs np[acc]    & np[nom] & np[acc]      & (((s\bs (s\bs np[nom]) \bs np[acc])\bs np[nom]) \bs np[acc]\\
&&\mc{2}{c@{}}{\backwardapp}\\
&&\mc{2}{c@{}}{(s\bs (s\bs np[nom]) \bs np[acc])\bs np[nom]}\\
&\mc{3}{c@{}}{\backwardapp}\\
&\mc{3}{c@{}}{s\bs ((s\bs np[nom]) \bs np[acc])}\\
\mc{4}{@{}c@{}}{\backwardapp}\\
\mc{4}{@{}c@{}}{s}\\
}
%\end{sideways}
}
\caption{\label{Abbildung-CG-isst-der-junge-den-kuchen-jacobs}Analysis of verb-initial sentences following \citet{Jacobs91a}}
\end{figure}%
The trace is the head in the entire analysis: it is first combined with the accusative object and then with the subject. In a final step,
it is combined with the transitive verb in initial-position.\footnote{ See \citew{Netter92} for a similar analysis in HPSG\indexhpsg.
} 
A problem with this kind of analysis is that the verb \emph{isst} `eats', as well as \emph{er} `he' and
\emph{ihn} `him'/`it', are arguments of the verb trace in (\mex{1}).
\ea
\gll Morgen [isst [er [ihn \_]]]\\
	 tomorrow \spacebr{}eats \spacebr{}he \spacebr{}him\\
\glt `He will eat it/him tomorrow.'
\z
Since adjuncts can occur before, after or between arguments of the verb in German, one would expect that \emph{morgen} `tomorrow' can occur before the verb
\emph{isst}, since \emph{isst} is just a normal argument of the verbal trace in final position. As adjuncts do not change the categorial status of a projection, the phrase \emph{morgen isst er ihn} `tomorrow he eats him' should be able to
occur in the same positions as \emph{isst er ihn}. This is not the case, however. If we replace
\emph{isst er ihn} by \emph{morgen isst er ihn} in (\mex{1}a), the result is (\mex{1}b), which is ungrammatical.
\eal
\ex[]{
\gll Deshalb isst er ihn.\\
     therefore eats he him\\
\glt `Therefore he eats it/him.'
}
\ex[*]{
\gll Deshalb morgen isst er ihn.\\
	 therefore tomorrow eats he him\\
}
\zl
%Normalerweise geht man davon aus, dass das \vf durch Voranstellung einer Konstituente besetzt wird
%und nicht durch ein
An approach which avoids this problem comes from \citet{KW91a} (see
Section~\ref{Abschnitt-Verbstellung-HPSG}). Here, the authors assume that there is a verb in initial
position which selects a projection of the verb trace. If adverbials are only combined with verbs in
final-position, then a direct combination of \emph{morgen} `tomorrow' and \emph{isst er ihn} `eats he
it' is ruled out. If one assumes that the verb in first-position is the functor, then it is possible to capture the
parallels between complementizers\is{complementizer} and verbs in initial position
\citep{Hoehle97a}: finite verbs in initial position differ from complementizers only in requiring
a projection of a verb trace, whereas complementizers require projections of overt verbs: 
\eal
\ex 
\gll dass [er ihn isst]\\
     that \spacebr{}he it eats\\
\ex 
\gll Isst [er ihn \_ ]\\
     eats \spacebr{}he it\\
\zl
This description of verb position in German captures the central insights of the GB analysis in Section~\ref{Abschnitt-Verbstellung-GB}.\is{verb position|)}

\section{Local reordering}
\label{Abschnitt-CG-lokale-Umstellung}

Up to now,\is{constituent order|(} we have seen combinations of functors and arguments where the arguments were either to the left or to the right of the functor. The
saturation of arguments always took place in a fixed order: the argument furthest to the right was
combined first with the functor, \eg, (s\bs np)/pp first combined with
the PP, and the result of this combination was combined with the NP.

There are a number of possibilities to analyze ordering variants in German: \citet{Uszkoreit86b} suggests accounting for possible orders lexically; that is, that
each possible order corresponds to a lexical item. One would therefore have at least six lexical items for a ditransitive verb. \citet[\page 257]{Briscoe2000a} and 
\citet[\page 96--98]{Villavicencio2002a} propose a variant of this analysis where the order of arguments is modified in the syntax: a syntactic rule can, for example, change
the order (S/PRT)/NP into (S/NP)/PRT.

A different approach is suggested by \citet{SB2006a-u}. They discuss various options for ordering arguments attested in the languages of the world.
This includes languages in which the order of combination is free, as well as languages where
the direction of combination is free.\is{constituent order!free}\is{constituent
  order!fixed} Steedman and Baldridge introduce the following convention for
representing categories: elements in curly brackets can be discharged in any order. `$|$'\is{$\vert$} in place of `$\backslash$'\is{$\backslash$} 
or `/'\is{/} serves to indicate that the direction of combination is free. Some prototypical examples are shown in (\mex{1}):

\ea
\begin{tabular}[t]{@{}lll@{}}
English\il{English}   & (S$\backslash$NP)/NP     & S(VO)\\
Latin\il{Latin}       & S\{$|$NP[nom], $|$NP[acc] \} & free order\\
Tagalog\il{Tagalog}     & S\{/NP[nom], /NP[acc] \} & free order, verb-initial\\
Japanese\il{Japanese} & S\{$\backslash$NP[nom], $\backslash$NP[acc] \} & free order, verb-final\\
\end{tabular}
\z
\citet[Section~3.1]{Hoffmann95a-u} has proposed an analysis analogous to that of Japanese for Turkish\il{Turkish} and this could also be used in conjunction with an analysis
of verb position for German. This would correspond to the GB/MP\indexgb analysis of \citet{Fanselow2001a} or the HPSG analysis\indexhpsg presented in Section~\ref{Abschnitt-HPSG-lokale-Umstellung}.
\is{constituent order|}

\section{Long"=distance dependencies}
\label{Abschnitt-UDC-KG}\label{sce-nld-cg}

\mbox{}\citet[Section~1.2.4]{Steedman89a}\is{long"=Distance dependency|(} proposes an analysis of long"=distance dependencies without movement or empty elements. For examples such as (\mex{1}), he assumes that
the category of \emph{Harry must have been eating} or \emph{Harry devours} is s/np.\todostefan{CUP
  reviewer: No explanation of islands}
\eal
\ex\label{Bsp-these-apples}
These apples, Harry must have been eating.
\ex apples which Harry devours
\zl
The fronted NP \emph{these apples} and the relative pronoun \emph{which} are both functors in the analysis of (\mex{0}) which take s/np as their argument.
Using the machinery introduced up to now, we cannot assign the category s/np to the strings
\emph{Harry must have been eating} and \emph{Harry devours} in (\mex{0}) although it is intuitively the case that \emph{Harry devours}
is a sentence missing an NP. We still require two further extensions of Categorial Grammar: type raising\is{type raising} and forward\is{composition!forward}
and backward composition\is{composition!backward}. Both of these operations will be introduced in the following sections.

\subsection{Type Raising}

The category np can be transformed into the category (s/(s\bs np)) by \emph{type raising}\is{type raising}. If we combine this category with (s\bs np), then we
get the same result as if we had combined np and (s\bs np) with the forward application rule in (\ref{vorwaertsapplikation}). (\mex{1}a) shows the combination of an NP
with a VP (a sentence missing an NP to its left). The combination of the type"=raised NP with the VP is given in (\mex{1}b).
\eal
\ex np $*$ s\bs np = s 
\ex s/(s\bs np) $*$ s\bs np = s
\zl
In (\mex{0}a), a verb or verb phrase selects an NP to its left (s\bs np). In (\mex{0}b), an NP having undergone type raising selects a verb or verb phrase to its right which requires
an NP to its left (s\bs np). 

Type raising simply reverses the direction of selection: the VP in (\mex{0}a) is the functor and the
NP is the argument, whereas in  (\mex{0}b), it is the type raised NP which acts as the functor,
and the VP is the argument. In each case, the result of the combination is the same. This change of selectional direction may just seem like a trick at first glance, but as we will see, this
trick can be extremely useful. First, however, we will introduce forward and backward composition.
\is{type raising|)}

\subsection{Forward and backward composition}
\label{Kategorialgrammatik-Komposition}

(\mex{1})\is{composition|(} shows the rules for forward and backward composition.
\eal
\ex\label{Regel-Vorwaertskomposition}
 Forward composition\is{composition!forward} (> B)\\
    X/Y $*$ Y/Z = X/Z 
\ex Backward composition\is{composition!backward} (< B)\\
    Y\bs Z $*$ X\bs Y = X\bs Z
\zl 
These rules will be explained using forward composition as an example. (\mex{0}a) can be understood as follows: X/Y more or less means; if I find a Y, then I am a complete X.
In the combinatorial rule, X/Y is combined with Y/Z. Y/Z stands for a Y that is not yet complete and is
still missing a Z. The requirement that Y must find a Z in order
to be complete is postponed: we pretend that Y is complete and use it anyway, but we still bear in mind that
something is actually still missing. Hence, if we combine X/Y with Y/Z, we get something which becomes an X when combined with a Z. 
\is{composition|)} 

\subsection{Analysis of long"=distance dependencies}
\label{Abschnitt-CG-UDC}

By using forward composition, we can assign \emph{Harry must have been eating} the category
s/np. Figure~\vref{abb-CG-Komposition} shows how this works.
\begin{figure}
\centerline{%
\deriv{6}{
These\;apples  & Harry                & must & have & been & eating\\
\hr           & \forwardt            & \hr  & \hr  & \hr  & \hr\\
%
%
np            & s/{(s\bs np)}        & {(s\bs np)}/vp & vp/vp\mathdash en & vp\mathdash en/vp\mathdash ing & vp\mathdash ing/np\\
              & \multicolumn{2}{c}{\forwardc}\\
              & \multicolumn{2}{c}{{{s}/vp}}\\
              & \multicolumn{3}{c}{\forwardc}\\
              & \multicolumn{3}{c}{{{s}/vp\mathdash en}}\\
              & \multicolumn{4}{c}{\forwardc}\\
              & \multicolumn{4}{c}{{{s}/vp\mathdash ing}}\\
              & \multicolumn{5}{c@{}}{\forwardc}\\
              & \multicolumn{5}{c@{}}{{{s}/np}}\\
}}
\caption{\label{abb-CG-Komposition}Application of forward composition to VP-chains}
\end{figure}%
\emph{must} is a verb which requires an unmarked infinitive form,  \emph{have} requires a participle and \emph{been} must combine with a
gerund. In the above figure, the arrow with a small `T' stands for type raising, whereas the arrows with a `B' indicate composition.
The direction of composition is shown by the direction of the arrow.

For the analysis of (\ref{Bsp-these-apples}), we are still missing one small detail, a rule that turns the NP at the beginning of the sentence
into a functor which can be combined with s/np. Normal type raising cannot handle this because it would produce s/(s\bs np) when s/(s/np)
is required.

\citet[\page 217]{Steedman89a} suggests the rule in (\mex{1}):
\ea
\label{Regel-Topikalisierung}
Topicalization\is{topicalization} ($\uparrow$\is{$\uparrow$}):\\
X $\Rightarrow$ st/(s/X)\\
where X $\in$ \{ np, pp, vp, ap, s$'$ \}
\z
st stands for a particular type of sentence, namely one with topicalization.

If we replace X with np, we can turn \emph{these apples} into st/(s/np) and complete the analysis of 
(\ref{Bsp-these-apples}) as shown in Figure~\vref{abb-CG-UDC}.
\begin{figure}
\centerline{%
\deriv{6}{
These\;apples  & Harry                & must & have & been & eating\\
\forwardtop   & \forwardt            & \hr  & \hr  & \hr  & \hr\\
%
%
st/(s/np)\;\;     & s/{(s\bs np)}        & {(s\bs np)}/vp & vp/vp\mathdash en & vp\mathdash en/vp\mathdash ing & vp\mathdash ing/np\\
              & \multicolumn{2}{c}{\forwardc}\\
              & \multicolumn{2}{c}{{{s}/vp}}\\
              & \multicolumn{3}{c}{\forwardc}\\
              & \multicolumn{3}{c}{{{s}/vp\mathdash en}}\\
              & \multicolumn{4}{c}{\forwardc}\\
              & \multicolumn{4}{c@{}}{{{s}/vp\mathdash ing}}\\
              & \multicolumn{5}{c@{}}{\forwardc}\\
              & \multicolumn{5}{c@{}}{{{s}/np}}\\
\multicolumn{6}{@{}c@{}}{\forwardapp}\\
\multicolumn{6}{@{}c@{}}{{st}}\\
}}
\caption{\label{abb-CG-UDC}Analysis of long"=distance dependencies in Categorial Grammar}
\end{figure}%

\noindent
The mechanism presented here will of course also work for dependencies that cross sentence boundaries.
Figure~\vref{abb-CG-UDC-lang} shows the analysis for (\mex{1}):
\ea
Apples, I believe that Harry eats.
\z
\begin{figure}
\centerline{%
\deriv{6}{
Apples        & I             & believe        & that & Harry         & eats\\
\forwardtop   & \forwardt     & \hr            & \hr  & \hr           & \hr\\
%
%
st/(s/np)\;\;     & s/(s\bs np)  & (s\bs np)/s'    & s'/s & s/(s\bs np) & (s\bs np)/np\\
              & \multicolumn{2}{c@{}}{\forwardc}  &      & \multicolumn{2}{c@{}}{\forwardc}\\
              & \multicolumn{2}{c@{}}{{{s}/s'}}   &      & \multicolumn{2}{c@{}}{{{s}/np}}\\
              & \multicolumn{3}{c@{}}{\forwardc}\\
              & \multicolumn{3}{c@{}}{{{s}/s}}\\
              & \multicolumn{5}{c@{}}{\forwardc}\\
              & \multicolumn{5}{c@{}}{{{s}/np}}\\
\multicolumn{6}{@{}c@{}}{\forwardapp}\\
\multicolumn{6}{@{}c@{}}{{st}}\\
}}
\caption{\label{abb-CG-UDC-lang}Analysis of long"=distance dependencies across sentence boundaries}
\end{figure}%
%
Using the previously described tools, it is, however, only possible to describe extractions where the fronted element
in the sentence would have occurred at the right edge of the phrase without fronting. This means it is not possible to analyze sentences where the
middle argument of a ditransitive verb has been extracted \citep[\page 532]{Steedman85a-u}. \citet[\page 406]{Pollard88a} provides the derivation
in Figure~\vref{abb-CG-UDC-Ditrans} for (\mex{1}).
\ea
Fido we put downstairs.
\z
\begin{figure}
\centerline{%
\deriv{4}{
Fido                 & we            & put        & downstairs\\
\forwardtoptop       & \forwardt     & \hr        & \hr  \\
%
%
(st/pp)/((s/pp)/np)  & s/(s\bs np)  & ((s\bs np)/pp)/np    & pp\\
                     & \multicolumn{2}{c@{}}{\forwardczwei}  \\
                     & \multicolumn{2}{c@{}}{(s/pp)/np}   \\
%
\multicolumn{3}{@{}c@{}}{\forwardapp}  \\
\multicolumn{3}{@{}c@{}}{{st/pp}}   \\
\multicolumn{4}{@{}c@{}}{\forwardapp}\\
\multicolumn{4}{@{}c@{}}{{st}}\\
}}
\caption{\label{abb-CG-UDC-Ditrans}Analysis of long"=distance dependencies across sentence boundaries}
\end{figure}%
In this analysis, it is not possible to combine \emph{we} and \emph{put} using the rule in
(\ref{Regel-Vorwaertskomposition}) since (s\bs np) is not directly accessible: 
breaking down ((s\bs np)/pp)/np into functor and argument gives us ((s\bs np)/pp) and np. 
In order to deal with such cases, we need another variant of composition:
\ea
Forward composition for n=2 (> BB)\\
X/Y $*$ (Y/Z1)/Z2 = (X/Z1)/Z2
\z
With this addition, it is now possible to combine the type-raised \emph{we} with \emph{put}. The result is (s/pp)/np.
The topicalization rule in (\ref{Regel-Topikalisierung}), however, requires an element to the right of st with the form
(s/X). This is not the case in Figure~\ref{abb-CG-UDC-Ditrans}. For the NP \emph{Fido}, we need a functor category
which allows that the argument itself is complex. The rule which is needed for the case in
(\mex{-1}) is given in (\mex{1}).

\ea
\label{Regel-Topikalisierung-zwei}
Topicalization\is{topicalization} for n=2 ($\uparrow\uparrow$\is{$\uparrow\uparrow$}):\\
X2 $\Rightarrow$ (st/X1)/((s/X1)/X2)\\
where X1 and X2 $\in$ \{ NP, PP, VP, AP, S$'$ \}
\z

\noindent
If we assume that verbs can have up to four arguments (\zb \emph{buy}: buyer,
seller, goods, price), then it would be necessary to assume a further rule for composition as well as another topicalization rule.
Furthermore, one requires a topicalization rule for subject extraction \citep[\page 405]{Pollard88a}. Steedman has developed a notation which
provides a compact notation of the previously discussed rules, but if one considers what exactly these representations stand for, one still arrives at the
same number of rules that have been discussed here.%
\is{long"=distance dependency|)}

\section{Summary and classification}
\label{Abschnitt-Relativsaetze-CG}\label{Abschnitt-Ratte-CG}\label{sec-pied-piping-cg}

The operations of Combinatory Categorial Grammar, which go beyond those of standard Categorial Grammar, allow for so much flexibility that it is even possible
to assign a category to sequences of words that would not normally be treated as a constituent. This is an advantage for the analysis of coordination\is{coordination}
(see Section~\ref{Abschnitt-Koordination}) and furthermore, \citet{Steedman91a} has argued that intonation data\is{prosody} support the constituent status of these strings.
See also Section~\ref{Abschnitt-Inkrementelle-Verarbeitung} for a direct model of incremental language processing in Categorial Grammar. In phrase structure grammars,
it is possible to use GPSG mechanisms to pass information about relative pronouns contained in a phrase up the tree. These techniques are not used in CG and this leads
to a large number of recategorization rules for topicalization and furthermore leads to inadequate analyses of pied-piping constructions in relative clauses. As the
topicalization analysis was already discussed in Section~\ref{Abschnitt-UDC-KG}, I will briefly elaborate on relative clauses here.

\citet[\page 614]{SB2006a-u} present an analysis of long"=distance dependencies using the following relative clause in (\mex{1}):
\ea
the man  that Manny says Anna married\\
\z
The relative pronoun is the object of \emph{married} but occurs outside the clause \emph{Anna married}.

Steedman assumes the lexical entry in (\mex{1}) for relative pronouns:

\ea
\label{le-Relativpronomen-CG}
(n$\backslash$n)/(s/np)
\z
This means the following: if there is a sentence missing an NP to the right of a relative pronoun, then the relative pronoun can form an
N-modifier (n$\backslash$n) with this sentence. The relative pronoun is the head (functor) in this analysis.

Utilizing both additional operations of type raising and composition, the examples with relative clauses can be analyzed as shown in
Figure~\vref{abb-CG-Relativsatz}.
%
\begin{figure}
\centerline{%
\deriv{5}{
that                                & Manny                                                   & says                              & Anna                 & married\\
\hr                                 & \forwardt                                               & \hr                               & \forwardt            & \hr \\
%
%
(n\bs n)/(s/np) & s/{(s\bs np)}                           & {(s\bs np)}/s     & s/{(s\bs np)} & {(s\bs np)}/np\\
                                    & \multicolumn{2}{c}{\forwardc} & \multicolumn{2}{c@{}}{\forwardc}\\
%
%
                                    & \multicolumn{2}{c}{{s/{s}}}                    & \multicolumn{2}{c@{}}{{{s}/np}}\\
                                    & \multicolumn{4}{c@{}}{\forwardc}\\
                                    & \multicolumn{4}{c@{}}{{{s/np}}}\\
\multicolumn{5}{c@{}}{\forwardapp}\\
\multicolumn{5}{c@{}}{{n\bs n}}\\
}}
\caption{\label{abb-CG-Relativsatz}Categorial Grammar analysis of a relative clause with long"=distance dependency}
\end{figure}%
%
The lexical entry for the verbs corresponds to what was discussed in the preceding sections: \emph{married} is a normal transitive verb and 
 \emph{says} is a verb that requires a sentential complement and forms a VP (s\bs np) with it. This VP yields a sentence when combined
 with an NP. The noun phrases in Figure~\ref{abb-CG-Relativsatz} have been type raised. Using forward composition, it is possible to combine 
 \emph{Anna} and \emph{married} to yield s/np. This is the desired result: a sentence missing an NP to its right. \emph{Manny}
and \emph{says} and then \emph{Manny says} and \emph{Anna married} can also be combined via forward composition and we then have the category
s/np for \emph{Manny says Anna married}. This category can be combined with the relative pronoun using forward application and we then arrive at
n\bs n, which is exactly the category for postnominal modifiers.

However, the assumption that the relative pronoun constitutes the head is problematic since one has to then go to some lengths to explain pied-piping\is{pied-piping}
constructions such as those in (\mex{1}).

\eal
\ex\label{Beispiel-Minister}
Here's the minister [[in [the middle [of [whose sermon]]]] the dog barked].\footnote{
\citew[\page 212]{ps2}.
}
\ex Reports [the height of the lettering on the covers of which] the government prescribes should be
abolished.\label{Ross-reports}\footnote{
\citew[\page 109]{Ross67}.\nocite{Ross86a-u}
}
\zl
In (\mex{0}), the relative pronoun is embedded in a phrase that has been extracted from the rest of the relative clause. The relative pronoun
in (\mex{0}a) is the determiner of \emph{sermon}. Depending on the analysis, \emph{whose} is the head of the phrase \emph{whose sermon}.
The NP is embedded under \emph{of} and the phrase \emph{of whose sermon} depends on \emph{middle}. The entire NP \emph{the middle of the sermon}
is a complement of the preposition \emph{in}. It would be quite a stretch to claim that \emph{whose} is the head of the relative clause in (\mex{0}a).
The relative pronoun in (\mex{0}b) is even more deeply embedded.
\citet[\page 50]{Steedman97a} gives the following lexical entries for \emph{who},
\emph{whom} and \emph{which}:
\eal
\label{le-relpron-Steedman}
\settowidth\jamwidth{(komplexe extrahierte NP-Relativphrase)}
\ex ((n$\backslash$n)/(s\bs np))\bs (np/np)       \jambox{(complex subject-relative phrase)}
\ex ((n$\backslash$n)/(s/pp))$\backslash$(pp/np)  \jambox{(complex extracted PP-relative phrase)}
\ex ((n$\backslash$n)/(s/np))$\backslash$(np/np)  \jambox{(complex extracted NP-relative phrase)}
\zl
Using (\mex{0}b) and (\mex{0}c), it is possible to analyze (\mex{1}a) and (\mex{1}b):
\eal
\ex a report the cover of which Keats (expects that Chapman) will design
\ex a subject on which Keats (expects that Chapman) will speak
\zl
In the analysis of (\mex{0}b), \emph{which} requires a preposition to its left (pp/np) so it can form the category
(n$\backslash$n)/(s/pp). This category needs a sentence lacking a PP to its right in order to form a post-nominal
modifier (n$\backslash$n). In the analysis of (\mex{0}a), \emph{the cover of} becomes np/np by means
of composition and \emph{which} with the lexical entry (\mex{-1}c) can combine with \emph{the cover of} to its left. The result is
the category (n$\backslash$n)/(s/np), that is, something that requires a sentence missing an NP.

Ross' examples (\ref{Ross-reports}) can also be analyzed as follows (\mex{-1}c):
\ea
reports [the height of the lettering on the covers of]\sub{np/np}
which]\sub{(n$\backslash$n)/(s/np)} the government prescribes
\z
The complex expression \emph{the height of the lettering on the covers of} becomes np/np after composition and the rest of the analysis
proceeds as that of (\mex{-1}a).

In addition to entries such as those in (\ref{le-relpron-Steedman}), we also need further entries to analyze sentences such as (\mex{1}), where
the relative phrase has been extracted from the middle of the clause (see \citealp[\page 410]{Pollard88a}):
\ea
Fido is the dog which we put downstairs.
\z
The problem here is similar to what we saw with topicalization: \emph{we put} does not have the cateory
s/np but rather (s/pp)/np and as such, cannot be directly combined with the relative pronoun in (\ref{le-Relativpronomen-CG}).

\citet[\page 204]{Morrill95a} discusses the lexical entry in (\ref{le-relpron-Steedman}b) for the relative pronoun in (\mex{1}):
\ea
\label{Beispiel-about-which}
about which John talked
\z
In the lexical entry (\ref{le-relpron-Steedman}b), \emph{which} requires something to the left of it, which requires a noun phrase in order to form a complete
prepositional phrase; that is, \emph{which} selects a preposition. Morrill noted that there is a need
to postulate further lexical items for cases like (\mex{1}) in which the relative pronoun occurs in the middle of the fronted phrase.
\ea
the contract [the loss of which after so much wrangling] John would finally have to pay for
\z
These and other cases could be handled by additional lexical stipulations. Morrill instead proposes additional types of the combination of functors and
arguments, which allow a functor B \up\is{$\uparrow$} A to enclose its argument A and produce B, or a functor A $\downarrow$\is{$\downarrow$}
B to enclose its argument to then yield B (p.\,190). Even with these additional operations, he still needs the two lexical items in (\mex{1})
for the derivation of a pied-piping construction with an argument NP or a PP:
\eal
\ex (NP \up\ NP) $\downarrow$ (N$\backslash$N)/(S/NP)
\ex (PP \up\ NP) $\downarrow$ (N$\backslash$N)/(S/PP)
\zl
These lexical items are still not enough, however, as (\mex{0}b) contains a PP but this PP corresponds to an argument PP, which is required
for (\ref{Beispiel-about-which}). To analyze (\ref{Beispiel-Minister}), which involves a PP adjunct, we need to assume the category (s\bs np)/(s\bs np)  for the prepositional
phrase \emph{in the middle of whose sermon}. We, therefore, also require at least three additional items for relative pronouns.

By introducing new operations, Morrill manages to reduce the number of lexical entries for \emph{which}; however, the fact remains
that he has to mention the categories which can occur in pied-piping constructions in the lexical entry of the relative pronoun.

Furthermore, the observation that relative clauses consist of a phrase with a relative pronoun plus a sentence missing a relative phrase
is lost. This insight can be kept if one assumes a GPSG"=style\indexgpsg analysis where information about whether there is a relative pronoun
in the relative phrase can be passed up to the highest node of the relative phrase.
The relative clause can then be analyzed as the combination of a sentence with a gap and an appropriately marked relative phrase.
For the discussion of such analyses in the framework of \gbt and HPSG/CxG, see Section~\ref{Abschnitt-Relativ-Interrogativsaetze}.
\is{Categorial Grammar (CG)|)}

\section*{Comprehension questions}

\begin{enumerate}
\item Identify the functors and arguments in Figures~\ref{abb-cg-transitives-Verb}
  and~\ref{abb-the-cat-chased-Mary}.
\item Which combination operations do you know?
\item What is composition used for?
\end{enumerate}

\section*{Exercises}

\begin{enumerate}
\item Analyze the following sentence:
\ea
The children in the room laugh loudly.
\z
\item\label{ue-Xbar-CG} Analyze the noun phrase in (\mex{1}):
\ea
the picture of Mary
\z
Compare the resulting analysis with the structure given in Figure~\vref{Abbildung-NP-mit-PP-Argument} and think about which
categories of \xbar syntax the categories in Categorial Grammar correspond to. 
\end{enumerate}

\section*{Further reading}

\begin{sloppypar}
Mark Steedman discusses a variant of Categorial Grammar, \emph{Combinatory Categorial Grammar}, in a series of books and articles:
\citet{Steedman91a,Steedman2000a-u,SB2006a-u}.
\end{sloppypar}
\citet{Lobin2003a} compares Categorial Grammar with Dependency Grammar and \citet{PB93a} suggest a
combination of Dependency Grammar and Categorial Grammar, which they call Dependency Categorial Grammar\is{Dependency Categorial Grammar}.

\citet{Briscoe2000a} and \citet{Villavicencio2002a} discuss UG"=based acquisition models in the framework of Categorial Grammar.





% Yusuke Kubota ESSLLI 2013: Extraktion geht nicht, wenn etwas in der Mitte fehlt
% The book that he read _ yesterday.
% Grund: S/NP ist etwas, dem rechts was fehlt, nicht in der Mitte.

%      <!-- Local IspellDict: en_US-w_accents -->
