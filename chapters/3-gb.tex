%% -*- coding:utf-8 -*-


\chapter{Transformational Grammar -- Government \& Binding}
\label{Kapitel-GB}\label{chap-GB}\label{chap-gb}

Transformational Grammar\is{Transformational Grammar|(} and its subsequent incarnations (such as Government and Binding Theory
and Minimalism) were developed by Noam Chomsky at MIT in Boston \citep{Chomsky57a,Chomsky65a,Chomsky75a,Chomsky81a,Chomsky86b,Chomsky95a-u}.
Manfred \citet{Bierwisch63} was the first to implement Chomsky's ideas for German. In the 60s, the decisive impulse came from the 
\emph{Arbeitsstelle Strukturelle Grammatik} `Workgroup for Structural Grammar', which was part of the Academy of Science of the GDR. (See
\citew{Bierwisch92} and \citew{Vater2010a} for a historic overview.)
As well as Bierwisch's work, the following other works focussing on German, which have been written as part of this research
program, should also be mentioned: \citew{Fanselow87a}, \citew{FF87}, \citew{SS88a}, \citew{Grewendorf88a}, \citew{Haider93a},
\citew{Sternefeld2006a-u}.

The different implementations of Chomskyan theories are often grouped under the heading \emph{Generative Grammar}\index{Generative Grammar}.
This term comes from the fact that phrase structure grammars and the augmented frameworks that were suggested by Chomsky can generate
sets of well"=formed expressions (see p.\,\pageref{Seite-generiert}). It is such a set of sentences that constitutes a language (in the formal
sense) and one can test if a sentence forms part of a language by checking if a particular sentence is in the set of sentences generated
by a given grammar. In this sense, simple phrase structure grammars and, with corresponding formal assumptions, GPSG, LFG, 
HPSG and Construction Grammar (CxG) are generative theories.
In recent years, a different view of the formal basis of theories such as LFG, HPSG and CxG has
emerged such that the aforementioned theories are now \emph{model theoretic}
theories\is{model"=theoretic grammar} rather than generative"=enumerative ones\footnote{
Model theoretic approaches are always constraint"=based and the terms \emph{model theoretic} and \emph{constraint"=based} are 
sometimes used synonymously. 
} (See Chapter~\ref{Abschnitt-Generativ-Modelltheoretisch} for discussion). In 1957, Chomsky defined the term \emph{Generative Grammar}
in the following way (see also \citealp[\page 162]{Chomsky95a-u}):
\begin{quote}
A grammar of a language purports to be a description of the ideal speaker-hearer's intrinsic
competence. If the grammar is, furthermore, perfectly explicit\,--\,in other words, if it does not
rely on the intelligence of the understanding reader but rather provides an explicit analysis of
his contribution\,--\,we may call it (somewhat redundantly) a \emph{generative
  grammar}. \citep[\page 4]{Chomsky65a}
\end{quote}
In this sense, all grammatical theories discussed in this book would be viewed as generative grammars.
To differentiate further, sometimes the term \emph{Mainstream Generative Grammar}\is{Mainstream
  Generative Grammar} (MGG) is used \citep[\page 3]{CJ2005a} for Chomskyan models.
In this chapter, I will discuss a well"=developed and very influential version of Chomskyan
grammar, \gbt. More recent developments following Chomsky's Minimalist Program are dealt with in Chapter~\ref{chapter-mp}.

\section{General remarks on the representational format}
\label{Abschnitt-GB-allgemein}

This section provides an overview of general assumptions. I introduce the concept of transformations
in Section~\ref{Abschnitt-Transformationen}. Section~\ref{Abschnitt-GB-Paramater} provides
background information about assumptions regarding language acquisition, which shaped the theory
considerably, Section~\ref{Abschnitt-T-Modell} introduces the so"=called T model, the basic
architecture of \gbt. Section~\ref{Abschnitt-X-Bar} introduces the \xbart in the specific form that is used in \gb
and Section~\ref{sec-GB-CP-IP-System-English} shows how this version of the \xbart can be applied to English. The discussion of
the analysis of English sentences is an important prerequisite for the understanding of the analysis
of German, since many analyses in the \gb framework are modeled in parallel to the analyses of
English. Section~\ref{sec-German-clause} introduces the analysis of German clauses in a parallel way
to what has been done for English in Section~\ref{sec-GB-CP-IP-System-English}.

\subsection{Transformations}
\label{Abschnitt-Transformationen}

In the previous\is{transformation|(} chapter, I introduced simple phrase structure
grammars. \citet[Chapter~5]{Chomsky57a} criticized this kind of rewrite grammars since -- in his
opinion -- it is not clear how one can capture the relationship between active and passive sentences or
the various ordering possibilities of constituents in a sentence. While it is of course possible to
formulate different rules for active and passive sentences in a phrase structure grammar, it would
not adequately capture the fact that the same phenomenon occurs in the example pairs in (\mex{1})--(\mex{3}):
\eal
\ex 
\gll weil dort noch jemand arbeitet\\
     because there still somebody works\\
\glt `because somebody is still working there'
\ex 
\gll weil dort noch gearbeitet wurde\\
     because there still worked was\\
\glt `because work was still being done there'	 
\zl
\eal
\ex 
\gll weil er den Weltmeister schlägt\\
	 because he the world.champion beats\\
\glt `because he beats the world champion'
\ex 
\gll weil der Weltmeister geschlagen wurde\\
	 because the world.champion beaten was\\
\glt `because the world champion was beaten'
\zl
\eal
\ex 
\gll weil der Mann der Frau den Schlüssel stiehlt\\
	 because the man the woman the key steals\\
\glt `because the man is stealing the key from the woman'
\ex 
\gll weil der Frau der Schlüssel gestohlen wurde\\
	 because the woman the key stolen was\\
\glt `because the key was stolen from the woman'
\zl

\noindent
\citet[\page 43]{Chomsky57a} suggests a transformation that creates a connection between active and
passive sentences. The passive transformation for English\il{English} that he suggested has the form in (\mex{1}):
\ea
\begin{tabular}[t]{@{}l@{~}l@{~}l@{~}l}
NP& V &NP & $\to$ 3 [\sub{AUX} be] 2en [\sub{PP} [\sub{P} by] 1]\\
1 & 2 &3\\
\end{tabular}
\z
This transformational rule maps a tree with the symbols on the left"=hand side of the rule onto a tree with the symbols on the
right"=hand side of the rule. Accordingly, 1, 2 and 3 on the right of the rule correspond to symbols, which are under the numbers on the
left"=hand side. \emph{en} stands for the morpheme which forms the participle (\emph{seen}, \emph{been}, \ldots, but also \emph{loved}).
Both trees for (\mex{1}a,b) are shown in Figure~\ref{fig-Passivtransformation}.

\eal
\ex John loves Mary.
\ex Mary is loved by John.
\zl
\begin{figure}
\hfill
\begin{forest}
%sn edges
[S, for tree={parent anchor=south, child anchor=north}
  [NP [John] ]
  [VP
    [V [loves] ]
    [NP [Mary] ] 
  ]]
\end{forest}
\hspace{1em}
\raisebox{6\baselineskip}{$\leadsto$}
\hspace{1em}
  \begin{forest}
  %sn edges
  [S, for tree={parent anchor=south, child anchor=north}
  	[NP[Mary]]
	[VP
	[Aux[is, tier=word]]
	[V[loved, tier=word]]
	[PP
	[P[by, tier=word]]
	[NP[John, tier=word]]]]]
\end{forest}
\hfill\mbox{}
\caption{\label{fig-Passivtransformation}Application of passive transformation}
\end{figure}%
The symbols on the left of transformational rules do not necessarily have to be in a local tree, that is, they can be daughters of different mothers
as in Figure~\ref{fig-Passivtransformation}.

Rewrite grammars were divided into four complexity classes\is{complexity class} based on the properties they
have. The simplest grammars are assigned to the class 3, whereas the most complex are of Type-0. The so"=called 
context"=free grammars\is{context-free grammar} we have dealt with thus far are of Type-2. Transformational grammars which allow symbols to
be replaced by arbitrary other symbols are of Type-0 \citep{PR73a-u}.\label{page-TG-Typ0} Research on the complexity
of natural languages shows that the highest complexity level (Type-0) is too complex for natural language. It follows from this
-- assuming that one wants to have a restrictive formal apparatus for the description of grammatical knowledge \citep[\page 62]{Chomsky65a} -- that
the form and potential power of transformations has to be restricted.\footnote{
	For more on the power of formal languages, see Chapter~\ref{sec-generative-capacity}.
} 
Another criticism of early versions of transformational grammar was that, due to a lack of restrictions, the way in which transformations interact was not clear. 
Furthermore, there were problems associated with transformations which delete material (see
\citealp[Section~3.1.4]{Klenk2003a})\todostefan{provide English ref}. For this reason, new theoretical approaches such 
as Government \& Binding \citep{Chomsky81a} were developed. In this model, the form that grammatical rules can take is restricted (see Section~\ref{Abschnitt-X-Bar}). Elements
moved by transformations are still represented in their original position, which makes them
recoverable at the original position and hence the necessary information is available for semantic interpretation. 
There are also more general principles, which serve to restrict transformations\is{transformation|)}.

After some initial remarks on the model assumed for language acquisition in \gbt, we will take a closer look at phrase structure rules,
transformations and constraints.


\subsection{The hypothesis regarding language acquisition: Principles \& Parameters}
\label{Abschnitt-GB-Paramater}

\citet[Section~I.8]{Chomsky65a}\is{Principles \& Parameters|(} assumes that linguistic knowledge must be innate since the language system is,
in his opinion, so complex that it would be impossible to learn a language from the given input using more general cognitive principles alone
(see also Section~\ref{Abschnitt-PSA}). If it is not possible to learn language solely through interaction with our environment, then at least part of
our language ability must be innate. The question of exactly what is innate and if humans actually have an innate capacity for language remains
controversial and the various positions on the question have changed over the course of the last decades. Some notable works on this topic are \citew{Pinker94a}, 
\citew{Tomasello95a}, \citew{Wunderlich2004a}, \citew*{HCF2002a} and \citew{Chomsky2007a}. For more on this discussion, see Chapter~\ref{chap-innateness}.

\citet{Chomsky81a} also assumes that there are general, innate principles which linguistic structure cannot violate. These principles are parameterized, that is,
there are options. Parameter\is{parameter!head position}\is{parameter} settings can differ between languages.
An example for a parameterized principle is shown in (\mex{1}):
\ea
Principle: A head occurs before or after its complement(s) depending
on the value of the parameter \textsc{position}.
%Prinzip: Ein Kopf steht in Abhängigkeit vom Parameter \textsc{stellung}
%vor oder nach seinen Komplementen.
\z
The Principles and Parameters model (P\&P model) assumes that a significant part of language acquisition consists of extracting enough information
from the linguistic input in order to be able to set parameters. \citet[\page 8]{Chomsky2000a-u} compares the setting of parameters to
flipping a switch. For a detailed discussion of the various assumptions about language acquisition in the P\&P"=model, see
Chapter~\ref{sec-acquisition}. Speakers of English\il{English} have to learn that heads occur before their
complements in their language, whereas a speaker of Japanese\il{Japanese} has to learn that heads follow their complements. (\mex{1}) gives the 
respective examples:
\eal
\label{Bsp-Kopfstellungsparameter}
\ex be showing pictures of himself
\ex
\gll zibun -no syasin-o mise-te iru\\
     \refl{}  from picture     showing be\\
\zl
As one can see, the Japanese verb, noun and prepositional phrases are a mirror image of the corresponding phrases in English.
(\mex{1}) provides a summary and shows the parametric value for the position parameter:
\ea
\begin{tabular}[t]{@{}lll@{}}
Language                & Observation                      & Parameter: head initial\\
English\il{English}   & Heads occur before complements    & +\\
Japanese\il{Japanese} & Heads occur after complements   & $-$\\
\end{tabular}
\z
Investigating languages based on their differences with regard to certain assumed parameters has proven to be a very
fruitful line of research in the last few decades and has resulted in an abundance of comparative cross"=linguistic studies.

After these introductory comments on language acquisition, the following sections will discuss the basic assumptions of \gbt.%
\is{Principles \& Parameters|)}

\subsection{The T model}
\label{Abschnitt-T-Modell}

Chomsky\is{T model|(}
criticized simple PSGs for not being able to adequately capture certain correlations. An example of this is the relationship between
active and passive sentences. In phrase structure grammars, one would have to formulate active and passive rules for intransitive,
transitive and ditransitive verbs. The fact that the passive can otherwise be consistently described as the suppression of the most
prominent argument is not captured by phrase structure rules. Chomsky therefore assumes that there is an underlying structure, the
so"=called \emph{Deep Structure}\is{Deep Structure|see{D"=Structure}}\is{D"=structure}, and that other structures are derived from this. 
The general architecture of the so-called T~model is discussed in the following subsections.

\subsubsection{D-structure and S-structure}

During the derivation of new structures, parts of the Deep Structure can be deleted or moved. In
this way, one can explain the relationship between active and passive sentences. As the result of
this kind of manipulation of structures, also called transformations, one derives a new
structure, the  \emph{Surface Structure}\is{Surface Structure|see{S"=structure}}\is{S"=structure},
from the original Deep Structure. Since the Surface Structure does not actually mirror the actual
use of words in a sentence in some versions of the theory, the term \emph{S"=structure} is sometimes
used instead as to avoid misunderstandings. 
\ea
\begin{tabular}[t]{@{}l@{~=~}l@{}}
\emph{Surface Structure} & S-structure\\
\emph{Deep Structure} & D"=structure\\
\end{tabular}
\z
\noindent
Figure~\vref{Abb-T-Modell} gives an overview of the GB architecture: phrase structure
rules and the lexicon license the D"=structure from which the S"=structure is derived by means of transformations.
\begin{figure}
\centering
\begin{forest}
for tree = {edge={->},l=4\baselineskip}
[D-structure
     [S-structure,edge label={node[midway,right]{move $\alpha$}} 
            [Deletion rules{,}\\Filter{,} phonol.\ rules
                    [Phonetic\\Form (PF)]]
            [Anaphoric rules{,}\\rules of quantification and control
                    [Logical\\Form (LF)]]]]
    \end{forest}

\caption{\label{Abb-T-Modell}The T~model}
\end{figure}%

S"=structure feeds into Phonetic Form (PF)\is{Phonetic Form (PF)} and Logical Form (LF)\is{Logical Form (LF)}.
The model is referred to as the \emph{T-model} (or Y-model)
because D"=structure, S"=structure, PF and LF form an upside"=down T (or Y). We will
look at each of these individual components in more detail.

Using phrase structure rules, one can describe the relationships between individual elements (for
instance words and phrases, sometimes also parts of words). The format for these rules
is \xbar syntax (see Section~\ref{sec-xbar}). The lexicon\is{lexicon|(}, together with the structure licensed by \xbar syntax,
forms the basis for D"=structure. D"=structure is then a syntactic representation of the selectional grid (=~valence classes\is{valence!classes})
of individual word forms in the lexicon. 

The lexicon contains a lexical entry for every word which comprises information about morphophonological structure, syntactic features
and selectional properties. This will be explained in more detail in Section~\ref{Abschnitt-GB-Lexikon}. Depending on one's exact theoretical
assumptions, morphology\is{morphology} is viewed as part of the lexicon. Inflectional morphology\is{inflection} is, however, mostly consigned
to the realm of syntax. The lexicon is an interface for semantic interpretation of individual word forms.\is{lexicon}

The surface position in which constituents are realized is not necessarily the position they have in
D"=structure. For example, a sentence with a ditransitive verb has the following ordering variants:
\eal
\ex 
\gll {}[dass] der Mann der Frau das Buch gibt\\
	 {}\spacebr{}that the.\nom{} man the.\dat{} woman the.\acc{} book gives\\
\glt `that the man gives the woman the book'
\ex 
\gll Gibt der Mann der Frau das Buch?\\
	 gives the.\nom{} man the.\dat{} woman the.\acc{} book\\
\glt `Does the man give the woman the book?'
\ex 
\gll Der Mann gibt der Frau das Buch.\\
	 the.\nom{} man gives the.\dat{} woman the.\acc{} book\\
\glt `The man gives the woman the book.'
\zl
The following transformational rules for the movements above are assumed: (\mex{0}b) is derived from (\mex{0}a) by fronting the verb, 
and (\mex{0}c) is derived from (\mex{0}b) by fronting the nominative noun phrase. In \gbt, there is only one very general transformation:
Move-$\alpha$\is{Move-$\alpha$} = ``Move anything anywhere!''. The nature of what exactly can be moved where and for which reason is determined
by principles. Examples of such principles are the Theta Criterion and the Case Filter, which will be
dealt with below.

The relations between a predicate and its arguments that are determined by the lexical entries have to be accessible for semantic interpretation at all 
representational levels. For this reason, the base position of a moved element is marked with a
trace. This means, for instance, that the position in which the
fronted \emph{gibt} `gives' originated is indicated in (\mex{1}b). The respective marking is
referred to as a \emph{trace}\is{trace} or a \emph{gap}\is{gap}. Such empty elements may be
frightening when one encounters them first, but I already motivated the assumption
of empty elements in nominal structures in Section~\ref{sec-psg-np}  (page~\pageref{np-epsilon}). 
\eal
\ex 
\gll {}[dass] der Mann der Frau das Buch gibt\\
	 {}\spacebr{}that the man the woman the book gives\\
\glt `that the man gives the woman the book'
\ex 
\gll Gibt$_i$ der Mann der Frau das Buch \_$_i$?\\
	 gives the man the woman the book\\
\glt `Does the man give the woman the book?'
\ex 
\gll {}[Der Mann]$_j$ gibt$_i$ \_$_j$ der Frau das Buch \_$_i$.\\
	 {}\spacebr{}the man gives {} the woman the book\\
\glt `The man gives the woman the book.'
\zl
(\mex{0}c) is derived from (\mex{0}a) by means of two movements, which is why there are two traces in (\mex{0}c). The traces are marked with
indices so it is possible to distinguish the moved constituents. The corresponding indices are then present on the moved constituents. Sometimes,
\emph{e} (for \emph{empty}) or \emph{t} (for \emph{trace}) is used to represent traces.

The S"=structure derived from the D"=structure is a surface"=like structure but should not be equated with the structure of actual utterances.
\is{T model|)}

\subsubsection{Phonetic Form}

%https://en.wikipedia.org/wiki/Phonetic_Form
Phonological operations are represented at\is{Phonetic Form (PF)|(} the level of Phonetic Form (PF). PF is responsible for creating the form which
is actually pronounced. For example, so"=called \emph{wanna}"=contraction takes place at PF \citep[\page 20--21]{Chomsky81a}.
\eal
\ex The students want to visit Paris.
\ex The students wannna visit Paris.
\zl
The contraction\is{contraction} in (\mex{0}) is licensed by the optional rule in (\mex{1}):
\ea
want + to $\to$ wanna
\z
\is{Phonetic Form (PF)|)}

\subsubsection{Logical Form}

Logical Form\is{Logical Form (LF)|(} is the syntactic level which mediates between S"=structure and the semantic interpretation of
a sentence. Some of the phenomena which are dealt with by LF are anaphoric reference of pronouns, quantification and control. 

Syntactic factors\label{Seite-Bindungstheorie} play a role in resolving anaphoric dependencies.
An important component of \gbt is Binding Theory\is{Binding Theory}, which seeks to explain what a
pronoun can or must refer to and when a reflexive pronoun can or must be used. 
(\mex{1}) gives some examples of both personal and reflexive pronouns:
\eal
\ex 
\gll Peter kauft einen Tisch. Er gefällt ihm.\\
	 Peter buys a table.\mas{} he likes him\\
\glt `Peter is buying a table. He likes it/him.'
\ex 
\gll Peter kauft eine Tasche. Er gefällt ihm.\\
	 Peter buys a bag.\fem{} he likes him\\
\glt `Peter is buying a bag. He likes it/him.'
\ex 
\gll Peter kauft eine Tasche. Er gefällt sich.\\
	 Peter buys a bag.\fem{} he likes himself\\
\glt `Peter is buying a bag. He likes himself.'
\zl
In the first example, \emph{er} `he' can refer to either Peter, the table or something/someone else that was previously mentioned
in the context. \emph{ihm} `him' can refer to Peter or someone in the context. Reference to the table is restricted by world knowledge.
In the second example, \emph{er} `he' cannot refer to \emph{Tasche} `bag' since \emph{Tasche} is feminine and \emph{er} is masculine.
\emph{er} `he' can refer to Peter only if \emph{ihm} `him' does not refer to Peter. \emph{ihm} would otherwise have to refer to a person
in the wider context. This is different in (\mex{0}c). In (\mex{0}c), \emph{er} `he' and \emph{sich} `himself' must refer to the same
object. This is due to the fact that the reference of reflexives such as \emph{sich} is restricted to a particular local domain. Binding Theory
attempts to capture these restrictions.

LF is also important for quantifier scope. Sentences such as (\mex{1}a) have two readings. These are given in (\mex{1}b) and (\mex{1}c).

\eal
\label{Beispiel-Every-man-loves-a-woman}
\ex Every man loves a woman.
\ex $\forall x \exists y (man(x) \to (woman(y) \wedge love(x,y)))$
\ex $\exists y \forall x (man(x) \to (woman(y) \wedge love(x,y)))$
\zl
The symbol $\forall$\is{$\forall$} stands for a \emph{universal quantifier}\is{quantifier!universal} and $\exists$\is{$\exists$} stands for an
\emph{existential quantifier}\is{quantifier!existential}. The first formula corresponds to the reading that for every man, there is a woman who he loves
and in fact, these can be different women. Under the second reading, there is exactly one woman such that all men love her. The question of when such an
ambiguity arises and which reading is possible when depends on the syntactic properties of the given utterance. LF is the level which is important for the
meaning of determiners such as \emph{a} and \emph{every}.

Control Theory\is{Control Theory} is also specified with reference to LF. Control Theory deals with the question of how the semantic role of the infinitive
subject in sentences such as (\mex{1}) is filled.
\eal
\ex 
\gll Der Professor schlägt dem Studenten vor, die Klausur noch mal zu schreiben.\\
	 the professor suggests the student \textsc{prt} the test once again to write\\
\glt `The professor advises the student to take the test again.'
\ex 
\gll Der Professor schlägt dem Studenten vor, die Klausur nicht zu bewerten.\\
	 the professor suggests the student \textsc{prt} the test not to grade\\
\glt `The professor suggests to the student not to grade the test.'
\ex 
\gll Der Professor schlägt dem Studenten vor, gemeinsam ins Kino zu gehen.\\
	 the professor suggests the student \textsc{prt} together into cinema to go\\
\glt `The professor suggests to the student to go to the cinema together.'
\zl
\is{Logical Form (LF)|)}

\subsubsection{The lexicon}
\label{Abschnitt-GB-Lexikon}

The meaning of words tells us that they have to be combined with certain roles like ``acting person'' or ``affected thing'' when creating more complex phrases.
For example, the fact that the verb \emph{know} needs two arguments belongs to its semantic contribution. The semantic representation of the contribution of the verb \emph{know} in (\mex{1}a) is given in (\mex{1}b):
\eal
\ex Maria knows the man.
\ex \relation{know}(x,y)
\zl
\noindent
Dividing heads into valence classes is also referred to as \emph{subcategorization}\is{subcategorization}:\label{Seite-Subkategoriesierung} 
\emph{know} is subcategorized for a subject and an object.
This term comes from the fact that a head is already categorized with regard to its
part of speech (verb, noun, adjective, \ldots) and then further sub"-classes (\eg, intransitive or
transitive verb) are formed with regard to valence information. Sometimes the phrase \emph{X subcategorizes for Y} is used, which means \emph{X selects Y}.
% or even \emph{X governs Y}. Martin: government ist in GB was anderes.
\emph{know} is referred to as the predicate\is{predicate} since 
\relation{know} is the logical predicate.
The subject and object are the arguments\is{argument} of the predicate. There are several terms used
to describe the set of selectional requirements such
as \emph{argument structure}\is{argument structure}, \emph{valence frames}\is{valence frame},
\emph{subcategorization frame}\is{subcategorization frame}, \emph{thematic grid}\is{thematic grid} and \emph{theta grid} or $\theta$-grid\is{theta grid}. 
%\todostefan{ich habe hier einige uebersetzungen erstmal weggelassen, weil ich mir nicht sicher bin wie die richtige uebersetzung ist.}
%Subjekt und Objekt sind die Argumente\is{Argument} des Prädikats. Spricht man von der Gesamtheit
%der Selektionsanforderungen, verwendet man Begriffe wie Argumentstruktur\is{Argumentstruktur}, Valenzrahmen\is{Valenz!-rahmen},
%Selektionsraster\is{Selektion!-sraster}, Subkategorisierungsrahmen\is{Subkategorisierung!-srahmen}, thematisches Raster\is{Raster!thematisches} 
%oder Theta-Raster = $\theta$-Raster\is{theta-Raster@$\theta$-Raster}
%(\emph{thematic grid}, \emph{Theta-grid}). 

Adjuncts\is{adjunct} modify semantic predicates and when the semantic aspect is emphasized they are
also called \emph{modifiers}\is{modifier}. Adjuncts are not present in the argument structure
of predicates.

Following GB assumptions, arguments occur in specific positions in the clause -- in so"=called argument positions\is{argument!position} (\eg, the sister of 
an \xnull element, see Section~\ref{sec-xbar}). The Theta Criterion\is{Theta Criterion} states that elements in argument positions have to be assigned
a semantic role\is{semantic role} -- a so"=called theta role\is{theta role@$\theta$-role} -- and
each role can be assigned only once \citep[\page 36]{Chomsky81a}: 
\begin{principle-break}[Theta Criterion]\label{theta-Kriterium}
\begin{itemize}
\item Each theta role is assigned to exactly one argument position.
\item Every phrase in an argument position receives exactly one theta role.
\end{itemize}
\end{principle-break}
\noindent
The arguments of a head are ordered, that is, one can differentiate between higher- and lower"=ranked arguments. The highest"=ranked
argument of verbs and adjectives has a special status. Since GB assumes that it is often (and always in some languages) realized in a position
outside of the verb or adjective phrase, it is often referred to as the \emph{external argument}\is{argument!external}. The remaining
arguments occur in positions inside of the verb or adjective phrase. These kind of arguments are dubbed 
\emph{internal arguments}\is{argument!internal} or \emph{complements}\is{complement}. For simple sentences, this often means that the subject
is the external argument.

When discussing types of arguments, one can identify three classes of theta roles:
\begin{itemize}
\item Class 1: agent\is{agent} (acting individual), the cause of an action or feeling (stimulus), holder of a certain property
\item Class 2: experiencer\is{experiencer} (perceiving individual), the person profiting from something (beneficiary\is{beneficiary})
(or the opposite: the person affected by some kind of damage), possessor (owner or soon"=to"=be owner of something, or the opposite:
someone who has lost or is lacking something) 
\item Class 3: patient\is{patient} (affected person or thing), theme\is{theme}
\end{itemize}
If a verb has several theta roles of this kind to assign, Class~1 normally has the highest rank, whereas Class~3 has the lowest.
Unfortunately, the assignment of semantic roles to actual arguments of verbs has received a rather inconsistent treatment in the 
literature. This problem has been discussed by \citet{Dowty91a}, who suggests using proto"=roles. An argument is assigned the
proto"=agent role if it has sufficiently many of the properties that were identified by Dowty as prototypical properties of agents (\eg, animacy\is{animacy}, volitionality).%
\nocite{Gruber65a-u,Fillmore68,Fillmore71a-u,Jackendoff72a-u,Dowty91a}

The mental lexicon contains \emph{lexical entries} with the specific properties of syntactic words needed to use that word grammatically.
Some of these properties are the following:
\begin{itemize}
\item form
\item meaning (semantics)
\item grammatical features:\\
      syntactic word class + morphosyntactic features   
\item theta grid
\end{itemize}

\noindent
(\mex{1}) shows an example of a lexical entry:
\ea
\raisebox{1.2ex}{%
\begin{tabular}[t]{|l|ll|}
\hline
form     & \emph{hilft} `helps'&\\\hline
semantics & \relation{helfen}     &\\\hline
grammatical features  & \multicolumn{2}{l|}{verb,}\\
                       & \multicolumn{2}{l|}{3rd person singular indicative present active}\\\hline\hline

%\setlength{\arrayrulewidth}{9pt}
theta grid                &&\\\hline
theta roles                & \underline{agent} & beneficiary\\[2mm]\hline
grammatical particularities &                   & dative\\\hline
\end{tabular}}
\z
Assigning semantic roles to specific syntactic requirements (beneficiary = dative) is also called \emph{linking}.\is{linking}

Arguments are ordered according to their ranking: the highest argument is furthest left. In the case
of \emph{helfen}, the highest argument is the external argument\is{external argument}, which is why the agent is underlined. With so"=called unaccusative verbs,\is{verb!unaccusative}\footnote{
See \citew{Perlmutter78} for a discussion of unaccusative verbs. The term \emph{ergative verb}\is{verb!ergative} is also common, albeit
a misnomer. See \citew{Burzio81-u,Burzio86a-u-gekauft} for the earliest work on unaccusatives in the Chomskyan framework and 
\citew{Grewendorf89a} for German. Also, see \citew{Pullum88a} on the usage of these terms and for a historical evaluation.
}
the highest argument is not treated as the external argument. It would therefore not be underlined in the corresponding lexical entry.

\subsection{\xbart}
\label{Abschnitt-X-Bar}

In\is{X theory@\xbar theory|(} GB, it is assumed that all syntactic structures licensed by the core grammar\is{core grammar}\footnote{
 \citet[\page 7--8]{Chomsky81a} distinguishes between a regular area of language that is determined by a grammar that
 can be acquired using genetically determined language"=specific knowledge and a periphery\is{periphery}, to which irregular parts
 of language such as idioms (\eg, \emph{to pull the wool over sb.'s eyes}) belong. See Section~\ref{Abschnitt-musterbasiert}.
}
correspond to the \xbar schema (see Section~\ref{sec-xbar}).\footnote{
   \citet[\page 210]{Chomsky70a} allows for grammatical rules that deviate from the \xbar schema.
   It is, however, common practice to assume that languages exclusively use \xbar structures.
}
In the following sections, I will comment on the syntactic categories assumed and the basic assumptions with regard to
the interpretation of grammatical rules. 

\subsubsection{Syntactic categories}
\label{GB-syntaktische-categoryn}

The categories which can be used for the variable X in the \xbar schema are divided into lexical\is{category!lexical} and functional\is{category!functional}
categories. This correlates roughly with the difference between open and closed word classes. The following are lexical categories: 
\begin{itemize}
\item V = verb\is{verb}
\item N = noun\is{noun}
\item A = adjective\is{adjective}
\item P = preposition\is{preposition}/postposition
\item Adv = adverb\is{adverb}
\end{itemize}
Lexical categories can be represented using binary features and a cross"=classification:\footnote{%
   See \citew[\page 199]{Chomsky70a} for a cross"=classification of N, A and V, and 
   \citew[Section~3.2]{Jackendoff77} for a cross"=classification that additionally includes P but has a different feature assignment.
} 
\LATER{\citet{Wunderlich96a} has criticized this feature decomposition}
\begin{table}[H]
\centerline
{\renewcommand{\arraystretch}{1.5}%
\begin{tabular}[t]{ccc}
\lsptoprule
 & $-$V & +V \\
\midrule
$-$N & P = [ $-$N, $-$V ] &  V = [ $-$N, +V ] \\
  +N & N = [ +N, $-$V ]    &  A = [ +N, +V ]\\
\lspbottomrule
\end{tabular}}
\caption{\label{Tabelle-Merkmalszerlegung-Wortarten}Representation of four lexical categories using two binary features}
\end{table}%

\noindent
Adverbs are viewed as intransitive prepositions\label{Seite-Adverbien-PP} and are therefore captured by the decomposition in the
table above.

Using this cross"=classification, it is possible to formulate generalizations. One can, for example, simply refer to adjectives and
verbs: all lexical categories which are [~+V~] are either adjectives or verbs. Furthermore, one can
say of [~+N~] categories (nouns and adjectives) that they can bear case. 

Apart from this, some authors have tried to associate the head position with the feature values in Table~\ref{Tabelle-Merkmalszerlegung-Wortarten} (see \eg, \citealp[\page 52]{Grewendorf88a};
\citealp[\page 124]{Haftka96a}; G.\ \citealp[\page 238]{GMueller2011a}). With prepositions and nouns,
the head precedes the complement in German:
\eal
\ex
\gll \emph{für} Marie\\
	 for Marie\\
\ex 
\gll \emph{Bild} von Maria\\
	 picture of Maria\\
\zl
With adjectives and verbs, the head is final:
\eal
\ex 
\gll dem König \emph{treu}\\
     the king loyal\\
\glt `Loyal to the king'
\ex 
\gll der [dem Kind \emph{helfende}] Mann\\
     the the child helping man\\
\glt `the man helping the child'
\ex 
\gll dem Mann \emph{helfen}\\
     the man help\\
\glt `help the man'
\zl
This data seems to suggest that the head is final with [~+V~] categories and initial with [~$-$V~] categories. Unfortunately, this
generalization runs into the problem that there are also postpositions in German. These are, like
prepositions, not verbal, but do occur after the NP they require: 
\eal
\ex 
\gll des Geldes \emph{wegen}\\
     the money because\\
\glt `because of the money'
\ex 
\gll die Nacht \emph{über}\\
     the night during\\
\glt `during the night'
\zl
Therefore, one must either invent a new category, or abandon the attempt to use binary category features to describe ordering restrictions.
If one were to place postpositions in a new category, it would be necessary to assume another binary
feature.\footnote{
Martin Haspelmath\aimention{Martin Haspelmath} has pointed out that one could assume a rule that moves a
post-head argument into a pre-head position (see \citealp[\page 89]{Riemsdijk78a} for the discussion
of a transformational solution). This would be parallel to the realization of
prepositional arguments of adjectives in German:
\eal
\ex
\gll auf seinen Sohn stolz\\
     on  his son proud\\
\glt `proud of his son'
\ex 
\gll stolz auf seinen Sohn\\
     proud of his son\\
\zl
But note that the situation is different with postpositions here, while all adjectives that take
prepositional objects allow for both orders, this is not the case for prepositions. Most
prepositions do not allow their object to occur before them. It is an idiosyncratic feature of some
postpositions that they want to have their argument to the left.%
} 
Since this feature can have either a negative or a positive value, one would then have four
additional categories. There are then eight possible feature combinations, some of which would not
correspond to any plausible category.

For functional categories, GB does not propose a cross"=classification. Usually, the following categories are assumed:
\begin{table}[H]
\begin{tabular}{lp{65ex}@{}}
C   & Complementizer\is{category!functional!C} (subordinating conjunctions such as \emph{dass} `that')\\
I   & Finiteness\is{category!functional!I} (as well as Tense and Mood);\\
    & also Infl in earlier work (inflection),\\
    & T in more recent work (Tense)\is{category!functional!T} \\
D   & Determiner\is{category!functional!D} (article, demonstrative)\\
\end{tabular}
\end{table}%

\subsubsection{Assumptions and rules}

In GB, it is assumed that all rules must follow the \xbar format discussed in Section~\ref{sec-xbar}. In other theories, rules which correspond
to the \xbar format are used along other rules which do not. If the strict version of \xbart is
assumed, this comes with the assumption of \emph{endocentricity}\is{endocentricity}: every phrase
has a head and every head is part of a phrase (put more technically: every head
projects\is{projection} to a
phrase). 

Furthermore, as with phrase structure grammars, it is assumed that the branches of tree structures
cannot cross (\emph{Non-Tangling Condition}\is{Non-Tangling Condition}). This assumption is made by
the majority of theories discussed in this book. There are, however, some variants of TAG\indextag,
HPSG\indexhpsg, Construction Grammar\indexcxg, and Dependency Grammar\indexdg which allow crossing branches and therefore
discontinuous\is{constituent!discontinuous} constituents
(\citealp*{BJR91a,Reape94a,BC2005a}; \citealp[\page 261]{Heringer96a-u}; \citealp[Section~9.6.2]{Eroms2000a}).

In \xbart, one normally assumes that there are at most two projection levels (X$'$ and X$''$). However, there are some versions of Mainstream
Generative Grammar and other theories which allow three or more levels \citep{Jackendoff77,Uszkoreit87a}. In this chapter, I follow the
standard assumption that there are two levels, that is, phrases have at least three projection levels:


\begin{itemize}
\item X$^0$ = head
\item X$'$ = intermediate projection (X bar) 
\item XP = highest projection (=~X$''$ = $\overline{\overline{\mbox{X}}}$), also called \emph{maximal projection}\is{projection!maximal} 
\end{itemize}
\is{X theory@\xbar theory|)}

\subsection{CP and IP in English}
\label{Abschnitt-GB-CP-IP-System-Englisch}\label{sec-GB-CP-IP-System-English}

Most work in Mainstream Generative Grammar is heavily influenced by previous publications dealing with English. If one wants to understand GB analyses of
German and other languages, it is important to first understand the analyses of English and, for this reason, this will be the focus of this section.
The CP/IP system is also assumed in LFG grammars of English and thus the following section also provides a foundation for understanding some of the
fundamentals of LFG presented in Chapter~\ref{Kapitel-LFG}.

In earlier work, the rules in (\mex{1}a) and (\mex{1}b) were proposed for English sentences \citep[\page 19]{Chomsky81a}.

\eal
\ex S $\to$ NP VP
\ex S $\to$ NP Infl VP
\zl

Infl stands for \emph{Inflection} as inflectional affixes are inserted at this position in the
structure. The symbol AUX was also used instead
of Infl in earlier work, since auxiliary verbs are treated in the same way as inflectional
affixes. Figure~\vref{Abb-Old-School-Hilfsverb} shows a sample 
analysis of a sentence with an auxiliary, which uses the rule in (\mex{0}b). 
%
\begin{figure}
\centerline{%
\begin{forest}
sn edges
[S
  [NP
  	[Ann]]
  [INFL
  	[will]]
  [VP
  	[V$'$
		[V$^0$[read]]
		[NP[the newspaper, triangle]]]]]
\end{forest}
}
\caption{\label{Abb-Old-School-Hilfsverb}English sentence with an auxiliary verb following \citet[\page 19]{Chomsky81a}}
\end{figure}%

Together with its complements, the verb forms a structural unit: the VP. The constituent status of
the VP is supported by several constituent tests and further differences
between subjects and objects regarding their positional restrictions.

The rules in (\mex{0}) do not follow the \xbar template since there is no symbol on the right"=hand side of the rule with the same
category as one on the left"=hand side, that is, there is no head. In order to integrate rules like
(\mex{0}) into the general theory, \citet[\page 3]{Chomsky86b} developed a rule system with
two layers above the verb phrase (VP), namely the CP/""IP system\is{category!functional!I|(}. CP stands for \emph{Complementizer Phrase}.
The head of a CP can be a complementizer. Before we look at CPs in more detail, I will discuss an example of an IP in this new system. 
Figure~\vref{Abb-GB-Hilfsverb} shows an IP with an auxiliary in the \inull position. As we can see, this corresponds to the structure of
the \xbar template: \inull is a head, which takes the VP as its complement and thereby forms I$'$. The subject\is{subject} is the specifier\is{specifier} 
of the IP.
%
\begin{figure}
\centerline{%
\begin{forest}
sn edges
[IP
  [NP
  	[Ann]]
  [I$'$
  	[I$^0$
  		[will]]
	[VP
  	[V$'$
		[V$^0$[read]]
		[NP[the newspaper, triangle]]]]]]
\end{forest}}
\caption{\label{Abb-GB-Hilfsverb}English sentence with auxiliary verb in the CP/IP system}
\end{figure}%
% nach vorn verschoben
% Wie man sieht, entspricht die Struktur dem \xbars: \inull ist ein Kopf, der die VP als Komplement
% nimmt und eine I$'$ bildet. Das Subjekt\is{Subjekt} ist der Spezifikator\is{Spezifikator} der IP.
% %Die Hilfsverben stehen in \inull (=~Aux). Satzadverbien können zwischen Hilfsverb und Vollverb stehen.

The sentences in (\mex{1}) are analyzed as complementizer phrases (CPs), the complementizer is the head:

\eal
\ex that Ann will read the newspaper
\ex\label{ex-that-ann-reads-the-newspaper}that Ann reads the newspaper
\zl
In sentences such as (\mex{0}), the CPs do not have a specifier. Figure~\vref{Abb-GB-Englisch-CP} shows the analysis
of (\mex{0}a). 
\begin{figure}
% node labels for moving elements will be typeset by the \tmove command
% here we have to provide invisible boxes to get the line drawing right.
\centerline{%
\begin{forest}
sn edges
[CP
[C$'$
	[C$^0$[that]]
	[IP
		[NP[Ann]]
		[I$'$
			[I$^0$[will]]
			[VP
				[V$'$
					[V$^0$[read]]
					[NP[the newspaper, triangle]]]]]]]]
\end{forest}}
\caption{\label{Abb-GB-Englisch-CP}English complementizer phrase}
\end{figure}%


Yes/no-questions\label{Seite-GB-Entscheidungsfragen-Englisch} in English such as those in (\mex{1}) are formed by moving the 
auxiliary verb\is{auxiliary movement} in front of the subject.
\ea
Will Ann read the newspaper?
\z
Let us assume that the structure of questions corresponds to the structure of sentences with complementizers. This means that questions are also
CPs. Unlike the sentences in (\mex{-1}), however, there is no subordinating conjunction. In the D"=structure of questions, the \cnull position is
empty and the auxiliary verb is later moved to this position. Figure~\vref{Abb-GB-Ja-Nein} shows an analysis of (\mex{0}).

\begin{figure}
\centerline{%
\begin{forest}
sn edges
[CP
[C$'$
	[C$^0$[will$_k$]]
	[IP
		[NP[Ann]]
		[I$'$
			[I$^0$[\trace$_k$]]
			[VP
				[V$'$
					[V$^0$[read]]
					[NP[the newspaper, triangle]]]]]]]]
\end{forest}
}
\caption{\label{Abb-GB-Ja-Nein}English polar question}
\end{figure}%
The original position of the auxiliary is marked by the trace \_$_k$, which is coindexed with the moved auxiliary.\is{auxiliary movement}



\emph{wh}-questions are formed by the additional movement of a constituent in front of the
auxiliary; that is into the specifier position of the CP. 
Figure~\vref{Abb-GB-Wh} shows the analysis of (\mex{1}): 
\ea
What will Ann read?
\z
%
\begin{figure}
\centerline{%
\begin{forest}
sn edges
[CP
[NP[what$_i$]]
[C$'$
	[C$^0$[will$_k$]]
	[IP
		[NP[Ann]]
		[I$'$
			[I$^0$[\trace$_k$]]
			[VP
				[V$'$
					[V$^0$[read]]
					[NP[\trace$_i$]]]]]]]]
\end{forest}
}
\caption{\label{Abb-GB-Wh}English \emph{wh}"=question}
\end{figure}%
As before, the movement of the object of \emph{read} is indicated by a trace. This is important when constructing the meaning of
the sentence. The verb assigns some semantic role to the element in its object position. Therefore,
one has to be able to ``reconstruct'' the fact that \emph{what} actually originates in this position. This is ensured by coindexation of the trace with \emph{what}.\il{English|)}

Until now, I have not yet discussed sentences without auxiliaries such as (\ref{ex-that-ann-reads-the-newspaper}). In order to analyze this kind of sentences, one has to assume that the
inflectional affix is present in the \inull position. An example analysis is given in Figure~\vref{Abb-GB-englischer-Satz-ohne-Hilfsverb}.
\begin{figure}
\centerline{%
\begin{forest}
sn edges
[IP
	[NP[Ann]]
	[I$'$
		[I$^0$[-s]]
		[VP
			[V$'$
				[V$^0$[read-]]
				[NP[the newspaper, triangle]]]]]]
\end{forest}}
\caption{\label{Abb-GB-englischer-Satz-ohne-Hilfsverb}English sentence without auxiliary}
\end{figure}%
Since the inflectional affix precedes the verb, some kind of movement operation still needs to take place.
For theory"=internal reasons, one does not wish to assume movement operations to positions lower in the
tree, hence the verb has to move to the affix and not the other way around.
%% \begin{itemize}
%% \item Die {D"=Struktur} ist die Phrasenstruktur,\\
%% die sich aus den Theta-Rastern der beteiligten \bf{lexikalischen Einheiten} ergibt.
%% \item Die {S-Struktur} berücksichtigt
%% zusätzlich die Anforderungen der {functionaln categoryn}.
%
%% Besonders wichtig:\\
%% functional categoryn können {Bewegungen} auslösen. 
%% \end{itemize}

Following this excursus on the analysis of English sentences, we can now turn to German.



\subsection{The structure of the German clause}
\label{sec-German-clause}

The CP/IP model has been adopted by many scholars for the analysis of German.\footnote{
  For GB analyses without IP, see \citew{BK89a}, \citew[\page 157]{Hoehle91},
  \citew{Haider93a,Haider97a} and \citew[Section~IV.3]{Sternefeld2006a-u}. Haider assumes
  that the function of I is integrated into the verb. In LFG\indexlfg, an IP is assumed for English
  (\citealp[Section~6.2]{Bresnan2001a}; \citealp[Section~3.2.1]{Dalrymple2001a-u}), but not for German \citep[Section~3.2.3.2]{Berman2003a}.
  In HPSG\indexhpsg, no IP is assumed.
} 
The categories C, I and V, together with their specifier positions, can be linked to the
topological\is{topology} fields as shown in Figure~\vref{Abb-GB-Topo}.

\begin{figure}
    \centering
        \begin{forest}
            sn edges original,empty nodes
            [CP
              [{}
                [XP,terminus
                  [SpecCP\\prefield, name=p1
                  ]
                ]
              ]
              [C$'$
                    [{}
                      [C$^0$, terminus
                        [C$^0$\\left SB, name=c0
                        ]
                      ]
                    ]
                    [IP
                      [{}
                        [XP, terminus
                          [{IP (without I$^0$, V$^0$)\\middle field}
                            [SpecIP\\subject position, set me left, name=specip
                            ]
                            [phrases inside\\the VP, name=p3
                            ]
                          ]
                        ]
                      ]
                      [I$'$
                              [VP, name=vp
                                [V$^0$, name=v0, terminus, no path, anchor=east
                                  [{V$^0$, I$^0$\\right SB}, name=p2, set me left
                                  ]
                                ]
                              ]
                              [{}
                                    [I$^0$, terminus, name=io
                                    ]
                              ]
                      ]
                    ]
              ]
            ]
            \draw [thick]
              (p1.north west) rectangle (io.east |- p3.south);
            \draw
              ($(c0.north east)!1/2!(specip.west |- c0.north east)$) coordinate (p6) -- (p6 |- p3.south)
              ($(p1.north east)!1/2!(c0.north west)$) coordinate (p4) -- (p3.south -| p4)
              ($(specip.north east)!1/2!(p3.north west)$) coordinate (p5) -- (p3.south -| p5)
              ($(p2.north west)!1/2!(p2.north west -| p3.east)$) coordinate (p7) -- (p3.south -| p7)
              (p6 |- p2.south) -- (p2.south -| p7)
              (vp.south) -- (v0.center -| p3.west) -- (v0.west)
              (v0.east) -- +(4.5pt,0) -- (vp.south)
              ;
        \end{forest}
\is{field!pre-}\is{field!middle-}\is{sentence bracket}
\caption{\label{Abb-GB-Topo}CP, IP and VP and the topological model of German}
\end{figure}%
%% \begin{itemize}
%% \item Zuordnung zum Vorfeld dadurch motiviert, dass im Bairischen
%%       zusätzlich zur w/d-Phrase ein Komplementierer auf"|treten kann.
%% \item außerdem theorieinterne Gründe (Phrasenposition vs.\ Kopfposition)
%% \item Diese Zuordnung schafft allerdings empirische Probleme:
%%       \begin{itemize}
%%       \item Komplementierer kann mit w-Phrase koordiniert werden \citep{Reis85} %S. 301
%%       \item w/d-Phrasen verhalten sich in Bezug auf Verum-Fokus genauso wie Komplementierer
%%       \item w/d-Phrasen können in Dialekten wie Komplementierer flektiert werden
%%       \end{itemize}
%% \end{itemize}
Note that SpecCP and SpecIP are not category symbols. They do not occur in grammars with rewrite rules. Instead, they simply describe
positions in the tree.%
\is{category!functional!I|)}


As shown in Figure~\ref{Abb-GB-Topo}, it is assumed that the highest argument of the verb (the subject in simple sentences) 
has a special status. It is taken for granted that the subject always occurs outside of the VP, which is why it is referred
to as the external argument.\is{argument!external} The VP itself does not have a specifier. In more recent work, however, the subject
is generated in the specifier of the VP \citep{FS86a-u,KS91a-u}. In some languages, it is assumed that it moves to a position
outside of the VP. In other languages such as German, this is the case at least under certain conditions (\eg, definiteness, see
\citealp{Diesing92a}). I am presenting the classical GB analysis here, where the subject is outside the VP. All arguments other than
the subject are complements of the V, that are realized within the VP, that is, they are internal arguments. If the verb requires just one complement, then this is the
sister of the head V$^0$ and the daughter of V$'$ according to the \xbar schema. The accusative object is the prototypical complement.

Following the \xbar template, adjuncts\is{adjunct|(} branch off above the complements of V$'$. The analysis of a VP
with an adjunct is shown in Figure~\vref{GB-Adjunkte}.
\ea
\gll weil der Mann morgen den Jungen trifft\\
	 because the man tomorrow the boy meets\\
\glt `because the man is meeting the boy tomorrow'
\z

\begin{figure}
\centerline{%
\begin{forest}
sn edges
[VP
[V$'$
	[Adv[morgen;tomorrow]]
	[V$'$
		[NP[den Jungen;the boy, triangle]]
		[V[triff-;meet]]]]]
\end{forest}}
\caption{\label{GB-Adjunkte}Analysis of adjuncts in \gbt}
\end{figure}%
\is{adjunct|)}


\section{Verb position}
\label{Abschnitt-Verbstellung-GB}

In German, the position of the heads of VP and IP (\vnull and \inull) are to the right of their
complements and \vnull and \inull form part of the right sentence bracket.
The subject and all other constituents (complements and adjuncts) all occur to the left of \vnull
and \inull and form the middle field. It is assumed
that German -- at least in terms of D"=structure -- is an SOV language (=~ a language with the base order Subject--Object--Verb).
The analysis of German as an SOV language is almost as old as Transformational Grammar itself. It was originally proposed by
\citet*[\page34]{Bierwisch63}.\footnote{
	Bierwisch attributes the assumption of an underlying verb"=final order to \citet{Fourquet57a}. A German translation of the
	French manuscript cited by Bierwisch can be found in \citew[\page117--135]{Fourquet70a}. For other proposals, see \citew{Bach62a},
\citew{Reis74a}, \citew{Koster75a} and \citew[Chapter~1]{Thiersch78a}. Analyses which assume that
German has an underlying SOV pattern were also suggested in \gpsg \citep[\page110]{Jacobs86a}, 
LFG \citep[Section~2.1.4]{Berman96a-u} and HPSG   (\citealp*{KW91a}; \citealp{Oliva92a}; \citealp*{Netter92};   
\citealp*{Kiss93}; \citealp*{Frank94}; \citealp*{Kiss95a}; \citealp{Feldhaus97};
\citealp{Meurers2000b}; \citealp{Mueller2005c,MuellerGS}). 
}
Unlike German, Germanic languages like Danish\il{Danish}, English\il{English} and Romance languages
like French\il{French} are SVO languages, whereas Welsh\il{Welsh} and Arabic\il{Arabic}
are VSO languages. Around 40\,\% of all languages belong to the SOV languages, around 35\,\% are
SVO \citep{wals-81}.
% 41,03 SOV
% 35,4  SVO
%  6,9  VSO

The assumption of verb"=final order\label{page-verbletzt} as the base order is motivated by the following observations:\footnote{
	For points 1 and 2, see \citew[\page34--36]{Bierwisch63}. For point~\ref{SOV-Skopus} see \citew[Section~2.3]{Netter92}.%
}

% Ich glaube, über Deutsch würde man sagen, dass Prominenz rechts in der phonologischen Phrase ist,
% aber nicht ausschließlich durch Längung markiert, sondern vor allem durch pitch. Demnach wäre
% deutsch wieder irgendwie dazwischen... Außerdem kommt bei Deutsch (wie bei Englisch) noch
% Deakzentuierung von Gegebenem dazu, da kann Prominenz auch mal nicht-final sein. Oder man sagt,
% dass in diesen Fällen dann automatisch immer die phonologische Phrase zuende ist, dann ist der
% Akzent wieder final. Caroline Féry hat dazu Papiere. Diese Deakzentuierung gibt es im Spanischen
% (und ich denke auch im Italienischen) nicht in dieser Art. Japanisch weiß ich nicht.
%
% Viele Grüße,
% Felix
%
%
%  Am 30.09.11 21:00, schrieb Stefan Müller:
% >
% > Habe gerade folgendes gefunden:
% >
% >> Interestingly, because head-final languages (\eg, Turkish and
% >> Japanese) mark prominence in phonological phrases initially through
% >> pitch and intensity and head-initial languages (\eg, Italian and
% >> English) mark prominence in phonological phrases finally through
% >> duration (Nespor et al., 2008),
% >
% >
% > Wie macht das Deutsche das und ist es dann also head-final oder
% > -initial. Das wollte ich ja schon immer mal wissen ...
% >
% > Viele Grüße
% >
% >     Stefan


\begin{enumerate}
\item Verb particles form a close unit with the verb.
\eal
\ex 
\gll weil er morgen an-fängt\\
     because he tomorrow \textsc{prt}-starts\\
\glt `because he is starting tomorrow'
\ex 
\gll Er fängt morgen an.\\
	 he starts tomorrow \textsc{prt}\\
\glt `He is starting tomorrow.'
\zl
This unit can only be seen in verb"=final structures, which speaks for the fact that this structure reflects the base order.

Verbs which are derived from a noun by back-formation\is{back-formation} (\eg, \emph{uraufführen} 
`to perform something for the first time', can often not be divided into their component parts and
V2 clauses are therefore ruled out (This was first mentioned by \citet{Hoehle91b} in unpublished
work. The first published source is \citew[\page 62]{Haider93a}):
\eal
\ex[]{
\gll weil sie das Stück heute ur-auf-führen\\
	 because they the play today \textsc{pref}-\textsc{part}-lead\\
\glt `because they are performing the play for the first time today'
}
\ex[*]{
\gll Sie ur-auf"|führen heute das Stück.\\
	 they \textsc{pref}-\textsc{part}-lead today the play\\
}
\ex[*]{
\gll Sie führen heute das Stück ur-auf.\\
	 they lead today the play \textsc{pref}-\textsc{part}\\
}
\zl
The examples show that there is only one possible position for the verb. This order is the one that is assumed to be the base order.
\item Verbs in non"=finite clauses and in finite subordinate clauses with a conjunction are
always in final position (I am ignoring the possibility of extraposing constituents):
\eal
\ex 
\gll Der Clown versucht, Kurt-Martin die Ware zu geben.\\
     the clown tries Kurt-Martin the goods to give\\
\glt `The clown is trying to give Kurt-Martin the goods.'
\ex 
\gll dass der Clown Kurt-Martin die Ware gibt\\
	 that the clown Kurt-Martin the goods gives\\
\glt `that the clown gives Kurt-Martin the goods'
\zl

\item If one compares the position of the verb in German to Danish\il{Danish} (Danish is an SVO language
like English), then one can clearly see that the verbs in German form a cluster at the end of the sentence,
whereas they occur before any objects in Danish \citep[\page 146]{Oersnes2009b}:
\eal
\ex 
\gll dass er ihn gesehen$_3$ haben$_2$ muss$_1$\\
	 that he him seen have must\\
\ex 
\gll at han må$_1$ have$_2$ set$_3$ ham\\
     that he must have seen him\\
\glt `that he must have seen him'
\zl
\item\label{SOV-Skopus}\is{scope|(} The scope relations of the adverbs in (\ref{bsp-absichtlich-nicht-anal}) depend on their order:
the left"=most adverb has scope over the two following elements.\footnote{%
At this point, it should be mentioned that there seem to be exceptions from the rule that modifiers to the left take scope over those to
their right. \citet*[\page47]{Kasper94a} discusses examples such as (i), which go back to \citet*[\page137]{BV72}.
\eal
\label{bsp-peter-liest-gut-wegen}
\ex 
\gll Peter liest gut wegen der Nachhilfestunden.\\
	 Peter reads well because.of the tutoring\\
\ex 
\gll Peter liest wegen der Nachhilfestunden gut.\\
	 Peter reads because.of the tutoring well\\
\glt `Peter can read well thanks to the tutoring.'
\zl
% Kiss95b:212
	As \citet[Section~6]{Koster75a} and \citet*[\page67]{Reis80a} have shown, these are not particularly convincing counter"=examples
	as the right sentence bracket is not filled in these examples and therefore the examples are
        not necessarily instances of normal reordering inside of the middle field, but could instead
        involve extraposition\is{extraposition} of the PP.
	As noted by Koster and Reis, these examples become ungrammatical if one fills the right bracket and does not extrapose the causal adjunct:
\eal
\ex[*]{
\gll Hans hat gut  wegen      der Nachhilfestunden gelesen.\\
     Hans has well because.of the tutoring read\\
}
\ex[]{
\gll Hans hat gut gelesen wegen der Nachhilfestunden.\\
	 Hans has well read because.of the tutoring\\
\glt `Hans has been reading well because of the tutoring.'
}
\zl
However, the following example from \citet[\page 383]{Crysmann2004a} shows that, even with the right bracket occupied, one can still have an
order where an adjunct to the right has scope over one to the left:
\ea
\gll Da muß es schon erhebliche Probleme mit der Ausrüstung gegeben haben, da wegen schlechten
  Wetters ein Reinhold Messmer niemals aufgäbe.\\
  there must it already serious problems with the equipment given have since because.of bad weather a Reinhold Messmer never
  would.give.up\\
 \glt `There really must have been some serious problems with the equipment because someone like Reinhold Messmer would never give
  up just because of some bad weather.'
%\ex Stefan  ist wohl deshalb krank geworden, weil er äußerst hart wegen der Konferenz in Bremen gearbeitet hat.
\z
Nevertheless, this does not change anything regarding the fact that the corresponding cases in (\ref{bsp-absichtlich-nicht-anal}) 
and (\ref{bsp-absichtlich-nicht-anal-v1}) have the same meaning regardless of the position of the verb. The general means of semantic
composition may well have to be implemented in the same way as in Crysmann's analysis.%
}
This was explained by assuming the following structure:
\eal
\label{bsp-absichtlich-nicht-anal}
\ex 
\gll weil er [absichtlich [nicht lacht]]\\
	 because he \spacebr{}intentionally \spacebr{}not laughs\\
\glt `because he is intentionally not laughing'
\ex 
\gll weil er [nicht [absichtlich lacht]]\\
     because he \spacebr{}not \spacebr{}intentionally laughs\\
\glt `because he is not laughing intentionally'
\zl
It is interesting to note that scope relations are not affected by verb position. If one assumes that sentences with verb"=second
order have the underlying structure in (\mex{0}), then this fact requires no further explanation. (\mex{1}) shows
the derived S"=structure for (\mex{0}):
\eal
\label{bsp-absichtlich-nicht-anal-v1}
\ex 
\gll Er lacht$_i$ [absichtlich [nicht \_$_i$]].\\
     he laughs \spacebr{}intentionally \spacebr{}not\\
\glt `He is intentionally not laughing.'
\ex 
\gll Er lacht$_i$  [nicht [absichtlich \_$_i$]].\\
     he laughs \spacebr{}not \spacebr{}intentionally\\
\glt `He is not laughing intentionally.'
\zl\is{scope}
%\item Verum-Fokus
\nocite{Hoehle88a,Hoehle97a}
\end{enumerate}\is{verb final language}
%% Nebeneffekt der SOV-Struktur: Je enger
%% sich ein Satzglied auf das Verb bezieht, desto näher steht es an der rechten Satzklammer und auch
%% dann, wenn das Verb wegbewegt wurde.

\noindent
After motivating and briefly sketching the analysis of verb"=final order, I will now look 
at the CP/IP analysis of German in more detail. \cnull corresponds to the left sentence bracket and can be filled
in two different ways: in subordinate clauses introduced by a conjunction, the subordinating conjunction (the
complementizer\is{complementizer}) occupies \cnull as in English. The verb remains in the right
sentence bracket, as illustrated by (\mex{1}).
\ea 
\gll dass jeder diesen Mann kennt\\
     that everybody this man knows\\
\glt `that everybody knows this man'
\z
Figure~\vref{Abb-GB-Komplementierer} gives an analysis of (\mex{0}).
\begin{figure}
\centering
\begin{forest}
sn edges
[CP
[C$'$
	[C$^0$[dass;that]]
	[IP
		[NP [jeder;everybody]]
		[I$'$
			[VP
				[V$'$
					[NP[diesen Mann;this man, triangle]]
					[V$^0$[\trace$_j$]]]]
			[I$^0$[kenn-$_j$ -t;know- -s]]]]]]
\end{forest}
\caption{\label{Abb-GB-Komplementierer}Sentence with a complementizer in \cnull}
\end{figure}%
In verb"=first and verb"=second clauses, the finite verb is moved to \cnull via the \inull position:
\vnull $\to$  \inull $\to$ \cnull.  Figure~\vref{Abb-GB-Verberststellung} shows the analysis of (\mex{1}):
\ea
\gll Kennt jeder diesen Mann?\\
	 knows everybody this man\\
\glt `Does everybody know this man?'
\z
\begin{figure}
\centerline{%
\begin{forest}
sn edges
[CP
[C$'$
	[C$^0$[(kenn-$_j$ -t)$_k$;knows]]
	[IP
		[NP [jeder;everybody]]
		[I$'$
			[VP
				[V$'$
					[NP [diesen Mann; this man, triangle]]
					[V$^0$[\trace$_j$]]]]
			[I$^0$ [\trace$_k$]]]]]]
\end{forest}
}
\caption{\label{Abb-GB-Verberststellung}Verb position in GB}
\end{figure}%
The \cnull position is empty in the D"=structure of (\mex{0}). Since it is not occupied by a complementizer, the
verb can move there.\is{verb position}


\section{Long"=distance dependencies}


The\is{long"=distance dependency|(}\is{verb position!-second-} SpecCP position corresponds to the prefield and can
be filled by any XP in declarative clauses\is{declarative clause} in German. In this way, one can derive the
sentences in (\mex{2}) from (\mex{1}) by moving a constituent in front of the verb:
\ea
\gll Gibt der Mann dem Kind jetzt den Mantel?\\
     gives the.\nom{} man the.\dat{} child now the.\acc{} coat\\
\glt `Is the man going to give the child the coat now?'
\z

\eal
\ex 
\gll Der Mann gibt dem Kind jetzt den Mantel.\\
     the.\nom{} man gives the.\dat{} child now the.\acc{} coat\\
\glt `The man is giving the child the coat now.'
\ex 
\gll Dem Kind gibt der Mann jetzt den Mantel.\\
     the.\dat{} child gives the.\nom{} man now the.\acc{} coat\\
\ex 
\gll Den Mantel gibt der Mann dem Kind jetzt.\\
	 the.\acc{} coat gives the.\nom{} man the.\dat{} child now\\
\ex 
\gll Jetzt gibt der Mann dem Kind den Mantel.\\
	 now gives the.\nom{} man the.\dat{} child the.\acc{} coat\\
\zl
Since any constituent can be placed in front of the finite verb, German is treated typologically as one of the
verb"=second languages\is{verb"=second language} (V2). Thus, it is a verb"=second language with SOV base order.
English, on the other hand, is an SVO language without the V2 property, whereas Danish is a V2 language with SVO
as its base order (see \citew{Oersnes2009b} for Danish).

Figure~\vref{Abb-GB-Vorfeldbesetzung} shows the structure derived from Figure~\ref{Abb-GB-Verberststellung}.
\begin{figure}
\centering
\begin{forest}
sn edges
[CP
[NP [diesen Mann$_i$;this man, triangle]]
[C$'$
	[C$^0$[(kenn-$_j$ -t)$_k$; know- -s]]
	[IP
		[NP [jeder;everybody]]
		[I$'$
			[VP
				[V$'$
					[NP[\trace$_i$]]
					[V$^0$[\trace$_j$]]]]
			[I$^0$ [\trace$_k$]]]]]]
\end{forest}
\caption{\label{Abb-GB-Vorfeldbesetzung}Fronting in \gbt}
\end{figure}%
The crucial factor for deciding which phrase to move is the \emph{information structure}\is{information structure}
of the sentence. That is, material connected to previously mentioned or otherwise"=known information is 
placed further left (preferably in the prefield) and new information tends to occur to the right. Fronting to the
prefield in declarative clauses is often referred to as
\emph{topicalization}\label{Seite-Topikalisierung}\is{topicalization}. But this is rather a
misnomer, since the focus\is{focus} (informally: the constituent being asked for) can also occur in the prefield. Furthermore, expletive
pronouns\is{pronoun!expletive} can occur there and these are non"=referential and as such cannot be
linked to preceding or known information, hence expletives can never be topics.

Transformation"=based analyses also work for so"=called \emph{long"=distance dependencies}, that is, dependencies
over several phrase boundaries:
\eal
\label{bsp-Fernabhaengigkeit}
\ex\label{bsp-um-zwei-millionen}
\gll {}[Um zwei Millionen Mark]$_i$ soll er versucht haben, [eine Versicherung \_$_i$ zu betrügen].\footnotemark\\
     {}\spacebr{}around two million Deutsche.Marks should he tried have \spacebr{}an insurance.company {} to deceive\\
\footnotetext{
         taz, 04.05.2001, p.\,20.
}
\glt `He apparently tried to cheat an insurance company out of two million Deutsche Marks.'
\ex
\gll "`Wer$_i$, glaubt er, daß er \_$_i$ ist?"' erregte sich ein Politiker vom Nil.\footnotemark\\
     \spacebr{}who believes he that he {} is retort \textsc{refl} a politician from.the Nile\\
\footnotetext{
        Spiegel, 8/1999, p.\,18.
}
\glt `\,``Who does he think he is?'', a politician from the Nile exclaimed.'
\ex\label{ex-wen-glaubst-du-dass}
\gll Wen$_i$ glaubst du, daß ich \_$_i$ gesehen habe?\footnotemark\\
     who believe you that I {} seen have\\
\footnotetext{
    \citew[\page84]{Scherpenisse86a}.
    }
\glt `Who do you think I saw?'
\ex 
\gll {}[Gegen ihn]$_i$ falle es den Republikanern hingegen schwerer, [~[~Angriffe \_$_i$] zu lancieren].\footnotemark\\
	 {}\spacebr{}against him fall it the Republicans however more.difficult \hspaceThis{[~[~}attacks {} to launch\\
\footnotetext{
  taz, 08.02.2008, p.\,9.
}
\glt `It is, however, more difficult for the Republicans to launch attacks against him.'
\zl
The elements in the prefield in the examples in (\mex{0}) all originate from more deeply embedded phrases. In GB,
it is assumed that long"=distance dependencies across sentence boundaries are derived in steps 
\citep[\page 75--79]{Grewendorf88a}, that is, in the analysis of (\ref{ex-wen-glaubst-du-dass}), the interrogative
pronoun is moved to the specifier position of the \emph{dass}"=clause and is moved from there to the specifier of
the matrix clause.\todostefan{Gianina: Shouldn't then (37c) indicate this trace in fromt of dass?} The reason for this is that there are certain restrictions on movement which must be checked
locally.%
\is{long"=distance dependency|)}

\section{Passive}
\label{Abschnitt-GB-Passiv}\label{sec-passive-gb}

Before\is{passive} I turn to the analysis of the passive in Section~\ref{sec-case-assignment}, the first subsection will elaborate on the differences
between structural\is{case!structural|(} and lexical\is{case!lexical|(} case.

\subsection{Structural and lexical case}
\label{Abschnitt-struktureller-Kasus}

The case of many case-marked arguments is dependent on the syntactic environment in which the head
of the argument is realized. These arguments are referred to as arguments with \emph{structural case}. Case"=marked arguments, which
do not bear structural case, are said to have \emph{lexical case}.\footnote{%
Furthermore, there is a so"=called \emph{agreeing case}\is{case!agreement} (see page~\pageref{page-Kasuskongruenz}) and \emph{semantic case}. Agreeing case is
found in predicatives. This case also changes depending on the structure involved, but the change is
due to the antecedent element changing its case. Semantic case depends on the function of certain
phrases (\eg, temporal accusative adverbials). Furthermore, as with lexical case of objects, semantic case does not change
depending on the syntactic environment. For the analysis of the passive, which will be discussed in this section, only
structural and lexical case will be relevant.}

The following are examples of structural case:\footnote{
        Compare \citew*[\page 200]{HM94a}.

		(\mex{1}b) is a so"=called AcI\is{verb!AcI|uu} construction. AcI stands for \emph{Accusativus cum infinitivo}, which means ``accusative
		with infinitive''. The logical subject of the embedded verb (\emph{kommen} `to come'
                in this case) becomes the accusative object of the matrix verb \emph{lassen} `to let'.
		Examples for AcI"=verbs are perception verbs\is{verb!perception} such as \emph{hören} `to hear' and \emph{sehen} `to see'
		as well as \emph{lassen} `to let'.
}
\eal
\ex 
\gll Der Installateur kommt.\\
	 the.\nom{} plumber comes\\
\glt `The plumber is coming.'
\ex 
\gll Der Mann lässt den Installateur kommen.\\
	 the man lets the.\acc{} plumber come\\
\glt `The man is getting the plumber to come.'
\ex 
\gll das Kommen des Installateurs\\
	 the coming of.the plumber\\
\glt `the plumber's visit'
\zl
In the first example, the subject is in the nominative case\is{case!nominative}, whereas \emph{Installateur} `plumber' is
in accusative\is{case!accusative} in the second example and even in the genitive\is{case!genitive} in the third following nominalization.\is{nominalization}
The accusative case of objects is normally structural case. This case becomes nominative under passivization\is{passive}:
\eal
\ex 
\gll Karl schlägt den Weltmeister.\\
     Karl beats the.\acc{} world.champion\\
\glt `Karl beats the world champion.'
\ex 
\gll Der Weltmeister wird geschlagen.\\
	 the.\nom{} world.champion is beaten\\
\glt `The world champion is being beaten.'
\zl

\noindent
Unlike the accusative, the genitive\is{case!genitive} governed by a verb is a lexical case. The case of a genitive object does not change when
the verb is passivized.
\eal
\ex
\gll Wir gedenken der Opfer.\\
     we remember the.\gen{} victims\\
\ex 
\gll Der Opfer wird gedacht.\\
	 the.\gen{} victims are remembered\\
\glt `The victims are being remembered.'
\zl
(\mex{0}b) is an example of the so"=called \emph{impersonal passive}\is{passive!impersonal}. Unlike example (\mex{-1}b), where the accusative
object became the subject\is{subject}, there is no subject in (\mex{0}b). See Section~\ref{Abschnitt-Subjekt}.

Similarly, there is no change in case with dative objects\is{case!dative|(}:
\eal
\ex 
\gll Der Mann hat ihm geholfen.\\
	 the man has him.\dat{} helped\\
\glt `The man has helped him.'
\ex 
\gll Ihm wird geholfen.\\
	 him.\dat{} is helped\\
\glt `He is being helped.'
\zl
It still remains controversial as to whether some or all of the datives in verbal environments should be treated as instances of structural case.
For reasons of space, I will not recount this discussion but instead refer the interested reader to Chapter~14 of \citew{MuellerLehrbuch1}.
In what follows, I assume -- like \citet[\page 20]{Haider86} -- that the dative is in fact a lexical case.\is{case!lexical} 



\subsection{Case assignment and the Case Filter}
\label{sec-case-assignment}

In GB, it is assumed that the subject receives case from (finite) I\is{category!functional!I} and
that the case of the remaining arguments comes from
V (\citealp[\page 50]{Chomsky81a}; \citealp[\page
26]{Haider84b}; \citealp[\page 71--73]{FF87}). 
\begin{principle-break}[Case Principle]\label{Kasusprinzip-GB}
\begin{itemize}
% hier stand `seinem', aber in den Abbildungen gibt es zwei Komplemente
\item V assigns objective case (accusative) to its complement if it bears structural case.
\item When finite, INFL assigns case to the subject.
\end{itemize}
\end{principle-break}
The Case Filter\is{case!filter} rules out structures where case has not been assigned to an NP.

Figure~\vref{Abb-GB-Aktiv} shows the Case Principle in action with the example in 
(\mex{1}a).\footnote{%
The figure does not correspond to \xbar theory in its classic form, since \emph{der Frau} `the woman' 
is a complement which is combined with V$'$.  In classical \xbar theory, all complements have to be combined
with \vnull. Furthermore, in the following figures the verb has been left in \vnull for reasons of clarity. In order
to create a well"=formed S"=structure, the verb would have to move to its affix in \inull.%
}
\eal
\ex 
\gll {}[dass] der Mann der Frau den Jungen zeigt\\
     \spacebr{}that the man the.\dat{} woman the.\acc{} boy shows\\
\glt `that the man shows the boy to the woman'
\ex 
\gll{}[dass] der Junge der Frau gezeigt wird\\
      \spacebr{}that the boy.\nom{} the.\dat{} woman shown is\\
\glt `that the boy is shown to the woman'
\zl
\begin{figure}
\hfill
\begin{forest}
sn edges
[IP
  [NP, name=subject [der Mann;the man, triangle]]
  [I$'$
	[VP
		[V$'$
			[NP, name=dobject [der Frau;the woman, triangle]]
			[V$'$
				[NP,   name=aobject [den Jungen;the boy, triangle]]
				[V$^0$,name=verb    [zeig-;show-]]]]]
	[I$^0$, name=Infl [-t;-s]]]]
\draw[->,dotted] (Infl.north) .. controls (3.5,-0.3) and (-1.5,.5) .. ($(subject.north)+(-.1,.1)$);
\draw[->]        (verb.north) .. controls (2.8,-2.9) and (-.4,.5) .. ($(subject.north)+(0,.1)$);
\draw[->,dashed] (verb.north) .. controls (2.8,-3.0) and (0,-3.3) .. ($(dobject.north)+(0,.1)$);
\draw[->,dashed] (verb.north) .. controls (2.3,-4.3) and (1.8,-4.5) .. ($(aobject.north)+(0,.1)$);
%\draw (-4,-7) to[grid with coordinates] (4,0.5);
\end{forest} \hspace{1cm}
\begin{tabular}[b]{ll@{}}
\tikz[baseline]\draw[dotted](0,1ex)--(1,1ex);&just case\\
\tikz[baseline]\draw(0,1ex)--(1,1ex);&just theta role\\
\tikz[baseline]\draw[dashed](0,1ex)--(1,1ex);&case and theta role
\\
\\
\end{tabular}
\caption{\label{Abb-GB-Aktiv}Case and theta-role assignment in active clauses}
\end{figure}%
%
The passive morphology blocks the subject. The object that would get accusative in the active
receives a semantic role in passives, but it does not get case. Therefore, it has to move to a position where case can be assigned to it 
\citep[\page 124]{Chomsky81a}. Figure~\vref{Abb-GB-Passiv} shows how this works for example
(\mex{0}b).
\begin{figure}
\hfill
\begin{forest}
sn edges
[IP
[NP, name=subject [der Junge$_i$;the boy ,triangle]]
[I$'$
	[VP
		[V$'$
			[NP, name=dobject [der Frau;the woman, triangle]]
			[V$'$
				[NP,   name=aobject [\_$_i$]]
				[V$^0$,name=verb [gezeigt wir-;shown is, triangle]]]]]
	[I$^0$ ,name=Infl [-\/d]]]]
\draw[->,dotted] (Infl.north) .. controls (2.5,.35)   and (-1.5,-.05) .. ($(subject.north)+(0,.1)$);
\draw[->,dashed] (verb.north) .. controls (2.2,-3.4) and (0,-3.1) .. ($(dobject.north)+(0,.1)$);
\draw[->]        (verb.north) .. controls (2.0,-4.55) and (1.2,-4.35) .. ($(aobject.north)+(0,.1)$);
%\draw (-4,-7) to[grid with coordinates] (4,0.5);
\end{forest}\hspace{1cm}
\begin{tabular}[b]{ll@{}}
\tikz[baseline]\draw[dotted](0,1ex)--(1,1ex);&just case\\
\tikz[baseline]\draw(0,1ex)--(1,1ex);&just theta role\\
\tikz[baseline]\draw[dashed](0,1ex)--(1,1ex);&case and theta role
\\
\\
\end{tabular}
\caption{\label{Abb-GB-Passiv}Case and theta-role assignment in passive clauses}
\end{figure}%

This movement"=based analysis works well for English\il{English} since the underlying object always has to
be moved:
\eal
\ex[]{
The mother gave [the girl] [a cookie].
}
\ex[]{
{}[The girl] was given [a cookie] (by the mother).
}
\ex[*]{
It was given [the girl] [a cookie].
}
\zl
%
(\mex{0}c) shows that filling the subject position with an expletive is not possible, so the object
really has to move. However, \citet[Section~4.4.3]{Lenerz77} showed that such a movement is not obligatory in German:

\eal
\label{ex-passive-German-no-movement}
\ex 
\gll weil das Mädchen dem Jungen den Ball schenkte\\
     because the.\nom{} girl the.\dat{} boy the.\acc{} ball gave\\
\glt `because the girl gave the ball to the boy'
\ex 
\gll weil dem Jungen der Ball geschenkt wurde\\
	 because the.\dat{} boy the.\nom{} ball given was\\
\glt `because the ball was given to the boy'
\ex 
\gll weil der Ball dem Jungen geschenkt wurde\\
     because the.\nom{} ball the.\dat{} boy given was\\
\zl
In comparison to (\mex{0}c), (\mex{0}b) is the unmarked order. \emph{der Ball} `the ball' in (\mex{0}b) occurs
in the same position as \emph{den Ball} in (\mex{0}a), that is, no movement is necessary. Only the case differs.
(\mex{0}c) is, however, somewhat marked in comparison to (\mex{0}b). The analysis which has been proposed for
cases such as (\mex{0}b) involves abstract movement: the elements stay in their positions, but are connected to
the subject position and receive their case information from there. \citet[\page 1311]{Grewendorf93}
assumes that there is an empty expletive pronoun\is{empty element}\is{pronoun!expletive}
in the subject position of sentences such as (\mex{0}b) as well as in the subject position of sentences with an
impersonal passive\is{passive!impersonal} such as (\mex{1}):\footnote{
	See \citew[\page 11--12]{Koster86a} for a parallel analysis for Dutch\il{Dutch} as well as 
	\citew{Lohnstein2014a} for a movement"=based account of the passive that also involves an
        empty expletive for the analysis of the impersonal passive.
}\todostefan{Gianina: to difficult to parse}
\ea
\gll weil heute nicht gearbeitet wird\\
	 because today not worked is\\
\glt `because there will be no work done today'
\z
A silent expletive pronoun is something that one cannot see or hear and that does not carry any meaning. For discussion of 
this kind of empty elements, see Section~\ref{Abschnitt-UG-EPP} and Chapter~\ref{Abschnitt-Diskussion-leere-Elemente}.

In the following chapters, I describe alternative treatments of the passive that do without mechanisms such as
empty elements that are connected to argument positions and seek to describe the passive in a more
general, cross"=linguistically consistent manner as the suppression of the most prominent argument.

A further question which needs to be answered is why the accusative object does not receive case from the verb.
This is captured by a constraint, which goes back to \citet[\page 178--185]{Burzio86a-u-gekauft} and is therefore
referred to as \emph{Burzio's Generalization}.\is{Burzio's Generalization}\footnote{
Burzio's original formulation was equivalent to the following: a verb assigns accusative if and only if it assigns
a semantic role to its subject.

This claim is problematic from both sides. In (i), the verb does not assign a semantic role to the subject, however
there is nevertheless accusative case:
\ea
\gll Mich friert.\\
	 me.\acc{} freezes\\
\glt `I am freezing.'
\z
One therefore has to differentiate between structural and lexical accusative and modify Burzio's Generalization
accordingly. The existence of verbs like \emph{begegnen} `to bump into' is problematic for the other side of
the implication. \emph{begegnen} has a subject but still does not assign accusative but rather
dative:
\ea
\gll Peter begegnete einem Mann.\\
     Peter met a.\dat{} man\\
\glt `Peter met a man.'
\z
\citet[\page 185]{Burzio86a-u-gekauft} assumes that one"=place intransitive verbs have the potential to assign
accusative. He supports this claim by pointing out the existence of the resultative
constructions\is{resultative construction}, in which additional accusatives can be realized. (iii) is
an example:
\ea
He talked my head off.
\z
However, there are also verbs such as \emph{verschwinden} `to disappear' which never assign
accusative, not even in such constructions.

See \citew{Haider99a} and \citew[\page 89]{Webelhuth95a} as well as the references cited there for further problems
with Burzio's Generalization.
}
\ea
Burzio's Generalization (modified):\\
If V does not have an external argument, then it does not assign (structural) accusative case.
\z

\noindent
%\interfootnotelinepenalty=100%
\citet[\page 12]{Koster86a} has pointed out that the passive in English\il{English} cannot be derived by Case
Theory since if one allowed empty expletive subjects for English as well as German and Dutch\il{Dutch}, then it would be possible
to have analyses such as the following in (\mex{1}) where np is an empty expletive:
\ea
np was read the book.
\z
Koster rather assumes that subjects in English are either bound by other elements (that is, non"=expletive) or lexically filled, that
is, filled by visible material.
Therefore, the structure in (\mex{0}) would be ruled out and it would be ensured that \emph{the book} would have to be placed in front
of the finite verb so that the subject position is filled.
\is{passive|)}

\section{Local reordering}
\label{sec-GB-lokale-Umstellung}

Arguments in the middle field can, in principle, occur in an almost arbitrary order. (\mex{1}) exemplifies this:
\eal
\label{ex-gb-umstellung}
\ex 
\gll {}[weil] der Mann der Frau das Buch gibt\\
     \spacebr{}because the man the woman the book gives\\
\glt `because the man gives the book to the woman'
\ex 
\gll {}[weil] der Mann das Buch der Frau gibt\\
     \spacebr{}because the man the book the woman gives\\
\ex\label{ex-das-buch-der-mann-der-frau-gibt} 
\gll {}[weil] das Buch der Mann der Frau gibt\\
     \spacebr{}because the book the man the woman gives\\
\ex 
\gll {}[weil] das Buch der Frau der Mann gibt\\
     \spacebr{}because the book the woman the man gives\\
\ex 
\gll {}[weil] der Frau der Mann das Buch gibt\\
     \spacebr{}because the woman the man the book gives\\
\ex 
\gll {}[weil] der Frau das Buch der Mann gibt\\
     \spacebr{}because the woman the book the man gives\\
\zl

\noindent
In (\mex{0}b--f), the constituents receive different stress and the number of contexts in which each
sentence can be uttered is more restricted than in (\mex{0}a) \citep{Hoehle82}. The order in (\mex{0}a)
is therefore referred to as the \emph{neutral order}\is{neutral order} or \emph{unmarked order}\is{order!unmarked}.

Two proposals have been made for analyzing these orders: the first suggestion assumes that the five orderings in (\mex{0}b--f) are derived from
a single underlying order by means of \movea \citep{Frey93a}. As an example, the analysis of
(\ref{ex-das-buch-der-mann-der-frau-gibt}) is given in Figure~\vref{fig-das-buch-der-mann-der-frau-gibt-movement}.
\begin{figure}
\begin{forest}
sn edges
[IP
  [NP$_i$ [das Buch;the book, triangle]]
  [IP
    [NP [der Mann;the man, triangle]]
    [I$'$
 	[VP
		[V$'$
			[NP [der Frau;the woman, triangle]]
			[V$'$
				[NP   [\trace$_i$]]
				[V$^0$  [gib-;give-]]]]]
	[I$^0$, name=Infl [-t;-s]]]] ]
\end{forest}
\caption{Analysis of local reordering as adjunction to IP}\label{fig-das-buch-der-mann-der-frau-gibt-movement}
\end{figure}%
The accusative object \emph{das Buch} `the book' is moved to the left and adjoined to the top"=most IP.

\is{scope|(}%
An argument that has often been used to support this analysis is the fact that scope ambiguities
exist in sentences with reorderings which are not present in sentences in the base order. The explanation of such ambiguities comes from the assumption that the scope of quantifiers
can be derived from their position in the surface structure as well as their position in the deep structure. If the position in both the surface
and deep structure are the same, that is when there has not been any movement, then there is only one reading possible. If movement has taken place,
however, then there are two possible readings \citep[\page ]{Frey93a}:
\eal
\ex 
\gll Es ist nicht der Fall, daß er mindestens einem Verleger fast jedes Gedicht anbot.\\
     it is not the case that he at.least one publisher almost every poem offered\\
\glt `It is not the case that he offered at least one publisher almost every poem.'
\ex 
\gll Es ist nicht der Fall, daß er fast jedes Gedicht$_i$ mindestens einem Verleger \_$_i$ anbot.\\
	 it is not the case that he almost every poem at.least one publisher {} offered\\
\glt `It is not the case that he offered almost every poem to at least one publisher.'
\zl

\noindent
It turns out that approaches assuming traces run into problems as they predict certain readings for sentences with multiple traces which
do not exist (see \citealp[\page 146]{Kiss2001a} and \citealp[Section~2.6]{Fanselow2001a}). 
For instance in an example such as (\mex{1}), it should be possible to interpret \emph{mindestens einem Verleger} `at least one publisher' at
the position of \_$_i$, which would lead to a reading where \emph{fast jedes Gedicht} `almost every poem' has scope over \emph{mindestens einem Verleger} 
`at least one publisher'.
\ea
\gll Ich glaube, dass mindestens einem Verleger$_i$ fast jedes Gedicht$_j$ nur dieser Dichter \_$_i$ \_$_j$ angeboten hat.\\
	 I believe that at.least one publisher almost every poem only this poet {} {} offered has\\
\glt `I think that only this poet offered almost every poem to at least one publisher.'
\z
This reading does not exist, however.

\citet[\page 308]{SE2002a} discuss analogous examples from Japanese\il{Japanese}, which they credit to Kazuko 
Yatsushiro\aimention{Kazuko Yatsushiro}. They develop an analysis where the first step is to move the accusative object in front of the subject.
Then, the dative object is placed in front of that and then, in a third movement, the accusative is moved once more. The last movement can
take place to construct either the S"=structure\footnote{
	The authors are working in the Minimalist framework. This means there is no longer S"=structure strictly speaking. I
	have simply translated the analysis into the terms used here.
}
or as a movement to construct the phonological form. In the latter case, this movement will not have any semantic effects.
While this analysis can predict the correct available readings, it does require a number of additional movement operations with intermediate steps.
\is{scope|)}

The alternative to a movement analysis is so"=called \emph{base generation}\is{base generation}: the starting structure generated by phrase structure
rules is referred to as the \emph{base}. One variant of base generation assumes that the verb is
combined with one argument at a time and that $\theta$"=roles are assigned in parallel to
head"=argument combinations. The order in which
arguments are combined with the verb is not specified, which means that all of the orders in (\ref{ex-gb-umstellung}) can be
generated directly without any transformations (compare this to the grammar in (\ref{psg-binaer}) on page~\pageref{psg-binaer}). This kind of analysis has
been proposed for GB by \citet{Fanselow2001a}.\footnote{%
	The base generation analysis is the natural analysis in the HPSG\indexhpsg framework. It has already been developed by Gunji\nocite{Gunji86a}
	in 1986 for Japanese\il{Japanese} and will be discussed in more detail in Section~\ref{sec-HPSG-lokale-Umstellung}. \citet[\page 313--314]{SE2002a}
	claim that they show that syntax has to be derivational, that is, a sequence of syntactic
        trees has to be derived. I am of the opinion that this cannot generally be shown to be the
        case. There is, for example, an analysis by \citet{Kiss2001a} which shows that scope
        phenomena can be explained well by constraint"=based approaches.
}
For the discussion of different approaches to describing constituent position, see 
\citew{Fanselow93a}.\is{constituent order|)}




\section{Summary and classification}
\label{sec-summary-gb}

Works in GB and some contributions to the Minimalist Program (see Chapter~\ref{chap-mp}) have led to a number of new discoveries in both language"=specific and cross"=linguistic research. In
the following, I will focus on some aspects of German syntax.

The analysis of verb movement developed in Transformational Grammar by \citet*[\page34]{Bierwisch63}, \citet{Reis74a},
\citet{Koster75a}, \citet[Chapter~1]{Thiersch78a} and \citet{denBesten83a} has become the standard analysis in almost
all grammar models (possibly with the exception of Construction Grammar\indexcxg and Dependency Grammar\indexdg).

The work by \citet{Lenerz77} on constituent order has influenced analyses in other frameworks
(the linearization rules in GPSG and HPSG go back to Lenerz' descriptions). Hai\-der's work on constituent order,
case and passive \citeyearpar{Haider84b,Haider85,Haider85b,Haider86,Haider90a,Haider93a} has had a significant influence on LFG
and HPSG analyses of German.

The entire configurationality discussion\is{configurationality}, that is, whether it is better to assume that the 
subject of finite verbs in German is inside or outside the VP, was important
(for instance \citealp*{Haider82,Grewendorf83a,Kratzer84a,Webelhuth85a,%
Sternefeld85b,%
Scherpenisse86a,%S. 31, Chapter~4
Fanselow87a,Grewendorf88a,Duerscheid89a,Webelhuth90,%
Oppenrieder91a,%
Wilder91a,Haider93a,Grewendorf93,%
Frey93a,%
%Duerscheid89a:60
Lenerz94a,%
Meinunger2000a%S. 30
}) and German unaccusative verbs\is{verb!unaccusative} received their first detailed discussion in GB circles 
\citep{Grewendorf89a,Fanselow92}. The works by Fanselow and Frey on constituent order, in 
particular with regard to information structure, have advanced German syntax quite considerably
\citep{Fanselow88,Fanselow90,Fanselow93a,Fanselow2000a,Fanselow2001a,Fanselow2003d,Fanselow2003a,Fanselow2004a,Frey2000a-u,Frey2001a,Frey2004a,Frey2005a}.
Infinitive constructions, complex predicates and partial fronting have also received detailed and successful treatments
in the GB/MP frameworks
(\citealp{Bierwisch63,Evers75a,Haider82,Haider86c,Haider90b,Haider91,Haider93a,Grewendorf83a,Grewendorf87a,Grewendorf88a,denBesten85b,Sternefeld85b,Fanselow87a,Fanselow2002a,SS88a,BK89a}; G.\,\citealp{GMueller96a,GMueller98a,VS98a}).
%,WL2001a,Wurmbrand2001a-u}). 
In the area of secondary predication, the work by \citet{Winkler97a} is particularly noteworthy.

This list of works from subdisciplines of grammar is somewhat arbitrary (it corresponds more or less to my own
research interests) and is very much focused on German. There are, of course, a wealth of other articles on other
languages and phenomena, which should be recognized without having to be individually listed here.

In the remainder of this section, I will critically discuss two points: the model of language acquisition of the Principles
\& Parameters framework and the degree of formalization inside Chomskyan linguistics (in particular the last few decades
and the consequences this has). Some of these points will be mentioned again in Part~\ref{part-discussion}. 

\subsection{Explaining language acquisition}
\label{sec-acquisition-gb}

One of the aims of Chomskyan research on grammar is to explain language acquisition. In GB, one
assumed a very simple set of rules, which was the same for all languages (\xbart), as well as
general principles that hold for all languages, but which could be parameterized for individual
languages or language classes. It was assumed that a parameter was relevant for multiple phenomena.
The Principles \& Parameters model was particularly fruitful and led to a number of interesting
studies in which commonalities and differences between languages were uncovered. From the point of
view of language acquisition, the idea of a parameter which is set according to the input has often
been cricitized as it cannot be reconciled with observable facts: after setting a parameter, a
learner would have to immediately have mastered certain aspects of that language. \citet[\page
  146]{Chomsky86} uses the metaphor of switches which can be flipped one way or the other. As it is
assumed that various areas of grammar are affected by parameters, setting one parameter should have
a significant effect on the rest of the grammar of a given learner.  However, the linguistic
behavior of children does not change in an abrupt fashion as would be expected (\citealp[\page
  731]{Bloom93a}; \citealp[\page 6]{Haider93a}; \citealp[\page 3]{Abney96a};
\citealp[Section~9.1]{AW98a}; \citealp{Tomasello2000a,Tomasello2003a};
\citealp{Newmeyer2005a})\NOTE{Seite}.  Furthermore, it has not been possible to prove that there is
a correlation between a certain parameter and various grammatical phenomena. For more on this, see
Chapter~\ref{chap-acquisition}.


The Principles and Parameters model nevertheless remains interesting for cross"=linguistic
research. Every theory has to explain why the verb precedes its objects in English\il{English} and follows them in 
Japanese\il{Japanese}. One can name this difference a parameter and then classify languages
accordingly, but whether this is actually relevant for language acquisition is being increasingly called in question.\is{language acquisition|)}

\subsection{Formalization}
\label{sec-formalization-gb}

In his 1963 work on Transformational Grammar, Bierwisch writes the following:\footnote{
Es ist also sehr wohl möglich, daß mit den formulierten Regeln Sätze erzeugt werden können,
die auch in einer nicht vorausgesehenen Weise aus der Menge der grammatisch richtigen Sätze herausfallen,
die also durch Eigenschaften gegen die Grammatikalität verstoßen, die wir nicht wissentlich aus
der Untersuchung ausgeschlossen haben. Das ist der Sinn der Feststellung, daß eine Grammatik
eine Hypothese über die Struktur einer Sprache ist. Eine systematische Überprüfung der Implikationen
einer für natürliche Sprachen angemessenen Grammatik ist sicherlich eine mit Hand nicht mehr
zu bewältigende Auf"|gabe. Sie könnte vorgenommen werden, indem die Grammatik als Rechenprogramm in einem
Elektronenrechner realisiert wird, so daß überprüft werden kann, in welchem Maße das Resultat
von der zu beschreibenden Sprache abweicht.}
\begin{quote}
It is very possible that the rules that we formulated generate
sentences which are outside of the set of grammatical sentences in an
unpredictable way, that is, they violate grammaticality due to
properties that we did not deliberately exclude in our examination. This
is meant by the statement that a grammar is a hypothesis about the
structure of a language. A systematic check of the implications of a
grammar that is appropriate for natural languages is surely a task that
cannot be done by hand any more. This task could be solved by
implementing the grammar as a calculating task on a computer so that it
becomes possible to verify to which degree the result deviates from the
language to be described. \citep*[\page 163]{Bierwisch63}
\end{quote}
Bierwisch's claim is even more valid in light of the empirical progress made in the last decades. For example,
\citet{Ross67} identified restrictions for movement and long"=distance dependencies and \citet{Perlmutter78} discovered
unaccusative verbs in the 70s. For German, see \citew{Grewendorf89a} and \citew{Fanselow92}.
Apart from analyses of these phenomena, restrictions on possible constituent positions have been developed
\citep{Lenerz77}, as well as analyses of case assignment \citep*{YMJ87,Meurers99b,Prze99} and theories of
verbal complexes\is{verbal complex} and the fronting of parts of phrases  (\citealp{Evers75a,Grewendorf88a,HN94a,Kiss95a}; G.\ \citealp{GMueller98a};
\citealp{Meurers99c}; \citealp{Mueller99a,Mueller2002b}; \citealp{deKuthy2002a}). All these
phenomena interact!

Consider another quote:
\begin{quote}
A goal of earlier linguistic work, and one that is still a central goal of the
linguistic work that goes on in computational linguistics, is to develop grammars
that assign a reasonable syntactic structure to every sentence of English,
or as nearly every sentence as possible. This is not a goal that is currently much
in fashion in theoretical linguistics. Especially in Government-Binding theory
(GB), the development of large fragments has long since been abandoned in
favor of the pursuit of deep principles of grammar.
The scope of the problem of identifying the correct parse cannot be appreciated
by examining behavior on small fragments, however deeply analyzed.
Large fragments are not just small fragments several times over\,--\,there is a
qualitative change when one begins studying large fragments. As the range of
constructions that the grammar accommodates increases, the number of undesired parses for sentences
increases dramatically. \citep[\page 20]{Abney96a}
\end{quote}
%
So, as Bierwisch and Abney point out, developing a sound theory of a large fragment of a human
language is a really demanding task. But what we aim for as theoretical linguists is much more: the aim is to formulate restrictions which ideally hold for all languages or at least
for certain language classes. It follows from this, that one has to have an overview of the interaction of various phenomena in
not just one but several languages. This task is so complex that individual researchers cannot manage it. This is
the point at which computer implementations become helpful as they immediately flag inconsistencies in a theory. 
After removing these inconsistencies, computer implementations can be used to systematically analyze test data
or corpora and thereby check the empirical adequacy of the theory
(Müller, \citeyear[Chapter~22]{Mueller99a}; \citeyear{MuellerCoreGram}; \citeyear{MuellerKernigkeit}; \citealp{OF98}; \citealp{Bender2008c}, see Section~\ref{sec-formal}).

More than 50 years after the first important published work by Chomsky, it is apparent that there has not been
one large"=scale implemented grammatical fragment on the basis of Transformational Grammar analyses. Chomsky
has certainly contributed to the formalization of linguistics and developed important formal foundations which
are still relevant in the theory of formal languages\is{language!formal} in computer science\is{computer science} and
in theoretical computational linguistics \citep{Chomsky59a-u}. However, in 1981, he had already turned his back on rigid
formalization:
\begin{quote}
%It is this point of view that lies behind the rough distinction between leading ideas and execution, 
%and that motivates much of what follows. 
I think that we are, in fact, beginning to approach a grasp of certain 
basic principles of grammar at what may be the appropriate level of abstraction. At the same time, 
it is necessary to investigate them and determine their empirical adequacy by developing quite specific mechanisms.
We should, then, try to distinguish as clearly as we can between discussion that bears on leading ideas and
discussion that bears on the choice of specific realizations of them. \citep*[\page 2--3]{Chomsky81a}
\end{quote}
This is made explicit in a letter to \emph{Natural Language and Linguistic Theory}:
\begin{quote}
Even in mathematics, the concept of formalization in our sense was not
developed until a century ago, when it became important for advancing research
and understanding. I know of no reason to suppose that linguistics is so much
more advanced than 19th century mathematics or contemporary molecular
biology that pursuit of Pullum's injunction would be helpful, but if that can be
shown, fine. For the present, there is lively interchange and exciting progress
without any sign, to my knowledge, of problems related to the level of formality
of ongoing work. \citep[\page 146]{Chomsky90a}
\end{quote}
This departure from rigid formalization has led to there being a large number of publications inside
Mainstream Generative Grammar with sometimes incompatible assumptions to the point where it is no longer clear
how one can combine the insights of the various publications. 
An example of this is the fact that the central notion of government\is{government} has several different definitions
(see \citew{AS83a} for an overview\footnote{%
A further definition can be found in \citew{AL84a-u}. This is, however, equivalent to an earlier version as shown
by \citet[\page 104--106]{PP86a}.%
}).

This situation has been cricitized repeatedly since the 80s and sometimes very harshly by proponents of GPSG 
(\citealp*[\page 6]{GKPS85a};
\citealp{Pullum85a,Pullum89b}; \citealp[\page 48]{Pullum91b}; \citealp{KP90a}). 

The lack of precision and working out of the details\footnote{
	See \eg, \citew[\page 550]{Kuhns86a}, \citew[\page 508]{CL92a}, \citew[\page262]{KT91a}, \citew[\page
  3]{Kolb97a} and \citew[\page 580]{Freidin97a-u}, \citew[\page 25, 47]{Veenstra98a}, \citew[\page 888]{LLJ2000b} and
  \citew[\page 397, 399, 400]{Stabler2010a} for the latter. 
} and the frequent modification of basic assumptions\footnote{
	See \eg, \citew[\page 4]{Kolb97a}, \citew{Fanselow2009a} and the quote from Stabler on page~\pageref{Zitat-Stabler}.
} has led to insights gained by Mainstream Generative Grammar rarely being translated into computer implementations.
There are some implementations that are based on Transformational Grammar/GB/MP models or borrow ideas from Mainstream
Generative Grammar
\citep*{Petrick65a-u,ZFHW65a,Kay67a,Friedman69a,FBDPM71a-u,Morin73a-u,Marcus80a-u,AC86a,Kuhns86a,Correra87a,Stabler87a,Stabler92a-u,Stabler2001a,KT91a,Fong91a-u,CL92a,Lohnstein93a-u,FC94a,Nordgard94a,Veenstra98a,%
FG2012a% 
},\footnote{
  See \citew{FC94a} for a combination of a GB approach with statistical methods.
}
but these implementations often do not use transformations or differ greatly from the theoretical assumptions of the
publications. For example, \citet[\page 102--104]{Marcus80a-u} and
\citet[\page 5]{Stabler87a} use special purpose rules for auxiliary movement\is{auxiliary movement}.\footnote{
  \citet{NF86a-u,NF87a-u} has shown that Marcus' parser can only parse context"=free languages. Since natural languages
  are of a greater complexity (see Chapter~\ref{sec-generative-capacity}) and grammars of corresponding complexity
  are allowed by current versions of Transformational Grammar, Marcus' parser can be neither an adequate implementation of
  the Chomskyan theory in question nor a piece of software for analyzing natural language in general.
}
These rules reverse the order of \emph{John} and \emph{has} for the analysis of sentences such as (\mex{1}a) so that we get the order in
  (\mex{1}b), which is then parsed with the rules for non"=inverted structures.
\eal
\ex Has John scheduled the meeting for Wednesday?
\ex John has scheduled the meeting for Wednesday?
\zl
These rules for auxiliary movement are very specific and explicitly reference the category of the auxiliary. This does not correspond
to the analyses proposed in GB in any way. As we have seen in Section~\ref{Abschnitt-GB-CP-IP-System-Englisch}, there are no special
transformational rules for auxiliary movement. Auxiliary movement is carried out by the more general transformation \movealpha and
the associated restrictive principles. It is not unproblematic that the explicit formulation of the rule refers to the category \emph{auxiliary}
as is clear when one views Stabler's GB"=inspired phrase structure grammar: 
\eal
\ex\label{Regel-Aux-inv-Stabler} s $\to$ switch(aux\_verb,np), vp.
\ex s([First$|$L0],L,X0,X) :- \begin{tabular}[t]{@{}l@{}}
                              aux\_verb(First),\\
                              np(L0,L1,X0,X1),\\
                              vp([First$|$L1],L,X1,X).\\
                              \end{tabular}
\zl
%
The rule in (\mex{0}a) is translated into the Prolog predicate in (\mex{0}b). The expression [First$|$L0] after the s corresponds to the string, which
is to be processed. The `$|$'"=operator divides the list into a beginning and a rest. \emph{First} is the first word to be processed
and L0 contains all other words. In the analysis of (\mex{-1}a), First is \emph{has} and L0 is \emph{John scheduled the meeting for Wednesday}.
In the Prolog clause, it is then checked whether First is an auxiliary (aux\_verb(First)) and if
this is the case, then it will be tried to prove that the list L0 begins with a noun phrase. Since \emph{John} is an NP, this is successful. L1 is the sublist of L0 which remains after the analysis of L0, that is
\emph{scheduled the meeting for Wednesday}. This list is then combined with the auxiliary (First) and now it will be checked whether the resulting
list \emph{has scheduled the meeting for Wednesday} begins with a VP. This is the case and the remaining list L is empty. As a result, the
sentence has been successfully processed.

The problem with this analysis is that exactly one word is checked in the lexicon. Sentences such as (\mex{1}) can not be analyzed:\footnote{
  For a discussion that shows that the coordination of lexical elements has to be an option in linguistic theories, see \citew{Abeille2006a}.
} 
\ea
Could or should we pool our capital with that of other co-ops to address the needs of a regional
``neighborhood''?\footnote{
  \url{http://www.cooperativegrocer.coop/articles/index.php?id=595}. 28.03.2010.
}
\z
In this kind of sentence, two modal verbs have been coordinated\is{coordination}. They then form an \xzero and -- following GB analyses -- can be
moved together. If one wanted to treat these cases as Stabler does for the simplest case, then we would need to divide the list of words to be
processed into two unlimited sub"=lists and check whether the first list contains an auxiliary or several coordinated auxiliaries. We would
require a recursive predicate aux\_verbs which somehow checks whether the sequence \emph{could or should} is a well"=formed sequence of 
auxiliaries. This should not be done by a special predicate but rather by syntactic rules responsible for the coordination of auxiliaries.
The alternative to a rule such as (\mex{-1}a) would be the one in (\mex{1}), which is the one that
is used in theories like GPSG \citep[\page 62]{GKPS85a}, LFG \citep[\page 491]{Falk84a-u}, some HPSG analyses
(\citealp[\page 36]{GSag2000a-u}), and Construction Grammar
\citep{Fillmore99a}:%% \footnote{
%%   \citealp[\page 42]{ps2} Pollard \& Sag suggest a general schema that combines a head with all its arguments in one
%%   go. This general schema also combines v(aux+), np, and vp directly and hence accounts for
%%   auxiliary inversion without any movement transformation.
%% }
\ea
s $\to$ v(aux+), np, vp.
\z
This rule would have no problems with coordination data like (\mex{-1}) as coordination of multiple auxiliaries would produce an object with the
category v(aux+) (for more on coordination see Section~\ref{Abschnitt-Koordination}). If inversion
makes it necessary to stipulate a special rule like (\ref{Regel-Aux-inv-Stabler}), then it is not clear why one could not simply use the transformation"=less rule in (\mex{0}).

In the MITRE system \citep{ZFHW65a}, there was a special grammar for the surface structure, from which the deep structure was derived via
reverse application of transformations, that is, instead of using one grammar to create deep structures which are then transformed into
other structures, one required two grammars. The deep structures that were determined by the parser
were used as input to a transformational component since this was the only way to ensure that
the surface structures can actually be derived from the base structure \citep[\page 10]{Kay2011a}.

There are other implementations discussed in this chapter that differ from transformation"=based analyses. For example, \citet[\page 265, Section~4]{KT91a}
arrive at the conclusion that a declarative, constraint"=based approach to GB is more appropriate than a derivational one. \citet{Johnson89a}
suggests a \emph{Parsing as Deduction} approach\is{Parsing as Deduction} which reformulates sub"=theories of GB (\xbart,
Theta"=Theory\is{theta-Theory@$\theta$-Theory}, Case Theory, \ldots) as logical expressions.\footnote{
	See \citew[\page 511]{CL92a} and \citew[\page 38]{FC94a} for another constraint"=based Parsing-as-Deduction approach.
}
These can be used independently of each other in a logical proof. In Johnson's analysis, \gbt is understood as a constraint"=based system. 
More general restrictions are extracted from the restrictions on S"= and D"=structure which can then be used directly for parsing. This means that 
transformations are not directly carried out by the parser. As noted by Johnson, the language fragment he models is very small. It contains no
description of \emph{wh}"=movement, for example (p.\,114). 

Probably the most detailed implementation of the GB/Barriers tradition is Stabler's Prolog implementation \citeyearpar{Stabler92a-u}.
Stabler's achievement is certainly impressive, but his book confirms what has been claimed thus far: Stabler has to simply stipulate many
things which are not explicitly mentioned in \emph{Barriers} (\eg, using feature"=value pairs when
formalizing \xbar theory, a practice that was borrowed from GPSG\indexgpsg) and some assumptions
cannot be properly formalized and are simply ignored (see \citealp{Briscoe97a} for details). 


GB analyses\label{Seite-Representationelle-GB} which fulfill certain requirements can be reformulated so that they no longer make use of transformations.
These transformation"=less approaches are also called \emph{representational}\is{representational model}, whereas the transformation"=based approaches are referred to as
\emph{derivational}\is{derivation}. For representational analyses, there are only surface structures augmented by traces but none of these structures is connected
to an underlying structure by means of transformations (see %\citew{McCawley68a}; Pullum2007a:3 sagt, dass das nicht MTS ist
Koster \citeyear[\page ]{Koster78b-u}; \citeyear[\page 235]{Koster87a-u}; 
%\citealp[\page 66, Fußnote~4]{Bierwisch83a}; 
\citealp{KT91a}; \citealp[Section~1.4]{Haider93a}; 
\citealp[\page 14]{Frey93a}; \citealp[\page 87--88, 177--178]{Lohnstein93a-u}; \citealp[\page 38]{FC94a}; \citealp[\page 58]{Veenstra98a}, for example).
These analyses can be implemented in the same way as corresponding HPSG analyses\indexhpsg (see
Chapter~\ref{Kapitel-HPSG}) as computer-processable fragments and this has in fact been carried out
for example for the analysis of verb position in German.\footnote{%
	This shows that ten Hacken's contrasting of HPSG with GB and LFG \citep[Section~4.3]{TenHacken2007a}
        and the classification of these frameworks as belonging to different research paradigms is
        completely mistaken. In his classification, ten Hacken refers mainly to the model"=theoretic
        approach that HPSG assumes. However, LFG also has a model"=theoretic formalization
        \citep{Kaplan95a}. Furthermore, there is also a model"=theoretic variant of GB
	\citep{Rogers98a-u}. For further discussion, see Chapter~\ref{Abschnitt-Generativ-Modelltheoretisch}. 
}
However, such implemented analyses differ from GB analyses with regard to their basic architecture and in small, but important details such as how one deals with
the interaction of long"=distance dependencies and coordination \citep{Gazdar81}. For a critical discussion and classification of movement analyses
in Transformational Grammar, see \citew{Borsley2012a}. 

Following this somewhat critical overview, I want to add a comment in order to avoid being misunderstood:
I do not demand that all linguistic work shall be completely formalized. There is simply
no space for this in a, say, thirty page essay. Furthermore, I do not believe that all linguists
should carry out formal work and implement their analyses as computational models. However, there
has to be \emph{somebody} who works out the formal details and these basic theoretical assumptions
should be accepted and adopted for a sufficient amount of time by the research community in
question.



\section*{Comprehension questions}

\begin{enumerate}
\item Give some examples of functional and lexical categories.
\item How can one represent lexical categories with binary features and what advantages does this have?
\end{enumerate}

\section*{Exercises}


\begin{enumerate}
\item Draw syntactic trees for the following examples:
\eal
\ex 
\gll dass die Frau den Mann liebt\\
     that the.\nom{} woman the.\acc{} man loves\\
\glt `that the woman loves the man'
\ex 
\gll dass der Mann geliebt wird\\
     that the.\nom{} man loved is\\
\glt `that the man is loved'
\ex 
\gll Der Mann wird geliebt.\\
     the.\nom{} man is loved\\
\glt `The man is loved.'
\ex 
\gll dass der Mann der Frau hilft\\
     that the.\nom{} man the.\dat{} woman helps\\
\glt `that the man helps the woman'
\ex 
\gll Der Mann hilft der Frau.\\
     the man.\nom{} helps the.\dat{} woman\\
\glt `The man is helping the woman.'
\zl
For the passive sentences, use the analysis where the subject noun phrase is moved from the object position, that is, the analysis
without an empty expletive as the subject.
\end{enumerate}



\section*{Further reading}

For Sections~\ref{Abschnitt-GB-allgemein}--\ref{sec-GB-lokale-Umstellung}, I used material from Peter Gallmann\aimention{Peter Gallmann}
from 2003 \citep{Gallmann2003a}. This has been modified, however, at various points. I am solely responsible for any mistakes or inadequacies.
For current materials by Peter Gallmann, see \url{http://www.syntax-theorie.de}. 

In the book \emph{Syntaktische Analyseperspektiven}, \citet{Lohnstein2014a} presents a variant of GB
which more or less corresponds to what is discussed
in this chapter (CP/IP, movement"=based analysis of the passive). The chapters in said book have been written by proponents of various theories
and all analyze the same newspaper article. This book is extremely interesting for all those who wish to compare the various theories out there.

\citew{Haegeman90a-u} is a comprehensive introduction to GB. Those who do read German may consider
the textbooks by \citew{FF87}, \citew{SS88a} and \citew{Grewendorf88a} since they are also
addressing the phenomena that are covered in this book.

In many of his publications, Chomsky discusses alternative, transformation"=less approaches as ``notational variants''.
This is not appropriate, as analyses without transformations can make different predictions to transformation"=based approaches (\eg, w.\,r.\,t.\ coordination
and extraction. See Section~\ref{Abschnitt-Einordnung-GPSG} for a discussion of GPSG in this respect). In \citew{Gazdar81b}, one can find a comparison
of GB and GPSG\indexgpsg as well as a discussion of the classification of GPSG as a notational variant of Transformational Grammar with contributions
from Noam Chomsky, Gerald Gazdar and Henry Thompson\aimention{Henry S. Thompson}.


\citet{Borsley99a-u} and \citet{KS2008a-u} have parallel textbooks for GB and HPSG\indexhpsg in English. For the comparison of Transformational Grammar
and LFG, see \citew{BK82a}. \citew{Kuhn2007a} offers a comparison of modern derivational analyses with constraint"=based LFG\indexlfg and HPSG approaches.
\citet{Borsley2012a} contrasts analyses of long"=distance dependencies in HPSG with movement"=based analyses as in GB/Minimalism. Borsley discusses
four types of data which are problematic for movement"=based approaches: extraction without fillers,
extraction with multiple gaps (see also the discussion of (\ref{ex-atb-minimalism}) on
p.\,\pageref{ex-atb-minimalism} and of (\ref{ex-atb-gazdar}) on p.\,\pageref{ex-atb-gazdar} of this
book), extractions where fillers and gaps do not match and extraction without gaps.





%%% Local Variables: 
%%% mode: latex
%%% TeX-master: "grammatiktheorie-2"
%%% End: 



%      <!-- Local IspellDict: en_US-w_accents -->


% 29:36

% 1:04:00 Sport-Teams und Autoritäten

% 1:30:00

% 2:19:23

% 2:31:56


