%% -*- coding:utf-8 -*-
\exewidth{(235)}


\chapter{Phrasal vs.\ lexical analyses}
\label{Abschnitt-Phrasal-Lexikalisch}

\chaptersubtitle{coauthored with Stephen Wechsler}

This section deals with a rather crucial aspect when it comes to the comparison of the theories
described in this book: valence and the question whether sentence structure, or rather syntactic
structure in general, is determined by lexical information or whether syntactic structures have an
independent existence (and meaning) and lexical items are just inserted into them. Roughly speaking,
frameworks like GB/Minimalism, LFG, CG, HPSG, and DG are lexical, while GPSG and Construction
Grammar (\citealp{Goldberg95a,Goldberg2003b,Tomasello2003a,Tomasello2006c,Croft2001a}) are
phrasal approaches. This categorization reflects tendencies, but there are non-lexical 
approaches in Minimalism (Borer's exoskeletal approach, \citeyear{Borer2003a-u}) and LFG
(\citealp{Alsina96a}; \citealp{ADT2008a,ADT2013a}) and there are lexical approaches in Construction
Grammar (Sign"=Based Construction Grammar, see Section~\ref{sec-SBCG}). The phrasal approach is
wide"=spread also in frameworks like Cognitive Grammar (\citealp{Dabrowska2001a}; \citealp[\page 169]{Langacker2009a})
 and Simpler Syntax \citep{CJ2005a,Jackendoff2008a} that could not be discussed in this book.

The question is whether the meaning of an utterance like (\mex{1}a) is contributed by the verb
\emph{give} and the structure is needed for the NPs around the verb does not contribute any meaning
or whether there is a phrasal pattern X Verb Y Z that contributes some `ditransitive meaning'
whatever this may be.\footnote{
Note that the prototypical meaning is a transfer of possession in which Y receives Z from X, but the
reverse holds in (i.b):
\eal
\ex 
\gll Er gibt ihr den Ball.\\
     he.\nom{} gives her.\dat{} the.\acc{} ball\\
\ex
\gll Er stiehlt ihr den Ball.\\
     he.\nom{} steals  her.\dat{} the.\acc{} ball\\
\glt `He steals the ball from her.'
\zllast
}
\eal
\ex Peter gives Mary the book.
\ex Peter fishes the pond empty.
\zl
Similarly, there is the question of how the constituents in (\mex{0}b) are licensed. This sentence is
interesting since it has a resultative meaning that is not part of the meaning of the verb
\emph{fish}: Peter's fishing causes the pond to become empty. Nor is this additional meaning part of
the meaning of any other item in the sentence. On the lexical account,
there is a lexical rule that licenses a lexical item that selects for \emph{Peter}, \emph{the pond},
and \emph{empty}. This lexical item also contributes the resultative meaning. On the phrasal
approach, it is assumed that there is a pattern Subj V Obj Obl. This pattern contributes the
resultative meaning, while the verb that is inserted into this pattern just contributes its
prototypical meaning, \eg the meaning that \emph{fish} would have in an
intransitive construction. I call such phrasal approaches \emph{plugging approaches}, since lexical
items are plugged into ready"=made structures that do most of the work.

In what follows I will examine these proposals in more detail and argue that the lexical approaches
to valence are the correct ones. The discussion will be based on earlier work of mine
\citep{Mueller2006d,Mueller2007d,MuellerPersian} and work that I did together with Steve Wechsler
\citep{MWArgSt,MWArgStReply}. Some of the sections in \citet{MWArgSt} started out as translations of
\citet{MuellerGTBuch2}, but the material was reorganized and refocused due to intensive discussion
with Steve Wechsler. So rather than using a translation of Section~11.11 of \citew{MuellerGTBuch2},
I use parts of \citew{MWArgSt} here and add some subsections that had to be left out of the article
due to space restrictions (Subsections~\ref{Abschnitt-Diskussion-Haugereid} and~\ref{sec-neuro-linguistics}).
Because there have been misunderstandings in the past (\eg \citew{Boas2014a}, see \citew{MWArgStReply}), a disclaimer is necessary
here. This section is not an argument against Construction Grammar. As was mentioned above
Sign"=Based Construction Grammar is a lexical variant of Construction Grammar and hence compatible
with what I believe to be correct. This section is also not against phrasal constructions in
general, since there are phenomena that seem to be best captured with phrasal constructions. These are
discussed in detail in Subsection~\ref{Abschnitt-Phrasale-Konstruktionen}. What I will argue against in
the following subsections is a special kind of phrasal construction, namely phrasal argument
structure constructions. I believe that all phenomena that have to do with valence and valence
alternations should be treated lexically.





\section{Some putative advantages of phrasal models}
\label{Abschnitt-Stoepselei}
%
%The previous section reviewed earlier
%arguments for needing lexical representations of valence.  In this section we present more detailed
%arguments specifically directed against the claim that lexical valence representations
%(i.\,e.\ predicate argument structures) can or should be replaced by what we call a plugging proposal,
%that is, a system in which a verb or other predicator is plugged into a meaningful construction.

%As noted in Section~\ref{PAS-sec} above, we believe that grammars include meaningful phrasal
%constructions.  Our purpose is not to argue against their existence, but rather to argue that they
%cannot replace lexical valence representations.  So the existence of such phrasal constructions does not bear
%on the issue at stake here.  
In this section we examine certain claims to purported advantages of phrasal versions of Construction Grammar over lexical rules.  
Then in the following section, we will turn to positive arguments for lexical rules. 

\subsection{Usage-based theories}\label{usage-based-sec}

For many practitioners of Construction Grammar, their approach to syntax is deeply rooted in the
ontological strictures of \emph{usage-based} theories of language \citep{Langacker87a-u, Goldberg95a,
Croft2001a, Tomasello2003a}.  Usage-based theorists oppose the notion of `linguistic rules conceived
of as algebraic procedures for combining symbols that do not themselves contribute to meaning'
\citep[\page 99]{Tomasello2003a}. All linguistic entities are symbolic of things in the realm of denotations;
`all have communicative significance because they all derive directly from language use' (\emph{ibid}). Although the formatives of language may be rather abstract, they can never be divorced
from their functional origin as a tool of communication.  The usage-based view of constructions is
summed up well in the following quote:
\begin{quote}
The most important point is that constructions are nothing more or less than patterns of usage,
which may therefore become relatively abstract if these patterns include many different kinds of
specific linguistic symbols.  But never are they empty rules devoid of semantic content or
communicative function. \citep[\page 100]{Tomasello2003a}
\end{quote}

\noindent 
Thus constructions are said to differ from grammatical rules in two ways: they must carry meaning;
and they reflect the actual `patterns of usage' fairly directly.

Consider first the constraint that every element of the grammar must carry meaning, which we call
the \emph{semiotic dictum}.  Do lexical or phrasal theories hew the most closely to this dictum?
Categorial Grammar, the paradigm of a lexical theory (see Chapter~\ref{chap-CG}), is a strong
contender: it consists of meaningful words, with only a few very general combinatorial rules such as
X/Y $*$ Y = X.  Given the rule-to-rule assumption, those combinatorial rules specify the meaning of the
whole as a function of the parts.  Whether such a rule counts as meaningful in itself in Tomasello's
sense is not clear.

What does seem clear is that the combinatorial rules of Construction Grammar, such as Goldberg's
Correspondence Principle for combining a verb with a construction \citeyearpar[\page 50]{Goldberg95a},
have the same status as those combinatorial rules:

%\begin{quote}
\ea
The Correspondence Principle:  each participant that is lexically profiled and expressed must be
fused with a profiled argument role of the construction.  If a verb has three profiled participant
roles, then one of them may be fused with a non-profiled argument role of a construction. \citep[\page 50]{Goldberg95a}\label{tcp}
\z
%\end{quote}
Both verbs and constructions are specified for participant roles, some of which are \emph{profiled}.
Argument profiling for verbs is `lexically determined and highly conventionalized'
\citep[\page 46]{Goldberg95a}.  Profiled argument roles of a construction are mapped to direct
grammatical functions, i.\,e., SUBJ, OBJ, or OBJ2.  By the Correspondence Principle the lexically profiled argument
roles must be direct, unless there are three of them, in which case one may be indirect.\footnote{We
  assume that the second sentence of (\ref{tcp}) provides for exceptions to the first sentence.}
With respect to the semiotic dictum, the Correspondence Principle has the same status as the
Categorial Grammar combinatorial rules: a meaningless algebraic rule that specifies the way to
combine meaningful items.   

Turning now to the lexicalist syntax we favor, some elements abide by the semiotic dictum while
others do not.  Phrase structure rules for intransitive and transitive VPs (or the respective HPSG ID schema)
do not.  Lexical valence structures clearly carry meaning since they are associated with particular
verbs.  In an English ditransitive, the first object expresses the role of `intended recipient' of
the referent of the second object.
Hence \emph{He carved her a toy} entails that he carved a toy with the intention that she receive
it.  So the lexical rule that adds a benefactive recipient argument to a verb adds meaning.  Alternatively, a phrasal ditransitive construction might
contribute that `recipient' meaning.\footnote{In Section~\ref{coordination-sec} we argue that the
  recipient should be added in the lexical argument structure, not through a phrasal construction.
  See Wechsler (\citeyear[\page 111--113]{Wechsler91a-u}; \citeyear[\page 88--89]{Wechsler95a-u}) for an
  analysis of English ditransitives with elements of both constructional and lexical approaches.  It
  is based on Kiparsky's notion of a \emph{thematically
    restricted positional linker} (\citeyear{Kiparsky87a-u, Kiparsky88a-u}).}  Which structures have
meaning is an empirical question for us. 



In Construction Grammar, however, meaning is assumed for all constructions  \emph{a priori}.  But
while the ditransitive construction plausibly contributes meaning, no truth-conditional meaning has
yet been discovered for either the intransitive or bi"=valent transitive constructions.  Clearly the
constructionist's evidence for the meaningfulness of \emph{certain} constructions such as the
ditransitive does not constitute evidence that \emph{all} phrasal constructions have meaning.  So
the lexical and phrasal approaches seem to come out the same, as far as the semiotic dictum is
concerned.


Now consider the second usage-based dictum, that the elements of the grammar directly reflect
patterns of usage, which we call \emph{the transparency dictum}.  The Construction Grammar
literature often presents their constructions informally in ways that suggest that they represent
surface constituent order patterns: the transitive construction is `X VERB Y' (Tomasello) or `Subj V Obj'
\citep{Goldberg95a,Goldberg2006a}\footnote{
  \citet[\page 300]{GCS2004a} report about a language acquisition experiment that involves an SOV
  pattern. The SOV order is mentioned explicitly and seen as part of the construction.
}; the passive construction is `X \emph{was} VERB\emph{ed by} Y'
\citep[\page 100]{Tomasello2003a} or `Subj aux Vpp (PPby)' \citep[\page 5]{Goldberg2006a}.  But a theory
in which constructions consist of surface patterns was considered in detail and rejected by
Müller (\citeyear[Section~2]{Mueller2006d}), and does not accurately reflect Goldberg's actual
theory.\footnote{
  This applies to argument structure constructions only. In some of her papers Goldberg assumes that
  very specific phrase structural configurations are part of the constructions. For instance in her
  paper on complex predicates in Persian \citep{Goldberg2003a} she assigns \vnull and \vbar categories. See
  \citew[Section~4.9]{MuellerPersian} for a critique of that analysis.} 
The more detailed discussions present \emph{argument structure
  constructions}, which are more abstract and rather like the lexicalists' grammatical elements (or
perhaps an LFG f-structure): the transitive construction resembles a transitive valence structure
(minus the verb itself); the passive construction resembles the passive lexical rule.

With respect to fulfilling the desiderata of usage-based theorists, we do not find 
any significant difference between the
non-lexical and lexical approaches.  
  
\subsection{Coercion}
\label{coercion-sec}

Researchers working with plugging proposals usually take coercion as an indication of the usefulness of phrasal
constructions. For instance, Anatol Stefanowitsch (Lecture in the lecture series \emph{Algorithmen und Muster –-
  Strukturen in der Sprache}, 2009) discussed the example in (\mex{1}):
\ea
Das Tor zur Welt Hrnglb öffnete sich ohne Vorwarnung
und verschlang [sie] \ldots{} die Welt Hrnglb wird von Magiern
erschaffen, die Träume zu Realität formen können, aber
nicht in der Lage sind zu träumen. Haltet aus, Freunde.
Und ihr da draußen, bitte träumt ihnen ein Tor.\footnote{
\href{http://www.elbenwaldforum.de/showflat.php?Cat=&Board=Tolkiens_Werke&Number=1457418&page=3&view=collapsed&sb=5&o=&fpart=16}{\nolinkurl{http://www.elbenwaldforum.de/showflat.php?Cat=&Board=Tolkiens_Werke&}}
\href{http://www.elbenwaldforum.de/showflat.php?Cat=&Board=Tolkiens_Werke&Number=1457418&page=3&view=collapsed&sb=5&o=&fpart=16}{\nolinkurl{Number=1457418&page=3&view=collapsed&sb=5&o=&fpart=16}}. 27.02.2010.

`The gate to the world Hrnglb opened without warning and swallowed them. The world Hrnglb is created
by magicians that can form reality from dreams but cannot dream themselves. Hold out, friends! And
you out there, please, dream a gate for them.'
}
\z
The crucial part is \emph{bitte träumt ihnen ein Tor} `Dream a gate for them'. In this fantasy
context the word \emph{träumen}, which is intransitive, is forced into the ditransitive construction
and therefore gets a certain meaning. This forcing of a verb corresponds to overwriting or rather extending properties
of the verb by the phrasal construction.

%But it is possible to find other explanations for such cases. Instead of forcing an object into a
%hole in which it does not fit one could adapt the object and insert it then. The second proposal can
%be modeled by lexical rules, that is, 

In cases in which the plugging proposals assume that
information is over-written or extended, lexical approaches assume mediating lexical
rules. \citet[Section~4]{BC99a} have worked out a lexical approach in detail.\footnote{
\citet{Kay2005a}, working in the framework of CxG, also suggests unary constructions.}
They discuss the ditransitive sentences in (\mex{1}), which either correspond to the prototypical
ditransitive construction (\mex{1}a) or deviate from it in various ways.
\eal
\ex Mary gave Joe a present.
\ex\label{paint} Joe painted Sally a picture.
\ex Mary promised Joe a new car.
\ex He tipped Bill two pounds.
\ex The medicine brought him relief.
\ex The music lent the party a festive air.
\ex Jo gave Bob a punch.
\ex He blew his wife a kiss.
\ex\label{ex-smiled-herself-an-upgrade} She smiled herself an upgrade.
\zl
For the non-canonical examples they assume lexical rules that relate transitive (\emph{paint}) and intransitive (\emph{smile}) 
verbs to ditransitive ones and contribute the respective semantic information or the respective
metaphorical extension. The example in (\ref{ex-smiled-herself-an-upgrade}) is rather similar to the
\emph{träumen} example discussed above and is also analyzed with a lexical rule (page~509). Briscoe
and Copestake note that this lexical rule is much more restricted in its productivity than the other lexical
rules they suggest. They take this as motivation for developing a representational
format in which lexical items (including those that are derived by lexical rules) are
associated with probabilities, so that differences in productivity of various patterns can be captured.

Looking narrowly at such cases, it is hard to see any rational grounds for choosing between the phrasal analysis and the lexical rule.  But if we broaden our view, the lexical rule approach can be seen to have a much wider application. 
Coercion is a very general pragmatic process, occurring in many contexts where no construction seems
to be responsible  \citep{Nunberg95a-u}.  Nunberg cites many cases such as the restaurant waiter
asking \emph{Who is the ham sandwich?} \citep[\page 115]{Nunberg95a-u}.  
\citet[\page 116]{CB92a} discuss the conversion of terms for animals to mass nouns (see also \citet[\page 36--43]{CB95a-u}). Example (\mex{1}) is about a substance, not about a cute bunny.
\ea
After several lorries had run over the body, there was rabbit splattered all over the road.
\z
The authors suggest a lexical rule that maps a count noun onto a mass noun. This analysis is also
assumed by \citet[\page 114--115]{Fillmore99a}.
Such coercion can occur without any syntactic context: one can answer the question \emph{What's that
  stuff on the road?} or \emph{What are you eating?} with the one-word utterance \emph{Rabbit.}
Some coercion happens to affect the complement structure of a verb, but this is simply a special
case of a more general phenomenon that has been analyzed by rules of systematic polysemy.      

\subsection{Aspect as a clause level phenomenon}
\label{sec-aspect-at-clause-level}

\citet{Alsina96a}, working in the framework of LFG\indexlfg, argues for a phrasal analysis of resultative constructions based on the aspectual properties
of sentences, since aspect\is{aspect} is normally viewed as a property that is determined by the sentence syntax. Intransitive verbs such as \emph{bark}
refer to activities\is{activity}, a resultative construction with the same verb, however, stands for an accomplishment\is{accomplishment} (an extended
change of state\is{change of state}).
Alsina supports this with the following data:
\eal
\judgewidth{(*)}
\ex[(*)]{
The dog barked in five minutes.
}
\ex[]{
The dog barked the neighbors awake in five minutes.
}
\zl
The latter sentence means that the \emph{barking} event was completed after five minutes. A reading referring to the time span of the event
is not available for (\mex{0}a). If (\mex{0}a) is grammatical at all, then a claim is being made about the time frame in which the event begun.

If we now consider examples such as (\mex{1}c), however, we see that Alsina's argumentation is not cogent since the resultative
meaning is already  present at the word"=level in nominalizations. As the examples in (\mex{1}) show, this contrast can be observed in nominal constructions
and is therefore independent of the sentence syntax:
\eal
\judgewidth{\#}
\ex[]{
\gll weil sie die Nordsee in fünf Jahren leer fischten\\
	 because they the North.Sea in five years empty fished\\
\glt `because they fished the North Sea (until it was) empty in five years'
}
\ex[\#]{
\gll weil sie in fünf Jahren fischten\\
	 because they in five years fished\\
}
\ex[]{
\gll das Leerfischen der Nordsee in fünf Jahren\\
	 the empty.fishing of.the North.Sea in five years\\
}
\ex[\#]{
\gll das Fischen in fünf Jahren\\
	 the fishing in five years\\
}
\zl
%
In a lexical approach there is a verb stem selecting for two NPs and a resultative predicate. This
stem has the appropriate meaning and can be inflected or undergo derivation und successive
inflection. In both cases we get words that contain the resultative semantics and hence are
compatible with respective adverbials. 





\subsection{Simplicity and polysemy}\label{polysemy-subsec}

Much of the intuitive appeal of the plugging approach stems from its apparent simplicity relative to
the use of lexical rules.  But the claim to greater simplicity for Construction Grammar is based on
misunderstandings of both lexical rules and Construction Grammar (specifically of Goldberg's \citeyearpar{Goldberg95a,Goldberg2006a} version).   It draws the distinction in the wrong place and misses the real differences
between these approaches.  This argument from simplicity is often repeated and so it is important to
understand why it is incorrect.    
%\NOTE{Is this too combative?  maybe need to tone it down}

\citet{Tomasello2003a} presents the argument as follows.  Discussing first the lexical rules approach, \citet[\page 160]{Tomasello2003a} writes that 

\begin{quote}
One implication of this view is that a verb must have listed in the lexicon a different meaning for
virtually every different construction in which it participates [\ldots].  For example, while the
prototypical meaning of \emph{cough} involves only one participant, the cougher, we may say such
things as \emph{He coughed her his cold}, in which there are three core participants.  In the
lexical rules approach, in order to produce this utterance the child's lexicon must have as an entry
a ditransitive meaning for the verb \emph{cough}. \citep[\page 160]{Tomasello2003a}
\end{quote}
\citet[\page 160]{Tomasello2003a} then contrasts a Construction Grammar approach, citing \citet{FKoC88a}, \citet{Goldberg95a}, and \citet{Croft2001a}.  He concludes as follows:

\begin{quote}
The main point is that if we grant that constructions may have meaning of their own, in relative
independence of the lexical items involved, then we do not need to populate the lexicon with all
kinds of implausible meanings for each of the verbs we use in everyday life.  The construction
grammar approach in which constructions have meanings is therefore both much simpler and much more
plausible than the lexical rules approach.  \citep[\page 161]{Tomasello2003a}
\end{quote}

\noindent
This reflects a misunderstanding of lexical rules, as they are normally understood.  There is no implausible sense populating the lexicon.
The lexical rule approach to \emph{He coughed her his cold} states that when the word \emph{coughed} appears with
two objects, the whole complex has a certain meaning (see \citealp[\page 876]{Mueller2006d}). Furthermore we explicitly distinguish between listed elements
(lexical entries) and derived ones. The general term subsuming both is \emph{lexical item}.

%Adopting lexical rules does not mean that we `populate the
%lexicon with all kinds of implausible meanings'; quite the contrary.  
%that could not otherwise be expressed by means of forms listed in the
%lexicon.  Tomasello seems to be confusing simplicity of the grammar with simplicity of the language
%licensed by the grammar.\NOTE{St.Mü.: I do not understand this.}  (By the same flawed reasoning one could complain that every grammar is
%populated with infinitely many sentences with all kinds of implausible meanings.)   

The simplicity argument also relies on a misunderstanding of a theory Tomasello advocates, namely the
theory due to \citet{Goldberg95a, Goldberg2006a}.  For his argument to go through, Tomasello must tacitly assume
that verbs can combine freely with constructions, that is, that the grammar does not place extrinsic
constraints on such combinations.  If it is necessary to also stipulate which verbs can appear in
which constructions, then the claim to greater simplicity collapses: each variant lexical item with
its ``implausible meaning'' under the lexical rule approach corresponds to a verb-plus-construction
combination under the phrasal approach. 

Passages such as the following may suggest that verbs and constructions are assumed to combine
freely:\footnote{The context of these quotes makes clear that the verb and the argument structure construction are considered 
constructions.  See \citet[\page 21, ex.~(2)]{Goldberg2006a}.} 

%\begin{quote}
%Constructions are combined freely to form actual expressions as long
%as they are not in conflict.  Unresolved conflicts result in judgments
%of ill-formedness.  (Goldberg 2006, p.\,10)
%\end{quote}

\begin{quote}
Constructions are combined freely to form actual expressions as long
as they can be construed as not being in conflict (invoking the notion
of construal is intended to allow for processes of accommodation or
coercion).  \citep[\page 22]{Goldberg2006a} 
\end{quote}

\begin{quote}
Allowing constructions to combine freely as long as there are no
conflicts, allows for the infinitely creative potential of language.
[\ldots] That is, a speaker is free to creatively combine constructions as
long as constructions exist in the language that can be combined
suitably to categorize the target message, given that there is no
conflict among the constructions.  \citep[\page 22]{Goldberg2006a} 
\end{quote}

\noindent
But in fact Goldberg does not assume free combination, but rather that a verb is ``conventionally
associated with a construction'' \citep[\page 50]{Goldberg95a}: verbs specify their participant roles and which
of those are obligatory direct arguments (\emph{profiled}, in Goldberg's terminology).  In fact, Goldberg herself \citeyearpar[\page 211]{Goldberg2006a}
argues against Borer's putative assumption of free combination \citeyearpar{Borer2003a-u} on the grounds that Borer is
unable to account for the difference between \emph{dine} (intransitive), \emph{eat} (optionally
transitive), and \emph{devour} (obligatorily transitive).\footnote{Goldberg's critique cites a 2001
  presentation by Borer with the same title as \citew{Borer2003a-u}.  See
  Section~\ref{sec-idiosyncratic-case-and-PP} for more discussion of this issue.  As far as
we know, the \emph{dine / eat / devour} minimal triplet originally came from \citet[\page 89--90]{Dowty89b-u}. }
Despite Tomasello's comment above,
Construction Grammar is no simpler than the lexical rules.   

The resultative construction is often used to illustrate the simplicity argument.  For example,  
\citet[Chapter~7]{Goldberg95a} assumes that the same lexical item for the verb \emph{sneeze}
is used in (\mex{1}a) and (\mex{1}b). It is simply inserted into different constructions:
\eal
\ex He sneezed.
\ex He sneezed the napkin off the table.
\zl
The meaning of (\mex{0}a) corresponds more or less to the verb meaning, since the verb is used in
the Intransitive Construction. But the Caused-Motion Construction in (\mex{0}b) contributes
additional semantic information concerning the causation and movement: his sneezing caused the
napkin to move off the table.  \emph{sneeze} is plugged into the Caused Motion Construction, which
licenses the subject of \emph{sneeze} and additionally provides two slots: one for the theme
(\emph{napkin}) and one for the goal (\emph{off the table}).  The lexical approach is essentially parallel,
except that the lexical rule can feed further lexical processes like passivization (\emph{The napkin
  was sneezed off the table}), and conversion to nouns or adjectives (see Sections
\ref{sec-val-morph} and \ref{sec-acquisition}).   

In a nuanced comparison of the two approaches, \citet[\page 139--140]{Goldberg95a}
considers again the added recipient argument in \emph{Mary kicked Joe the ball}, where \emph{kick}
is lexically a 2-place verb.  She notes that on the constructional view, ``the composite fused
structure involving both verb and construction is stored in memory''.  
The verb itself retains its original meaning as a 2-place verb, so that ``we
avoid implausible verb senses such as `to cause to receive by kicking'.''  The idea seems to be that
the lexical approach, in contrast, must countenance such implausible verb senses since a lexical
rule adds a third argument.  

But the lexical and constructional approaches are actually indistinguishable on this point.  The lexical rule does not
produce a verb with the ``implausible sense'' in (\mex{1}a).  Instead it produces the sense in (\mex{1}b):
\eal
\ex cause-to-receive-by-kicking(x, y, z) 
\ex cause(kick(x, y),receive(z,y))
\zl
The same sort of ``composite fused structure'' is assumed under either view.  
With respect to the semantic structure, the number and plausibility of senses, and the polyadicity of the semantic relations, the two
theories are identical.  They mainly differ in the way this representation fits into the larger theory of syntax.  
They also differ in another respect: on the lexical view, the derived three"=argument valence
structure is associated with the phonological string
\emph{kicked}.  Next, we present evidence for this claim.
 

\section{Evidence for lexical approaches}

\subsection{Valence and coordination}
\label{coordination-sec}

On the lexical account, the verb \emph{paint} in (\ref{paint}), for example, is lexically a
2"=argument verb, while the unary branching node immediately dominating it is effectively a
3"=argument verb.  On the constructional view there is no such   predicate seeking three arguments
that dominates only the verb.  Coordination provides evidence for the lexical account.   

A generalization about coordination is that two constituents which have compatible syntactic
properties can be coordinated and that the result of the coordination is an object that has the
syntactic properties of each of the conjuncts. This is reflected by the
Categorial Grammar analysis which assumes the category (X\bs X)/X for the conjunction: the
conjunction takes an X to the right, an X to the left and the result is an X.

For example, in (\mex{1}a) we have a case of the coordination of two lexical
verbs. The coordination \emph{know and like} behaves like the coordinated simplex verbs: it takes a
subject and an object. Similarly, two sentences with a missing object are coordinated in (\mex{1}b)
and the result is a sentence with a missing object. 
\eal
\ex[]{
I know and like this record.
}
\ex[]{
Bagels, I like and Ellison hates.
}
\zl
The German examples in (\mex{1}) show that the case requirement of the involved verbs has to be
respected. In (\mex{1}b,c) the coordinated verbs require accusative and dative respectively and since
the case requirements are incompatible with unambiguously case marked nouns both of these examples are out.
\eal
\ex[]{
\gll Ich kenne und unterstütze diesen Mann.\\
     I know and support this man.\acc\\
}
\ex[*]{
\gll Ich kenne und helfe diesen Mann.\\
     I know and help this man.\acc\\
}
\ex[*]{
\gll Ich kenne und helfe diesem Mann.\\
     I know and help this man.\dat\\
}
\zl

\noindent
Interestingly, it is possible to coordinate basic ditransitive verbs with verbs that have
additional arguments licensed by the lexical rule. (\mex{1}) provides examples in English and German
((\mex{1}b) is quoted from \citew[\page 420]{MuellerGTBuch2}):
%\NOTE{TL: \emph{offer} as transitive verb}

\eal
\label{promise-make}
\ex She then offered and made me a wonderful espresso -- nice.\footnote{
\url{http://www.thespinroom.com.au/?p=102} 07.07.2012}
\ex 
\label{ex-gebacken-und-gegeben}
\gll ich hab ihr jetzt diese Ladung Muffins mit den Herzchen drauf gebacken und gegeben.\footnotemark\\
     I have her now this load Muffins with the little.heart there.on~~~~ baked and given\\
\footnotetext{
\url{http://www.musiker-board.de/diverses-ot/35977-die-liebe-637-print.html}. 08.06.2012
}
\glt `I have now baked and given her this load of muffins with the little heart on top.'
\zl
\noindent
These sentences show that both verbs are 3"=argument verbs at the $V^0$ level, since they involve $V^0$ coordination: 
\ea
{}[\sub{\vnull} offered and made] [\sub{NP} me]    [\sub{NP} a wonderful espresso] 
\z

\noindent
This is expected under the lexical rule analysis but not the non-lexical constructional one.\footnote{
One might wonder whether these sentences could be instances of Right Node Raising (RNR) out of coordinated VPs \citep{Bresnan74a-u, Abbott76a-u}:  
\ea \label{rnr}
She $[$ offered  \_\_\_  $]$ and $[$ made me \_\_\_ $]$  a wonderful espresso. 
\z
But this cannot be right.  
Under such an analysis the first verb has been used without a
benefactive or recipient object.  But \emph{me} is interpreted as the recipient of both the offering and making.
Secondly, the second object can be an unstressed pronoun (\emph{She offered and made me it}), which is not possible in RNR.  Note that \emph{offered and made} cannot be a pseudo-coordination meaning `offered to make'.  This is possible only with stem forms of certain verbs such as \emph{try}.}  
%Also the verb \emph{offer} without the recipient (\emph{?She offered
%  a special sauce}) is somewhat more awkward than the sentences in (\ref{promise-make}).

Summarizing the coordination argument:  coordinated verbs generally must have compatible syntactic properties like valence properties.  This means that in (\ref{promise-make}b), for example,
\emph{gebacken} `baked' and \emph{gegeben} `given' have the same valence properties. 
On the lexical approach the creation verb
\emph{gebacken}, together with a lexical rule, licenses a ditransitive verb.  It can therefore be coordinated with \emph{gegeben}. On the phrasal
approach however, the verb \emph{gebacken} has two argument roles and is not compatible with the verb
\emph{gegeben}, which has three argument roles. In the phrasal model, \emph{gebacken} can only realize three arguments when it
enters the ditransitive phrasal construction or argument structure construction.  But in sentences like (\ref{promise-make}) it is not
\emph{gebacken} alone that enters the phrasal syntax, but rather the combination of \emph{gebacken} and
\emph{gegeben}. On this view, the verbs are incompatible as far as the semantic roles are concerned. 

To fix this under the phrasal approach, one could posit a mechanism such that the semantic roles that are required for the coordinate phrase \emph{baked and
  given} %percolate down to 
  are shared by each of its conjunct verbs and that they are therefore compatible.  But this would
  amount to saying that there are several verb senses for \emph{baked}, something that the
  anti-lexicalists claim to avoid, as discussed in the next section.

%A reviewer of Theoretical Linguistics suggested an approach in which lexical items are
%underspecified with regard to their valence structure.  The valence information is added by the
%phrasal construction and the coordination construction has to make sure that the valence information
%on the conjuncts matches. This is an interesting suggestion but it requires the introduction of
%valence representations into the phrasal approach that are not needed for other reasons than the analysis of
%coordination. This seems to be an unwanted consequence of the phrasal analysis.

A reviewer of Theoretical Linguistics correctly observes that a version of the ASC approach could work in the exactly same way as our lexical analysis.  
Our ditransitive lexical rule would simply be rechristened as a `ditransitive ASC'.  This construction would combine with \emph{baked}, thus adding the
third argument, prior to its coordination with \emph{gave}.  As long as the ASC approach is a non-distinct notational
variant of the lexical rule approach then of course it works in exactly the same way.  But the literature on the ASC approach represents
it as a radical alternative to lexical rules, in which constructions are combined through inheritance hierarchies, instead of allowing lexical rules 
to alter the argument structure of a verb prior to its syntactic combination with the other words and phrases.  

The reviewer also remarked that examples like (\mex{1}) show that the benefactive argument has to be
introduced on the phrasal level.
\ea
I designed and built him a house.
\z
Both \emph{designed} and \emph{built} are bivalent verbs and \emph{him} is the benefactive that
extends both \emph{designed} and \emph{built}. However, we assume that sentences like (\mex{0}) can
be analyzed as coordination of two verbal items that are licensed by the lexical rule that
introduces the benefactive argument. That is, the benefactive is introduced before the coordination.

The coordination facts illustrate a more general point.  The output of a lexical rule such as the one that would
apply in the analysis of \emph{gebacken} in (\ref{ex-gebacken-und-gegeben}) is just a word (an
\xzero), so it has the same syntactic distribution as an underived word with the same category and
valence feature.  This important generalization follows from the lexical account while on the
phrasal view, it is mysterious at best.  The point can be shown with any of the lexical rules that
the anti-lexicalists are so keen to eliminate in favor of phrasal constructions.  For example,
active and passive verbs can be coordinated, as long as they have the same valence properties, as in
this Swedish example: 

\ea
\gll Golfklubben beg\"arde och beviljade-s marklov f\"or banbygget efter en hel del f\"orhandlingar och kompromisser med L\"ansstyrelsen och 
Naturv\aa rdsverket.\footnotemark\\
golf.club.\textsc{def} requested and granted-\textsc{pass} ground.permit for track.build.\textsc{def} after a whole part negotiations and compromises with county.board.\textsc{def} and nature.protection.agency.\textsc{def} \\
\footnotetext{http://www.lyckselegolf.se/index.asp?Sida=82}
\glt `The golf club requested and was granted a ground permit for fairlane construction after a lot of negotiations and compromises with the County Board and the Environmental Protection Agency.'
\z
\noindent
(English works the same way, as shown by the grammatical translation line.)  
The passive of the ditransitive verb \emph{bevilja} `grant' retains one object, so it is effectively
transitive and can be coordinated with the active transitive \emph{beg\"ara} `request'. 

Moreover, the English passive verb form, being a participle, can feed a second lexical rule deriving
adjectives from verbs.  All categories of English participles can be converted to adjectives
(Bresnan, \citeyear{Bresnan82a}, \citeyear[Chapter~3]{Bresnan2001a}):

\eal
\ex active present participles (cf.\,The leaf is falling): \emph{the falling leaf} 
\ex active past participles (cf.\,The leaf has fallen): \emph{the fallen leaf} 
\ex passive participles (cf.\,The toy is being broken (by the child).): \emph{the broken toy} 
\zl

\noindent
That the derived forms are adjectives, not verbs, is shown by a host of properties, including
negative \emph{un-} prefixation: \emph{unbroken} means `not broken', just as \emph{unkind} means
`not kind', while the \emph{un-} appearing on verbs indicates, not negation, but action reversal, as
in \emph{untie} (Bresnan, \citeyear[\page 21]{Bresnan82a}, \citeyear[Chapter~3]{Bresnan2001a}).  Predicate adjectives preserve the subject of predication of the verb and for
prenominal adjectives the rule is simply that the role that would be assigned to the subject goes to
the modified noun instead (\emph{The toy remained (un-)broken.}; \emph{the broken toy}).  Being an
$A^0$, such a form can be coordinated with another $A^0$, as in the following:

\eal
\ex The suspect should be considered [armed and dangerous].
\ex any [old, rotting, or broken] toys
\zl

\noindent
In (\mex{0}b), three adjectives are coordinated, one underived (\emph{old}), one derived from a
present participle (\emph{rotting}), and one from a passive participle (\emph{broken}).  Such
coordination is completely mundane on a lexical theory.  Each \azero conjunct has a valence feature
(in HPSG it would be the \textsc{spr} feature for predicates or the \textsc{mod} feature for the prenominal
modifiers), which is shared with the mother node of the coordinate structure.  But the point of the
phrasal (or ASC) theory is to deny that words have such valence features.   

The claim that lexical derivation of valence structure is distinct from phrasal combination is
further supported with evidence from deverbal nominalization \citep{Wechsler2008a}.  To derive nouns
from verbs, \emph{-ing} suffixation productively applies to all declinable verbs (\emph{the shooting
  of the prisoner}), while morphological productivity is severely limited for various other suffixes
such as \emph{-(a)tion} (\emph{*~the shootation of the prisoner}).  So forms such as \emph{destruction}
and \emph{distribution} must be retrieved from memory while \emph{-ing} nouns such as \emph{looting} or
\emph{growing} could be (and in the case of rare verbs or neologisms, must be) derived from the verb
or the root through the application of a rule \citep{Zucchi93a-u}.  
This difference explains why
\emph{ing}-nominals always retain the argument structure of the cognate verb, while other forms show
some variation.  A famous example is the lack of the agent argument for the noun \emph{growth} versus
its retention by the noun \emph{growing}: \emph{*~John's growth of tomatoes} versus \emph{John's growing
  of tomatoes} \citep{Chomsky70a}.\footnote{See Section~\ref{deverbal-sec} for further discussion.} 
  
But what sort of rule derives the \emph{-ing} nouns, a lexical rule or a phrasal one?  
In Marantz's \citeyearpar{Marantz97a} phrasal analysis,  a phrasal
construction (notated as \emph{vP}) is responsible for assigning the agent role 
of  \emph{-ing} nouns such as \emph{growing}.  For him, none of the words directly selects an agent via its argument structure.
The \emph{-ing} forms are
permitted to appear in the \emph{vP} construction, which licenses the possessive agent.  
Non-\emph{ing} nouns such as \emph{destruction} and  \emph{growth} do not appear in \emph{vP}.  Whether they allow
expression of the agent depends on semantic and pragmatic properties of the word: \emph{destruction} involves external 
causation so it does allow an agent, while \emph{growth} involves internal causation so it does not allow an agent.

However, a problem for Marantz is that these two types of nouns can coordinate and share dependents (example
(\mex{1}a) is from \citew[Section~7]{Wechsler2008a}): 

\eal
\ex With nothing left after the soldier's [destruction and looting] of their home, they reboarded
their coach and set out for the port of Calais.\footnote{\url{http://www.amazon.com/review/R3IG4M3Q6YYNFT}, 21.07.2012}
\ex  The [cultivation, growing or distribution] of medical marijuana within the County shall at all
times occur within a secure, locked, and fully enclosed structure, including a ceiling, roof or top,
and shall meet the following
requirements.\footnote{\href{http://www.scribd.com/doc/64013640/Tulare-County-medical-cannabis-cultivation-ordinance\#page=1}{http://www.scribd.com/doc/64013640/Tulare-County-medical-cannabis-cultivation-}\newline
  \href{http://www.scribd.com/doc/64013640/Tulare-County-medical-cannabis-cultivation-ordinance\#page=1}{ordinance\#page=1}, 22.10.2012}  
\zl
%\ex I believe it is time in the USA voting population to have the opportunity to vote on adding an
%amendment to the Bill of Rights to legalize the [use, growth and selling] of
%marijuana.\footnote{\url{http://signon.org/sign/constitutional-amendment-28}} 

On the phrasal analysis, the nouns \emph{looting} and \emph{growing} occur in one type
of syntactic environment (namely \emph{vP}), while forms \emph{destruction}, \emph{cultivation}, 
 and \emph{distribution} occur in a different syntactic environment.  This places contradictory
demands on the structure of coordinations like those in (\mex{0}).  As far as we know, neither this problem nor
the others raised by \citet{Wechsler2008a} have even been addressed by advocates of the phrasal theory of
argument structure.    

Consider one last example.  In an influential phrasal analysis, Hale and Keyser (\citeyear{HK93a-u})
derived denominal verbs like \emph{to saddle} through noun incorporation out of a structure akin to
[PUT a saddle ON x].  Again, verbs with this putative derivation routinely coordinate and share
dependents with verbs of other types: 

\ea
Realizing the dire results of such a capture and that he was the only one to prevent it, he quickly
[saddled and mounted] his trusted horse and with a grim determination began a journey that would
become legendary.\footnote{\url{http://www.jouetthouse.org/index.php?option=com_content&view=article&id=56&Itemid=63},
  21.07.2012}  
\z

\noindent
As in all of these \xnull coordination cases, under the phrasal analysis the two verbs place
contradictory demands on a single phrase structure.   

A lexical valence structure is an abstraction or generalization over various occurrences of the verb
in syntactic contexts.  To be sure, one key use of that valence structure is simply to indicate what
sort of phrases the verb must (or can) combine with, and the result of semantic composition; if that
were the whole story then the phrasal theory would be viable.  But it is not.  As it turns out, this
lexical valence structure, once abstracted, can alternatively be used in other ways: among other
possibilities, the verb (crucially including its valence structure) can be coordinated with other
verbs that have a similar valence structure; or it can serve as the input to lexical rules
specifying a new word bearing a systematic relation to the input word.  The coordination and lexical
derivation facts follow from the lexical view, while the phrasal theory at best leaves these facts
as mysterious and at worst leads to irreconcilable contradictions for the phrase structure.   
%In Section~\ref{relations-sec} we consider what the phrasal analysis replace lexical rules.  


%The passive is not a syntactic construction, but rather a verbal valence pattern.  Passive verbs appear in many contexts:
%
%\eal
%\ex Fred got kicked by the mule.
%\ex Nina got Bill elected to the committee.
%\ex Sharon had the carpet cleaned.
%\ex Smith wants the picture removed from the office.
%\ex George saw his brother beaten by the soldiers.
%\ex Any boy handed a worm would scream.
%\ex Handed a worm, the boy screamed.
%\zl
%(examples a-e from Baker 1995, 259)
%
%If we posit many constructions then the fact that they all share the same valence structure, as well as the same morphological form, becomes a highly improbable coincidence.  A super-type construction for passive, with sub-types for get, have, want, etc, would be equivalent to a verb's valence structure.  
%
%passivization of expletives:  thus no fixed semantic content of the 'construction'


\subsection{Valence and derivational morphology}
\label{sec-val-morph}\label{sec-phrasal-LI}\label{sec-inheritance-passive-LFG}

\citet{GJ2004a}, \citet{Alsina96a}, and \citet*{ADT2008a,ADT2013a} suggest analyzing resultative
constructions and/or caused motion constructions as phrasal constructions.\footnote{%
\citet[Section~2.3]{AT2014a} argue that their account is not constructional. If a construction is a
form-meaning pair, their account is constructional, since a certain c"=structure is paired with a
semantic contribution. Asudeh and Toivonen compare their approach with approaches in Constructional
HPSG \citep{Sag97a} and Sign"=Based Construction Grammar (see Section~\ref{sec-SBCG}), which they term constructional. The only difference
between these approaches and the approach by Asudeh, Dalrymple \& Toivonen is that the constructions in the HPSG"=based theories are modeled using types and
hence have a name.%
} As was argued in
\citew{Mueller2006d} this is incompatible with the assumption of lexical integrity, that is, that
word formation happens before syntax  and that the morphological structure is inaccessible to
syntactic processes \citep{BM95a}.\footnote{
  \citet[\page 14]{ADT2013a} claim that the Swedish Directed Motion Construction does not interact
  with derivational morphology. However, the parallel German construction does interact with
  derivational morphology. The absence of this interaction in Swedish can be explained by other
  factors of Swedish\il{Swedish} grammar and given this I believe it to be more appropriate to assume an
  analysis that captures both the German and the Swedish data in the same way.%
}
% \eal
% \ex
% \gll Er fährt den Wagen zu Schrott.\\
%      he drives the car to scrap.metal\\
% \glt `He drives the car to a wreck.'
% \ex
% \gll der zu Schrott gefahrene Wagen\\
%      the to scrap.metal driven car\\ 
% \glt `the car that was driven to a wreck'
% % Blood Red Shoes - der Name bezieht sich auf die blutig getanzten Schuhe Ginger Rogers
% \zl
Let us consider a concrete example, such as (\mex{1}):
\eal
\label{ex-tanzt-schuhe-blutig}
\ex[]{
\gll Er tanzt die Schuhe blutig / in Stücke.\\
     he dances the shoes bloody {} into pieces\\
}
\ex[]{
\gll die in Stücke / blutig getanzten Schuhe\\
     the into pieces {} bloody danced shoes\\
}
\ex[*]{
\gll die getanzten Schuhe\\
     the danced    shoes\\
}
\zl
The shoes are not a semantic argument of \emph{tanzt}. Nevertheless the referent of the NP that is realized as
accusative NP in (\mex{0}a) is the element the adjectival participle in (\mex{0}b) predicates
over. Adjectival participles like the one in (\mex{0}b) are derived from a passive participle of a
verb that governs an accusative object. If the accusative object is licensed phrasally by
configurations like the one in (\mex{0}a), then it is not possible to explain why the participle \emph{getanzte}
can be formed despite the absence of an accusative object in the valence specification of the verb. See \citew[Section~5]{Mueller2006d} for
further examples of the interaction of resultatives and morphology.
% Other valence-dependent derivations are the \bard (\suffix{able}). Resultatives appear in
% German \bards: \emph{leerfischbar} `empty.fishable' and \emph{Leerfischbarkeit}
% `empty.fishability'. The object of \emph{leer fischen} `to fish empty' is not the object of
% \emph{fischen} and hence it cannot be explained why \emph{fischbar}
The conclusion drawn by \citet[\page 412]{Dowty78a}
and \citet[\page 21]{Bresnan82a} in the late 70s and early 80s is that phenomena that feed morphology should be treated
lexically. The natural analysis in frameworks like HPSG, CG, CxG, and LFG is therefore one that assumes
a lexical rule for the licensing of resultative constructions. See
\citew{Verspoor97a}, \citew{Wechsler97a}, \citew{WN2001a}, Wunderlich (\citeyear[\page
  45]{Wunderlich92a-u-kopiert}; \citeyear[\page 120--126]{Wunderlich97c}), \citew{KW98a},
 \citew[Chapter~5]{Mueller2002b}, \citew{Kay2005a}, and \citew{Simpson83a} for lexical proposals in some of
 these frameworks. 

This argument is similar to the one that was discussed in connection with the GPSG representation of
valence in Section~\ref{sec-derivation-GPSG}: morphological processes have to be able to see the valence of the element
they attach to. This is not the case if arguments are introduced by phrasal configurations after the
level of morphology.

Asudeh, Dalrymple \& Toivonen's papers are about the concept of lexical integrity and about
constructions. \citet{AT2014a} replied to our target article and pointed out (again) that their
template approach makes it possible to specify the functional structure of words and phrases
alike. In the original paper they discussed the Swedish word \emph{vägen}, which is the definite
form of \emph{väg} `way'. They showed that the f"=structure is parallel to the f"=structure for the
English phrase \emph{the way}. 
In our reply, \citeyearpar{MWArgStReply} we gave in too early, I believe. Since the point is
not about being able to provide the f"=structure of words, the point is about morphology, that is
-- in LFG terms -- about deriving the f"=structure by a morphological analysis. More generally
speaking, one wants to derive all properties of the involved words, that is, their valence, their
meaning, and the linking of this meaning to their dependents. What we used in our argument based on
the sentences in (\ref{ex-tanzt-schuhe-blutig}) was parallel to what Bresnan (\citeyear[\page
  21]{Bresnan82a}; \citeyear[\page 31]{Bresnan2001a}) used in her classical argument for a lexical
treatment of passive. So either Bresnan's argument (and ours) is invalid or both arguments are valid and there is a problem
for Asudeh, Dalrymple \& Toivonen's approach and for phrasal approaches in general. I want to
give another example that was already discussed in \citew[\page 869]{Mueller2006d} but was omitted in
\citew{MWArgSt} due to space limitations. I will first point out why this example is problematic for
phrasal approaches and then explain why it is not sufficient to be able to assign certain
f"=structures to words: in (\mex{1}a), we are dealing with a resultative construction\is{construction!resultative|(}.
According to the plugging approach, the resultative meaning is contributed by a phrasal construction into which the
verb \emph{fischt} is inserted. There is no lexical item that requires a resultative predicate as
its argument. If no such lexical item exists, then it is unclear how the relation between (\mex{1}a)
and (\mex{1}b) can be established: 

\eal
\ex 
\gll {}[dass] jemand die Nordsee leer fischt\\
     {}\spacebr{}that somebody the North.Sea empty fishes\\
\glt `that somebody fishes the North Sea empty'
\ex\label{bsp-leerfischung}
\gll wegen      der \emph{Leerfischung}  der    Nordsee\footnotemark\\
     because of.the empty.fishing of.the North.Sea\\
\footnotetext{
        taz, 20.06.1996, p.\,6.%
}
\glt `because of the fishing that resulted in the North Sea being empty'
\zl
As Figure~\vref{Abbildung-Resultativkonstruktion-Nominalisierung} shows, both the arguments selected by the heads and the structures are completely different.
In (\mex{0}b), the element that is the subject of the related construction in (\mex{0}a) is not realized. As is normally the case in nominalizations,
it is possible to realize it in a PP with the preposition \emph{durch} `by':
\ea
\gll wegen der Leerfischung der Nordsee durch die Anrainerstaaten\\
     because of.the empty.fishing of.the North.Sea by the neighboring.countries\\
\glt `because of the fishing by the neighboring countries that resulted in the North Sea being empty'
\z
%
\begin{figure}
%\hfill
\begin{forest}
sn edges
[S
	[NP{[\textit{nom}]}
		[jemand;somebody]]
	[NP{[\textit{acc}]}
		[die Nordsee;the North.Sea, triangle]]
	[Adj
		[leer;empty]]
	[V
		[fischt;fishes]]]
\end{forest}
\hfill
\begin{forest}
sn edges
[NP
	[Det
		[die;the]]
	[N$'$
		[N
			[Leerfischung;empty.fishing]]
		[NP{[\textit{gen}]}
			[der Nordsee;of.the North.Sea, triangle]]]]
\end{forest}
%\hfill\mbox{}
\caption{\label{Abbildung-Resultativkonstruktion-Nominalisierung}Resultative construction and nominalization}
\end{figure}%
%
If one assumes that the resultative meaning comes from a particular configuration in which a verb
is realized, there would be no explanation for (\mex{-1}b) since no verb is involved in the analysis
of this example. One could of course assume that a verb stem is inserted into a construction both in
(\mex{-1}a) and (\mex{-1}b). The inflectional morpheme \suffix{t} and the derivational
morpheme \suffix{ung} as well as an empty nominal inflectional morpheme would then be independent syntactic
components of the analysis. However, since \citet[\page 119]{Goldberg2003a} and \citet{ADT2013a}
assume lexical integrity, only entire words can be inserted into syntactic constructions and hence
the analysis of the nominalization of resultative constructions sketched here is not an option for them.

One might be tempted to try and account for the similarities between the phrases in (\mex{-1}) using
inheritance\is{inheritance}. One would specify a general resultative construction standing in an inheritance relation
to the resultative construction with a verbal head and the nominalization construction. I have discussed this proposal in more detail in
\citew[Section~5.3]{Mueller2006d}. It does not work as one requires embedding for derivational morphology and this cannot be modeled
in inheritance hierarchies (\citew{KN93a}, see also \citew{Mueller2006d} for a detailed discussion).

It would also be possible to assume that both constructions  in (\mex{1}), for which structures such as those in
Figure~\ref{Abbildung-Resultativkonstruktion-Nominalisierung} would have to be assumed, are connected via metarules.\footnote{
  Goldberg (p.\,c.\ 2007, 2009) suggests connecting certain constructions using GPSG"=like metarules.
  \citet[\page 51]{Deppermann2006a}, who has a more Croftian view of CxG, rules this out.
 He argues for active/passive\is{passive} alternations that the passive construction has other information
structural\is{information structure} properties.  Note also that GPSG metarules relate phrase
structure rules, that is, local trees. The structure in
Figure~\ref{Abbildung-Resultativkonstruktion-Nominalisierung-Construction}, however, is highly complex.
}$^,$\footnote{
  The structure in (\mex{1}b) violates a strict interpretation of lexical integrity as is commonly assumed in
  LFG\indexlfg. \citet{Booij2005a,Booij2009a}, working in Construction Grammar\indexcxg, subscribes to a somewhat
  weaker version, however.%
}
\eal
\ex {}[ Sbj Obj Obl V ]
\ex {}[ Det [ [ Adj V -ung ] ] NP[\type{gen}] ]
\zl
The construction in (\mex{0}b) corresponds to
Figure~\vref{Abbildung-Resultativkonstruktion-Nominalisierung-Construction}.\footnote{
  I do not assume zero affixes for inflection. The respective affix in
  Figure~\ref{Abbildung-Resultativkonstruktion-Nominalisierung-Construction} is there to show that
  there is structure. Alternatively one could assume a unary branching rule/construction as is
  common in HPSG/Construction Morphology.
}
\begin{figure}
\centering
\begin{forest}
%sn edges
for tree={fit=rectangle}
[NP
	[Det]
	[N$'$
		[N 
                   [N-Stem
			[Adj]
			[V-Stem]
			[-ung] ]
                   [N-Affix [\trace] ]]
		[{NP[\textit{gen}]}] ] ]
\end{forest}
\caption{\label{Abbildung-Resultativkonstruktion-Nominalisierung-Construction}Resultative construction and nominalization}
\end{figure}%
The genitive NP is an argument of the adjective. It has to be linked semantically to the subject slot of the adjective.
Alternatively, one could assume that the construction only has the form [Adj V \suffix{ung}], that
is, that it does not include the genitive NP. But then one could also assume that the verbal variant
of the resultative construction has the form [OBL V] and that Sbj and Obj are only represented in
the valence lists. This would almost be a lexical analysis, however.

Turning to lexical integrity again, I want to point out that all that Asudeh \& Toivonen can do is
assign some f"=structure to the N in
Figure~\ref{Abbildung-Resultativkonstruktion-Nominalisierung-Construction}. What is needed, however,
is a principled account of how this f"=structure comes about and how it is related to the
resultative construction on the sentence level.

%% They could assume allo-constructions and make both c"=structures inherit from the same super construction.
%%
Before I turn to approaches with radical underspecification of argument structure in the next
section, I want to comment on a more recent paper by \citet*{AGT2014a}. The authors discuss the
phrasal introduction of cognate objects and benefactives\is{benefactive|(}. (\mex{1}a) is an example of the latter construction. 
\eal
\ex The performer sang the children a song.
\ex The children were sung a song. 
\zl
According to the authors, the noun phrase \emph{the children} is not an argument of \emph{sing} but
contributed by the c"=structure rule that optionally licenses a benefactive.
\ea\label{c-struc-vp-benefactive}
\phraserule{V$'$}{
\rulenode{V\\* \up~=~\down\\*( @\textsc{Benefactive} )}
\rulenode{DP\\*(\up\ \obj) = \down}
\rulenode{DP\\*(\up\ \objtheta) = \down}
}
\z
Whenever this rule is evoked, the template \textsc{Benefactive} can add a benefactive role and the
respective semantics if this is compatible with the verb that is inserted into the structure. The
authors show how the mappings for the passive\is{passive|(} example in (\mex{-1}b) work, but they do not provide
the c"=structure that licenses such examples. In order to analyze these examples one would need a
c"=structure rule for passive VPs and this rule has to license a benefactive as well. So it would
be:\todostefan{Is it \objtheta or \obj? If it could be \obj, the verb would have to be marked
  passive, since otherwise the benefactive could be introduced on intransitive verbs He laughed the children.}
\ea\label{c-struc-vp-benefactive-passive}
\phraserule{V$'$}{
\rulenode{V[pass]\\* \up~=~\down\\*( @\textsc{Benefactive} )}
\rulenode{DP\\*(\up\ \objtheta) = \down}
}
\z
Note that a benefactive cannot be added to any verb: adding a benefactive to an intransitive verb as
in (\mex{1}a) is out and the passive that would correspond to (\mex{1}a) is ungrammatical as well,
as (\mex{1}b) shows:
\eal
\ex[*]{
He laughed the children.
}
\ex[*]{
The children were laughed.
}
\zl
So one could not just claim that all c"=structure rules optionally introduce a benefactive
argument. Therefore there is something special about the two rules in (\ref{c-struc-vp-benefactive})
and (\ref{c-struc-vp-benefactive-passive}). The problem is that there is no relation between these
rules. They are independent statements saying that there can be a benefactive in the active and that
there can be one in the passive. This is what \citet[\page 43]{Chomsky57a} criticized in 1957 and
this was the reason for the introduction of transformations (see
Section~\ref{Abschnitt-Transformationen} of this book). Bresnan"=style LFG captured the
generalizations by lexical rules and later by Lexical Mapping Theory. But if elements are added
outside the lexical representations, the representations where these elements are added 
have to be related too. One could say that our knowledge about formal tools has changed since
1957. We now can use inheritance hierarchies to capture generalizations. So one can assume a type
(or a template) that is the supertype of all those c"=structure rules that introduce a
benefactive. But since not all rules allow for the introduction of a benefactive element, this
basically amounts to saying: c"=structure rule A, B, and C allow for the introduction of a
benefactive. In comparison, lexical rule"=based approaches have one statement introducing the
benefactive. The lexical rule states what verbs are appropriate for adding a benefactive and
syntactic rules are not affected.\is{passive|)}

In \citet{MWArgSt} we argued that the approach to Swedish caused motion constructions in
\citet{ADT2008a,ADT2013a} would not carry over to German since the German construction interacts with derivational
morphology. \citet{AT2014a} argued that Swedish is different from German and hence there would not
be a problem. However, the situation is different with the benefactive constructions. Although
English and German do differ in many respects, both languages have similar dative constructions:
\eal
\ex He baked her a cake.
\ex
\label{ex-er-buk-ihr-einen-kuchen} 
\gll Er buk   ihr        einen Kuchen.\\
     he baked her.\dat{} a.\acc{} cake\\
\zl
Now, the analysis of the free constituent order was explained by assuming binary branching
structures in which a VP node is combined with one of its arguments or adjuncts (see
Section~\ref{Abschnitt-LFG-Umstellung}). The c"=structure rule is repeated in (\mex{1}):
\ea
\label{lfg-vp-regel-two}
\phraserule{VP}{
\rulenode{NP\\* (\upsp \subj|\obj|\objtheta) = \down}
\rulenode{VP\\* \up~=~\down}}
\z
The dependent elements contribute to the f"=structure of the verb and coherence/""completeness ensure that all
arguments of the verb are present. One could add the introduction of the benefactive argument to
the VP node of the right-hand side of the rule. However, since the verb-final variant of
(\ref{ex-er-buk-ihr-einen-kuchen}) would have the structure in (\mex{1}), one would get spurious
ambiguities, since the benefactive could be introduced at every node:
\ea
\gll weil    [\sub{VP} er [\sub{VP} ihr [\sub{VP} einen Kuchen [\sub{VP} [\sub{V} buk]]]]]\\
     because {}        he {}        her {}        a cake       {}        {}       baked\\
\z
So the only option seems to be to introduce the benefactive at the rule that got the recursion
going, namely the rule that projected the lexical verb to the VP level. The rule (\ref{LFG-v-vp}) is
repeated as (\ref{LFG-v-vp-two}) for convenience.
\ea
\label{LFG-v-vp-two}
\phraserule{VP}{
\rulenode{(V)\\* \up~=~\down}}
\z
Note also that benefactive datives appear in adjectival environments as in (\mex{1}):
\eal
\ex
\gll der seiner Frau einen Kuchen backende Mann\\
     the his.\dat{} wife a.\acc{} cake backing man\\
\glt `the man who is baking a cake for her'
\ex
\gll der einen Kuchen seiner Frau backende Mann\\
     the a.\acc{} cake  his.\dat{} wife backing man\\
\glt `the man who is baking a cake for her'
\zl
In order to account for these datives one would have to assume that the adjective to AP rule that
would be parallel to (\ref{LFG-v-vp-two}) introduces the dative. The semantics of the benefactive
template would have to somehow make sure that the benefactive argument is not added to intransitive
verbs like \emph{lachen} `to laugh' or participles like \emph{lachende} `laughing'. While this may
be possible, I find the overall approach unattractive. First it does not have anything to do with
the original constructional proposal but just states that the benefactive may be introduced at
several places in the syntax, secondly the unary branching syntactic rule is applying to a lexical
item and hence is very similar to a lexical rule and thirdly the analysis does not capture cross"=linguistic commonalities of the
construction. In a lexical rule"=based approach as the one that was suggested by \citet[Section~5]{BC99a}, a benefactive argument is added to certain verbs
and the lexical rule is parallel in all languages that have this phenomenon. The respective
languages differ simply in the way the arguments are realized with respect to their heads. In languages
that have adjectival participles, these are derived from the respective verbal stems. The
morphological rule is the same independent of benefactive arguments and the syntactic rules for
adjectival phrases do not have to mention benefactive arguments.\is{benefactive|)}



\section{Radical underspecification: the end of argument structure?}
\label{radical-sec}

\subsection{Neo-Davidsonianism}

In the last section we examined proposals that assume that verbs come with certain argument roles
and are inserted into prespecified structures that may contribute additional arguments. While we
showed that this is not without problems, there are even more radical proposals that the
construction adds all agent arguments, or even all arguments.  The notion that the agent argument
should be severed from its verbs is put forth by \citet{Marantz84a, Marantz97a}, \citet{Kratzer96a},
\citet{Embick2004a} and others.  Others suggest that no arguments are selected by the verb.
\citet{Borer2003a-u} calls such proposals \emph{exoskeletal} since the structure of the clause is
not determined by the predicate, that is, the verb does not project an inner ``skeleton'' of the
clause.  Counter to such proposals are \emph{endoskeletal} approaches, in which the structure of the
clause is determined by the predicate, that is, lexical proposals.  The radical exoskeletal
approaches are mainly proposed in Mainstream Generative Grammar
\citep{Borer94a-u,Borer2003a-u,Borer2005a-u,Schein93a-u,HK97a-u,Lohndal2012a} but can also be found
in HPSG \citep{Haugereid2009a}.  We will not discuss these proposals in detail here, but we review
the main issues insofar as they relate to the question of lexical argument structure.\footnote{
  See \citew[Section~11.11.3]{MuellerGTBuch1} for a detailed discussion of Haugereid's approach.%
} We conclude that the available empirical evidence favors the lexical argument structure approach over such
alternatives.

Exoskeletal approaches usually assume some version of Neo-Davidsonianism. \citet{Davidson67a-u}
argued for an event variable in the logical form of action sentences (\mex{1}a).
\citet{Dowty89b-u} coined the term \emph{neo-Davidsonian} for the variant in (\mex{1}b), in which
the verb translates to a property of events, and the subject and complement dependents are
translated as arguments of secondary predicates such as \emph{agent} and
\emph{theme}. (\citet{Dowty89b-u} called the system in (\mex{1}a) an \emph{ordered argument
  system}.) \citet{Kratzer96a} further noted the possibility of mixed accounts such as (\mex{1}c),
in which the agent (subject) argument is severed from the \relation{kill} relation, but the theme (object) remains an
argument of the \relation{kill} relation.\footnote{%
  The event variable is shown as existentially bound, as in Davidson's original account.  
  As discussed below, in Kratzer's version it must be bound by a lambda operator instead.} 

\eal\settowidth\jamwidth{(neo-Davidsonian)} \label{neokill1}
\ex \emph{kill}: $\lambda y\lambda x\exists e[kill(e, x, y)]$  \jambox{(Davidsonian)}
\ex \emph{kill}: $\lambda y\lambda x\exists e[kill(e) \wedge agent(e, x) \wedge theme(e, y)]$ \jambox{(neo-Davidsonian)}
\ex \emph{kill}: $\lambda y\lambda x\exists e[kill(e,y) \wedge agent(e, x)]$ \jambox{(mixed)}
\zl
\citet{Kratzer96a} observed that a distinction between Davidsonian, neo-Davidsonian and mixed can be
made either ``in the syntax'' or ``in the conceptual structure'' \citep[\page 110--111]{Kratzer96a}.  For
example, on a lexical approach of the sort we advocate here, any of the three alternatives in
(\mex{0}) could be posited as the semantic content of the verb \emph{kill}.  A lexical entry for
\emph{kill} in the mixed model is given in (\ref{kill-argst-two}). 

\ea\label{kill-argst-two}
\ms{
phon & \phonliste{ kill }\\[1mm]
%head & verb\\
arg-st & \liste{ NP$_x$, NP$_y$ }\\[2mm]
content  & kill(e, y) $\wedge$ agent(e, x)\\ 
}
\z
In other words, the lexical approach is neutral on the question of the `conceptual structure' of eventualities, as noted already in a different connection in 
Section~\ref{polysemy-subsec}.  For this reason, certain semantic arguments for the neo-Davidsonian approach, such as those put forth by  \citet[Chapter~4]{Schein93a-u} 
and \citet{Lohndal2012a}, do not directly bear upon the issue of lexicalism, as far as we can tell.  

But \citet{Kratzer96a}, among others, has gone further and argued for an account that is neo-Davidsonian (or rather, mixed) ``in the syntax''.  
Kratzer's claim is that the verb specifies only the internal argument(s), as in (\mex{1}a) or (\mex{1}b), while the agent (external argument) role is assigned by the phrasal structure.  
On the `neo-Davidsonian in the syntax' view, the lexical representation of the verb has no arguments at all, except the event variable, as shown in (\mex{1}c).

\eal
\label{neokill}\settowidth\jamwidth{(all arguments severed)}
\ex \emph{kill}: $\lambda y\lambda e[kill(e, y)]$                         \jambox{(agent is severed)}
\ex \emph{kill}: $\lambda y\lambda e[kill(e) \wedge theme(e, y)]$ \jambox{(agent is severed)}
\ex \emph{kill}: $\lambda e[kill(e))]$                                  \jambox{(all arguments severed)}
\zl
On such accounts, the remaining dependents of the verb receive their semantic roles from silent secondary predicates,
which are usually assumed to occupy the positions of functional heads in the phrase structure.  An
Event Identification rule identifies the event variables of the verb and the silent light verb
\citep[\page 22]{Kratzer96a}; this is why the existential quantifiers in (\ref{neokill1}) have been
replaced with lambda operators in  (\ref{neokill}).  A standard term for the agent-assigning silent
predicate is `little \emph{v}'.  These extra-lexical dependents are the analogs of the ones
contributed by the constructions in Construction Grammar.   

In the following subsections we address arguments that have been put forth in favor of the `little v'
hypothesis, from idiom asymmetries (Section~\ref{idiom-asym}) and deverbal nominals
(Section~\ref{deverbal-sec}).  We argue that the evidence actually favors the lexical view.  Then we
turn to problems for exoskeletal approaches, from idiosyncratic syntactic selection
(Section~\ref{sec-idiosyncratic-case-and-PP}) and expletives (Section~\ref{sec-expletives}).  We
conclude with a look at the treatment of idiosyncratic syntactic selection under Borer's exoskeletal theory (Section~\ref{sec-borer}), and a summary
(Section~\ref{sec-underspec-summary}).

\subsection{Little \emph{v} and idiom asymmetries}
\label{idiom-asym}

\mbox{}\citet{Marantz84a} and \citet{Kratzer96a} argued for severing the agent from the argument structure as in (\mex{0}a), on the basis of putative idiom asymmetries.
\citet{Marantz84a} observed that while English has many idioms and specialized meanings for verbs in
which the internal argument is the fixed part of the idiom and the external argument is free, the
reverse situation is considerably rarer. To put it differently, the nature of the role played by the
subject argument often depends on the filler of the object position, but not vice versa. To take
Kratzer's examples \citep[\page 114]{Kratzer96a}: 

\eal
\ex kill a cockroach
\ex kill a conversation
\ex kill an evening watching TV 
\ex kill a bottle (i.e. empty it) 
\ex kill an audience (i.e., wow them)
\zl
On the other hand, one does not often find special meanings of a verb associated with the choice of subject, leaving the object position open (examples from \citew[\page 26]{Marantz84a}):

\eal
\ex Harry killed NP.
\ex Everyone is always killing NP. 
\ex The drunk refused to kill NP. 
\ex Silence certainly can kill NP.
\zl
Kratzer observes that a mixed representation of \emph{kill} as in (\mex{1}a) allows us to specify varying meanings that depend upon its sole NP argument.  

\eal
\ex \emph{kill}: $\lambda y\lambda e[kill(e, y)]$ 
\ex If \emph{a} is a time interval, then kill(e, a) = truth if e is an event of wasting \emph{a} \\
If \emph{a} is animate, then kill(e, a) = truth if e is an event in which \emph{a} dies \\
\ldots{} etc.
\zl
On the polyadic (Davidsonian) theory, the meaning could similarly be made to depend upon the filler of the agent role.  On the polyadic view, `there is no technical obstacle' \citep[\page 116]{Kratzer96a} to conditions like those in (\mex{0}b), except reversed, so that it is the filler of the agent role instead of the theme role that affects the meaning.  But, she writes, this could not be done if the agent is not an argument of the verb.  According to Kratzer, the agent-severed representation (such as (\mex{0}a)) disallows similar constraints on the meaning that depend upon the agent, thereby capturing the idiom asymmetry.  

But as noted by \citet{Wechsler2005a}, `there is no technical obstacle' to specifying
agent-dependent meanings even if the Agent has been severed from the verb as Kratzer proposes.  It
is true that there is no variable for the agent in (\mex{0}a).  But there is an event variable
\emph{e}, and the language user must be able to identify the agent of \emph{e} in order to interpret
the sentence.  So one could replace the variable \emph{a} with `the agent of \emph{e}' in the
expressions in (\mex{0}b), and thereby create verbs that violate the idiom asymmetry.

While this may seem to be a narrow technical or even pedantic point, it is nonetheless crucial.  Suppose we try to repair Kratzer's argument with an additional assumption: that modulations in the meaning of a polysemous verb can only depend upon arguments of the \emph{relation} denoted by that verb, and not on other participants in the event.  Under that additional assumption, it makes no difference whether the agent is severed from the lexical entry or not.   For example, consider the following (mixed) neo-Davidsonian representation of the semantic content in the lexical entry of \emph{kill}:    
\ea 
\emph{kill}: $\lambda y\lambda x\lambda e[kill(e,y) \wedge agent(e, x)]$ 
\z
Assuming that sense modulations can only be affected by arguments of the \emph{kill(e,y)} relation,
we derive the idiom asymmetry, even if (\mex{0}) is the lexical entry for \emph{kill}.  So suppose
that we try to fix Kratzer's argument with a different assumption: that modulations in the meaning
of a polysemous verb can only depend upon an argument of the lexically denoted function.  Kratzer's
`neo-Davidsonian in the syntax' lexical entry in (\ref{neokill}a) lacks the agent argument, while
the lexical entry in (\mex{0}) clearly has one.  But Kratzer's entry still fails to predict the
asymmetry because, as noted above, it has the \emph{e} argument and so the sense modulation can be
conditioned on the `agent of \emph{e}'.  As noted above, that event argument cannot be eliminated
(for example through existential quantification) because it is needed in order to undergo event
identification with the event argument of the silent light verb that introduces the agent
\citet[\page 22]{Kratzer96a}.

Moreover, recasting Kratzer's account in lexicalist terms allows for verbs to vary.  This is an important advantage, because the putative asymmetry is only a tendency.  The following are examples in which the subject is a fixed part of the idiom and there are open slots for non-subjects:
\eal
\ex\label{bird}
 A little bird told X that S.
\glt `X heard the rumor that S'   \citep[\page 526]{NSW94a} 
\ex\label{cat-tounge}
The cat's got x's tongue.
\glt `X cannot speak.'     \citep[\page349--350]{Bresnan82c}
\ex\label{what-is-eating-x}
What's eating x?
\glt `Why is X so galled?'  \citep[\page349--350]{Bresnan82c}
\zl
Further data and discussion of subject idioms in English and German can be found in \citew[Section~3.2.1]{MuellerLehrbuch1}.
%the Appendix below.  

The tendency towards a subject-object asymmetry plausibly has an independent explanation.
\citet*{NSW94a} argue that the subject-object asymmetry is a side-effect of an animacy asymmetry.
The open positions of idioms tend to be animate while the fixed positions tend to be inanimate.
\citet{NSW94a} derive these animacy generalizations from the figurative and proverbial nature of the
metaphorical transfers that give rise to idioms.  If there is an independent explanation for this
tendency, then a lexicalist grammar successfully encodes those patterns, perhaps with a mixed
neo-Davidsonian lexical decomposition, as explained above (see \citet{Wechsler2005a} for such a
lexical account of the verbs \emph{buy} and \emph{sell}).  But the `little v' hypothesis rigidly
predicts this asymmetry for all agentive verbs, and that prediction is not borne out.

\subsection{Deverbal nominals}
\label{deverbal-sec}

An influential argument against lexical argument structure involves English deverbal nominals and
the causative alternation.  It originates from a mention in \citet{Chomsky70a}, and is developed in
detail by \citet{Marantz97a}; see also \citet{Pesetsky96a-u} and \citet{HN2000a}.  The argument is
often repeated, but it turns out that the empirical basis of the argument is incorrect, and the
actual facts point in the opposite direction, in favor of lexical argument structure
\citep{Wechsler2008b, Wechsler2008a}.

Certain English causative alternation verbs allow optional omission of the agent argument  (\ref{grow1}), while the cognate nominal disallows expression of the agent (\ref{growth1}):

\eal
\label{grow1}
\ex[]{
that John grows tomatoes
}
\ex[]{
that tomatoes grow
}
\zl

\eal
\label{growth1}
\ex[*]{
John's growth of tomatoes
}
\ex[]{
the tomatoes' growth, the growth of the tomatoes
}
\zl
%
In contrast, nominals derived from obligatorily transitive verbs such as \emph{destroy} allow expression of the agent, as shown in (\ref{destruc1}a):  

\eal
\label{destroy1}
\ex[]{
that the army destroyed the city
}
\ex[*]{
that the city destroyed
}
\zl

\eal
\label{destruc1}
\ex[]{
the army's destruction of the city
}
\ex[]{
the city's destruction
}
\zl

\noindent
Following a suggestion by \citet{Chomsky70a}, \citet{Marantz97a} argued on the basis of these data
that the agent role is lacking from lexical entries. In verbal projections like (\ref{grow1}) and
(\ref{destroy1}) the agent role is assigned in the syntax by little \emph{v}.  Nominal projections
like (\ref{growth1}) and (\ref{destruc1}) lack little  \emph{v}.  Instead, pragmatics takes over to
determine which agents can be expressed by the possessive phrase: the possessive can express `the
sort of agent implied by an event with an external rather than an internal cause' because only the
former can `easily be reconstructed' (quoted from \citet[\page 218]{Marantz97a}).
The destruction of a city has a cause external to the city, while the growth of tomatoes is
internally caused by the tomatoes themselves \citep{Smith70a-u}.  Marantz points out that this
explanation is unavailable if the noun is derived from a verb with an argument structure specifying
its agent, since the deverbal nominal would inherit the agent of a causative alternation verb.   

The empirical basis for this argument is the putative mismatch between the allowability of agent arguments, across some verb-noun cognate pairs: \eg \emph{grow} allows the agent but \emph{growth} does not.  But it turns out that the \emph{grow/growth} pattern
is rare.   Most deverbal nominals precisely parallel the cognate verb: if the verb has an agent, so does the noun.  Moreover, there is a ready explanation for the exceptional cases that exhibit the \emph{grow/growth} pattern \citep{Wechsler2008a}.  First consider non-alternating theme-only intransitives (`unaccusatives'), as in (\ref{arrive1}) and non-alternating transitives as in (\ref{trans}).  The pattern is clear: if the verb is agentless, then so is the noun:

\begin{exe}\ex
\label{arrive1} 
\emph{arriv(al), disappear(ance), fall} etc.:
\begin{xlist}[iv.]
\ex[]{
A letter arrived.
}
\ex[]{
the arrival of the letter
}
\ex[*]{
The mailman arrived a letter.
}
\ex[*]{
the mailman's arrival of the letter
}
\zl

\begin{exe}\ex
\label{trans}
\emph{destroy/destruction, construct(ion), creat(ion), assign(ment)} etc.:
\begin{xlist}[iv.]
\ex The army is destroying the city.
\ex the army's destruction of the city
\zl

\noindent
This favors the view that the noun inherits the lexical argument structure of the verb.  For the
anti-lexicalist, the badness of (\ref{arrive1}c) and (\ref{arrive1}d), respectively, would have to
receive independent explanations.  For example, on Harley and Noyer's \citeyear{HN2000a} proposal,
(\ref{arrive1}c) is disallowed because a feature of the root ARRIVE prevents it from appearing in
the context of \emph{v}, but (\ref{arrive1}d) is instead ruled out because the cause of an event of
arrival cannot be easily reconstructed from world knowledge.  This exact duplication in two separate
components of the linguistic system would have to be replicated across all non-alternating
intransitive and transitive verbs, a situation that is highly implausible.

Turning to causative alternation verbs, Marantz's argument is based on the implicit generalization
that noun cognates of causative alternation verbs (typically) lack the agent argument.  But apart
from the one example of \emph{grow/growth}, there do not seem to be any clear cases of this pattern.
Besides \emph{grow(th)}, Chomsky \citeyear[examples (7c) and (8c)]{Chomsky70a} cited two experiencer
predicates, \emph{amuse} and \emph{interest}:  \emph{John amused (interested) the children with his
  stories}  versus  \emph{*John's amusement (interest) of the children with his stories}.   But this
was later shown by \citet{Rappaport83a-u} and \citet{Dowty89b-u} to have an
independent aspectual explanation.  Deverbal experiencer nouns like \emph{amusement} and
\emph{interest} typically denote a mental state, where the corresponding verb denotes an event in
which such a mental state comes about or is caused.   These result nominals lack not only the agent
but all the eventive arguments of the verb, because they do not refer to events.  Exactly to the
extent that such nouns can be construed as representing events, expression of the agent becomes
acceptable.   

In a response to \citew{Chomsky70a}, Carlota Smith (\citeyear{Smith72a-u}) surveyed
Webster's dictionary and found no support for Chomsky's claim that deverbal nominals do not inherit
agent arguments from causative alternation verbs.  She listed many counterexamples, including
``\emph{explode, divide, accelerate, expand, repeat, neutralize, conclude, unify}, and so on at
length.'' \citep[\page 137]{Smith72a-u}.  Harley and Noyer (\citeyear{HN2000a}) also noted many so-called
``exceptions'':  \emph{explode, accumulate, separate, unify, disperse, transform,
  dissolve/dissolution, detach(ment), disengage-(ment)}, and so on.  The simple fact is that these are not
exceptions because there is no generalization to which they can be exceptions.  These long lists of
verbs represent the norm, especially for suffix-derived nominals (in \suffix{tion}, \suffix{ment}, etc.).
Many zero-derived nominals from alternating verbs also allow the agent, such as  \emph{change,
  release}, and \emph{use}: \emph{my constant change of mentors from 1992--1997}; \emph{the frequent
  release of the prisoners by the governor};  \emph{the frequent use of sharp tools by underage children}
(examples from \citet[fn.\,13]{Borer2003a-u}).\footnote{\citet[\page 79, ex.~(231)]{Pesetsky96a-u} assigns a star
to \emph{the thief's return of the money}, but it is acceptable to many speakers. The \emph{Oxford
  English Dictionary} lists a transitive sense for the noun \emph{return} (definition 11a), and
corpus examples like \emph{her return of the spoils} are not hard to find.}   

Like the experiencer nouns mentioned above, many zero-derived nominals lack event readings.  Some
reject all the arguments of the corresponding eventive verb, not just the agent: \emph{*the freeze
  of the water, *the break of the window}, and so on.  According to Stephen Wechsler,
\emph{his drop of the ball} is slightly odd, but \emph{the drop of the ball} has exactly the same
degree of oddness.  The locution \emph{a drop in temperature} matches the verbal one \emph{The
  temperature dropped}, and both verbal and nominal forms disallow the agent: \emph{*The storm
  dropped the temperature. *the storm's drop of the temperature}.  In short, the facts seem to point
in exactly the opposite direction from what has been assumed in this oft-repeated argument against
lexical valence.  Apart from the one isolated case of \emph{grow/growth}, event-denoting deverbal
nominals match their cognate verbs in their argument patterns.

Turning to \emph{grow/growth} itself, we find a simple explanation for its unusual behavior \citep{Wechsler2008a}.  When the noun \emph{growth} entered the English language,  causative (transitive)  \emph{grow} did not exist.  The OED provides these dates of the earliest attestations of \emph{grow} and \emph{growth}:	

\ea
\label{oed}
\begin{tabular}[t]{@{}l@{~}lrl@{}} 
a. & intransitive \emph{grow}: &  c725	& `be verdant' \ldots{} `increase' (intransitive)\\
b. & the noun \emph{growth}:   &  1587	& `increase' (intransitive)\\
c. & transitive \emph{grow}:   &  1774	& `cultivate (crops)'\\
\end{tabular}
\z

\noindent
Thus \emph{growth} entered the language at a time when transitive \emph{grow} did not exist. The argument structure and meaning were inherited by the noun from its source verb, and then preserved into present-day English.  This makes perfect sense if, as we claim, words have predicate argument structures.  Nominalization by \emph{-th} suffixation is not productive in English, so \emph{growth} is listed in the lexicon.  To explain why \emph{growth} lacks the agent we need only assume that a lexical entry's predicate argument structure dictates whether it takes an agent argument or not.   So even this one word provides evidence for lexical argument structure.  




\subsection{Idiosyncratic syntactic selections}
\label{sec-idiosyncratic-case-and-PP}

%As was mentioned at the beginning of this section, proponents of so-called neo-constructivist
%approaches assume that roots are stored in the lexicon and connected to encyclopedic knowledge that
%helps to determine which arguments may be or have to be present. The arguments are licensed by
%functional projections that may contribute meaning to the core meaning contributed by the
%root or in Haugereid's proposal by binary branching ID schemata that license an argument that fills
%one of five argument roles. 

The notion of lexical valence structure immediately explains why the argument realization patterns
are strongly correlated with the particular lexical heads selecting those arguments.  
It is not sufficient to have general lexical items without valence information
and let the syntax and world knowledge decide about argument realizations, 
%We show that such
%approaches are not sufficient for describing language in total and that 
%The concept of valence is needed, 
because not all 
realizational patterns are determined by the meaning. 
The form of the preposition of a prepositional object is sometimes loosely semantically motivated but in
other cases arbitrary.  For example, the valence structure of the English verb \emph{depend} captures the fact that it selects an \emph{on}-PP to express one of its semantic arguments: 

\eal\label{depends-on-ex}
\ex John depends on Mary.  (\emph{counts, relies,} etc.)
\ex John trusts (*on) Mary.  
\ex 
\ms{
phon & \phonliste{ depend }\\[1mm]
arg-st & \liste{ NP$_x$ , PP[\type{on}]$_y$ }\\[2mm]
content  & depend\textrm{(}x,y\textrm{)}\\ 
}
\zl
Such idiosyncratic lexical selection is utterly pervasive in human language.  The verb or other
predicator often determines the choice between direct and oblique morphology, and for obliques, it
determines the choice of adposition or oblique case.  In some languages such as Icelandic even the
subject case can be selected by the verb \citep*{ZMT85a}.

Selection is language-specific.  English \emph{wait} selects \emph{for} (German \emph{für}) while German \emph{warten} selects \emph{auf} `on' with an accusative object:
\eal \label{loureed}
\ex I am waiting for my man.
\ex 
\gll Ich warte auf meinen Mann.\\
     I   wait  on  my     man.\acc\\
\zl
%A learner has to acquire that \emph{warten}
%has to be used with \emph{auf} + accusative and not with other prepositions or other
%case. 
It is often impossible to find semantic motivation for case.  In German there is a
tendency to replace genitive (\mex{1}a) with dative (\mex{1}b) with no apparent semantic motivation:  
%Instead of the genitive as in
%(\mex{1}a) one also finds examples with the dative as in (\mex{1}b):
\eal
\ex 
\gll dass der Opfer gedacht werde\\
     that the victims.\gen{} remembered was\\
\glt `that the victims would be remembered'
\ex 
\gll daß auch hier den Opfern des Faschismus gedacht werde [\ldots]\footnotemark\\
     that also here the victims.\dat{} of.the fascism remembered was\\
\glt `that the victims of fascism would be remembered here too'
\footnotetext{
Frankfurter Rundschau, 07.11.1997, p.\,6.
}
\zl
The synonyms \emph{treffen} and \emph{begegnen} `to meet' govern different cases (example from \citet[\page 126]{ps}).
%\eal
%\ex 
%\gll Er unterstützt ihn.\\
%     he supports him.\acc\\
%\ex 
%\gll Er hilft ihm.\\
%     he helps him.\dat{}\\
%\zl
\eal
\ex 
\gll Er traf den Mann.\\
     he.\nom{} met the man.\acc{}\\
\ex 
\gll Er begegnete dem Mann.\\
     he.\nom{} met the man.\dat{}\\
\zl
%Similarly, \emph{helfen} `to help' governs dative while and \emph{unterstützen} `to support' governs accusative.
%In order to avoid that the verb \emph{helfen} appears in the syntactic environment that licenses (\mex{0}a)
%and that the verb \emph{unterstützen} appears in the construction that licenses (\mex{0}b), 
One has to specify the case that the respective verbs require in the lexical items of the verbs.\footnote{
  Or at least mark the fact that \emph{treffen} takes an object with the default case for
  objects and \emph{begegnen} takes a dative object in German. See \citew{Haider85b}, \citew{HM94a}, and
  \citew{Mueller2001a} on structural and lexical case.
}
% PS87: S. 126
% Wem begegneten Sie / Wen trafen Sie? -> nicht auf Semantik reduzierbar
%
%Without any semantic motivation one verb takes an accusative object and the other one takes a dative.

A radical variant of the plugging approach is suggested by \citet{Haugereid2009a}. Haugereid
(pages\,12--13) assumes that the syntax combines a verb with an arbitrary combination of a subset of
five different argument roles. Which arguments can be combined with a verb is not restricted by the
lexical item of the verb.\footnote{ 
  Haugereid has the possibility to impose valence restrictions on verbs, but he claims that he uses
  this possibility just in order to get a more efficient processing of his computer implementation (p.\,13).
}
%\settowidth\jamwidth{(Max, 4;9)}
A problem for such views is that the meaning of an ambiguous verb sometimes depends on which of its arguments are expressed. 
The German verb
\emph{borgen}
has the two translations `borrow' and `lend', which basically are two different perspectives on the same event (see
\citew{Kunze91,Kunze93} for an extensive discussion of verbs of exchange of possession). 
Interestingly, the dative object is obligatory only with the `lend' reading \citep[\page 403]{MuellerGTBuch1}:
\eal
\ex 
\gll Ich borge ihm das Eichhörnchen.\\
     I   lend  him the squirrel\\
\glt `I lend the squirrel to him.'
\ex 
\gll Ich borge (mir) das Eichhörnchen.\\
     I borrow \hspaceThis{(}me the squirrel\\
\glt `I borrow the squirrel.'
\zl
If we omit it, we get only the `borrow' reading. 
%So, instead of (\ref{max-lend-borrow}), Max should have
%uttered (\mex{1}a) or (\mex{1}b):
%\eal
%\ex 
%\gll Ich verspreche dir, das niemandem zu borgen.\\
%     I promise you it nobody to lend\\
%\ex 
%\gll Ich verspreche dir, das nicht zu verborgen.\\
%     I promise you it not to lend.out\\
%\zl
%It follows that all theories have to have a place where it is fixed that
So the grammar must specify for specific verbs that
 certain arguments are
necessary for a certain verb meaning or a certain perspective on an event.

Synonyms with differing valence specifications include the minimal triplet mentioned earlier: \emph{dine} is obligatorily intransitive (or takes an \emph{on-}PP), \emph{devour} is transitive, and \emph{eat} can be used either intransitively or transitively \citep[\page 89--90]{Dowty89b-u}.  Many other examples are given in  \citet{Levin93a-u} and \citet{LRH2005a-u}.
%Here, we have another example of different valence
%frames, without there being any possibility to reduce this to a semantic contrast with regard to the
%core meaning of the involved predicates: all three involve an eating frame.
%
%The problem of the argument realization in the triplet \emph{dine}, \emph{devour}, and \emph{eat}
%and other examples by \citet{LRH2005a-u} that show that certain arguments are obligatory is sometimes noted in the literature, but they are simply ignored. 


In a phrasal constructionist approach one would have to assume
phrasal patterns with the preposition or case, into which the verb is inserted.  For (\ref{loureed}b), the pattern includes a prepositional object with \emph{auf} and an
accusative NP, plus an entry for \emph{warten} specifying that it can be inserted into such a structure (see \citew[Section~5.2]{KJ85a} for such a proposal in the framework of TAG). Since there are 
generalizations regarding verbs with such valence representations, one would be forced
to have two inheritance hierarchies: one for lexical entries with their valence properties and
another one for specific phrasal patterns that are needed for the specific constructions in which
these lexical items can be used.  

More often, proponents of neo-constructionist approaches either 
make proposals that are difficult to distinguish from lexical valence structures (see Section~\ref{sec-borer} below)
or simply decline to address the problem.  For instance, \citet{Lohndal2012a} writes:
\begin{quote}
An unanswered question on this story is how we ensure that the functional heads occur together with
the relevant lexical items or roots. This is a general problem for the view that Case is assigned by
functional heads, and I do not have anything to say about this issue here. \citep{Lohndal2012a} % p.\,18
\end{quote}
We think that 
%this view is  inadequate given 
%the current state of linguistics.  
getting case assignment
right in simple sentences, without vast overgeneration of ill-formed word sequences, is a minimal
requirement for a linguistic theory.  % that is asked to be taken seriously.

\subsection{Expletives}
\label{sec-expletives}

A final example for the irreducibility of valence to semantics are verbs that select for expletives
and reflexive arguments of inherently reflexive verbs in German:
\eal
\ex 
\gll weil es regnet\\
     because it rains\\
\ex 
\gll weil (es) mir (vor der Prüfung) graut\\
     because \hspaceThis{(}\textsc{expl} me.\dat{} \hspaceThis{(}before the exam dreads\\
\glt `because I am dreading the exam'
\ex\label{ex-zum-Professor}
\gll weil er es bis zum Professor bringt\\
     because he \textsc{expl} until to.the professor brings\\
\glt `because he made it to professor'
\ex 
\gll weil es sich um den Montag handelt\\
     because \textsc{expl} \textsc{refl} around the Monday trades\\
\glt `It is about the Monday.'
\ex 
\gll weil ich mich (jetzt) erhole\\
     because I myself \hspaceThis{(}now recreate\\
\glt `because I am relaxing'
\zl
The lexical heads in (\mex{0}) need to contain information about the expletive subjects/""objects and/""or
reflexive pronouns that do not fill semantic roles. Note that German allows for subjectless
predicates and hence the presence of expletive subjects cannot be claimed to follow from general
principles. (\ref{ex-zum-Professor}) is an example with an expletive object. Explanations referring
to the obligatory presence of a subject would fail on such examples in any case. Furthermore it has
to be ensured that \emph{erholen} is not realized in the [Sbj IntrVerb] construction for
intransitive verbs or respective functional categories in a Minimalist setting although the relation
\relation{erholen} (\relation{relax}) is a one-place predicate and hence \emph{erholen} is
semantically compatible with the construction.  

% Note also that the psychological predicate \emph{fürchten} `to dread' is semantically similar to
% \emph{grauen} `to dread'. A grammar has to account for the fact that neither verb can be used in a
% different frame:
% \eal
% \ex[]{
% Ich fürchte mich (vor der Prüfung).
% }
% \ex[*]{
% Mir fürchtet (es) (vor der Prüfung).
% }
% Das gibt es ...
% \ex[]{
% Ich graue mich (vor der Prüfung).
% }
% \zl


\subsection{An exoskeletal approach}
\label{Abschnitt-Diskussion-Haugereid}

%% During the past years the phrasal analyses that were common in GPSG are reappearing in several
%% frameworks (almost all versions of Construction Grammar, some versions of LFG, Simpler Syntax). This
%% section discusses an extreme variant of HPSG, namely one that assumes that lexical items do not contain
%% valence information \citep{Haugereid2007a}. Conceptionally, such approaches are much nearer to Borer's exoskeletal approach
%% \citeyearpar{Borer2005a-u} than to HPSG, which is a strongly lexicallized theory.
%% There are many high-level

In what follows I discuss Haugereid's proposal in more detail. His analysis has all the
high"=level problems that were mentioned in the previous subsections, but since it is worked out in
detail it is interesting to see its predictions.

\mbox{}\citet{Haugereid2007a}, working in the framework of HPSG, suggests an analysis along the lines of \citet{Borer2005a-u} where the meaning of an expression is defined as depending
on the arguments that are present. He assumes that there are five argument slots that are assigned to semantic roles\is{semantic role}
as follows:\todostefan{Das gibt irgendwie einen Bruch, weil das so Kleinkram ist.}
\begin{itemize}
\item Arg1: agent or source
\item Arg2: patient
\item Arg3: benefactive or recipient
\item Arg4: goal
\item Arg5: antecedent
\end{itemize}
Here, antecedent is a more general role that stands for instrument, comitative, manner and source.
The roles Arg1--Arg3 correspond to subject and objects. Arg4 is a resultative predicate of the end of a path.
Arg4 can be realized by a PP, an AP or an NP. (\mex{1})
gives examples for the realization of Arg4:
\eal
\ex John smashed the ball \emph{out of the room}.
\ex John hammered the metal \emph{flat}.
\ex He painted the car \emph{a brilliant red}.
\zl
Whereas Arg4 follows the other participants in the causal chain of events, the antecedent precedes the patient in the order of
events. It is realized as a PP.
(\mex{1}) is an example of the realization of Arg5:
\ea
John punctured the balloon \emph{with a needle}.
\z

\noindent
Haugereid now assumes that argument frames consist of these roles. He provides the examples in 
(\mex{1}):
\eal
\settowidth\jamwidth{(arg12345-frame)}
\ex John smiles.           \jambox{(arg1-frame)}
\ex John smashed the ball. \jambox{(arg12-frame)}
\ex The boat arrived.      \jambox{(arg2-frame)}
\ex John gave Mary a book. \jambox{(arg123-frame)}
\ex John gave a book to Mary. \jambox{(arg124-frame)}
\ex John punctured the ball with a needle. \jambox{(arg125-frame)}
\zl

\noindent
Haugereid points out that multiple verbs can occur in multiple argument frames. He provides the variants in (\mex{1})
for the verb \emph{drip}:
\eal
\settowidth\jamwidth{(arg12345-frame)}
\ex The roof drips.                    \jambox{(arg1-frame)}
\ex The doctor drips into the eyes.    \jambox{(arg14-frame)}
\ex The doctor drips with water.       \jambox{(arg15-frame)}
\ex The doctor drips into the eyes with water. \jambox{(arg145-frame)}
\ex The roof drips water.                      \jambox{(arg12-frame)}
\ex The roof drips water into the bucket.      \jambox{(arg124-frame)}
\ex The doctor dripped the eyes with water.    \jambox{(arg125-frame)}
\ex The doctor dripped into the eyes with water. \jambox{(arg145-frame)}
\ex John dripped himself two drops of water.     \jambox{(arg123-frame)}
\ex John dripped himself two drops of water into his eyes. \jambox{(arg1234-frame)}
\ex John dripped himself two drops of water into his eyes with a drop counter. \jambox{(arg12345-frame)}
\ex Water dripped. \jambox{(arg2-frame)}
\ex It drips. \jambox{(arg0-frame)}
\zl
He proposes the inheritance hierarchy in Figure~\ref{Abbildung-Haugereid} in order to represent all possible argument combinations, whereby
the Arg5 role is omitted due to space considerations.

\begin{figure}
\oneline{%
\begin{tabular}{@{}cccccccc@{}}
\multicolumn{8}{c}{\mynode{link}{link}}\\[6ex]
\mynode{arg1p}{arg1+} & \mynode{arg4p}{arg4+} & \mynode{arg2p}{arg2+} & \mynode{arg3p}{arg3+} & \mynode{arg3m}{arg3$-$} & \mynode{arg4m}{arg4$-$} & \mynode{arg1m}{arg1$-$} & \mynode{arg2m}{arg2$-$}\\[8ex]
\mynode{arg12123124}{arg12-123-124} & \mynode{arg12124224}{arg12-124-2-24} & \mynode{arg112}{arg1-12} & \mynode{arg1223}{arg12-23} & \mynode{arg02}{arg0-2}\\[8ex]
\mynode{arg124}{arg124} & \mynode{arg123}{arg123} & \mynode{arg12}{arg12} & \mynode{arg24}{arg24} & \mynode{arg1}{arg1} & \mynode{arg2}{arg2} & \mynode{arg23}{arg23} & \mynode{arg0}{arg0}\\
\end{tabular}
% todo put south and north into a style
\begin{tikzpicture}[overlay,remember picture,shorten <=2pt,shorten >=2pt] 
\draw (link.south)--(arg1p.north)
(link.south)--(arg4p.north)
(link.south)--(arg2p.north)
(link.south)--(arg3p.north)
(link.south)--(arg3m.north)
(link.south)--(arg4m.north)
(link.south)--(arg1m.north)
(link.south)--(arg2m.north)
(arg1p.south)--(arg12123124.north)
(arg1p.south)--(arg112.north)
(arg4p.south)--(arg124.north)
(arg4p.south)--(arg24.north)
(arg2p.south)--(arg12123124.north)
(arg2p.south)--(arg12124224.north)
(arg2p.south)--(arg1223.north)
(arg3p.south)--(arg123.north)
(arg3p.south)--(arg23.north)
(arg3m.south)--(arg12124224.north)
(arg3m.south)--(arg112.north)
(arg3m.south)--(arg24.north)
(arg3m.south)--(arg02.north)
(arg4m.south)--(arg112.north)
(arg4m.south)--(arg1223.north)
(arg4m.south)--(arg02.north)
(arg1m.south)--(arg24.north)
(arg1m.south)--(arg02.north)
(arg1m.south)--(arg23.north)
(arg2m.south)--(arg1.north)
(arg2m.south)--(arg0.north)
(arg12123124.south)--(arg124.north)
(arg12123124.south)--(arg123.north)
(arg12123124.south)--(arg12.north)
(arg12124224.south)--(arg124.north)
(arg12124224.south)--(arg12.north)
(arg12124224.south)--(arg24.north)
(arg12124224.south)--(arg2.north)
(arg112.south)--(arg1.north)
(arg112.south)--(arg12.north)
(arg1223.south)--(arg12.north)
(arg1223.south)--(arg23.north)
(arg02.south)--(arg0.north)
(arg02.south)--(arg2.north);
\end{tikzpicture}

}
\caption{\label{Abbildung-Haugereid}Hierarchy of argument frames following \citet{Haugereid2007a}}
\end{figure}%
Haugereid assumes binary"=branching\is{branching!binary} structures where arguments can be combined with a head in any order.
There is a dominance schema for each argument role. The schema realizing the argument role~3 provides a link value \type{arg3+}.
If the argument role~2 is provided by another schema, we arrive at the frame
\type{arg23}. For unergative intransitive verbs, it is possible to determine that it has an argument frame of 
\type{arg1}. This frame is only compatible with the types \type{arg1+}, \type{arg2$-$},
\type{arg3$-$} and \type{arg4$-$}. Verbs that have an optional object are assigned to \type{arg1-12} according to Haugereid.
This type allows for the following combinations: \type{arg1+}, \type{arg2$-$},
\type{arg3$-$} and \type{arg4$-$} such as \type{arg1+}, \type{arg2$+$}, \type{arg3$-$} and \type{arg4$-$}.

This approach comes very close to an idea by Goldberg: verbs are underspecified with regard to the sentence structures in which they occur and
it is only the actual realization of arguments in the sentence that decides which combinations of arguments are realized.
One should bear in mind that the hierarchy in Figure~\ref{Abbildung-Haugereid} corresponds to a considerable disjunction:
it lists all possible realizations of arguments. If we say that \emph{essen} `to eat' has the type \type{arg1-12}, then this
corresponds to the disjunction \type{arg1} $\vee$ \type{arg12}. In addition to the information in the hierarchy above, one also requires information about the syntactic properties of
the arguments (case, the form of prepositions, verb forms in verbal complements). Since this information is in part specific to each verb
(see Section~\ref{Abschnitt-Stoepselei}), it cannot be present in the dominance schemata and must instead be listed in each individual
lexical entry. For the lexical entry for \emph{warten auf} `wait for', there must be information about the fact that the subject has to be an
NP and that the prepositional object is an \emph{auf}-PP with accusative. The use of a type hierarchy then allows one to elegantly encode
the fact that the prepositional object is optional. The difference to a disjunctively specified
\subcatl with the form of (\mex{1}) is just a matter of formalization.
\ea
\subcat \sliste{ NP[\str] } $\vee$ \sliste{ NP[\str], PP[\type{auf}, \type{acc}] }
\z
%
Since Haugereid's structures are binary"=branching, it is possible to derive all permutations of arguments (\mex{1}a--b), and adjuncts can be
attached to every branching node (\mex{1}c--d). 
\eal
\ex 
\gll dass [\sub{arg1} keiner [\sub{arg2} Pizza isst]]\\
     that {} nobody {} pizza eats\\
\glt `that nobody eats pizza'
\ex 
\gll dass [\sub{arg2} Pizza [\sub{arg1} keiner isst]]\\
	 that {} pizza {} nobody eats\\
\ex 
\gll dass [\sub{arg1} keiner [gerne [\sub{arg2} Pizza isst]]]\\
	 that {} nobody \spacebr{}gladly {} pizza eats\\
\ex 
\gll dass [\sub{arg1} [hier [keiner [\sub{arg2} Pizza isst]]]\\
	 that {} \spacebr{}here \spacebr{}nobody {} pizza eats\\
\zl
Haugereid has therefore found solutions for some of the problems in Goldberg's analysis that were
pointed out in \citew{Mueller2006d}.
Nevertheless, there are a number of other problems, which I will discuss in what follows.
In Haugereid's approach, nothing is said about the composition of meaning. He follows the so"=called Neo"=Davidsonian\is{Neo-Davidsonian semantics} approach.
In these kind of semantic representations, arguments of the verb are not directly represented on the verb.
Instead, the verb normally has an event argument and the argument roles belonging to the event in question are determined in a separate predication.
(\mex{1}) shows two alternative representations, where \emph{e} stands for the event variable.
\eal
\ex 
\gll Der Mann isst eine Pizza.\\
	 the man eats a pizza\\
\glt `The man is eating a pizza'
\ex \relation{eat}(e, x, y) $\wedge$ \relation{man}(x) $\wedge$ \relation{pizza}(y)
\ex \relation{eat}(e) $\wedge$ agent(e,x) $\wedge$ theme(e,y) $\wedge$ \relation{man}(x) $\wedge$ \relation{pizza}(y)
\zl
Haugereid adopts Minimal Recursion Semantics (MRS)\indexmrs as his semantic formalism (also, see Section~\ref{Abschnitt-HPSG-Semantik} und~\ref{Abschnitt-leere-Elemente-Semantik}). 
The fact that arguments belong to a particular predicate is represented by the fact that the relevant predicates have the same handle. The representation in (\mex{0}c) corresponds
to (\mex{1}):
\ea
h1:\relation{essen}(e), h1:arg1(x), h1:arg2(y), h2:\relation{mann}(x), h3:\relation{pizza}(y)
\z
This analysis captures Goldberg's main idea: meaning arises from particular constituents being realized together with a head.

For\is{raising|(} the sentence in (\mex{1}a), Haugereid (2007, p.\,c.) assumes the semantic representation in (\mex{1}b):\footnote{
  See \citew[\page 165]{Haugereid2009a} for an analysis of the Norwegian examples in (i).
\ea
\gll Jon maler veggen rød.\\
     Jon paints wall.\defsc{} red\\
\glt `Jon paints the wall red.'
\zlast
}\is{construction!resultative|(}
\eal
\ex 
\gll der Mann den Teich leer fischt\\
	 the man the pond empty fishes\\
\ex h1:\relation{mann}(x), h2:\relation{teich}(y), h3:\relation{leer}(e),\\
    h4:\relation{fischen}(e2), h4:arg1(x), h4:arg2(y), h4:arg4(h3)
\zl
In (\mex{0}b), the arg1, arg2 and arg4 relations have the same handle as \relation{fischen}. 
Following Haugereid's definitions, this means that arg2 is the patient of the event. In the case of
(\mex{0}a), this makes incorrect predictions since the accusative element is not a semantic argument of the main
verb. It is a semantic argument of the secondary predicate \emph{leer} `empty' and has been raised to the object
of the resultative construction. Depending on the exact analysis one assumes, the accusative object is either a syntactic
argument of the verb or of the adjective, however, it is never a semantic argument of the verb. In addition to this problem,
the representation in (\mex{0}b) does not capture the fact that \emph{leer} `empty' predicates over the object. Haugereid (2007, p.c.) suggests
that this is implicit in the representation and follows from the fact that all arg4s predicate over all arg2s.
Unlike Haugereid's analysis, analyses using lexical rules that relate a lexical item of a verb to
another verbal item with a resultative meaning allow for a precise specification of the semantic representation
that then captures the semantic relation between the predicates involved. In addition, the lexical
rule"=based analysis makes it possible to license lexical items  that do not establish a semantic relation between the accusative object and the verb
(\citealp{Wechsler97a,WN2001a};
\citealp[Chapter~5]{Mueller2002b}).\is{raising|)}\is{construction!resultative|)}

Haugereid sketches an analysis of the syntax of the German clause and tackles active/passive alternations.
However, certain aspects of the grammar are not elaborated on. In particular, it remains unclear how complex clauses containing AcI verbs such as
\emph{sehen} `to see' and \emph{lassen} `to let' should be analyzed. Arguments of embedded and embedding verbs can be permuted in
these constructions. Haugereid (2007, p.\,c.) assumes special rules that allow one to saturate arguments of more deeply embedded verbs, for example,
a special rule that combines an arg2 argument of an argument with a verb. In order to combine \emph{das Nilpferd} and \emph{nicht füttern helfen
  lässt} in sentences such as (\mex{1}), he is forced to assume a special grammatical rule that combines an argument of a doubly embedded verb with another verb:
\ea
\label{ex-nilpferd-fuettern-helfen-laesst}
\gll weil    Hans Cecilia John das Nilpferd nicht füttern helfen lässt\\
     because Hans Cecilia John the hippo not feed help let\\
\glt `because Hans is not letting Cecilia help John feed the hippo.'
\z
In \citet[\page 220]{Mueller2004b}, I have argued that embedding under complex"=forming predicates is only constrained by performance\is{performance} factors
(see also Section~\ref{Abschnitt-Kompetenz-Performanz-TAG}). In German, verbal complexes\is{verbal complex} with more than four verbs are barely acceptable.
\citet[\page 58--59]{Evers75a} has pointed out, however, that the situation in Dutch\il{Dutch} is different since Dutch verbal complexes have a different branching:
in Dutch, verbal complexes with up to five verbs are possible. Evers attributes this difference to a greater processing load for German verbal complexes
(see also \citealp[Section~3.7]{Gibson98a}). Haugereid would have to assume that there are more rules
for Dutch than for German. In this way, he would give up the distinction between competence and
performance and incorporate performance restrictions directly into the grammar. If he wanted to maintain a distinction between the two, then Haugereid would
be forced to assume an infinite number of schemata or a schema with functional uncertainty\is{functional uncertainty} since depth of embedding is only
constrained by performance factors. Existing HPSG approaches to the analysis of verbal complexes do
without functional uncertainty \citep{HN94a}.
Since such raising analyses are required for object raising anyway (as discussed above), they should be given preference.

Summing up, it must be said that Haugereid's exoskeletal approach does account for different
orderings of arguments, but it neither derives the correct semantic representations nor does it offer a
solution for the problem of idiosyncratic selection of arguments and the selection of expletives.

\subsection{Is there an alternative to lexical valence structure?}
\label{sec-borer}

The question for theories denying the existence of valence structure is what replaces it to explain
idiosyncratic lexical selection.  In her exoskeletal approach, \citet{Borer2005a-u} explicitly
rejects lexical valence structures.  But she posits post-syntactic interpretive rules that are
difficult to distinguish from them.  To explain the correlation of \emph{depend} with an
\emph{on}-PP, she posits the following interpretive rule \citep[Vol.\ II, p.\,29]{Borer2005a-u}:

\ea
MEANING $\Leftrightarrow$ $\pi_9 + [ \langle e^{on} \rangle ]$  
\z
Borer refers to all such cases of idiosyncratic selection as idioms.  In a rule such as (\mex{0}),
``MEANING is whatever the relevant idiom means'' \citep[Vol.\ II, p.\,27]{Borer2005a-u}.  In (\mex{0}),
$\pi_9$ is the ``phonological index'' of the verb \emph{depend} and $e^{on}$ `corresponds to an open
value that must be assigned range by the f-morph \emph{on}' \citep[Vol.\ II, p.\,29]{Borer2005a-u}, where f-morphs are function
words or morphemes.  Hence this rule brings together much the same information as the lexical
valence structure in (\ref{depends-on-ex}c).  Discussing such ``idiom'' rules, Borer writes  

\begin{quote}
Although by assumption a listeme cannot be associated with any grammatical properties, one device used in this work has allowed us to get around the formidable restrictions placed on the grammar by such a constraint\,--\,the formation of idioms.  [\ldots] 
%Potentially, then, within the system developed here, any syntactic or morphological property which does not reduce directly to some formal computational principle is to be captured by classifying the relevant item as an idiom\,--\,a partial representation of a phonological index with some functional value. \ldots 
Such idiomatic specification could be utilized, potentially, not just for \emph{arrive} and \emph{depend on}, but also for obligatorily transitive verbs [\ldots], for verbs such as \emph{put}, with their obligatory locative, and for verbs which require a sentential complement.

The reader may object that subcategorization, of sorts, is introduced here through the back door, with the introduction, in lieu of lexical syntactic annotation, of an articulated listed structure, called an \emph{idiom}, which accomplishes, de facto, the same task.  The objection of course has some validity, and at the present state of the art, the introduction of idioms may represent somewhat of a concession. 
%\ldots  On the positive side, we note that to the extent that the existence of idioms is costly, we have attempted to put in place here a system which at least potentially extricates from the costly component of language all properties of listemes which are otherwise derivable from the structure.
  \\ \citep[Vol. II, p.\,354--355]{Borer2005a-u}
\end{quote}
Borer goes on to pose various questions for future research, related to constraining the class of
possible idioms.   With regard to that research program it should be noted that a major focus of lexicalist research has been narrowing the class of subcategorization and extricating derivable properties from idiosyncratic subcategorization.  Those
are the functions of HPSG lexical hierarchies, for example.  
%Whether future research within the
%exoskeletal approach can improve upon the past research in the lexical approach remains to be seen.
%\NOTE{SW:This is a bit obnoxious but I couldn't figure out how else to say it.}  
%
%The valence structure is more explicit about the linking between argument roles and complements, but this linking must be assumed on either theory.  The lexical theory of complement selection is very well-developed and well-understood, and we are unaware of any problems with it that Borer's alternative addresses.\footnote{There are some differences.  The phonological index $\pi$ is not itself a phonological representation such as a sequence of phonemes or a phonological feature matrix, but rather an abstract pointer to the phonological representation of the word.  Supposing for the sake of argument that the structure of that phonological representation is irrelevant to the rules of complement selection, then one way to capture that (hypothetical) generalization is to use the phonological index.  Another way is to posit actual phonological structure in the rule but exclude conditions relating phonological structure to complement selection from the grammar.   In any case, this issue is orthogonal to the question of lexical valence structure because either approach is consistent with the assumption of lexical valence structure, since one could posit either a phonological representation or a phonological index as the value of \textsc{phon}.}  





\subsection{Summary}
\label{sec-underspec-summary}

In Sections~\ref{idiom-asym}--\ref{sec-expletives} we showed that the question of which
arguments must be realized in a sentence cannot be reduced to semantics and world knowledge or to
general facts about subjects. The consequence is that valence information has to be connected to
lexical items. One therefore must either assume a connection between a lexical item and a certain phrasal
configuration as in Croft's approach \citeyearpar{Croft2003a} and in LTAG or assume our lexical
variant. In a Minimalist setting the right set of features must be specified lexically to
ensure the presence of the right case assigning functional heads. This is basically %equivalent 
similar to the lexical valence structures we are proposing here, except that it needlessly introduces  
various problems discussed above, such as the problem of coordination raised in Section~\ref{coordination-sec}. 

\section{Relations between constructions}
\label{relations-sec}
On the lexical rules approach, word forms are related by lexical rules: a verb
stem can be related to a verb with finite inflection and to a passive verb form; verbs can be converted
to adjectives or nouns; and so on.  The lexical argument structure accompanies the word and can be manipulated by the lexical rule.  
%In Section~\ref{lex-deriv-sec} we briefly review this approach and the classic arguments for lexicalism that motivate it.  
In this section we consider what can replace such rules within a phrasal or ASC approach.  



\subsection{Inheritance hierarchies for constructions}
\label{inheritance-sec}
\label{Abschnitt-Croft}


For each valence structure that the lexicalist associates with a root lexeme (transitive, ditransitive, etc.), 
the phrasal approach requires multiple phrasal constructions, one to replace each lexical rule or combination of lexical rules that can apply to the word.  
Taking ditransitives, for example, the phrasal approach requires an active-ditransitive construction, a passive-ditransitive construction, and 
so on, to replace the output of every lexical rule or combination of lexical rules applied to a ditransitive verb.  
(Thus \citew[\page 169--170]{BC2005a} assume an active-ditransitive and a
passive-ditransitive construction and \citew[\page 171--172]{KO2012a} assume active and passive
variants of the transitive construction.)  On that view some of the active voice constructions for German would be:

\eal
\label{ex-active-valence}
\ex {}Nom V
\ex {}Nom Acc V
\ex {}Nom Dat V
\ex {}Nom Dat Acc V
\zl 
The passive voice constructions corresponding to (\mex{0}) would be:
\eal
\label{ex-passive-valence}
\ex {}V V-Aux
\ex {}Nom V V-Aux
\ex {}Dat V V-Aux
\ex {}Dat Nom V V-Aux
\zl  

\noindent
Merely listing all these constructions is not only uneconomical but fails to capture the obvious
systematic relation between active and passive constructions.  Since phrasalists reject both lexical rules and transformations, they need an alternative way to relate phrasal configurations and thereby explain the regular relation between active and passive.  
The only proposals to date involve the use of inheritance hierarchies, so let us examine them.

Researchers working in various frameworks, both with lexical and phrasal orientation, have tried to develop inheritance-based analyses that could
capture the relation between valence patterns such as those in (\mex{-1}) and (\mex{0}) (see for instance
\citew[\page 12]{KF99a}; \citew[Chapter~4]{MR2001a};
\citealp{Candito96a}; \citealp[\page 188]{CK2003a-u}; \citealp[\page 171--172]{KO2012a};
\citealp[Chapter~3]{Koenig99a}; \citealp{DK2000b-u,Kordoni2001b-u} for proposals in CxG, TAG, and HPSG).  The idea
is that a single representation (lexical or phrasal, depending on the theory) can inherit properties from multiple constructions.  
%So \emph{She hammered the metal flat} inherits from the resultative construction, \emph{The metal was hammered} inherits from the passive construction, and \emph{The metal was hammered flat} inherits from both constructions.  
In a phrasal approach the description of the pattern in (\mex{-1}b) inherits from the transitive and
the active construction and the description of (\mex{0}b) inherits from both the transitive and the
passive constructions.  Figure~\vref{fig-passive-inheritance} illustrates the inheritance"=based
lexical approach: a lexical entry for a verb such as \emph{read} or \emph{eat} is combined with either an active
or passive representation. The respective representations for the active and passive are responsible
for the expression of the arguments. 
\begin{figure}
\centering
\begin{forest}
typehierarchy
[lexeme, for descendants={l sep+=5mm}
  [passive,name=passive, [passive $\wedge$ read, name=pr]]
  [active, name=active,  [active $\wedge$  read,  name=ar]]
  [read,   name=read     [passive $\wedge$ eat,  name=pe, no edge]]
  [eat,    name=eat,     [active $\wedge$  eat,   name=ae]] ]
\draw (passive.south)--(pe.north)
      (active.south) --(ae.north)
      (read.south)   --(pr.north)
      (read.south)   --(ar.north)
      (eat.south)    --(pe.north);
\end{forest}
\caption{\label{fig-passive-inheritance}Inheritance Hierarchy for active and passive}
\end{figure}%
%

As was already discussed in Section~\ref{sec-passive-bcg}, inheritance"=based analyses cannot
account for multiple changes in valence as for instance the combination of passive and impersonal
construction that can be observed in languages like Lithuanian\il{Lithuanian}
\citep[Section~5]{Timberlake82a}, Irish\il{Irish} \citep{Noonan94a}, and Turkish\il{Turkish}
\citep{Ozkaragoez86a}. Özkaragöz's Turkish examples are repeated here with the original glossing as
(\ref{ex-double-passivization-two}) for convenience:
\eal\label{ex-double-passivization-two}
\ex\label{ex-double-passivization-strangle-two}
\gll Bu şato-da boğ-ul-un-ur.\\
     this château-\textsc{loc} strangle-\textsc{pass}-\textsc{pass}-\textsc{aor}\\\hfill(Turkish)
\glt `One is strangled (by one) in this château.'
\ex\label{ex-double-passivization-hit-two}
\gll Bu oda-da döv-ül-ün-ür.\\
     this room-\textsc{loc} hit-\textsc{pass}-\textsc{pass}-\textsc{aor}\\
\glt `One is beaten (by one) in this room.'
\ex
\gll Harp-te vur-ul-un-ur.\\
     war-\textsc{loc} shoot-\textsc{pass}-\textsc{pass}-\textsc{aor}\\
\glt `One is shot (by one) in war.'
\zl
Another example from Section~\ref{sec-passive-bcg} that cannot be handled with inheritance is multiple causativization in
Turkish. Turkish allows double and even triple causativization \citep[\page 146]{Lewis67a-u}: 
\ea
Öl-dür-t-tür-t- \hfill(Turkish)\\
`to cause somebody to cause somebody to kill somebody' 
\z 
An inheritance"=based analysis would not work, since inheriting the same information several times
does not add anything new. \citet{KN93a} make the same point with respect to derivational morphology
in cases like \emph{preprepreversion}: inheriting information about the prefix \prefix{pre} twice or
more often, does not add anything.

So assuming phrasal models, the only way to capture the generalization with regard to (\ref{ex-active-valence}) and
(\ref{ex-passive-valence}) seems to be to assume GPSG-like metarules that relate the constructions
in (\ref{ex-active-valence}) to the ones in (\ref{ex-passive-valence}). If the constructions are
lexically linked as in LTAG, the respective mapping rules would be lexical rules. For approaches
that combine LTAG with the Goldbergian plugging idea such as the one by \citet{KO2012a} one would have to
have extended families of trees that reflect the possibility of having additional arguments and
would have to make sure that the right morphological form is inserted into the respective trees. The
morphological rules would be independent of the syntactic structures in which the derived verbal
lexemes could be used. One would have to assume two independent types of rules: GPSG-like metarules
that operate on trees and morphological rules that operate on stems and words. We believe that this
is an unnecessary complication and apart from being complicated the morphological rules would not
be acceptable as form-meaning pairs in the CxG sense since one aspect of the form namely that additional
arguments are required is not captured in these morphological rules. If such morphological rules
were accepted as proper constructions then there would not be any reason left to require that the
arguments have to be present in a construction in order for it to be recognizable, and hence, the
lexical approach would be accepted. Compare the discussion of \emph{Totschießen} `shoot dead' in
example (\ref{ex-tot-schiessen}) below.


Inheritance hierarchies are the main explanatory device in Croft's Radical Construction Grammar
\citep{Croft2001a}\is{Construction Grammar(CxG)|(}\is{inheritance|(}. He also assumes phrasal constructions and suggests representing these in a taxonomic network (an inheritance hierarchy).
He assumes that every idiosyncrasy of a linguistic expression is represented on its own node in this kind of network. Figure~\vref{Abbildung-Vererbungshierarchie-Croft} 
shows part of the hierarchy he assumes for sentences.
\begin{figure}
\centering
\begin{forest}
for tree={draw,          % to get the boxes
          fit=rectangle, % tree layout with more space
          l+=5mm}
[Clause
  [Sbj IntrVerb
    [Sbj sleep]
    [Sbj run]]
  [Sbj TrVerb Obj
    [Sbj kick Obj
      [Sbj kick the bucket]
      [Sbj kick the habit]]
    [Sbj kiss Obj]]]
\end{forest}
\caption{\label{Abbildung-Vererbungshierarchie-Croft}Classification of phrasal patterns in \citew[\page 26]{Croft2001a}}
\end{figure}
There are sentences with intransitive verbs and sentences with transitive verbs. Sentences with the form
[Sbj kiss
  Obj] are special instances of the construction [Sbj TrVerb Obj]. The [Sbj kick Obj] construction also has further
sub"=constructions, namely the constructions [Sbj kick the bucket] and [Subj
  kick the habit]. 
Since constructions are always pairs of form and meaning, this gives rise to a problem: in a normal sentence with \emph{kick}, there is a kicking relation between the subject and the
object of\is{idiom} \emph{kick}. This is not the case for the idiomatic use of \emph{kick} in   
(\mex{1}):
\ea
He kicked the bucket.
\z
This means that there cannot be a normal inheritance relation between the [Sbj kick Obj] and the
[Sbj kick the bucket] construction. Instead, only parts of the information may be inherited from the [Sbj kick Obj] construction. The other parts
are redefined by the sub"=construction. This kind of inheritance is referred to as \emph{default inheritance}\is{inheritance!default}.

\emph{kick the bucket} is a rather fixed expression, that is, it is not possible to passivize it or front parts of it without losing
the idiomatic reading \citep*[\page
508]{NSW94a}. However, this is not true for all idioms. As \citet*[\page 510]{NSW94a} have shown, there are idioms that can be passivized
(\mex{1}a) as well as realizations of idioms where parts of idioms occur outside of the clause (\mex{1}b).
\eal
\ex The beans were spilled by Pat.
\ex The strings [that Pat pulled] got Chris the job.
\zl
%
%
The problem is now that one would have to assume two nodes in the inheritance hierarchy for idioms
that can undergo passivization since the realization of the constituents is different in active and
passive variants but the meaning is nevertheless idiosyncratic. The relation between the active and passive form would be not be captured.
\citet{Kay2002a} has proposed a process where one can computes objects (Construction"=like objects = CLOs) from hierarchies that then license active and passive variants. As I
have shown in \citet[Section~3]{Mueller2006d}, this process does not deliver the desired results and it is far from straightforward to improve the procedure to the point
that it actually works. Even if one were to adopt the changes I proposed, there are still phenomena that cannot be described using inheritance hierarchies
(see Section~\ref{Abschnitt-Passiv-CxG} in this book).

A further interesting point is that the verbs have to be explicitly listed in the constructions. This begs the question of how constructions should be represented where the verbs
are used differently. If a new node in the taxonomic network is assumed for cases like (\mex{1}),
then Goldberg's criticism of lexical analyses that assume several lexical entries for a verb that
can appear in various constructions\footnote{
  Note the terminology: I used the word \emph{lexical entry} rather than \emph{lexical item}. The
  HPSG analysis uses lexical rules that correspond to Goldberg's templates. What Goldberg criticizes
  is lexical rules that relate lexical entries, not lexical rules that licence new lexical items,
  which may be stored or not. HPSG takes the latter approach to lexical rules. See
  Section~\ref{sec-hpsg-passive},%
} will be applicable here: one would have to
assume constructions for every verb and every possible usage of that verb.
\ea
He kicked the bucket into the corner.
\z
%
%
For sentences with negation, Croft assumes the hierarchy with multiple inheritance given in Figure~\vref{Abbildung-Vererbungshierarchie-mehrfach-Croft}. 
\begin{figure}
\centering
\begin{forest}
% we have to mention the style here, since we are overriding the global defaults for anchoring
.style={for tree={parent anchor=north, child anchor=south,grow=north,
          draw,          % to get the boxes
          fit=rectangle, % tree layout with more space
          l+=2mm}}
[I didn't sleep
  [Sbj IntrVerb]
  [Sbj Aux-n't Verb]]
\end{forest}
\caption{\label{Abbildung-Vererbungshierarchie-mehrfach-Croft}Interaction of phrasal patterns following \citew[\page 26]{Croft2001a}}
\end{figure}%
The problem with this kind of representation is that it remains unclear as to how the semantic embedding of the verb meaning under negation can
be represented. If all constructions are pairs of form and meaning, then there would have to be a semantic representation for [Sbj IntrVerb]
(\contv\isfeat{cont} or \textsc{sem} value\isfeat{sem}). Similarly, there would have to be a meaning for [Sbj Aux-n't Verb].
The problem now arises that the meaning of [Sbj IntrVerb] has to be embedded under the meaning of the negation and this cannot be achieved directly
using inheritance since X and not(X) are incompatible. There is a technical solution to this problem using auxiliary features. Since there are a number
of interactions in grammars of natural languages, this kind of analysis is highly implausible if one claims that features are a direct reflection of
observable properties of linguistic objects. For a more detailed discussion of approaches with classifications of phrasal patterns, see \citew{MuellerPersian} as well as
\citew[Section~18.3.2.2]{MuellerLehrbuch1} and for the use of auxiliary features in inheritance"=based analyses of the lexicon, see
 \citew[Section~7.5.2.2]{MuellerLehrbuch1}.\is{inheritance|)}



\subsection{Mappings between different levels of representations}
\label{sec-mapping-between-levels}\label{sec-inheritance-passive-SimSyn}

\citet[Chapter~6.3]{CJ2005a} suggest that passive should be analyzed as one of several possible mappings from the
Grammatical Function tier to the surface realization of arguments. Surface realizations of
referential arguments can be NPs in a certain case, with certain agreement properties, or in a certain position. While such analyses that work by
mapping elements with different properties onto different representations are common in theories
like LFG and HPSG \citep*{Koenig99a,BMS2001a}, a general property of these analyses is that one
needs one level of representation per interaction of phenomena (\argst, \textsc{sem-arg}, \textsc{add-arg}
in Koenig's proposal, \argst, \textsc{deps}, \spr, \comps in Bouma, Malouf, and Sag's proposal). This
was discussed extensively in \citew[Section~7.5.2.2]{MuellerLehrbuch1} with respect to extensions
that would be needed for Koenig's analysis. 

Since Culicover and Jackendoff argue for a phrasal
model, we will discuss their proposal here. Culicover and Jackendoff assume a multilayered model in
which semantic representations are linked to grammatical functions, which are linked to tree
positions. Figure~\ref{fig-jackendoff-linking-active} shows an example for an active sentence.
\begin{figure}
\centering
%\scalebox{.7}
{%
\begin{tabular}{ccccc}
DESIRE(&{~\mynode{b}{BILL$_2$}}, && & ~{\mynode{sw}{[SANDWICH; DEF]$_3$}})\\
\\[1ex]
       &{\mynode{gf2}{GF$_2$}}    && & {\mynode{gf3}{GF$_3$}}~\\
\\[1ex]
~~~~~~~~~\hfill{}[\sub{S} & {\mynode{np2}{NP$_2$}}  & [\sub{VP} & V$_1$ & ~~{\mynode{np3}{NP$_3$}}]] \\
\\
              & Bill           &  & desires & the sandwich.\\
\end{tabular}
\begin{tikzpicture}[overlay,remember picture] 
\draw (b)--(gf2)
      (gf2)--(np2)
      (sw)--(gf3)
      (gf3)--(np3);
\end{tikzpicture}
}
\caption{\label{fig-jackendoff-linking-active}Linking grammatical functions to tree positions: active}
\end{figure}%
GF stands for Grammatical Function. \citet[\page 204]{CJ2005a} explicitly avoid names like Subject and
Object since this is crucial for their analysis of the passive to work. They assume that the first GF
following a bracket is the subject of the clause the bracket corresponds to (p.\,195--196) and hence has to be mapped to an appropriate tree position in
English. Note that this view of grammatical functions and obliqueness 
%is too simplistic since it cannot 
does not account for subjectless sentences that are possible in some languages, for instance in
German.\footnote{
  Of course one could assume empty expletive subjects, as was suggested by \citet[\page
    1311]{Grewendorf93}, but empty elements and especially those without meaning are generally
  avoided in the constructionist literature. See \citew[Section~3.4, Section~11.1.1.3]{MuellerGTBuch1} for further
  discussion.
}

Regarding the passive, the authors write:

\begin{quote}
we wish to formulate the passive not as an operation that deletes or alters part of the argument
structure, but rather as a piece of structure in its own right that can be unified with the other
independent pieces of the sentence. The result of the unification is an alternative licensing
relation between syntax and semantics. \citep[\page 203]{CJ2005a}
\end{quote}
They suggest the following representation of the passive:
\ea
\label{constraint-CJ-passive}
{}\emph{[GF}$_i$ > [\emph{GF} \ldots]\emph{]}$_k$ $\Leftrightarrow$ \emph{[} \ldots V$_k$ + pass \ldots (by NP$_i$) \ldots \emph{]}$_k$
\z
The italicized parts are the normal structure of the sentence and the non-italicized parts are an
overlay on the normal structure, that is, additional constraints that have to hold in passive
sentences. 
Figure~\ref{fig-jackendoff-linking-passive} shows the mapping of the example discussed above that
corresponds to the passive.

\begin{figure}
\centering
%\scalebox{.7}
{%
\begin{tabular}{ccccc}
DESIRE(&~{\mynode{b}{BILL$_2$},} & & & ~{}{\mynode{sw}{[SANDWICH; DEF]$_3$}})\\
\\[1ex]
       &{\mynode{gf2}{GF$_2$}}    &&  & {\mynode{gf3}{GF$_3$}}\\
\\[1ex]
~~~~~~~~~\hfill{}[\sub{S} & {\mynode{np3}{NP$_3$}}  & [\sub{VP} & V$_1$  & by {\mynode{np2}{NP$_2$}}]] \\
\\
              & the sandwich             & & is desired & by Bill.\\
\end{tabular}
\begin{tikzpicture}[overlay,remember picture] 
\draw (b)--(gf2)
      (gf2.south)--(np2.north)
      (sw)--(gf3)
      (gf3.south)--(np3.north);
\end{tikzpicture}
}
\caption{\label{fig-jackendoff-linking-passive}Linking grammatical functions to tree positions: passive}
\end{figure}%

Although Culicover and Jackendoff emphasize the similarity between their approach and Relational
Grammar \citep{Perlmutter83a-ed}, there is an important difference: in Relational Grammar additional levels (strata) can be stipulated
if additional remappings are needed. In Culicover and Jackendoff's proposal there is no additional
level. This causes problems for the analysis of languages which allow for double
passivization. Examples for such languages were already given in (\ref{ex-double-passivization-two}) in
the previous subsection and specific examples from Turkish were provided in
(\ref{ex-double-passivization-two}). Approaches that assume that the personal passive is the unification
of a general structure with a passive-specific structure will not be able to capture this, since they committed
to a certain structure too early. The problem for approaches that state syntactic structure for the
passive is that such a structure, once stated, cannot be modified. Culicover and Jackendoff's 
 proposal works in this respect since there are no strong constraints in the
right-hand side of their constraint in (\ref{constraint-CJ-passive}). But there is a different
problem: when passivization is applied the second time, it has to apply to the innermost bracket,
that is, the result of applying (\ref{constraint-CJ-passive}) should be:
\ea
{}\emph{[GF}$_i$ > [\emph{GF}$_j$ \ldots]\emph{]}$_k$ $\Leftrightarrow$ \emph{[} \ldots V$_k$ + pass \ldots (by NP$_i$) \ldots (by NP$_j$) \ldots\emph{]}$_k$
\z
This cannot be done with unification, since unification checks for compatibility and since the first
application of passive was possible it would be possible for the second time as well. Dots in
representations are always dangerous and in the example at hand one would have to make sure that
NP$_i$ and NP$_j$ are distinct, since the statement in (\ref{constraint-CJ-passive}) just says there
has to be a \emph{by}-PP somewhere. What is needed instead of unification would be something that takes a GF representation
and searches for the outermost bracket and then places a bracket to the left of the next GF. But
this is basically a rule that maps one representation onto another one, just like lexical rules do.

If Culicover and Jackendoff want to stick to a mapping analysis, the only option to analyze the data
seems to be to assume an additional level for impersonal passives from which the mapping to phrase
structure is done. In the case of Turkish sentences like (\mex{1}), which is a personal passive, the mapping
to this level would be the identity function.   
%% \ea
%% \gll Arkada-şım bu   şato-da           boğ-ul-ur.\\
%%      friend-my this chateau-\textsc{loc} strangle-\textsc{pass}-\textsc{aor}\\
%% \glt `My friend is strangled (by one) in this chateau.'
%% \z
\ea
\gll Arkada-şım bu oda-da döv-ül-dü.\\
     friend-my  this   room-\textsc{loc} hit-\textsc{pass}-\textsc{aor}\\
\glt `My friend is beaten (by one) in this room.'
\z

\noindent
In the case of double passivization, the correct mappings would be implemented by two mappings between the three levels
that finally result in a mapping as the one that is seen in (\ref{ex-double-passivization-hit-two}).  
Note that the double passivization is also problematic for purely inheritance based approaches. What
all these approaches can suggest though is that they just stipulate three different relations between
argument structure and phrase structure: active, passive, double passive. But this misses the fact
that (\ref{ex-double-passivization-hit-two}) is a further passivization of (\mex{0}).

In contrast, the lexical rule-based approach suggested by
\citet{Mueller2003e} does not have any problems with double passivization:
the first application of the passivization lexical rule suppresses the least oblique argument and
provides a lexical item with the argument structure of a personal passive. The second application
suppresses the now least oblique argument (the object of the active clause) and results in an
impersonal passive.




\subsection{Is there an alternative to lexical rules?}
 
In this section we have reviewed the attempts to replace lexical rules with methods of relating
constructions.  These attempts have not been successful, in our assessment.  We believe that the
essential problem with them is that they fail to capture the derivational character of the
relationship between certain word forms.  Alternations signaled by passive voice and causative
morphology are relatively simple and regular when formulated as operations on lexical valence
structures that have been abstracted from their phrasal context.  But non-transformational rules or
systems formulated on the phrasal structures encounter serious problems that have not yet been
solved.

\section{Further problems for phrasal approaches}

\citet{Mueller2006d} discussed the problems shared by proposals that assume phrasal constructions to
be a fixed configuration of adjacent material as for instance \citep{GJ2004a}. I showed that many
argument structure constructions allow great flexibility as far as the order of their parts is
concerned. Back then I discussed resultative constructions in their interaction with free datives,
passive and other valence changing phenomena and showed that for all these constructions that are
licensed by such interactions the construction parts can be scrambled, the verb can appear in different positions,
arguments can be extracted and so on. The following subsection discusses particle verbs, which pose
similar problems for theories that assume a phrasal construction with fixed order of verb and particle.

\subsection{Particle verbs and commitment to phrase structure configurations}
\label{sec-particle-verbs-phrasal}

A general problem of approaches that assume phrase structure configurations paired with meaning is
that the construction may appear in different contexts: the construction parts may be involved in
derivational morphology (as discussed in the previous subsection) or the construction parts may be
involved in dislocations. A clear example of the latter type is the phrasal analysis of particle
verbs that was suggested by Booij (\citeyear[Section~2]{Booij2002a}; \citeyear{Booij2012a-u}) and  \citet{Blom2005a}, working in the
frameworks of Construction Grammar\indexcxg and LFG\indexlfg, respectively. The authors working on Dutch\il{Dutch} and German assume that particle verbs are licensed by
phrasal constructions (pieces of phrase structure) in which the first slot is occupied by the particle. 
\ea
{}[ X [~]\sub{V} ]\sub{V$'$} where X = P, Adv, A, or N
\z
Examples for specific Dutch constructions are:
\eal
\label{particle-konstruktionen}
\ex {}[ af   [~]\sub{V} ]\sub{V$'$}
\ex {}[ door [~]\sub{V} ]\sub{V$'$}
\ex {}[ op   [~]\sub{V} ]\sub{V$'$}
\zl 
This suggestion comes with the claim that particles cannot be fronted.  This claim is made
frequently in the literature, but it is based on introspection and wrong for languages like Dutch\il{Dutch} and German. On Dutch see \citew[\page19]{Hoeksema91a}, on German,
\citew{Mueller2002b,Mueller2002d,Mueller2003a,Mueller2007c}.\footnote{%
Some more fundamental remarks on
introspection and corpus data with relation to particle verbs can also be found in
\citew{Mueller2007c,MM2009a}.
} 
A German example is given in (\mex{1}); several pages of attested examples can be found in the cited references and some more complex examples will
also be discussed in Section~\ref{sec-neuro-linguistics} on page~\pageref{ex-complex-vf}.
\ea\label{bsp-los-damit-zwei}
\gll \emph{Los} damit \emph{geht} es schon am 15. April.\footnotemark\\
     \textsc{part} there.with goes it already at.the 15 April\\%
\footnotetext{
        taz, 01.03.2002, p.\,8, see also \citew[\page313]{Mueller2005d}.%
    }%
\glt `It already starts on April the 15th.'
\z
Particle verbs are mini-idioms. So the conclusion is that idiomatic expressions that 
allow for a certain flexibility in order should not be represented as phrasal configurations describing adjacent
elements. For some idioms, a lexical analysis along the lines of \citew{Sag2007a} seems to be
required.\footnote{Note also that the German example is best described as a clause with a complex internally 
  structured constituent in front of the finite verb and it is doubtful whether linearization-based
  proposals like the ones in \citew [\page244--248]{Kathol95a} or \citew{Wetta2011a} can capture
  this. See also the discussion of multiple frontings\is{fronting!apparent multiple} in connection to Dependency Grammar in Section~\ref{sec-dg-multiple-frontings}.
}
The issue of particle verbs will be taken up in Section~\ref{sec-neuro-linguistics} again, where we
discuss evidence for/against phrasal analyses from neuro science.



\section{Arguments from language acquisition}
\label{sec-acquisition}

The question whether language acquisition is pattern"=based and hence can be seen as evidence for
the phrasal approach has already been touched upon in the Sections~\ref{Abschnitt-musterbasiert}
and~\ref{Abschnitt-Selektionsbasierter-Spracherwerb}. It was argued that constructions can be
realized discontinuously in coordinations and hence it is the notion of
dependency that has to be acquired, acquiring simple continuous patterns is not sufficient.

Since the present discussion about phrasal and lexical approaches deals with specific proposals, I would like
to add two more special subsections: Section~\ref{sec-recognizability-of-constructions} deals with
the recognizability of constructions and Section~\ref{Abschnitt-Koordination-diskont} discusses specific approaches to coordination
in order to demonstrate how frameworks deal with the discontinuous realization of constructions.


\subsection{Recognizability of constructions}
\label{sec-recognizability-of-constructions}
 
I think that a purely pattern-based approach is weakened by the existence of examples like (\mex{1}):
\eal
\ex John tried to sleep.
\ex John tried to be loved.
\zl
Although no argument of \emph{sleep} is present in the phrase \emph{to sleep} and neither a subject
nor an object is realized in the phrase \emph{to be loved}, both phrases are recognized as phrases
containing an intransitive and a transitive verb, respectively.\footnote{
Constructionist theories do not assume empty elements. Of course, in the GB framework the subject
would be realized by an empty element. So it would be in the structure, although inaudible.%
}  

The same applies to arguments that are supposed to be introduced/licensed by a phras\-al construction:
in (\mex{1}) the resultative construction is passivized and then embedded under a control
verb, resulting in a situation in which only the result predicate (\emph{tot} `dead') and the matrix verb (\emph{geschossen} `shot') are
realized overtly within the local clause, bracketed here:
\ea
\gll Der kranke Mann wünschte sich, [tot geschossen zu werden].\footnotemark\\
     the sick   man  wished   SELF  \spacebr{}dead shot      to be\\
\footnotetext{
\citew[\page 387]{Mueller2007d}.
}
\glt `The sick man wanted to be shot dead.'
% replaced "ill" by "sick" after submission
\z
Of course passivization and control are responsible for these occurrences, but the important point
here is that arguments can remain unexpressed or implicit and nevertheless a meaning that is usually
connected to some overt realization of arguments is present \citep[Section~4]{Mueller2007d}. So,
what has to be acquired by the language learner is that when a result predicate and a main verb are
realized together, they contribute the resultative meaning.  
To take another example, NP arguments that are usually realized in active resultative constructions may remain implicit
in nominalizations like the ones in (\mex{1}):
\eal
\label{ex-tot-schiessen}
\ex 
\gll dann scheint uns das Totschießen mindestens ebensoviel Spaß zu machen\footnotemark\\
     then seems   us  the dead-shooting at.least as.much    fun to make\\
\footnotetext{
  \href{https://www.elitepartner.de/forum/wie-gehen-die-maenner-mit-den-veraenderten-anspruechen-der-frauen-um-26421-6.html}{https://www.elitepartner.de/forum/wie-gehen-die-maenner-mit-den-veraenderten-anspruechen-der-}
  \href{https://www.elitepartner.de/forum/wie-gehen-die-maenner-mit-den-veraenderten-anspruechen-der-frauen-um-26421-6.html}{frauen-um-26421-6.html}. 26.03.0212.
}
\glt `then the shooting dead seems to us to be as least as much fun'
% added "to us" after submission
\ex
\gll Wir lassen heut das Totgeschieße,\\                   
we  let    today the annoying.repeated.shooting.dead\\\\
\gll  Weil  man sowas heut nicht tut.\\
      since one such.thing today not does\\\\
\gll Und wer einen Tag sich ausruht,\\
     and who a day \textsc{self} rests\\\\
\gll Der schießt morgen doppelt gut.\footnotemark\\
this shoots tomorrow twice good\\
\footnotetext{
  \url{http://home.arcor.de/finishlast/indexset.html?dontgetmestarted/091201-1.html}. 26.03.2012.
} 
\glt `We do not shoot anybody today, since one does not do this today, and those who rest a day shoot
twice as well tomorrow.'
\zl
The argument corresponding to the patient of the verb (the one who is shot) can remain unrealized,
because of the syntax of nominalizations.  The resultative meaning is still understood, which shows
that it does not depend upon the presence of a resultative construction involving Subj V Obj and Obl.  


\subsection{Coordination and discontinuousness}
\label{Abschnitt-Koordination}
\label{Abschnitt-Koordination-diskont}\label{sec-coordination-cg}

The following subsection deals with analyses of coordination in some of the frameworks that were
introduced in this book. The purpose of the section is to show that simple phrasal patterns have to
be broken up in coordination structures. This was already mentioned in Section~\ref{Abschnitt-musterbasiert}, but I think it
is illuminative to have a look at concrete proposals.

In Categorial Grammar, there is a very elegant treatment of coordination (see \citealp{Steedman91a}). 
A generalization with regard to so"=called symmetric coordination is that two objects with the same syntactic properties are combined to an object
with those properties. We have already encountered the relevant data in the discussion of the motivation for feature geometry in HPSG on
page~\pageref{Seite-HPSG-Koordination}. Their English versions are repeated below as (\mex{1}):
\eal
\ex the man and the woman
\ex He knows and loves this record.
\ex He is dumb and arrogant.
\zl
\citet{Steedman91a} analyzes examples such as those in (\mex{0}) with a single rule:
\ea
X conj X $\Rightarrow$ X
\z
This rule combines two categories of the same kind with a conjunction in between to form a category that has the same category as the conjuncts.\footnote{
Alternatively, one could analyze all three examples using a single lexical entry for the conjunction
\emph{and}: \emph{and} is a functor that takes a word or phrase
of any category to its right and after this combination then needs to be combined with an element of the same category to its left in order to form the relevant
category after combining with this second element. This means that the category for \emph{und} would have the form (X\bs X)/X. 
This analysis does not require any coordination rules. If one wants to assume, as is common in\indexgb GB/MP\indexmp, that every structure has a head, then a headless
analysis that assumes a special rule for coordination like the one in (\mex{0}) would be ruled out.
}
Figure~\vref{Abb-cg-np-koordination} shows the analysis of (\mex{-1}a) and
Figure~\vref{Abb-CG-Koordination-V} gives an analysis of the corresponding English\il{English} example of
(\mex{-1}b).
\begin{figure}
\centerline{%
\deriv{5}{
the & man & and        & the  & woman\\
\hr & \hr  & \hr        & \hr & \hr \\
np/n & n   & conj       & np/n & n\\
\multicolumn{2}{@{}c@{}}{\forwardapp} &  & \multicolumn{2}{c@{}}{\forwardapp} \\
\multicolumn{2}{@{}c@{}}{np}          &  & \multicolumn{2}{c@{}}{np}\\
\multicolumn{5}{@{}c@{}}{\conjapp}\\
\multicolumn{5}{@{}c@{}}{np}\\
}
}
\caption{\label{Abb-cg-np-koordination}Coordination of two NPs in Categorial Grammar}
\end{figure}%

\begin{figure}
\centerline{%
\deriv{6}{
he  & knows        & and        & loves        & this  & record\\
\hr & \hr          & \hr        & \hr          & \hr   & \hr         \\
np  & (s\bs np)/np & conj & (s\bs np)/np & n/np  & n\\
    &              &      &              & \multicolumn{2}{c@{}}{\forwardapp}\\
    &              &      &              & \multicolumn{2}{@{}c@{}}{np}\\
    & \multicolumn{3}{c@{}}{\conjapp}\\
    & \multicolumn{3}{c@{}}{(s\bs np)/np}\\
    & \multicolumn{5}{c@{}}{\forwardapp} \\
    & \multicolumn{5}{c@{}}{s\bs np}\\
\multicolumn{6}{@{}c@{}}{\backwardapp}\\
\multicolumn{6}{@{}c@{}}{s}\\
}
}
\caption{\label{Abb-CG-Koordination-V}Coordination of two transitive verbs in Categorial Grammar}
\end{figure}%

If we compare this analysis to the one that would have to be assumed in traditional phrase structure grammars, it becomes apparent
what the advantages are: one rule was required for the analysis of NP coordination where two NPs are coordinated to form an NP and another was required
for the analysis of V coordination. This is not only undesirable from a technical point of view, neither does it capture the basic property of symmetric coordination:
two symbols with the same syntactic category are combined with each other.

It is interesting to note that it is possible to analyze phrases such as (\mex{1}) in this way:
\ea
\label{Beispiel-Gapping-Steedman}
give George a book and Martha a record
\z
In Section~\ref{Abschnitt-K-Tests-Koordination}, we have seen that these kind of sentences are problematic for constituent tests. However, in Categorial Grammar, it is possible to
analyze them without any problems if one adopts rules for type raising and composition as \citet{Dowty88a-u} and \citet{Steedman91a} do.
In Section~\ref{Abschnitt-UDC-KG}, we have already seen forward type raising as well as forward and backward composition. In order to analyze
(\mex{0}), one would require backward type raising repeated in (\mex{1}) and backward composition repeated in
(\mex{2}):
\ea
Backward type raising\is{type raising!backward} (< T)\\
X $\Rightarrow$ T\bs (T/X)
\z
\ea
Backward composition\is{composition!backward} (< B)\\
    Y\bs Z $*$ X\bs Y = X\bs Z
\z

\noindent
Dowty's analysis of (\mex{-2}) is given in Figure~\ref{Abb-CG-Gapping}. VP stands for s\bs np.
% vref loopt
\begin{figure}
\oneline{%
\deriv{6}{
give       & George                  & a\;book            & and        & Martha                  & a\;record  \\
\hr        & \backwardt               & \backwardt         & \hr        & \backwardt               & \backwardt    \\
(vp/np)/np & (vp/np)\bs ((vp/np)/np) & vp\bs (vp/np)     & conj & (vp/np)\bs ((vp/np)/np) & vp\bs (vp/np)\\
           & \multicolumn{2}{c@{}}{\backwardc}        &      &  \multicolumn{2}{c@{}}{\backwardc}\\
           & \multicolumn{2}{c@{}}{vp\bs ((vp/np)/np)}&      & \multicolumn{2}{@{}c@{}}{vp\bs ((vp/np)/np)}\\
    & \multicolumn{5}{c@{}}{\conjapp}\\
    & \multicolumn{5}{c@{}}{vp\bs ((vp/np)/np)}\\
\multicolumn{6}{@{}c@{}}{\backwardapp} \\
\multicolumn{6}{@{}c@{}}{vp}\\
}
}
\caption{\label{Abb-CG-Gapping}Gapping in Categorial Grammar}
\end{figure}%

This kind of type"=raising analysis was often criticized because raising categories leads to many different analytical possibilities
for simple sentences. For example, one could first combine a type"=raised subject with the verb and then combine the resulting constituent
with the object. This would mean that we would have a [[S V]
O] in addition to the standard [S [V O]] analysis.
\citet{Steedman91a} argues that both analyses differ in terms of information structure and it is therefore valid to assume different structures
for the sentences in question.

I will not go into these points further here. However, I would like to compare Steedman's lexical approach to phrasal analyses: all approaches
that assume that the ditransitive construction represents a continuous pattern encounter a serious problem with the examples discussed above. This
can be best understood by considering the TAG analysis\indextag of coordination proposed by \citet{SJ96a}.
If one assumes that [Sbj TransVerb Obj] or [S [V O]] constitutes a fixed unit, then the trees in
Figure~\vref{Abbildung-knows-loves} form the starting point for the analysis of coordination. 

\begin{figure}
\hfill
\begin{forest}
tag
[S
	[NP$\downarrow$]
	[VP
		[V
			[knows]]
		[NP$\downarrow$]]]
\end{forest}
\hfill
\begin{forest}
tag
[S
	[NP$\downarrow$]
	[VP
		[V
			[loves]]
		[NP$\downarrow$]]]
\end{forest}
\hfill\mbox{}
\caption{\label{Abbildung-knows-loves}Elementary trees for \emph{knows} and \emph{loves}}
\end{figure}%

If one wants to use these trees/constructions for the analysis of (\mex{1}), there are in principle
two possibilities: one assumes that two complete sentences are coordinated or alternatively, one
assumes that some nodes are shared in a coordinated structure.  
\ea
\label{Beispiel-he-knows-and-loves}
He knows and loves this record.
\z
%
% Diese Argumentationen sind wohl allesamt gegen Old-School Transformationsanalysen.
% Wenn man nur eine Variable hat, dann kriegt man auch die falschen Lesarten nicht.
% St. Mü. 30.05.2010
%
% Ansätze zur Behandlung der Koordination, die Beispiele wie (\mex{0}) auf zwei vollständige
% koordinierte Sätze zurückführen, wären auch nicht auf alle
% Koordinationsdaten anwendbar, wie bereits \citet[\page 102]{BV72}, \citet[\page 143]{Dowty79a},
% \citet[\page 104--105]{denBesten83a}, \citet[\page 8--9]{Klein85} und \citet{Eisenberg94a} festgestellt haben.
% %siehe auch deGeest70a:40
% Das Problem ist, dass die beiden folgenden Sätze nicht bedeutungsgleich sind:
% \eal
% \ex[]{
% Ein Mädchen stand an der Ecke und winkte mir über die Straße.
% }
% \ex[]{
% Ein Mädchen stand an der Ecke und ein Mädchen winkte mir über die Straße.
% }
% \zl
% Der erste Satz ist wahr, wenn es ein Mädchen gibt, dass sowohl an der Ecke steht als auch winkt,
% der zweite Satz wäre auch wahr, wenn es zwei Mädchen gäbe, von denen das eine an der Ecke steht und
% das andere winkt.
%
\citet{Abeille2006a} has shown that it is not possible to capture all the data if one assumes that cases of coordination such as  those in
(\mex{0}) always involve the coordination of two complete clauses. It is also necessary to allow for lexical coordination of the kind we saw
in Steedman's analysis (see also Section~\ref{Abschnitt-Spezfikatoren-MP}).
\citet{SJ96a} develop a TAG analysis\indextag in which nodes are shared in coordinate structures.
The analysis of (\ref{Beispiel-he-knows-and-loves}) can be seen in Figure~\vref{Abbildung-He-knows-and-loves-this-record-TAG}.
%vref loopt hier
\begin{figure}
\centering
\begin{forest}
sn edges
[\phantom{S}
  [S, no edge
	[NP,name=np11
		[he]]
	[VP, name=vp1
          [V,name=v1    [knows]]]]
  [S, no edge, name=s2
        [V, name=vcoord, no edge [and, name=and, no edge]]
        [VP
           [V,name=v2 [loves]]
           [NP, name=np22 [this record, triangle]]]]]
\draw (s2.south)--(np11.north)
      (vp1.south)--(np22.north);
\draw[thick] (vcoord.south)--(v1.north)
             (vcoord.south)--(v2.north)
             (vcoord.south)--(and.north);
\end{forest}
\caption{\label{Abbildung-He-knows-and-loves-this-record-TAG}TAG analysis of \emph{He knows and
    loves this record.}}
\end{figure}%
The subject and object nodes are only present once in this figure. The S nodes of both elementary trees both dominate the \emph{he} NP.
In the same way, the object NP node belongs to both VPs. The conjunction connects two verbs indicated by the thick lines. Sarkar and Joshi provide an
algorithm that determines which nodes are to be shared. The structure may look strange at first, but for TAG purposes, it is not the derived tree but rather the derivation tree that is important, since this is the one that is used to compute the semantic interpretation. The authors show that the derivation trees
for the example under discussion and even more complex examples can be constructed correctly.

In theories such as HPSG\indexhpsg and LFG\indexlfg where structure building is, as in Categorial Grammar, driven by valence, the above sentence is unproblematic:
both verbs are conjoined and then the combination behaves like a simple verb. The analysis of this is given in Figure~\vref{Abbildung-He-knows-and-loves-this-record-HPSG}. 
This analysis is similar to the Categorial Grammar analysis in
Figure~\ref{Abb-CG-Koordination-V}.\footnote{
  A parallel analysis in Dependency Grammar\indexdg is possible as well. \tes's original analysis
  was different though. See Section~\ref{sec-dg-coordination} for discussion.
}
\begin{figure}
\centering
\begin{forest}
sn edges
[S
	[NP
		[he]]
	[VP
		[V
			[V
				[knows]]
			[and]
			[V
				[loves]]]
		[NP
			[this record,triangle]]]]
\end{forest}
\caption{\label{Abbildung-He-knows-and-loves-this-record-HPSG}Selection"=based analysis of \emph{He knows and
    loves this record.} in tree notation}
\end{figure}%
With Goldberg's plugging analysis one could also adopt this approach to coordination: here, \emph{knows}
and \emph{loves} would first be plugged into a coordination construction and the result would then be plugged into the transitive construction.
Exactly how the semantics of \emph{knows and loves} is combined with that of the transitive construction is unclear since the meaning of this phrase
is something like \relation{and}(\relation{know}(x, y), \relation{love}(x, y)), that is, a complex event with at least two open argument slots x and y 
(and possibly additionally an event and a world variable depending on the semantic theory that is used). Goldberg would probably have to adopt an analysis such as the one in 
Figure~\ref{Abbildung-He-knows-and-loves-this-record-TAG} in order to maintain the plugging analysis.

Croft would definitely have to adopt the TAG analysis since the verb is already present in his constructions. For the example in (\ref{Beispiel-Gapping-Steedman}),
both Goldberg and Croft would have to draw from the TAG analysis in Figure~\vref{Abbildung-He-gave-george-a-book-and-martha-a-record-TAG}.
\begin{figure}
\centering
\begin{forest}
sn edges
[\phantom{S}
  [S, no edge
	[NP, name=np11
		[he]]
	[VP, name=vp1
		[V, name=v1 [gave]]
		[NP [George]]
	        [NP [a book,triangle]]]]
  [VP,name=vpcoord, no edge [and, name=and, no edge]]
  [S, name=s2, no edge
    [VP, name=vp2
      [NP [Martha]]
      [NP [a record,triangle]]]]]
\draw (s2.south)--(np11.north)
      (vp2.south)--(v1.north);
\draw[thick] (vpcoord.south)--(vp1.north)
             (vpcoord.south)--(vp2.north)
             (vpcoord.south)--(and.north);
\end{forest}
\caption{\label{Abbildung-He-gave-george-a-book-and-martha-a-record-TAG}TAG analysis of \emph{He
    gave George a book and Martha a record.}}
\end{figure}%

\noindent
The consequence of this is that one requires discontinuous\is{constituent!discontinuous} constituents. Since coordination allows a considerable number
of variants, there can be gaps between all arguments of constructions. An example with a ditransitive verb is given in (\mex{1}):
\ea
He gave George and sent Martha a record.
\z
See \citew{Crysmann2003c} and \citew{BS2004a} for HPSG analyses\indexhpsg that assume discontinuous constituents for particular
coordination structures.

The result of these considerations is that the argument that particular elements occur next to each other and this occurrence is associated with a particular meaning
is considerably weakened. What competent speakers acquire is actually the knowledge that heads must
occur with their arguments somewhere in the utterance and that all the requirements of the heads involved have to somehow be satisfied ($\theta$ Criterion, coherence/completeness, empty \subcatl). 
The heads themselves must not necessarily occur directly adjacent to their arguments. See the discussion in Section~\ref{Abschnitt-musterbasiert} about pattern"=based
models of language acquisition.

The semantics construction for complex structures such as those in Figure~\ref{Abbildung-He-gave-george-a-book-and-martha-a-record-TAG} is by no means
trivial. In TAG\indextag, there is the derivation tree in addition to the derived tree that can then be used to compute the semantic contribution of a linguistic
object. Construction Grammar does not have this separate level of representation. The question of how the meaning of the sentences discussed here is derived from
their component parts still remains an open question for phrasal approaches.

Concluding the section on language acquisition, we assume that a valence representation is the
result of language acquisition, since this is necessary for establishing the dependency relations in
various possible configurations in an utterance. See also \citew[\page 439]{Behrens2009a} for a similar conclusion. 

\section{Arguments from psycho- and neurolinguistics}

This section has three parts: in the first part we compare approaches that assume that valence
alternations are modeled by lexical rules, underspecification, or disjunctions with phrasal
approaches. In Subsection~\ref{sec-psycho-lv} part we discuss approaches to light verb constructions and Subsection~\ref{sec-neuro-linguistics}
is devoted to neurolinguistic findings.


\subsection{Lexical rules vs.\ phrasal constructions}
\label{sec-lr-phrasal-psycho}

\mbox{}\citet[Section~1.4.5]{Goldberg95a} uses evidence from psycholinguistic experiments to argue against lexical
approaches that use lexical rules to account for argument structure alternations: \citet{CT88a}
showed that sentences with true lexical ambiguity like those in (\mex{1}) and sentences with two
verbs with the same core meaning have different processing times.
\eal
\ex Bill set the alarm clock onto the shelf.
\ex Bill set the alarm clock for six.
\zl
\eal
\ex Bill loaded the truck onto the ship.
\ex Bill loaded the truck with bricks.
\zl
Errors due to lexical ambiguity cause a bigger increase in processing time than errors in the use of
the same verb. Experiments showed that there was a bigger difference in processing times for the
sentences in (\mex{-1}) than for the sentences in (\mex{0}). The difference in processing times
between (\mex{0}a) and (\mex{0}b) would be explained by different preferences for phrasal
constructions. In a lexicon-based approach one could explain the difference by assuming that one
lexical item is more basic, that is, stored in the mental dictionary and the other is derived from
the stored one. The application of lexical rules would be time consuming, but since the lexical
items are related, the overall time consumption is smaller than the time that is needed to process
two unrelated items \citep[\page 405]{Mueller2002b}.

Alternatively one could assume that the lexical items for both valence patterns are the result of
lexical rule applications. As with the phrasal constructions, the lexical rules would have different
preferences. This shows that the lexical approach can explain the experimental results as well, so
that they do not force us to prefer phrasal approaches.

\citet[\page 18]{Goldberg95a} claims that lexical approaches have to assume two variants of \emph{load}
with different meaning and that this would predict that \emph{load} alternations would behave like
two verbs that really have absolutely different meanings. The experiments discussed above show that
such predictions are wrong and hence lexical analyses would be falsified. However, as was shown in
\citew[Section~11.11.8.2]{MuellerGTBuch1}, the argumentation contains two flaws: let's assume that the construction
meaning of the construction that licenses (\mex{0}a) is C$_1$ and the construction meaning of the
construction that licenses (\mex{0}b) is C$_2$. Under such assumptions the semantic contribution of
the two lexical items in the lexical analysis would be (\mex{1}). load(\ldots) is the contribution
of the verb that would be assumed in phrasal analyses.
\ea
\begin{tabular}[t]{@{}l@{~}l@{~}l@{}}
a. & load (onto): & C$_1$ $\wedge$ load(\ldots)\\
b. & load (with): & C$_2$ $\wedge$ load(\ldots)\\
\end{tabular}
\z
(\mex{0}) shows that the lexical items partly share their semantic contribution. We hence predict
that the processing of the dispreferred argument realization of \emph{load} is simpler than the
dispreferred meaning of \emph{set}: in the latter case a completely new verb has to be activated
while in the first case parts of the meaning are activated already. (See also \citew[\page
64--65]{Croft2003a} for a brief rejection of Goldberg's interpretation of the experiment that
corresponds to what is said here)

\citet[\page 107]{Goldberg95a} argues against lexical rule-based approaches for locative
alternations\is{locative alternation} like (\mex{1}), since according to her such approaches have to assume that one of the verb forms has to be the more
basic form.
\eal
\ex He loaded hay onto the wagon.
\ex He loaded the wagon with hay.
\zl
She remarks that this is problematic since we do not have clear intuitions about what the basic and
what the derived forms are. She argues that the advantage of phrasal approaches is that various
constructions can be related to each other without requiring the assumption that one of the
constructions is more basic than the other. There are two phrasal patterns and the verb is used in
one of the two patterns. This criticism can be addressed in two ways: first one could introduce two
lexical types (for instance \type{onto-verb} and \type{with-verb}) into a type hierarchy. The two
types correspond to two valence frames that are needed for the analysis of (\mex{0}a) and
(\mex{0}b). These types can have a common supertype (\type{onto-with-verb}) which is relevant for all
\emph{spray}/\emph{load} verbs. One of the subtypes or the respective lexical item of the verb is
the preferred one. This corresponds to a disjunction in the lexicon, while the phrasal approach
assumes a disjunction in the set of phrasal constructions. 
%BCPW2005 nehmen eine Familie von Lexikoneinträgen an.

A variant of this approach is to assume that the lexical description of \emph{load} just contains
the supertype, that describes all \emph{spray}/\emph{load} verbs. Since model theoretic approaches
assume that all structures that are models of utterances contain only maximally specific types (see
for instance \citew{King99a-u} and \citew[\page 21]{ps2}), it is sufficient to say about verbs like
\emph{load} that they are of type \type{onto-with-verb}. Since this type has exactly two subtypes,
\emph{load} has to be either \type{onto-verb} or \type{with-verb} in an actual model.\footnote{
  This analysis does not allow one to specify verb specific preferences for one of the realization
  patterns since the lexicon contains the general type only.
}

A second option is to stick with lexical rules and to assume a single representation for the root of
a verb that is listed in the lexicon. In addition, one assumes two lexical rules that map this basic
lexical item onto other items that can be used in syntax after being inflected. The two lexical
rules can be described by types that are part of a type hierarchy and that have a common
supertype. This would capture commonalities between the lexical rules. We therefore have the same
situation as with phrasal constructions (two lexical rules vs.\ two phrasal constructions). The only
difference is that the action is one level deeper in the lexical approach, namely in the lexicon \citep[\page 405--406]{Mueller2002b}. 

The argumentation with regard to the processing of resultative constructions like (\mex{1}c) is parallel:
\eal
\ex He drinks.
\ex He drinks the milk.
\ex He drinks the pub empty.
\zl
When humans parse a sentence they build up structure incrementally. If one hears a word that is
incompatible with the current hypothesis, the parsing process breaks down or the current hypothesis
is revised. In (\mex{0}c) \emph{the pub} does not correspond to the normal transitive use of
\emph{drink}, so the respective hypothesis has to be revised. In the phrasal approach the resultative
construction would have to be used instead of the transitive construction. In the lexical analysis
the lexical item that is licensed by the resultative lexical rule would have to be used rather than
the bi-valent one. Building syntactic structure and lexicon access in general place different
demands on our processing capacities. However, when (\mex{0}c) is parsed, the lexical items for
\emph{drink} are active already, we only have to use a different one. It is currently unclear to us
whether psycholinguistic experiments can differentiate between the two approaches, but it seems to
be unlikely.


\subsection{Light verbs}
\label{sec-psycho-lv}

\citet*{WJKP2014a} report on a number of experiments that test predictions that are made by
various approaches to light verb constructions. (\mex{1}a) shows a typical light verb construction:
\emph{take} is a light verb that is combined with the nominal that provides the main
predication. 
\eal
\ex take a walk to the park
\ex walk to the park
\zl
%% \citet{HK93a-u} assume that (\mex{1}b) is derived from (\mex{1}a).
%% 
%% The structure they assume for the light verb construction is more complex than the one for the non-light verb in (\mex{0}b). This is due
%% to the fact that it is assumed that the noun \emph{walk} incorporates into a v node by
%% head-movement. As \citet{WJKP2014a} point out, this makes wrong predictions as far as processing is concerned. This
%% approach predicts that light verb constructions should be easier to process, which is not borne
%% out. Furthermore, head-movement approaches assume that the light verb constructions differ in their
%% underlying and surface structure from the non-light verb constructions. However, to the extent that
%% structural priming reflects syntactic structure, data from priming experiments suggest that the
%% light verb constructions and non-light verb constructions share the same kind of syntax
%% \citep{Wittenberg:2012mz}.  

%\begin{sloppypar}
%% Therefore there remain two classes of approaches as psycholinguistically plausible candidates
%% \citep{WP2011a}
\citet{WP2011a} examined two psychologically plausible theories of light verb constructions.  The phrasal approach 
 assumes that light verb constructions are stored objects associated with semantics \citep{Goldberg2003a}.
The alternative compositional view assumes that the semantics is computed as a fusion of the
semantics of the event noun and the semantics of the
light verb \citep{Grimshaw97a-u,Butt2003a-u,Jackendoff2002a-u,CJ2005a,MuellerPersian,BPW2008a-u}.  
Since light verb constructions are extremely frequent (\citealp*{Pinango:2006qy};
\citealp[\page 399]{WP2011a}), the phrasal approaches that assume that
light verb constructions are stored items with the object and verb fixed predict that light verb
constructions should be retrievable faster than non-light verb constructions like (\mex{1}) \citep[\page
  396]{WP2011a}. 
\ea
take a frisbee to the park
\z
This is not the case. As Wittenberg and Piñango found, there is no difference in processing at the licensing
condition (the noun in VO languages like English and the verb in OV languages like German). 
%\end{sloppypar}

However, \citet{WP2011a} found an increased processing load 300ms \emph{after} the light verb construction is
processed. The authors explain this by assuming that semantic integration of the noun with the
verbal meaning takes place after the syntactic
combination. While the syntactic combination is rather fast, the semantic computation takes
additional resources and this is measurable at 300ms. The verb contributes aspectual information and integrates
the meaning of the nominal element. The semantic roles are fused. The resource consumption effect
would not be expected if the complete light verb construction were a stored item that is
retrieved together with the complete meaning (p.\,404). We can conclude that Wittenberg and
Piñango's results are compatible with the lexical proposal, but are
incompatible with the phrasal view. % suggested by \citet{Goldberg2003a}. 


\subsection{Arguments from Neurolinguistics}
\label{sec-neuro-linguistics}

%\subsection{Particle Verbs}

\mbox{}\citet*{PCShandbookCxG} discuss neurolinguistic facts and relate them to the CxG view of grammar
theory. One important finding is that deviant words (lexical items) cause brain responses that differ in polarity
from brain responses on incorrect strings of words, that is, syntactic combinations. This suggests
that there is indeed an empirical basis for deciding the issue.

Concerning the standard example of the caused motion construction in (\mex{1}) the authors write the
following:
\ea
She sneezed the foam off the cappuccino.\footnote{
\citew[\page 42]{Goldberg2006a}.
}
\z
\begin{quote}
  this constellation of brain activities may initially lead to the co"=activation of the verb \emph{sneeze}
  with the DCNAs for \emph{blow} and thus to the sentence mentioned. Ultimately, such co-activation of a
  one-place verb and DCNAs associated with other verbs may result in the former one-place verb being
  subsumed into a three-place verb category and DCNA set, a process which arguably has been
  accomplished for the verb \emph{laugh} as used in the sequence \emph{laugh NP off the stage}. \citep*{PCShandbookCxG}
\end{quote}
A DCNA is a discrete combinatorial neuronal assembly. Regarding the specifics of DCNAs the authors write that 
\begin{quote}
Apart from linking categories together, typical DCNAs establish a temporal order between the
category members they bind to. DCNAs that do not impose temporal order (thus acting, in principle,
as AND units for two constituents) are thought to join together constituents whose sequential order
is free or allow for scrambling. \citep*[\page 404]{PCShandbookCxG}
% allow for scrambling steht wirklich so im quote
\end{quote}
I believe that this view is entirely compatible with the lexical view outlined above: the lexical
item or DCNA requires certain arguments to be present. A lexical rule that relates an intransitive verb
to one that can be used in the caused motion construction is an explicit representation of what it
means to activate the valence frame of \emph{blow}.

The authors cite earlier work \citep*{CSP2010a} and argue that particle verbs are lexical objects,
admitting for a discontinuous realization of particle verbs despite their lexical status
(p.\,21). They restrict their claim to frequently occurring particle verbs. This claim is of course
compatible with our assumptions here, but the differences in brain behavior are interesting when it
comes to fully productive uses of particle verbs. For instance any semantically appropriate mono"=valent verb in German can
be combined with the aspectual particle \emph{los}: \emph{lostanzen} `start to dance',
\emph{loslachen} `start to laugh', \emph{lossingen} `start to sing', \ldots. Similarly, the
combination of mono-valent verbs with the particle \emph{an} with the reading \emph{directed-towards} is
also productive: \emph{anfahren} `drive towards', \emph{anlachen} `laugh in the direction of',
\emph{ansegeln} `sail towards', \ldots{} (see \citew{Stiebels96a} on various productive
patterns). 
%As was argued in Section~\label{sec-bar-derivation}, this pattern of particle verb formation
%interacts with derivational morphology.
The interesting question is how particle verbs behave that follow these patterns but occur with low
frequency. This is still an open question as far as the experimental evidence is concerned, but as
I argue below lexical proposals to particle verbs as the one suggested by \citet{Mueller2003a} are
compatible with both possible outcomes.

Summarizing the discussion so far, lexical approaches are compatible with the accumulated neurobiological evidence 
and as far as particle verbs are concerned they seem to be better suited than the phrasal proposals
by \citet[Section~2]{Booij2002a} and \citet{Blom2005a} (See Section~\ref{sec-particle-verbs-phrasal}
for discussion). However, in general, it remains an open question what 
it means to be a discontinuous lexical item. The idea of discontinuous words is pretty old
\citep{Wells47a}, but there have not been many formal accounts of this idea. \citet*{NSW94a} suggest
a representation in a linearization-based framework of the kind that was proposed by
\citet{Reape94a} and \citet*[\page 244--248]{Kathol95a} and \citet{Crysmann2002a} worked out such
analyses in detail. Kathol's lexical item for \emph{aufwachen} `to wake up' is given in (\mex{1}):
\eas
\label{le-aufwachen-Kathol}
\mbox{\emph{aufwachen} (following \citealp[\page 246]{Kathol95a}):}\\
\begin{tabular}{@{}l@{}}
\onems{
\ldots$|$head   \ibox{1} \type{verb}\\
\ldots$|$vcomp  \eliste\\
dom \liste{ \onems{ \phonliste{ wachen }\\
                      \ldots$|$head  \ibox{1}\\
                      \ldots$|$vcomp \sliste{ \ibox{2} }\\
                    }} $\bigcirc$
    \liste{ \onems[vc]{ \phonliste{ auf\/ }\\
                      synsem \ibox{2} \ms{ \ldots$|$head \onems[sepref~]{flip $-$\\
                                                                     }\\
                                         }\\
                    }
            }\\
}
\end{tabular}
\zs
The lexical representation contains the list-valued feature \textsc{dom} that contains a description of the
main verb and the particle (see Section~\ref{sec-discontinuous-constituents-HPSG} for details). The \doml is a list that contains the dependents of a head. The
dependents can be ordered in any order provided no linearization rule is violated
\citep{Reape94a}. The dependency between the particle and the main verb was characterized 
by the value of the \vcomp feature, which is a valence feature for the selection of arguments that
form a complex predicate with their head. The shuffle operator $\bigcirc$ concatenates two lists
without specifying an order between the elements of the two lists, that is, both \emph{wachen},
\emph{auf} and \emph{auf}, \emph{wachen} are possible. The little marking \type{vc} is an assignment
to a topological field in the clause.

I criticized such linearization-based proposals since it is unclear how
analyses that claim that the particle is just linearized in the domain of its verb can account for
sentence like (\mex{1}), in which complex syntactic structures are involved \citep{Mueller2007d}. German is a V2 language
and the fronting of a constituent into the position before the finite verb is usually described as
some sort of nonlocal dependency, that is, even authors who assume linearization-based analyses do
not assume that the initial position is filled by simple reordering of material
\citep{Kathol2000a,Mueller99a,Mueller2002b,TBjerre2006a}.\footnote{
  \citet[Section~6.3]{Kathol95a} working in HPSG\indexhpsg suggested such an analysis for simple sentences, but later
  changed his view. \citet{Wetta2011a} also working in HPSG assumes a purely linearization-based
  approach. Similarly \citet{GO2009a} working in  Dependency Grammar\indexdg assume that there is a
  simple dependency structure in simple sentences while there are special mechanisms to account for
  extraction out of embedded clauses. I argue against such proposals in \citew{MuellerGS} referring
  to the scope\is{scope} of adjuncts, coordination of simple with complex sentences and Across the Board
  Extraction\is{Across the Board Extraction} and apparent multiple frontings\is{fronting!apparent
    multiple}. See also Section~\ref{sec-linearization-problems-dg}.
}
\eal
\label{ex-complex-vf}
\ex
\gll {}[\sub{vf} [\sub{mf} Den Atem]  [\sub{vc} an]] hielt die ganze Judenheit.\footnotemark\\
       {}        {}        the breath {}    \partic{}  held  the whole Jewish.community\\
\glt `The whole Jewish community held their breath.'
\footnotetext{
Lion Feuchtwanger, \emph{Jud Süß}, p.\,276, quoted from \citew[\page 56]{Grubacic65a}.
}
\ex\label{bsp-wieder-an-tritt-zwei}
\gll {}[\sub{vf} [\sub{mf} Wieder] [\sub{vc} an]] treten auch die beiden Sozialdemokraten.\footnotemark\\
      {}         {}        again   {}        \partic{} kick also the two social.democrats\\
\footnotetext{
  taz, bremen, 24.05.2004, p.\,21.
}
\glt `The two Social Democrates are also running for office again.' % check

\ex
\gll {}[\sub{vf} [\sub{vc} Los]        [\sub{nf} damit]]    geht es schon   am 15. April.\footnotemark\\
       {}        {}        \textsc{part}  {}        there.with went it already at.the 15 April\\%
\footnotetext{
        taz, 01.03.2002, p.\,8.%
    }%
\glt `It already starts on April the 15th.'
\zl
The conclusion that has to be drawn from examples like (\mex{0}) is that particles interact in
complex ways with the syntax of sentences. This is captured by the lexical treatment that was
suggested in \citew[Chapter~6]{Mueller2002b} and \citew{Mueller2003a}: the main verb selects for the verbal
particle. By assuming that \emph{wachen} selects for \emph{auf} the tight connection between verb
and particle is represented.\footnote{
\citet[\page 197]{CSP2010a} write: \emph{the results provide neurophysiological evidence that
  phrasal verbs are lexical items. Indeed, the increased activation that we found for existing
  phrasal verbs, as compared to infelicitous combinations, suggests that a verb and its particle
  together form one single lexical representation, i.\,e.\ a single lexeme, and that a unified
  cortical memory circuit exists for it, similar to that encoding a single word} I believe that
  my analysis is compatible with this statement.
} Such a lexical analysis provides an easy way to account for fully
intransparent particle verbs like \emph{an-fangen} `to begin'. However, I also argued for a
lexical treatment of transparent particle verbs like \emph{losfahren} `to start to drive' and
\emph{jemanden/etwas anfahren} `drive directed towards somebody/something'. The
analysis involves a lexical rule that licenses a verbal item selecting for an adjunct
particle. The particles \emph{an} and \emph{los} can modify verbs and contribute arguments (in the
case of \emph{an}) and the particle semantics. This analysis can be shown to be compatible with the
neuro"=mechanical findings: if it is the case that even transparent particle verb combinations with
low frequency are stored, then the rather general lexical rule that I suggested in the works cited above is the
generalization of the relation between a large amount of lexical particle verb items and their respective main
verb. The individual particle verbs would be special instantiations that have the form of the
particle specified as it is also the case for non-transparent particle verbs like \emph{anfangen}.
If it should turn out that productive particle verb combinations with particle verbs of low
frequency cause syntactic reflexes in the brain, this could be explained as well: the lexical rule
licenses an item that selects for an adverbial element. This selection would then be seen as
parallel to the relation between the determiner and the noun in the NP \emph{der Mut} `the
courage', which \citet[\page 191]{CSP2010a} discuss as an example of a syntactic combination. Note
that my analysis is also compatible with another observation made by \citet*{SPP2005a-u}:
Morphological affixes also cause the lexical reflexes. In my analysis the stem of the main
verb is related to another stem that selects for a particle. This stem can be combined with
(derivational and inflectional) morphological affixes causing the lexical activation pattern in the
brain. After this combination the verb is combined with the particle and the dependency can be
either a lexical or a syntactic one, depending on the results of the experiments to be carried
out. The analysis is compatible with both results.
% CSP2010:198 say they have to check less frequent items.

Note that my analysis allows the principle of lexical integrity to be maintained. I therefore do
not follow \citep*[\page 198]{CSP2010a}, who claim that they \emph{provide proof that potentially
  separable multi-word items can nonetheless be word-like themselves, and thus against the validity
  of a once well-established linguistic principle, the Lexical Integrity Principle}. I agree that
non-transparent particle verbs are multi-word lexemes, but the existence of multi-word lexemes does
not show that syntax has access to the word internal morphological structure. The parallel between
particle verbs and clearly phrasal idioms was discussed in \citew{Mueller2002b,Mueller2002d} and it
was concluded that idiom-status is irrelevant for the question of wordhood. Since the interaction of
clearly phrasal idioms with derivational morphology as evidenced by examples like (\mex{1}) did not
force grammarians to give up on lexical integrity, it can be argued that particle verbs are not
convincing evidence for giving up the Lexical Integrity Principle either.\footnote{
  However, see \citew{Booij2009a} for some challenges to lexical integrity.
}
\eal
\ex
\gll Er hat ins Gras gebissen.\\
     he has in.the gras bit\\
\glt `He bit the dust.'
\ex 
\gll "`Heath Ledger"' kann ich nicht einmal schreiben, ohne dass mir sein ins Gras-Gebeiße wieder so
wahnsinnig leid tut% -- Den hatte ich so gerne.
\footnotemark\\
    \spacebr{}Heath Ledger can I not even write without that me his in.the grass.biting again so
    crazy sorrow does\\
\footnotetext{
\url{http://www.coffee2watch.at/egala}. 23.03.2012 
}
\glt `I cannot even write ``Heath Ledger'' without being sad again about his biting the dust.'
\zl
The example in (\mex{0}b) involves the discontinuous derivation with the circumfix \gee
(\citealp[Section~3.4.3]{Luedeling2001a}; \citealp[\page 324--327, 372--377]{Mueller2002b};
\citealp[Section~2.2.1, Section~5.2.1]{Mueller2003a}). Still the parts of the idiom \emph{ins Gras
  beiß-} `bite the dust' are present and with them the idiomatic reading. See \citew{Sag2007a} for a lexical analysis of idioms that can
explain examples like (\mex{0}).

% As was shown in \citew{Mueller2002b,Mueller2002d,Mueller2003a,Mueller2006c} (German) particle verbs
% pattern in many respects with verbal complexes, resultative predicates, and other predicative
% constructions like copula constructions and \emph{consider} type predications.

% In our model the verb selects for a particle and then the two parts can be realized discontinuously.

% We think that further experiments are needed in order to establish what exactly has been
% measured. The authors mention on page 21--22 that morphologically complex structures also cause a
% lexical response in the brain. This is interesting since clearly syntactic constructions as (\ref)
% repeated here as (\mex{1}) interact with morphology. So it would be interesting to see the
% neuro-imaging results and respective explanations.
% \ea
% \gll die in Stücke / blutig getanzten Schuhe\\
%      the into pieces {} bloody danced shoes\\
% \glt `the shoes that were danced bloody / into pieces'
% \z

%% Eva Witttenberg noted several problems with this paper.
%% \subsection{Light Verb Constructions}

%% The last subsection showed that certain results from neuro linguistics are compatible with both the
%% lexicalist and the phrasal constructionalist view. This subsection briefly discusses results of
%% research on light verb constructions. \citet{BBRBWA2009a} examined lightverb constructions
%% entertaining the hypotheis of \citet{Butt2010a} that the lightverb is underspecified with regard to
%% its meaning and that the argument roles are (partly) contributed by the non-verbal element (see also
%% Section~\ref{sec-psycho-lv}). This is in contrast with a view by \citet{Goldberg2003a} in which the
%% lightverb construction is a phrasal construction into which normal verbs are inserted and which
%% contributes the meaning.

%% The authors of the study examined MGG data of verbs that could be used in both light/non-light
%% constructions and verbs that are unambiguously non-light verbs. They examined the verbs in
%% isolation, in minimal context and in sentence context. They found that there are different
%% activations for light verbs and heavy verbs without any context (p.\,177). They interpreted their
%% findings as support for the lexical analysis suggested by \citet{Butt2010a}. This is also the
%% analysis that we assume \citep{MuellerPersian}.





So, while I think that it is impossible to distinguish phrasal and lexical approaches for phenomena
where heads are used with different valence patterns (Section~\ref{sec-lr-phrasal-psycho}), there seem to be ways to test whether patterns
with high frequency and strong collocations should be analyzed as one fixed chunk of material with a
fixed form and a fixed meaning or whether they should be analyzed compositionally.




\section{Arguments from statistical distribution}
\label{stat-sec}

In this section, we want to look at arguments from statistics that have been claimed to support a phrasal
view.  We first look at data-oriented parsing, a technique that was successfully used by
\citet{Bod2009a} to model language acquisition and then we turn to the collostructional analysis by
\citet{SG2009a}.  Lastly we argue that these distributional analyses cannot decide the question
whether argument structure constructions are phrasal or lexical.

\subsection{Unsupervised Data-Oriented Parsing}
\label{Abschnitt-U-Dop-phrasal}


In\is{statistics|(}\is{Unsupervised Data-Oriented Parsing (U-DOP)|(} Section~\ref{Abschnitt-UDOP}, we saw Rens Bod's approach to the structuring of natural language utterances.
If one assumes that language is acquired from the input without innate knowledge, the structure that Bod extracts from the distribution of words would have to be the ones
that children also learn (parts of speech, meaning and context would also have to be included).
These structures would then also have to be the ones assumed in linguistic theories. Since Bod does not have enough data, he carried out experiments under the assumption of binary"=branching
trees and for this reason, it is not possible to draw any conclusions from his work about whether rules license flat or binary"=branching structures.
There will almost certainly be interesting answers to this question in the future. What can certainly not be determined in a distribution"=based analysis is the exact node in the tree
where meaning is introduced. \citet[\page 132]{Bod2009b} claims that his approach constitutes ``a testable realization of CxG'' in the Goldbergian sense, but the trees that he can construct do not help us to decide between
phrasal or lexical analyses or analyses with empty heads\is{empty element}. These alternative analyses are represented in Figure~\vref{Abbildung-DOP-Resultatives}.\footnote{
The discussion is perhaps easier to follow if one assumes flat structures rather than binary"=branching ones.\\

\raisebox{2\baselineskip}{\begin{forest}
[X
       [X [er] ]
       [X [ihn] ]
       [X [leer] ]
       [X [fischt] ]
]
\end{forest}}\hfill
\begin{forest}
[X
       [X [er] ]
       [X [ihn] ]
       [X [leer] ]
       [X [X [fischt] ] ]
]
\end{forest}
\hfill
\begin{forest}
[X
       [X [er] ]
       [X [ihn] ]
       [X [leer] ]
       [X [X [fischt] ] 
           [X [\trace{}] ]]
]
\end{forest}

\noindent
The first figure corresponds to the Goldbergian view of phrasal constructions where the verb is inserted into the construction
and the meaning is present at the top"=most node. In the second figure, there is a lexical rule that provides the resultative semantics
and the corresponding valence information. In the third analysis, there is an empty head that combines with the verb and has ultimately
the same effect as the lexical rule.
}
\begin{figure}
\hfill
\begin{forest}
sn edges
[X
	[X
		[er]]
	[X
		[X
			[ihn]]
		[X
			[X
				[leer]]
			[X
				[fischt]]]]]
\end{forest}
\hfill
\begin{forest}
sn edges
[X
	[X
		[er]]
	[X
		[X
			[ihn]]
		[X
			[X
				[leer]]
			[X
				[X
					[fischt]]]]]]
\end{forest}
\hfill
\begin{forest}
sn edges
[X
	[X
		[er]]
	[X
		[X
			[ihn]]
		[X
			[X
				[leer]]
			[X
				[X
					[fischt]]
				[X
					[\trace]]]]]]
\end{forest}
%
\hfill\mbox{}
\caption{\label{Abbildung-DOP-Resultatives}Three possible analyses for resultative construction: holistic construction,
lexical rule, empty head}
\end{figure}%
The first figure stands for a complex construction that contributes the meaning as a whole. The second figure corresponds to the analysis
with a lexical rule and the third corresponds to the analysis with an empty head. A distributional analysis cannot decide between these theoretical
proposals.
Distribution is computed with reference to words, what the words actually mean is not taken into account. As such, it is only possible to say
that the word \emph{fischt} `fishes' occurs in a particular utterance, however it is not possible to see if this word contains resultative semantics or not. 
Similarly, a distribution analysis does not help one to distinguish between analyses with or without a lexical head.
The empty head is not perceptible in the signal. It is a theoretical construct and, as we have seen in Section~\ref{Abschnitt-leere-Elemente-LRs-Transformations},
it is possible to translate an analysis using an empty head into one with a lexical rule. For the present example, any argumentation for a particular analysis will
be purely theory"=internal.

Although U"=DOP cannot help us to decide between analyses, there are areas of grammar for which these structures are of interest: under the assumption of
binary"=branching structures, there are different branching possibilities depending on whether one assumes an analysis with verb movement or not. This means that
although one does not see an empty element in the input, there is a reflex in statistically"=derived trees. The left tree in
Figure~\vref{Abbildung-DOP-Verbbewegung} shows a structure that one would expect from an analysis following
\citew[\page 159]{Steedman2000a-u} (see Section~\ref{sec-Verbstellung-CG-Steedman}). The tree on the right shows a structure that would be expected
from a GB"=type verb movement analysis (see Section~\ref{Abschnitt-Verbstellung-GB}). 
\begin{figure}
\hfill%
\begin{forest}
sn edges
[X
	[X
		[X
			[kennt]]
		[X
			[er]]]
	[X
		[ihn]]]
\end{forest}
\hfill
\begin{forest}
sn edges
[X
	[X
		[kennt]]
	[X
		[X
			[er]]
		[X
			[ihn]]]]
\end{forest}
\hfill\mbox{}
\caption{\label{Abbildung-DOP-Verbbewegung}Structures corresponding to analysis with or without verb movement}
\end{figure}% 
But at present, there is no clear finding in this regard (Bod, p.\,c.\ 2009).\todostefan{ask again} There is a great deal of variance in the U"=DOP trees.
The structure assigned to an utterance depends on the verb (Bod referring to the Wall Street Journal). 
Here, it would be interesting to see if this changes with a larger data sample.
In any case, it would be interesting to look at how all verbs as well as particular verb classes behave. The U"=DOP procedure
applies to trees containing at least one word each. If one makes use of parts of speech in addition, the result is structures that correspond to
the ones we have seen in the preceding chapters.
Sub"=trees would then not have two Xs as their daughters but rather NP and V, for example. It is
also possible to do statistic work with these kind of subtrees and use the part of speech symbols of
words (the preterminal symbols) rather than the words themselves in the computation. For example, one would get trees for the symbol V instead of many trees for
specific verbs. So instead of having three different trees for \emph{küssen} `kiss', \emph{kennen} `know' and
\emph{sehen} `see', one would have three identical trees for the part of speech verb that corresponds to the
trees that are needed for transitive verbs. The probability of the V tree therefore is higher than
the probabilities of the trees for the individual verbs. Therefore one would have a better set of data
to compute structures for utterances such as those in Figure~\ref{Abbildung-DOP-Verbbewegung}. 
I believe that there are further results in this area to be found in years to come.\is{Unsupervised Data-Oriented Parsing (U-DOP)|)}

Concluding this subsection, we contend that Bod's paper is a milestone in the Poverty of the
Stimulus debate, but it does not and cannot show that a particular version of constructionist
theories, namely the phrasal one, is correct.


\subsection{Collostructions}


\mbox{}\citet[Section~5]{SG2009a} assume a plugging analysis: ``words
  occur in (slots provided by) a given   construction if their meaning matches that of the
  construction''. The authors claim that their \emph{collostructional analysis has confirmed}
  [\emph{the plugging analysis}] \emph{from various perspectives}. Stefanowitsch and Gries are able to show that certain verbs occur more often
than not in particular constructions, while other verbs never occur in the respective
constructions. For instance, \emph{give}, \emph{tell}, \emph{send}, \emph{offer} and \emph{show} are
attracted by the Ditransitive Construction, while \emph{make} and \emph{do} are repelled by this
construction, that is they occur significantly less often in this construction than what would be
expected given the overall frequency of verbs in the corpus. Regarding this distribution the authors write:
\begin{quote}
  These results are typical for collexeme analysis in that they show two things. First, there are
  indeed significant associations between lexical items and grammatical structures. Second, these
  associations provide clear evidence for semantic coherence: the strongly attracted collexemes all
  involve a notion of ‘transfer’, either literally or metaphorically, which is the meaning typically
  posited for the ditransitive. This kind of result is typical enough to warrant a general claim
  that collostructional analysis can in fact be used to identify the meaning of a grammatical
  construction in the first place. \citep[\page 943]{SG2009a}
\end{quote}

% SW: many 'latinate' verbs of transfer like donate, distribute, contribute, etc.-- do not occur in the double object construction.  Stefanowitsch and Gries miss this fact since they only look at the verbs that DO occur. (this is true even when they compare NP-NP to NP-PPto (Table 43.12): they only look at verbs that occur in both).  So their data do not support CG as they define it:
% Stefanowitsch and Gries, p. 941:  "According to Construction Grammar, this relationship 
% [between lexis and grammatical structure] is determined by semantic compatibility: words occur in 
% (slots provided by) a given construction if their meaning matches that of the construction."  
% This 'strong version' of CG is a myth.  Adele knows it doesn't work, as she noted already in
% 1995.  You have to stipulate which words go in which constructions.  
% p. 946:  "...clear evidence for the associations between words and constructions and for semantic compatibility as the main principle governing these associations." This is vaguer: "the main principle" rather than the conditional implication stated above.

\noindent
We hope that the preceding discussion made clear that the distribution of words in a corpus cannot
be seen as evidence for a phrasal analysis. The corpus study shows that \emph{give} usually is used
with three arguments in a certain pattern that is typical for English (Subject Verb Object1 Object2)
and that this verb forms a cluster with other verbs that have a transfer component in their meaning.
The corpus data do not show whether this meaning is contributed by a phrasal pattern or by lexical
entries that are used in a certain configuration.



\section{Conclusion}


%We have shown in this paper that there are no compelling arguments for assuming phrasal argument structure
%constructions, but that there are several arguments against them. Assuming a lexical or\,--\,in the
%terminology of \citet{Goldberg2013a}\,--\,template-based view solves all the problems that arise for
%phrasal approaches.
%
%Furthermore we showed that radically underspecified approaches in the sense of \citet{Borer2005a-u},
%\citet{Haugereid2007a}, and \citet{Lohndal2012a} are not restricted enough. The only way to establish
%the necessary restrictions is a lexical representation, since the information that has to be
%captured is in part lexeme dependent.


The essence of the lexical view is that a verb is stored with a valence structure indicating how it
combines semantically and syntactically with its dependents.  Crucially, that structure is
abstracted from the actual syntactic context of particular tokens of the verb.  Once abstracted,
that valence structure can meet other fates besides licensing the phrasal structure that it most
directly encodes: it can undergo lexical rules that manipulate that structure in systematic ways; it
can be composed with the valence structure of another predicate; it can be coordinated with similar
verbs; and so on.  Such an abstraction allows for simple explanations of a wide range of
robust, complex linguistic phenomena.  We have surveyed the arguments against the lexical valence
approach, and in favor of a phrasal representation instead.  We find the case for a phrasal
representation of argument structure to be unconvincing: there are no compelling arguments in favor
of such approaches, and they introduce a number of problems:
\begin{itemize}
\item They offer no account for the interaction of valence changing processes and derivational morphology.
\item They offer no account for the interaction of valence changing processes and coordination of words.
\item They offer no account for the iteration of valence changing processes.
\item They overgenerate, unless a link between lexical items and phrasal constructions is assumed.
\item They offer no account for distribution of arguments in partial fronting examples.
\end{itemize}
Assuming a lexical valence structure
allows us to solve all the problems that arise with phrasal approaches.



\section{Why (phrasal) constructions?}
\label{Abschnitt-Phrasale-Konstruktionen}\label{sec-why-phrasal}

In previous sections, I have argued against assuming too much phrasality in grammatical descriptions.
If one wishes to avoid transformations in order to derive alternative patterns from a single base structure, while still maintaining lexical integrity,
then phrasal analyses become untenable for analyzing all those phenomena where changes in valence and derivational morphology interact. There are, however, some areas
in which these two do not interact. In these cases, there is mostly a choice between analyses with silent heads and those with phrasal constructions. In this section,
I will discuss some of these cases.

\subsection{Verbless directives}
\label{Abschnitt-Phrasale-Konstruktionen-Jacobs}

\citet{Jacobs2008a} showed that there are linguistic phenomena where it does not make sense to assume that there is a head
in a particular group of words. These configurations are best described as phrasal constructions, in which the adjacency of particular
constituents leads to a complete meaning that goes beyond the sum of its parts. Examples of the
phenomena that are discussed by Jacobs are phrasal templates such as those in (\mex{1})
and verbless directives\is{directive} as in (\ref{Beispiel-Direktiva}):
\begin{exe}
%\ex Pro\sub{1/2pers} N                          \jambox{Ich Idiot!, Du Armer!, \ldots}
%% \begin{tabular}[t]{@{}l@{~}ll@{}}
%% a. & Pro\sub{+w,kaus/fin} NP      & Wozu Konstruktionen?, Warum ich?, \ldots\\
%%    &                              & `Why constructions?, Why me?'\\
%% b. & NP\sub{akk} Y\sub{PP/A/Adv}  & Den Hut in der Hand (kam er ins Zimmer).\\
%%    &                              & `(he came into the room) hat in hand'\\
%% \end{tabular}
\ex Pro\sub{+w,caus/purp} NP
\begin{xlist}
\ex
\gll  Wozu Konstruktionen?\\
      why constructions\\
\glt `Why constructions?'
\ex 
\gll Warum ich?\\
     why I.\nom\\
\glt `Why me?'
\end{xlist}
\end{exe}
\ea
NP\sub{acc} Y\sub{PP/A/Adv}\\
\gll Den Hut in der Hand (kam er ins Zimmer).\\
     the hat in the hand \hspaceThis{(}came he into.the room\\
\glt `(He came into the room) hat in hand.'

\z
In (\mex{-1}), we are dealing with abbreviated questions:
\eal
\ex 
\gll Wozu braucht man Konstruktionen? / Wozu sollte man Konstruktionen annehmen?\\
     to.what needs one constructions {} to.what should one constructions assume\\
\glt `Why do we need constructions?' / `Why should we assume constructions?'
\ex 
\gll Warum soll ich das machen? / Warum wurde ich ausgewählt? / Warum passiert mir sowas?\\
	 why should I that do {} why was I chosen {} why happens me something.like.that\\
\glt `Why should I do that?' / `Why was I chosen?' / `Why do things like that happen to me?'
\zl
In (\mex{-1}), a participle has been omitted:
\ea
\gll Den Hut in der Hand haltend kam er ins Zimmer.\\
	 the hat.\acc{} in the hand holding came he in.the room\\
\glt `He came into the room hat in hand.'
\z
Cases such as (\mex{-2}) can be analyzed with an empty head\is{empty head} that corresponds to \emph{haltend} `holding'.
For (\mex{-3}), on the other hand, one would require either a syntactic structure with multiple empty elements, or an empty head that
selects both parts of the construction and contributes the components of meaning that are present in (\mex{-1}).
If one adopts the first approach with multiple silent elements, then one would have to explain why these elements cannot occur in other
constructions. For example, it would be necessary to assume an empty element corresponding to
\emph{man} `one'/""`you'. But such an empty element could never occur in embedded clauses since subjects cannot simply be omitted there:
\ea[*]{
\gll weil dieses Buch gerne liest\\
	 because this book gladly reads\\
\glt Intended: `because he/she/it likes to read this book'
}
\z
If one were to follow the second approach, one would be forced to assume an empty head with particularly odd semantics.

The directives in (\mex{1}) and (\mex{2}) are similarly problematic (see also \citew[\page
  220]{JP2005a-u} for parallel examples in English\il{English}):
\eal
\label{Beispiel-Direktiva}
\ex 
\gll Her  mit  dem Geld   / dem gestohlenen Geld!\\
     here with the money {} the stolen money\\
\glt `Hand over the (stolen) money!'
\ex 
\gll Weg  mit  dem Krempel / dem alten Krempel!\\
     away with the junk   {} the old junk\\
\glt `Get rid off this (old) junk!'
\ex 
\gll Nieder mit den Studiengebühren / den sozialfeindlichen Studiengebühren!\\
     down with the tuition.fees  {} the antisocial tuition.fees\\
\glt `Down with (antisocial) tuition fees!'
\zl
\eal
\ex 
\gll In den Müll mit diesen Klamotten.\\
     in the trash with these clothes\\
\glt `Throw these clothes in the trash!'
\ex 
\gll Zur Hölle mit dieser Regierung.\\
	 to.the hell with this government\\
\glt `To hell with this government!'
\zl
Here, it is also not possible to simply identify an
elided\is{ellipsis} verb. It is, of course, possible to assume an empty head that selects an adverb or a 
\emph{mit}-PP, but this would be \emph{ad hoc}.
% Dann ist es ja nicht schlimm:
% (und ansonsten äquivalent zur phrasalen Analyse, siehe Abschnitt~\ref). 
Alternatively, it would be possible to assume that adverbs in (\mex{-1}) select the \emph{mit}-PP. Here, one would have to disregard the fact that adverbs
do not normally take any arguments. The same is true of Jacobs' examples in (\mex{0}). For these,
one would have to assume that \emph{in} and \emph{zur} `to the' are the respective heads. Each of
the prepositions would then have to select a noun phrase and a \emph{mit}-PP. While this is technically possible, it is as unattractive
as the multiple lexical entries that Categorial Grammar has to assume for pied"=piping constructions (see Section~\ref{Abschnitt-Relativsaetze-CG}). 

A considerably more complicated analysis has been proposed by G.\ \citet{GMueller2009a}. Müller treats verbless directives as antipassive constructions\is{antipassive|(}. 
Antipassive constructions involve either the complete suppression of the direct object or its realization as an oblique element (PP). There
can also be morphological marking on the verb. The subject is normally not affected by the antipassive but can, however, receive a different case
in ergative case systems due to changes in the realization of the object. According to
G.\ Müller\aimention{Gereon M{\"u}ller}, there is a relation between (\mex{1}a) and (\mex{1}b) that is similar to active"=passive pairs:

\eal
\ex 
\gll {}[dass] jemand diese Klamotten in den Müll schmeißt\\
     {}\spacebr{}that somebody these clothes in the trash throws\\
\glt `that somebody throws these clothes into the thrash'     
\ex\label{in-den-Muell-mit} 
\gll In den Müll mit diesen Klamotten!\\
in the rubbish with these clothes\\
\glt `Throw these clothes into the garbage!'
\zl
An empty passive morpheme absorbs the capability of the verb to assign accusative (see also Section~\ref{Abschnitt-GB-Passiv} 
on the analysis of the passive in \gbt). The object therefore has to be realized as a PP or not at all. It follows from Burzio's
Generalization\is{Burzio's Generalization} that as the accusative object has been suppressed, there cannot be an external argument.
G.\,Müller\aimention{Gereon M{\"u}ller} assumes, like proponents of Distributed Morphology\is{Distributed Morphology} (\eg \citealp{Marantz97a})\todostefan{Halle
  Marantz 93/94}, that lexical entries are inserted into complete trees post"=syntactically. The antipassive morpheme creates a feature bundle in the relevant
  tree node that is not compatible with German verbs such as \emph{schmeißen} `throw' and this is why only a null verb with the corresponding specifications can be
  inserted. Movement of the directional PP is triggered by mechanisms that cannot be discussed further here. The antipassive morpheme forces an obligatory
reordering of the verb\is{verb position} in initial position (to C, see
Section~\ref{Abschnitt-Verbstellung-GB} and Section~\ref{sec-verb-position-MP}). By stipulation, filling the prefield is only possible in sentences where the C position is filled by a visible verb and this is why
G.\,Müller's analysis does only derive V1 clauses. These are interpreted as imperatives or polar questions. Figure~\vref{abb-in-den-Muell-Gereon}
gives the analysis of (\mex{0}b).
\begin{figure}
\centering
\begin{forest}
[CP
	[C
		[v $+$ APASS
			[V, name=v1]
			[v $+$ APASS
				[$\varnothing$, name=zero, tier=word]]]
		[C]]
	[vP
		[PP$_2$
			[in den Müll,tier=word,triangle]]
		[v$'$
			[VP
				[DP$_1$
					[(mit) diesen Klamotten,triangle]]
				[V$'$
					[t$_2$]
					[t$_V$, tier=word]]]
			[v
				[t$_v$, tier=word]]]]]
\draw (v1.south)--(zero.north);
\end{forest}
\caption{\emph{In den Müll mit diesen Klamotten} `in the trash with these clothes' as an antipassive following Gereon Müller (2009)}\label{abb-in-den-Muell-Gereon}
\end{figure}%
%\noindent
\citet{Budde2010a} and \citet{Mache2010a} note that the discussion of the data has neglected the fact that there are also interrogative variants
of the construction:
\eal
\ex 
\gll Wohin mit den Klamotten?\\
	 where.to with the clothes\\
\glt `Where should the clothes go?'
\ex 
\gll Wohin mit dem ganzen Geld?\\
	 where.to with the entire money\\
\glt `Where should all this money go?'
\zl
Since these questions correspond to V2 sentences, one therefore does not require the constraint that the prefield can only be filled if the C position
is filled. 
% Das sind eigene Konstruktionen
%% Such a constraint would be problematic in any case since
%% there are sentences without copula\is{copula} (Müller:
%% \citeyear[\page 73--74]{Mueller2002b}; \citeyear{Mueller2004e}).

One major plus point of this analysis is that it derives the different sentence types that are possible with these kind of constructions:
the V1"=variants correspond to polar questions and imperatives, and the V2"=variants with a question word correspond to \emph{wh}"=questions.
A further consequence of the approach that was pointed out by Gereon Müller\aimention{Gereon M{\"u}ller} is that no further explanation is required for
other interactions with the grammar. For example, the way in which the constructions interact with adverbs follows from the analysis:
{\judgewidth{?*}
\eal
\ex[]{
\gll Schmeiß den Krempel weg!\\
	 throw the junk away\\
}
\ex[]{
\gll Schmeiß den Krempel schnell weg!\\
	 throw the junk quickly away\\
}
\ex[?*]{
\gll Schmeiß den Krempel sorgfältig weg!\\
	 throw the junk carefully away\\
}
\zl
\eal
\ex[]{
\gll Weg mit dem Krempel!\\
	 away with the junk\\
}
\ex[]{
\gll Schnell weg mit dem Krempel!\\
	 quickly away with the junk\\
}
\ex[?*]{
\gll Sorgfältig weg mit dem Krempel!\\
	 carefully away with the junk\\
}
\zl}

\noindent
Nevertheless one should still bear the price of this analysis in mind: it assumes an empty antipassive morpheme that is otherwise not
needed in German. It would only be used in constructions of the kind discussed here. This morpheme is not compatible with
any verb and it also triggers obligatory verb movement, which is something that is not known from any other morpheme that is used
to form verb diatheses.
% Imperativ?

The costs of this analysis are of course less severe if one assumes that humans already have this antipassive morpheme anyway, that is, this morpheme
is part of our innate Universal Grammar\indexug. But if one follows the argumentation from the earlier sections of this chapter, then one should only assume
innate linguistic knowledge if there is no alternative explanation\aimention{Gereon M{\"u}ller}.

G.\ Müller's analysis can be translated into HPSG. The result is given in (\mex{1}):
\ea
\oneline{%
\onems[verb-initial-lr]{
%synsem$|$loc$|$cat$|$subcat \sliste{ [ nonloc$|$inher$|$slash \eliste ] }\\
rels \relliste{ \ms[imperative-or-interrogative]{
                  event & \ibox{2}\\
                  } } $\oplus$ \etag\\[2mm]
lex-dtr  \onems{
          phon \eliste\\
          ss$|$loc \onems{ cat   \onems{ head$|$mod \type{none}\\
                                           subcat     \sliste{ XP[\textsc{mod} \ldots{}  ind \ibox{1}], (PP[\type{mit}]\ind{1}) }\\
                                         }\\
                            cont  \ms{
                                   ind & \ibox{2}\\
                                   rels & \liste{ \ms[directive]{
                                                  event       & \ibox{2}\\
                                                  patient    & \ibox{1}\\
                                                  }\\
                                                 }\\
                                      }\\
                          }\\
}\\
}}
\z
(\mex{0}) contains a lexical entry for an empty verb in verb"=initial position.\is{verb position} \relation{directive}
is a placeholder for a more general relation that should be viewed as supertype of all possible meanings of this
construction. These subsume both \emph{schmeißen} `to throw' and cases such as (\mex{1}) that were pointed out to me by Monika Budde:
\ea
\label{Klavier-durch-die-Tuer}
\gll Und mit dem Klavier ganz langsam durch die Tür!\\
	 and with the piano very slowly through the door\\
\glt `Carry the piano very slowly through the door!'
\z
Since only verb"=initial and verb"=second orders are possible in this construction, the application of the lexical rule for verb"=initial position
(see page~\pageref{lr-verb-movement}) is obligatory. This can be achieved by writing the result of the application of this lexical rule into the lexicon, without
having the object to which the rule should have applied actually being present in the lexicon itself. 
\citet[Section~3.4.2, 5.3]{Koenig99a} proposed something similar for English\il{English} \emph{rumored} `it is rumored that \ldots' and \emph{aggressive}. 
There is no active variant of the verb \emph{rumored}, a fact that can be captured by the assumption that only the result of applying a passive lexical rule
is present in the lexicon. The actual verb or verb stem from which the participle form has been
derived exists only as the daughter of a lexical rule but not as an independent linguistic
object. Similarly, the verb \noword{aggress} only exists as the daughter of a (non"=productive)
adjective rule that licenses \emph{aggressive} and a nominalization rule licensing \emph{aggression}.

The optionality of the \emph{mit}-PP is signaled by the brackets in (\mex{-1}). If one adds the information inherited from the type \type{verb-initial-lr}
under \synsem, then the result is (\mex{1}).
%\vpageref{in-den-muell-lexical}.
%\begin{figure}[hbp]
\ea
\label{in-den-muell-lexical}
\oneline{%
\onems[verb-initial-lr]{
synsem$|$loc \ms{ head & \ms[verb]{vform & fin\\
                                          initial & $+$\\
                                          dsl     & none\\
                                 }\\
                           subcat & \sliste{ \onems{ loc$|$cat \onems{ head  \ms[verb]{
                                                               dsl & \ibox{3}\\
                                                               }\\
                                                         subcat \eliste\\
                                                       }
                                              }}\\
                         }\\
rels \relliste{ \ms[imperative-or-interrogative]{
                  event & \ibox{2}\\
                  } } $\oplus$ \ibox{4}\\[5mm]
lex-dtr  \onems{
          phon \eliste\\
          ss$|$loc \ibox{3} \onems{ cat   \onems{ head$|$mod \type{none}\\
                                           subcat     \sliste{ XP[\textsc{mod} \ldots{}  ind \ibox{1}], (PP[\type{mit}]\ind{1}) }\\
                                         }\\
                            cont  \ms{
                                   ind & \ibox{2}\\
                                   rels & \ibox{4} \liste{ \ms[directive]{
                                                  event       & \ibox{2}\\
                                                  patient    & \ibox{1}\\
                                                  }\\
                                                 }\\
                                      }\\
                          }\\
}\\
}}
\z
%\vspace{-\baselineskip}
%\end{figure}%
%
The valence properties of the empty verb in (\mex{0}) are to a large extent determined by the lexical rule for verb"=initial order: the V1"=LR licenses a verbal head
that requires a VP to its right that is missing a verb with the local properties of the \textsc{lex-dtr} \iboxb{3}.
% Außerdem wird
% von der selegierten VP verlangt, dass sie eine leere \slashl hat. Daraus ergibt sich, dass kein
% Element aus der VP extrahiert werden darf, weshalb mit dem Eintrag in (\mex{0}) ausschließlich
% Verberstsätze abgeleitet werden können. 

Semantic information dependent on sentence type (assertion, imperative or question) is determined inside the V1"=LR depending on the morphological
make"=up of the verb and the \slashv of the selected VP (see Müller
\citeyear[Section~10.3]{MuellerLehrbuch1}; \citeyear{MuellerSatztypen}; \citeyear{MuellerGS}).
Setting the semantics to \type{imperative-or-interrogative}  rules out \emph{assertion} as it occurs in V2"=clauses.
Whether this type is resolved in the direction of \type{imperative} or
\type{interrogative} is ultimately decided by further properties of the utterance such as intonation or the use of interrogative pronouns.

The valence of the lexical daughters in (\mex{0}) as well as the connection to the semantic role (the linking to the patient role) are simply stipulated.
Every approach has to stipulate that an argument of the verb has to be expressed as a \emph{mit}-PP. Since there is no antipassive\is{antipassive|)} in German,
the effect that could be otherwise achieved by an antipassive lexical rule in (\mex{0}) is simply written into the \textsc{lex-dtr} of the verb movement rule.

The \subcatl of \textsc{lex-dtr} contains a modifier (adverb, directional PP) and the 
\emph{mit}-PP. This \emph{mit}-PP is co"=indexed with the patient of \relation{directive} and the modifier refers to the referent of the \emph{mit}-PP. The agent
of \relation{directive} is unspecified since it depends on the context (speaker, hearer, third person).

This analysis is shown in Figure~\vref{verb-movement-muell}.
\begin{figure}
\oneline{%
\begin{forest}
sn edges
[V{[\subcat \sliste{}]}
	[V{[\subcat \sliste{ \ibox{1} [\textsc{head$|$dsl} \ibox{2}] }]}
		[V{[\textsc{loc} \ibox{2} ]}, tier=pp, edge label={node[midway,right]{V1-LR}}
			[\trace]]]
	[\ibox{1} V\feattab{
                        \textsc{head$|$dsl} \ibox{2},\\
                        \subcat \sliste{} }
		[\ibox{3} PP, tier=pp
			[in den Müll;in the garbage,triangle]]
		[V\feattab{
                         \textsc{head$|$dsl} \ibox{2},\\
                         \subcat \sliste{ \ibox{3} }}
			[\ibox{4} PP{[\type{mit}]}
				[mit diesen Klamotten;with these clothes,triangle]]
			[V\ibox{2}\feattab{ \textsc{head$|$dsl} \ibox{2},\\
                                            \subcat \sliste{ \ibox{3}, \ibox{4} }}
				[\trace]]]]]
\end{forest}
}
\caption{\label{verb-movement-muell}HPSG variant of the analysis of \emph{In den Müll mit diesen Klamotten!/?}}
\end{figure}%
Here, V[\textsc{loc} \ibox{2}] corresponds to the \textsc{lex-dtr} in (\mex{0}). The V1-LR licenses an element that requires a maximal verb projection
with that exact \dslv \ibox{2}. Since \dsl is a head feature, the information is present along the head path. The \dslv is identified with the \localv
(\iboxt{2} in Figure~\ref{verb-movement-muell}) in the verb movement trace (see page~\pageref{le-verbspur}). 
This ensures that the empty element at the end of sentence has exactly the same local properties that the \textsc{lex-dtr} in (\mex{0}) has.
Thus, both the correct syntactic and semantic information is present on the verb trace and structure
building involving the verb trace follows the usual principles.
The structures correspond to the structures that were assumed for German sentences in Chapter~\ref{Kapitel-HPSG}.
Therefore, there are the usual possibilities for integrating adjuncts. The correct derivation of the semantics, in particular embedding under
imperative or interrogative semantics, follows automatically (for the semantics of adjuncts in conjunction with verb position, see  \citew[Section~9.4]{MuellerLehrbuch1}). 
Also, the ordering variants with the \emph{mit}-PP preceding the direction (\ref{Klavier-durch-die-Tuer}) and direction preceding the 
\emph{mit}-PP (\ref{in-den-Muell-mit}) follow from the usual mechanisms.

If one rejects the analyses discussed up to this point, then one is only really left with phrasal constructions or dominance schemata that connect parts
of the construction and contribute the relevant semantics. Exactly how one can integrate adjuncts into the phrasal construction in a non"=stipulative way
remains an open question, however, there are already some initial results by Jakob \citet{Mache2010a} that suggest that directives can still be sensibly integrated into
the entire grammar provided an appropriate phrasal schema is assumed.

\subsection{Serial verbs}

There\is{verb!serial|(}\il{Mandarin Chinese|(} are languages with so"=called serial verbs. For example, it is possible to form sentences in Mandarin Chinese where there is only one subject
and several verb phrases. There are multiple readings depending on the distribution of aspect marking inside the VP: if the first VP contains a perfect marker, then we have
the meaning `VP1 in order to do/achieve VP2' (\mex{1}a). If the second VP contains a perfect marker, then the entire construction means `VP2 because VP1' (\mex{1}b) and if the
first VP contains a durative marker and the verb \emph{hold} or \emph{use}, then the entire construction means `VP2 using VP1' (\mex{1}c). 
\eal
\ex
\gll Ta1 qu3 le qian2 qu4 guang1jie1. \\
     he withdraw \textsc{prf} money go shop \\
\glt `He withdrew money to go shopping.'

\ex
\gll Ta1 zhu4 Zhong1guo2 xue2 le Han4yu3. \\
     he  live China learn \textsc{prf} Chinese \\
\glt `He learned Chinese because he lived in China.'

\ex
\gll Ta1 na2 zhe kuai4zi chi1 fan4.\\
     he  take \textsc{dur} chopsticks eat food \\
\glt `He eats with chopsticks.'
\zl
If we consider the sentences, we only see two adjacent VPs. The meanings of the entire sentences, however, contain parts of meaning that go beyond the meaning
of the verb phrases. Depending on the kind of aspect marking, we arrive at different interpretations with regard to the semantic combination of verb phrases.
As can be seen in the translations, English sometimes uses conjunctions in order to express relations between clauses or verb phrases.

There are three possible ways to capture these data:
\begin{enumerate}
\item One could claim that speakers of Chinese simply deduce the relation between the VPs from the context,
\item one could assume that there are empty heads in Chinese corresponding to \emph{because} or \emph{to}, or
\item one could assume a phrasal construction for serial verbs that contributes the correct semantics for the complete
meaning depending on the aspect marking inside the VPs.
\end{enumerate}
The first approach is unsatisfactory because the meaning does not vary arbitrarily. There are grammaticalized conventions that
should be captured by a theory. The second solution has a stipulative character and thus, if one wishes to avoid empty elements, only
the third solution remains. \citet{ML2009a} have presented a corresponding analysis.\is{verb!serial|)}\il{Mandarin Chinese|)}

\subsection{Relative and interrogative clauses}
\label{Abschnitt-Relativ-Interrogativsaetze}

\mbox{}\citet{Sag97a}\is{relative clause|(}\is{interrogative clause|(}
develops a phrasal analysis of relative clauses as have \citet{GSag2000a-u} for interrogative clauses.
Relative and interrogative clauses consist of a fronted phrase and a clause or a verb phrase missing the fronted phrase.
The fronted phrase contains a relative or interrogative pronoun.
\eal
\ex the man [who] sleeps
\ex the man [who] we know
\ex the man [whose mother] visited Kim
\ex a house [in which] to live
\zl
\eal
\ex I wonder [who] you know.
\ex I want to know [why] you did this.
\zl
The GB analysis of relative clauses is given in Figure~\ref{Abbildung-GB-Relativsatz}.
In this analysis, an empty head\is{empty element} is in the C position and an element from the IP is moved
to the specifier position.%
\begin{figure}
\centering
\begin{forest}
sn edges, for tree={fit=rectangle}
[CP{[\type{rel}]}
	[NP
		[whose remarks,triangle]]
	[\cbar{[\type{rel}]}
		[\cnull{[\type{rel}]}
			[\trace]]
		[IP,l sep+=\baselineskip
			[they seemed to want to object to,triangle]]]]
\end{forest}
\caption{\label{Abbildung-GB-Relativsatz}Analysis of relative clauses in \gbt}
\end{figure}%

%\noindent
The alternative analysis shown in Figure~\vref{Abbildung-HPSG-Relativsatz} involves combining the subparts directly
in order to form a relative clause.
\begin{figure}
\begin{forest}
sn edges, for tree={fit=rectangle}
[S{[\type{rel}]}
	[NP
		[whose remarks,triangle]]
	[S,l sep+=\baselineskip
		[they seemed to want to object to,triangle]]]
\end{forest}
\caption{\label{Abbildung-HPSG-Relativsatz}Analysis of relative clauses in HPSG following \citew{Sag97a}}
\end{figure}%
\citet{Borsley2006a} has shown that one would require six empty heads in order to capture the various relative clauses possible in English, if one would
want to analyze them lexically. These heads can be avoided and replaced by corresponding schemata
(see Chapter~\ref{chap-empty} on empty elements). A parallel argument can also be found in \citet{Webelhuth2011a}
for German: grammars of German would also have to assume six empty heads for the relevant types of relative clause.%
\nocite{Borsley2007a}
% Generalisierungen über verschieden Relativsatzkonstruktionen kann man in Vererbungshierarchien
% erfassen. Natürlich könnte man genauso die Generalisierungen in Bezug auf die Eigenschaften der
% leeren Köpfe in Vererbungshierarchien erfassen.

Unlike the resultative constructions that were already discussed, there is no variability among interrogative and relative clauses with regard to the order of
their parts. There are no changes in valence\is{valence!change} and no interaction with derivational morphology\is{morphology}. Thus, nothing speaks against a phrasal
analysis.
% Aber spricht auch etwas dafür? Sag weißt auf folgende Daten aus
% dem Koreanischen\il{Koreanisch} hin. Im Beispiel (\mex{1}b) kommt das Verb \emph{legen} in einem
% Relativsatz vor. Es ist besonders flektiert, \dash, es ist für die Verwendung in Relativsätzen
% ausgezeichnet. 
% \eal
% \ex 
% \gll John-i chayk-ul ku sangca-ey neh-ess-ta.\\
%      John-nom Buch-acc die Schachtel-loc legen-past-decl\\
% \glt `John legte das Buch in die Schachtel.'
% \ex 
% \gll {}[[John-i chayk-ul neh-un] sangca-ka] khu-ta.\\
%        \hspaceThis{[[}John-nom Buch-acc legen-rel Schachtel-nom groß-decl\\
% \glt `Die Schachtel, in die John das Buch gelegt hat, ist groß.'
% \zl
% % Wenn man das komplett parallel machen wollte, müsste man für das Englische eine disjunktive
% % Spezifikation von MOD-Werten und entsprechendem semantischen Beitrag annehmen. In einem Fall
% % handelt es sich um das normale Verb und im anderen Fall um des Relativsatz-Verb mit nominaler Semantik.
% Allgemein gilt, dass immer das höchste Verb des Relativsatzes flektiert wird. Das kann man gut
% erklären, wenn man annimmt, dass dieses Verb der Kopf des Relativsatzes ist.
If one wishes to avoid the assumption of empty heads, then one should opt for the analysis of relative clauses 
by Sag, or the variant in Müller (\citeyear[Chapter~10]{Mueller99a}; \citeyear[Chapter~11]{MuellerLehrbuch1}). The latter analysis does without a special schema
for noun"=relative clause combinations since the semantic content of the relative clause is provided by the relative clause
schema.%

\citet{Sag2010b} discusses long"=distance dependencies in English\il{English} that are subsumed
under the term \emph{wh}"=movement in \gbt and the MP\indexmp. He shows that this is by no means a
uniform phenomenon.  He investigates \emph{wh}"=questions (\mex{1}),
\emph{wh}"=exclamatives\is{wh-exclamative@\emph{wh}"=exclamative} (\mex{2}),
topicalization\is{topicalization} (\mex{3}), \emph{wh}"=relative clauses\is{relative clause}
(\mex{4}) and \emph{the}"=clauses\is{the-clause@\emph{the}-clause} (\mex{5}):
\eal
\ex How foolish is he?
\ex I wonder \emph{how foolish he is}.
\zl

\eal
\ex What a fool he is!
\ex It's amazing \emph{how odd it is}.
\zl
\ea
The bagels, I like.
\z
\eal
\ex I met the person \emph{who they nominated}.
\ex I'm looking for a bank \emph{in which to place my trust}.
\zl
\eal
\ex The more people I met, \emph{the happier I became}.
\ex \emph{The more people I met}, the happier I became.
\zl
These individual constructions vary in many respects. Sag lists the following questions that have to be answered
for each construction:
\begin{itemize}
\item Is there a special \emph{wh}"=element in the filler daughter and, if so, what kind of element is it?
\item Which syntactic categories can the filler daughters have?
%\item Welche syntaktischen Kategorien kann die Kopf"|tochter haben?
\item Can the head"=daughter be inverted or finite? Is this obligatory?
\item What is the semantic and/or syntactic category of the mother node?
\item What is the semantic and/or syntactic category of the head"=daughter?
\item Is the sentence an island? Does it have to be an independent clause?
\end{itemize}
The variation that exists in this domain has to somehow be captured by a theory of grammar. Sag develops an analysis with multiple schemata
that ensure that the category and semantic contribution of the mother node correspond to the properties of both daughters. The constraints for both classes of
constructions and specific constructions are represented in an inheritance hierarchy\is{inheritance} so that the similarities between the constructions can be
accounted for. The analysis can of course also be formulated in a GB"=style using empty heads\is{empty head}. One would then have to find some way of capturing
the generalizations pertaining to the construction. This is possible if one represents the constraints on empty heads in an inheritance hierarchy. Then, the approaches
would simply be notational variants of one another. If one wishes to avoid empty elements in the grammar, then the phrasal approach would be preferable.
\is{relative clause|)}\is{interrogative clause|)}

\subsection{The N-P-N construction}
\label{Abschnitt-NPN-Konstruktion}

\mbox{}\citet{Jackendoff2008a}\is{construction!N-P-N|(} discusses the English N-P-N construction. Examples of this construction are given in (\mex{1}):
\eal
\ex day by day, paragraph by paragraph, country by country
\ex dollar for dollar, student for student, point for point
\ex face to face, bumper to bumper
\ex term paper after term paper, picture after picture
\ex book upon book, argument upon argument
\zl
This construction is relatively restricted: articles and plural nouns are not allowed. The phonological content of the first noun has to correspond to that
of the second. There are also similar constructions in German:
\eal
% Und du läufst Rüssel an Schwanz hinterher
%\ex Sie lagen Gesicht an Gesicht im Bett.
\ex 
\gll Er hat Buch um Buch verschlungen.\\
	 he has book around book swallowed\\
\glt `He binge-read book after book.'
\ex 
\gll Zeile für Zeile\footnotemark\\
	 line for line\\
\glt `line by line'
\footnotetext{
  \emph{Zwölf Städte}. Einstürzende Neubauten. Fünf auf der nach oben offenen Richterskala, 1987.
}
\zl
Determining the meaning contribution of this kind of N-P-N construction is by no means trivial. Jackendoff suggests the meaning
\emph{many Xs in succession} as an approximation.

Jackendoff points out that this construction is problematic from a syntactic perspective since it is not straightforwardly possible
to determine a head. It is also not clear what the structure of the remaining material is if one is working under assumptions of
\xbart. If the preposition \emph{um} were the head in (\mex{0}a), then one would expect that it is combined with an NP, however this is not possible:
\eal
\ex[*]{
\gll Er hat dieses Buch um jenes Buch verschlungen.\\
	 he has this book around this book swallowed\\
} 
\ex[*]{
\gll Er hat ein Buch um ein Buch verschlungen.\\
	 he has a book around a book swallowed\\
}
\zl
% Sie lagen sein Gesicht an ihrem Gesicht im Bett.
For these kind of structures, it would be necessary to assume that a preposition selects a noun to its right and, if it find this, it then requires
a second noun of this exact form to its left. For N-\emph{um}-N and N-\emph{für}-N, it is not entirely clear  what the entire construction has to do with
the individual prepositions. It could also try to develop a lexical analysis for this phenomenon, but the facts are different to those for resultative constructions:
in resultative constructions, the semantics of simplex verbs plays a clear role. Furthermore, unlike with the resultative construction, the order of the component
parts of the construction are fixed in the N-P-N construction. It is not possible to extract a noun or place the preposition in front of both nouns. Syntactically,
the N-P-N combination with some prepositions behaves like an NP \citep[\page 9]{Jackendoff2008a}:
\ea
Student after/upon/*by student flunked.
\z
This is also strange if one wishes to view the preposition as the head of the construction.

Instead of a lexical analysis, Jackendoff proposes the following phrasal construction for N-\emph{after}-N combinations:
\ea
\begin{tabular}[t]{@{}ll@{}}
Meaning: & MANY X$_i$s IN SUCCESSION [or however it is encoded]\\
Syntax:  & [\sub{NP} N$_i$ P$_j$ N$_i$]\\
Phonology: & Wd$_i$ after$_j$ Wd$_i$\\
\end{tabular}
\z
% Nimmt man die Merkmalsgeometrie von \citet{ps2} an, so kann man eine lexikalische Analyse nicht
% formulieren, da Köpfe die phonologischen Eigenschaften ihrer Argumente nicht selegieren
% können.\footnote{
%   Eine lexikalische Analyse wird doch möglich, wenn m
%
The entire meaning as well as the fact that the N-P-N has the syntactic properties of an NP would be captured on the construction level.

I already discussed examples by \citet{Bragmann2015a} in Section~\ref{sec-headless-constructions-dg}
that show that N-P-N constructions may be extended by further P-N combinations:
\ea
Day after day after day went by, but I never found the courage to talk to her.
\z
So rather than an N-P-N pattern Bragmann suggests the pattern in (\mex{1}), where `+'\is{$+$} stands for at
least one repetition of a sequence.
\ea
N (P N)+
\z
As was pointed out on page~\pageref{n-p-n-plus-cx} this pattern is not easy to cover in
selection"=based approaches. One could assume that an N takes arbitrarily many P-N combinations,
which would be very unusual for heads. Alternatively, one could assume recursion, so N would be
combined with a P and with an N-P-N to yield N-P-N-P-N. But such an analysis would make it
really difficult to enforce the restrictions regarding the identity of the nouns in the complete
construction. In order to enforce such an identity the N that is combined with N-P-N would have to
impose constraints regarding deeply embedded nouns inside the embedded N-P-N object (see also Section~\ref{sec-locality}).

G.\ \citet{GMueller2011a} proposed a lexical analysis of the N-P-N construction. He assumes that prepositions can have a feature \textsc{redup}.
In the analysis of \emph{Buch um Buch} `book after book', the preposition is combined with the right noun \emph{um Buch}. In the phonological component, reduplication of \emph{Buch} is triggered
  by the \textsc{redup} feature, thereby yielding \emph{Buch um Buch}.
This analysis also suffers from the problems pointed out by Jackendoff: in order to derive the semantics of the construction, the semantics would have to be present
in the lexical entry of the reduplicating preposition (or in a relevant subsequent component that
interprets the syntax).  Furthermore it is unclear how a reduplication analysis would deal with the
Bragmann data.
% Zwar ist es richtig, dass
% Präpositionen im Deutschen auch ohne Artikel verwendet werden können, aber 
\is{construction!N-P-N|)}
\is{Construction Grammar (CxG)|)}





\if 0

Draußen, auf der regennassen Friedrichstraße , quietschten die Straßenbahnen um die Kurve am
Oranienburger Tor.

Olaf Schwarzbach Forelle Grau: Die Geschichte von OL. Berlin: Berlin Verlag, 2015, S.287.

\fi
%      <!-- Local IspellDict: en_US-w_accents -->
