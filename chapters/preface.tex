\chapter*{Preface}

This book is an extended and revised version of my German book \emph{Grammatiktheorie}
\citep{MuellerGTBuch2}. It introduces various grammatical theories that play a role in current
theorizing or have made contributions in the past which are still relevant today. I explain some foundational
assumptions and then apply the respective theories to what can be called the ``core grammar'' of
German. I have decided to stick to the object language that I used in the German version of this
book since many of the phenomena that will be dealt with cannot be explained with English as the object
language. Furthermore, many theories have been developed by researchers with English as their native
language and it is illuminative to see these theories applied to another language.
I show how the theories under consideration deal with arguments and adjuncts, active/passive
alternations, local reorderings (so"=called scrambling), verb position, and fronting of phrases over
larger distances (the verb second property of the Germanic languages without English).

The second part deals with foundational questions that are important for developing theories.
This includes a discussion of the question of whether we have innate domain specific knowledge of
language (UG), the discussion of psycholinguistic evidence concerning the processing of language by
humans, a discussion of the status of empty elements and of the question whether we construct and perceive utterances 
holistically or rather compositionally, that is, whether we use phrasal or lexical constructions.

Unfortunately, linguistics is a scientific field 
with a considerable amount of terminological chaos. I therefore wrote an introductory
chapter that introduces terminology in the way it is used later on in the book. The second chapter
introduces phrase structure grammars, which plays a role for many of the theories that are covered
in this book. I use these two chapters (excluding the Section~\ref{sec-PSG-Semantik} on interleaving
phrase structure grammars and semantics) in introductory courses of our BA curriculum for German
studies. Advanced readers may skip these introductory chapters. The following chapters are
structured in a way that should make it possible to understand the introduction of the theories
without any prior knowledge. The sections regarding new developments and classification are more
ambitious: they refer to chapters still to come and also point to other publications that are
relevant in the current theoretical discussion but cannot be repeated or summarized in this
book. These parts of the book address advanced students and researchers. I use this book for teaching
the syntactic aspects of the theories in a seminar for advanced students in our BA. The slides are
available on my web page. The second part of the book, the general discussion, is more ambitious and contains the discussion
of advanced topics and current research literature.

This book only deals with relatively recent developments. For a historical overview, see for instance
\citew{Robins97a-u,JL2006a-u}. I am aware of the fact that chapters on
Integrational Linguistics\is{Integrational Linguistics}
\citep{Lieb83a-u,Eisenberg2004a,Nolda2007a-u}, Optimality Theory\indexot (\citealp{PS93a-u};
\citealp{Grimshaw97a-u}; G.\ \citealp{GMueller2000a-u}), Role and Reference Grammar\is{Role and
  Reference Grammar} \citep{vanValin93a-ed} and Relational Grammar\is{Relational Grammar}
\citep{Perlmutter83a-ed,Perlmutter84b-ed} are missing. I will leave these theories for later editions.

The original German book was planned to have 400 pages, but it finally was much bigger: the first
German edition has 525 pages and the second German edition has 564 pages. I
added a chapter on Dependency Grammar and one on Minimalism to the English version and now the
book has \pageref{LastPage} pages. I tried to represent the chosen theories appropriately and to cite all important work. Although the list of
references is over 85 pages long, I was probably not successful.
I apologize for this and any other shortcomings.

%      <!-- Local IspellDict: en_US-w_accents -->