%% -*- coding:utf-8 -*-
\chapter{Extraction, scrambling, and passive: one or several descriptive devices?}
\label{chap-scrambling-extraction-passive}

An anonymous reviewer suggested discussing one issue in which transformational theories differ from
theories like LFG and HPSG. The reviewer claimed that Transformational Grammars use just one tool
for the description of active/passive alternations, scrambling, and extraction, while theories like
LFG and HPSG use different techniques for all three phenomena. If this claim were correct and if
the analyses made correct predictions, the respective GB/Minimalism theories would be better
than their competitors, since the general aim in science
is to develop theories that need a minimal set of assumptions. I already commented on the analysis of
passive in GB in Section~\ref{sec-passive-gb}, but I want to extend this discussion here and include
a Minimalist analysis and one from Dependency Grammar. 

The task of any passive analysis is to explain the difference in argument realization in examples
like (\mex{1}):
\eal
\ex\label{ex-she-beats-him}
She beats him.
\ex\label{ex-he-was-beaten}
He was beaten.
\zl
In these examples about chess, the accusative object of \emph{beat} is realized as the nominative
in (\mex{0}b). In addition, it can be observed that the position of the elements is different: while
\emph{him} is realized postverbally in object position in (\mex{0}a), it is realized preverbally in
(\mex{0}b). In GB this is explained by a movement analysis. It is assumed that the object does not
get case in passive constructions and hence has to move into the subject position where case is
assigned by the finite verb. This analysis is also assumed in Minimalist work as in
David Adger's textbook \citeyearpar{Adger2003a}, for instance. Figure~\vref{fig-passive-mp} shows his analysis
of (\mex{1}):
\ea
Jason was killed.
\z
\begin{figure}
\centerfit{
\begin{forest}
for tree={fit=rectangle}
[TP
  [Jason]
  [{\tbar[\st{\textit{u}N*}]}
     [{T[past,\st{nom}]}
       [be{[Pass,\st{\textit{u}Infl}:past*]}]
       [{T[past]}]]
     [PassP
       [\phonliste{be}]
       [\vP
         [\textit{v}
           [\textit{kill}]
           [{\textit{v}[\st{\textit{u}Infl}:Pass]}]]
         [VP
           [\phonliste{kill}]
           [\phonliste{Jason}]]]]]]
]
\end{forest}
}
\caption{\label{fig-passive-mp}Adger's Minimalist movement-based analysis of the passive (p.\,231)}
\end{figure}%
TP stands for Tense Phrase and corresponds to the IP that was discussed in
Chapter~\ref{chap-GB}. PassP is a functional head for passives\is{category!functional!PassP}. \vP\is{category!functional!v@\textit{v}} is
a special category for the analysis of verb phrases that was originally introduced for the analysis
of ditransitives \citep{Larson88a}
%; \citealp[\page 315]{Chomsky95a}) auch für transitive Verben
and VP is the normal VP that consists of verb and object. In Adger's analysis, the
verb \emph{kill} moves from the verb position in VP to the head position of \textit{v}, the
passive auxiliary \emph{be} moves from the head position of PassP to the head position of the Tense
Phrase. Features like Infl are `checked' in combination with such movements. The exact implementation
of these checking and valuing operations does not matter here. What is important is that
\emph{Jason} moves from the object position to a position that was formerly known as the specifier
position of T (see Footnote~\ref{fn-Chomsky-on-Specifiers} on
page~\pageref{fn-Chomsky-on-Specifiers} on the notion of specifier). All these analyses assume that
the participle cannot assign accusative to its object and that the object has to move to another
position to get case or check features.
How exactly one can formally represents the fact that the participle cannot assign case is hardly ever made explicit in the GB literature.\todostefan{Baker, Johnson, Roberts 1989 sagen, dass der akkusativ dem pro externen Argument zugewiesen wird}
The following is a list of statements that can be found in the literature:
\eal
\ex We shall assume that a passivized verb loses the ability to assign structural ACCUSATIVE case to
its complement. \citep[\page 183]{Haegeman94a-u}

\ex das Objekt des Aktivsatzes wird zum Subjekt des Passivsatzes, weil die passivische Verbform
keinen Akkusativ"=Kasus regieren kann (Akk"=Kasus"=Absorption). \citep[\page 172]{Lohnstein2014a} 
\zl
In addition, it is sometimes said that the external theta role is absorbed by the verb morphology
(\citealp{Jaeggli86a}; 
%\citealp[]{Roberts87a}; 
\citealp[\page 183]{Haegeman94a-u}). Now, what would it entail if we made this explicit? There is some
lexical item for verbs like \emph{beat}. The active form has the ability to assign accusative to its
object, but the passive form does not. Since this is a property that is shared by all transitive
verbs (by definition of the term transitive verb), this is some regularity that has to be
captured. One way to capture this is the assumption of a special passive morpheme that suppresses
the agent and changes something in the case specification of the stem it attaches too. How this
works in detail was never made explicit.
Let us compare this morpheme"=based analysis with lexical rule"=based analyses: as was explained in
Section~\ref{Abschnitt-leere-Elemente-LRs-Transformations}, empty heads can be used instead of
lexical rules in those cases in which the phonological form of the input and the output do not
differ. So for example, lexical rules that license additional arguments as in
resultative constructions, for instance, can be replaced by an empty head. However, as was explained in Section~\ref{Abschnitt-HPSG-Passiv}, lexical 
rules are also used to model morphology. This is also true for Construction Grammar\indexcxg (see
Gert Booij's work on Construction Morphology \citeyearpar{Booij2010a}, which is in many ways similar to Riehemann's work in
HPSG \citeyearpar{Riehemann93a,Riehemann98a}). In the case of the passive lexical rule, the participle
morphology is combined with the stem and the subject is suppressed in the corresponding valence
list. This is exactly what is described in the GB/MP literature. The respective lexical rule for the
analysis of \emph{ge-lieb-t} `loved' is depicted in Figure~\vref{fig-morpheme-vs-lexical-rule} to
the left.
\begin{figure}
\hfill
\begin{forest}
[{[ \phon \phonliste{ ge } $\oplus$ \ibox{1} $\oplus$ \phonliste{ t } ]}
   [ {[ \phon \ibox{1} ]}   ]]
\end{forest}
\hfill
\begin{forest}
[V
  [V-Aff [ge]]
  [V-Stem]
  [V-Aff [t]]]
\end{forest}
\hfill\mbox{}
\caption{\label{fig-morpheme-vs-lexical-rule}Lexical rule"=based/constructionist
  vs.\ morpheme"=based analysis}
\end{figure}%
The morpheme"=based analysis is shown to the right. To keep things simple, I assume a flat analysis,
but those who insist on binary branching structures would have to come up with a way of deciding
whether the \prefix{ge} or the \suffix{t} is combined first with the stem and in which way selection
and percolation of features takes place. Independent of how morphology is done, the fact  that the inflected form (the top node in both figures) has different properties than the
verb stem has to be represented somehow. In the morpheme"=based world, the morpheme is responsible for suppressing the agent and
changing the case assignment properties, in the lexical rule/construction world this is done by the
respective lexical rule. There is no difference in terms of needed tools and necessary stipulations.

The situation in Minimalist theories is a little bit different. For instance, \citep[\page 229,
  231]{Adger2003a} writes the following:
\begin{quote}
Passives are akin to unaccusatives, in that they do not assign accusative case to their object,
and they do not appear to have a thematic subject. [\ldots] Moreover, the idea that the function of
this auxiliary is to select an unaccusative little \vP simultaneously explains the lack of
accusative case and the lack of a thematic subject. \citep[\page 229, 231]{Adger2003a}  
\end{quote}
So this is an explicit statement. The relation between a stem and a passive participle form that was
assumed in GB analyses is now a verb stem that is combined with two different versions of
\littlev. Which \textit{v} is chosen is determined by the governing head, a functional
Perf\is{category!functional!Perf} head or a Pass\is{category!functional!Pass} head. This can be
depicted as in Figure~\vref{fig-Pass-vs-Perf-and-little-v}.
\begin{figure}
\hfill
\begin{forest}
[\vP
     [DP]
     [\littlevbar
       [\textit{v}{[\st{\textit{u}D}]}]
       [VP
         [\textit{kill} {[V, \st{\textit{u}D}]}]
         [DP ]]]]
\end{forest}
\hfill
\begin{forest}
[\vP
       [\textit{v}]
       [VP
         [\textit{kill} {[V, \st{\textit{u}D}]}]
         [DP ]]]
\end{forest}
\hfill\mbox{}
\caption{\label{fig-Pass-vs-Perf-and-little-v}Analysis of the passive and the perfect and the
  passive in a Minimalist theory involving two different versions of \littlev}
\end{figure}%
When \emph{kill} is used in the perfect or the passive, it is spelled out as \emph{killed}. If it
is used in the active with a 3rd person singular subject it is spelled out as \emph{kills}. This can
be compared with a lexical analysis, for instance the one assumed in HPSG. The analysis
is shown in Figure~\vref{fig-LR-passive-HPSG}.
\begin{figure}
\hfill
\begin{forest}
[V\feattab{\spr   \sliste{ \ibox{1} },\\
           \comps \sliste{ \ibox{2} },\\
           \argst \sliste{ \ibox{1} NP[\str], \ibox{2} NP[\str] }}
 [V\feattab{
           \argst \sliste{ \ibox{1} NP[\str], \ibox{2} NP[\str] }}]]
\end{forest}
\hfill
\begin{forest}
[V\feattab{\spr   \sliste{ \ibox{2} },\\
           \comps \sliste{ },\\
           \argst \sliste{ \ibox{2} NP[\str] }}
 [V\feattab{
           \argst \sliste{ \ibox{1} NP[\str], \ibox{2} NP[\str] }}]]
\end{forest}
\hfill\mbox{}
\caption{\label{fig-LR-passive-HPSG}Lexical rule"=based analysis of the perfect and the passive in HPSG}
\end{figure}%
The left figure shows a lexical item that is licensed by a lexical rule that is applied to the stem
\stem{kill}. The stem has two elements in its argument structure list and for the active forms the
complete argument structure list is shared between the licensed lexical item and the stem. The first
element of the \argstl is mapped to \spr and the other elements to \comps (in English). Passive is
depicted in the right figure: the first element of the \argst with structural case is suppressed and
since the element that was the second element in the \argstl of the stem \iboxb{2} is now the first element,
this item is mapped to \spr. See Section~\ref{sec-hpsg-passive} for passive in HPSG and Section~\ref{Abschnitt-Arg-St} for comments on
\argst and the differences between German and English\il{English}. 

The discussion of Figures~\ref{fig-Pass-vs-Perf-and-little-v} and~\ref{fig-LR-passive-HPSG} are
a further illustration of a point made in Section~\ref{Abschnitt-leere-Elemente-LRs-Transformations}:
lexical rules can be replaced by empty heads and vice versa. While HPSG says there are stems that
are related to inflected forms and corresponding to the inflection the arguments are realized in a
certain way, Minimalist theories assume two variants of \littlev that differ in their selection of
arguments. Now, the question is: are there empirical differences between the two approaches? I think
there are differences if one considers the question of language acquisition. What children can
acquire from data is that there are various inflected forms and that they are related somehow. What
remains questionable is whether they really would be able to detect empty little \textit{v}s. One
could claim of course that children operate with chunks of structures such as the ones in
Figure~\ref{fig-Pass-vs-Perf-and-little-v}. But then a verb would be just a chunk consisting of
\littlev and V and having some open slots. This would be indistinguishable from what the HPSG
analysis assumes.


As far as the ``lexical rules as additional tool'' aspect is concerned, the discussion is closed, but
note that the standard GB/Minimalism analyses differ in another way from LFG and HPSG analyses,
since they assume that passive has something to do with movement, that is, they assume that the same mechanisms
are used that are used for nonlocal dependencies.\footnote{
  There is another option in Minimalist theories. Since Agree\is{Agree} can check features
  nonlocally, T can assign nominative to an embedded element. So, in principle the object may get
  nominative in the VP without moving to T. However, \citet[\page 368]{Adger2003a} assumes that German has a
  strong EPP feature on T, so that the underlying object has to move to the specifier of T. This is
  basically the old GB analysis of passive in German with all its conceptual problems and disadvantages.
}
This works for languages like English in which
the object has to be realized in postverbal position in the active and in preverbal position in the
passive, but it fails for languages like German in which the order of constituents is more free.
\citet[Section~4.4.3]{Lenerz77} discussed the examples in (\ref{ex-passive-German-no-movement}) on
page~\pageref{ex-passive-German-no-movement} -- which are repeated here as
(\ref{ex-passive-German-no-movement-two}) for convenience:
\eal
\label{ex-passive-German-no-movement-two}
\ex 
\gll weil das Mädchen dem Jungen den Ball schenkt\\
     because the girl the.\dat{} boy the.\acc{} Ball gives\\
\glt `because the girl gives the ball to the boy'
\ex 
\gll weil dem Jungen der Ball geschenkt wurde\\
     because the.\dat{} boy the.\nom{} ball given was\\
\ex 
\gll weil der Ball dem Jungen geschenkt wurde\\
     because the.\nom{} ball the.\dat{} boy given was\\
\glt `because the ball was given to the boy'
\zl
While both orders in (\mex{0}b) and (\mex{0}c) are possible, the one with dative--nominative order
in (\mex{0}b) is the unmarked one. There is a strong linearization preference in German demanding that
animate NPs be serialized before inanimate ones \citep[\page 46]{Hoberg81a}. This linearization rule is
unaffected by passivization. 
Theories that assume that passive is movement either have to
assume that the passive of (\mex{0}a) is (\mex{0}c) and (\mex{0}b) is derived from (\mex{0}c) by a
further reordering operation (which would be implausible since usually one assumes that more marked
constructions require more transformations), or they would have to come up with other
explanations for the fact that the subject of the passive sentence has the same position as the
object in active sentences. As was already explained in Section~\ref{sec-passive-gb}, one such explanation is to
assume an empty expletive subject that is placed in the position where nominative is assigned and
to somehow connect this expletive element to the subject in object position. While this somehow
works, it should be clear that the price for rescuing a movement"=based analysis of passive is rather
high: one has to assume an empty expletive element, that is, something that neither has a form nor a
meaning. The existence of such an object could not be inferred from the input unless it is assumed
that the structures in which it is assumed are given. Thus, a rather rich UG\indexug would have to be
assumed. 

The question one needs to ask here is: why does the movement"=based analysis have these problems and why
does the valence"=based analysis not have them? The cause of the problem is that the analysis
of the passive mixes two things: the fact that SVO languages like English encode subjecthood
positionally, and the fact that the subject is suppressed in passives. If these two things are
separated the problem disappears. The fact that the object of the active sentence in (\ref{ex-she-beats-him}) is
realized as the subject in (\ref{ex-he-was-beaten}) is explained by the assumption that the first NP on the
argument structure list with structural case is realized as subject and mapped to the respective valence feature: \spr in
English. Such mappings can be language specific (see Section~\ref{Abschnitt-Arg-St} and
\citew{MuellerGermanic} where I discuss Icelandic\il{Icelandic}, which is an SVO language with
subjects with lexical case).

In what follows, I discuss another set of examples that are sometimes seen as evidence for a
movement"=based analysis. The examples in (\mex{1}) are instances of the so"=called remote passive\is{passive!remote}
\citep[\page 175--176]{Hoehle78a}.\footnote{
  See \citew[Section~3.1.4.1]{Mueller2002b} and \citew{Wurmbrand2003a} for corpus examples.
}
\eal
\ex\iw{versuchen|(}
\gll daß er auch von mir zu überreden versucht wurde\footnotemark\\
     that he.\nom{} also from me to persuade tried got\\
\footnotetext{
        \citew*[\page 212]{Oppenrieder91a}\ia{Oppenrieder}.%
}
\glt `that an attempt to persuade him was also made by me'
\ex 
\gll weil    der Wagen oft zu reparieren versucht wurde\\
     because the car.\nom{}   often to repair   tried     was\\
\glt `because many attempts were made to repair the car'\label{bsp-zu-reparieren-versucht-wurde}
%,
\zl
What is interesting about these examples is that the subject is the underlying object of a deeply
embedded verb. This seems to suggest that the object is extracted out of the verb phrase. So the
analysis of (\mex{0}b) would be (\mex{1}):
\ea
\gll weil    [\sub{IP} der Wagen$_i$ [\sub{VP} oft   [\sub{VP} [\sub{VP} [\sub{VP} \_$_i$ zu reparieren] versucht] wurde]\\
     because {}        the car.\nom{} {}        often {}        {}        {}        {}    to repair       tried     was\\
\z
While this is a straight-forward explanation of the fact that
(\ref{bsp-zu-reparieren-versucht-wurde}) is grammatical, another explanation is possible as well. In
the HPSG analysis of German (and Dutch\il{Dutch}) it is assumed that verbs like those in
(\ref{bsp-zu-reparieren-versucht-wurde}) form a verbal complex, that is, \emph{zu reparieren
  versucht wurde} `to repair tried was' forms one unit. When two or more verbs form a complex, the
highest verb attracts the arguments from the verb it embeds \citep{HN89a,HN94a,BvN98}. A verb like
\emph{versuchen} `to try'
selects a subject, an infinitive with \emph{zu} `to' and all complements that are selected by this
infinitive. In the analysis of (\mex{1}), \emph{versuchen} `to try' selects for its subject, the
object of \emph{reparieren} `to repair' and for the verb \emph{zu reparieren} `to repair'.
\ea
\gll weil er den Wagen zu reparieren versuchen will\\
     because he.\nom{} the.\acc{} car to repair try wants\\
\glt `because he wants to try to repair the car'
\z
Now if the passive lexical rule applies to \stem{versuch}, it suppresses the first argument of
\stem{versuch} with structural case, which is the subject of \stem{versuch}. The next argument of
\stem{versuch} is the object of \emph{zu reparieren}. Since this element is the first NP with
structural case, it gets nominative as in (\ref{bsp-zu-reparieren-versucht-wurde}). So, this shows
that there is an analysis of the remote passive that does not rely on movement. Since
movement"=based analyses were shown to be problematic and since there are no data that cannot be
explained without movement, analyses without movement have to be preferred.

This leaves us with movement"=based accounts of local reordering (scrambling). The reviewer
suggested that scrambling, passive, and nonlocal extraction may be analyzed with the same
mechanism. It was long thought that scope facts made the assumption of movement"=based analyses of
scrambling necessary, but it was pointed out by \citew[\page 146]{Kiss2001a} and
\citew[Section~2.6]{Fanselow2001a} that the reverse is true: movement"=based accounts of scrambling
make wrong predictions with regard to available quantifier scopings. I discussed the respective
examples in Section~\ref{sec-GB-lokale-Umstellung} already and will not repeat the discussion
here. The conclusion that has to be drawn from this is that passive, scrambling, and long distance
extraction are three different phenomena that should be treated differently. The solution for the
analysis of the passive that is adopted in HPSG is based on an analysis by \citet{Haider86}, who
worked within the GB framework. The ``scrambling-as-base generation'' approach to local reordering
that was used in HPSG right from the beginning \citep{Gunji86a} is also adopted by some practitioners
of GB/Minimalism, \eg \citet{Fanselow2001a}.

Having discussed the analyses in GB/Minimalism, I now turn to Dependency Grammar. 
\citet{GO2009a} suggest that \emph{w}-fronting, topicalization, scrambling, extraposition,
splitting, and also the remote passive should be analyzed by what they call
\emph{rising}\is{rising}. The concept was already explained in Section~\ref{sec-nld-dg}. The
Figures~\ref{fig-die-idee-wird-jeder-verstehen-dg-rising}
and~\ref{fig-gestern-hat-sich-der-spieler-verletzt-dg-rising} show examples for the fronting and the scrambling of an object.
\begin{figure}
\centering
\begin{forest}
dg edges
[V
  [N, edge=dashed 
    [Det [die;the] ]
    [Idee;idea]] 
  [wird;will] 
  [N [jeder;everybody] ]
  [V$_g$ [verstehen;understand]]]
\end{forest}
\caption{\label{fig-die-idee-wird-jeder-verstehen-dg-rising}Analysis of \emph{Die Idee wird jeder
    verstehen.} `Everybody will understand the idea.' involving rising}
\end{figure}%%
\begin{figure}
\centering
\begin{forest}
dg edges
[V
  [Adv [Gestern;yesterday] ]
  [hat;has] 
  [N, edge=dashed [sich;himself] ]
  [N
    [Det [der;the]]
    [Spieler;player]]
  [V$_g$ [verletzt;injured]]]
\end{forest}
\caption{\label{fig-gestern-hat-sich-der-spieler-verletzt-dg-rising}Analysis of \emph{Gestern hat
    sich der Spieler verletzt.} `Yesterday, the player injured himself.' involving rising of the object of the main
  verb \emph{verletzt} `injured'}
\end{figure}%%
Groß and Osborne assume that the object depends on the main verb in sentences with auxiliary verbs,
while the subject depends on the auxiliary. Therefore, the object \emph{die Idee} `the idea' and
the object \emph{sich} `himself' have to rise to the next higher verb in order to keep the
structures projective\is{projectivity}.
Figure~\vref{fig-dass-der-wagen-zu-reparieren-versucht-wurde-dg-rising} shows the analysis of the
remote passive.
\begin{figure}
\centering
\begin{forest}
dg edges
[Subjunction
  [dass;that]
  [V
    [N, edge=dashed
      [Det [der;the]]
      [Wagen;car]]
    [V
      [V$_g$ [zu reparieren;to repair]]
      [versucht;tried]]
    [wurde;was]]]
\end{forest}
\caption{\label{fig-dass-der-wagen-zu-reparieren-versucht-wurde-dg-rising}Analysis of the remote
  passive \emph{dass der Wagen zu reparieren versucht wurde} `that it was tried to repair the car' involving rising}
\end{figure}%%
The object of \emph{zu reparieren} `to repair' rises to the auxiliary \emph{wurde} `was'.

Groß and Osborne use the same mechanism for all these phenomena, but it should be clear that there
have to be differences in the exact implementation. Groß and Osborne say that English does not have
scrambling, while German does. If this is to be captured, there must be a way to distinguish the two
phenomena, since if this were not possible, one would predict that English has scrambling as well,
since both German and English allow long distance fronting. \citet[\page 58]{GO2009a} assume
that object nouns that rise must take the nominative. But if the kind of rising that they assume
for remote passives is identical to the one that they assume for scrambling, they would predict that
\emph{den Wagen} gets nominative in (\mex{1}) as well:
\ea
\gll dass den Wagen niemand repariert hat\\
     that the.\acc{} car nobody.\nom{} repaired has\\
\glt `that nobody repaired the car'
\z
Since \emph{den Wagen} `the car' and \emph{repariert} `repaired' are not adjacent, \emph{den Wagen} has to rise to the
next higher head in order to allow for a projective realization of elements. So in order to assign
case properly, one has to take into account the arguments that are governed by the head to which a
certain element rises. Since the auxiliary \emph{hat} `has' already governs a nominative, the NP \emph{den
  Wagen} has to be realized in the accusative. An analysis that assumes that both the accusative and
nominative depend on \emph{hat} `has' in (\mex{0}) is basically the verbal complex analysis
assumed in HPSG and some GB variants.

Note, however, that this does not extend to nonlocal dependencies. Case is assigned locally by verbs or
verbal complexes, but not to elements that come from far away. The long distance extraction of
NPs is more common in southern variants of German and there are only a few verbs that do not take a
nominative argument themselves. The examples below involve \emph{dünken} `to think', which governs an
accusative and a sentential object and \emph{scheinen} `to seem', which governs a dative and a
sentential object. If (\mex{1}a) is analyzed with \emph{den Wagen} rising to
\emph{dünkt}, one might expect that \emph{den Wagen} `the car' gets nominative since there is no other element
in the nominative. However, (\mex{0}b) is entirely out.

\eal
\ex[]{ 
\gll Den Wagen dünkt mich, dass er repariert.\\
     the.\acc{} car thinks me.\acc{}     that he.\nom{} repairs\\
\glt `I think that he repairs the car'
}
\ex[*]{
\gll Der Wagen dünkt mich, dass er repariert.\\
     the.\nom{} car thinks me.\acc{}     that he.\nom{} repairs\\
}
\zl

Similarly there is no agreement between the fronted element and the verb to which it attaches:
\eal
\ex[]{
\gll Mir scheint, dass die Wagen ihm gefallen.\\
     me.\dat.1\pl{} seems.3\sg{} that the cars.3\pl{} him please.3\pl{} \\
\glt `He seems to me to like the cars.'
}
%% \ex 
%% \gll Der Wagen scheint mir, dass ihm gefällt.\\
%%      the car   seems   me   that him pleases\\
\ex[]{
\gll Die Wagen scheint mir, dass ihm gefallen.\\
     the cars.3\pl{}  seem.3\sg{}     me.\dat{}   that him please.3\pl{}\\
\glt `The cars, he seems to me to like.'
}

\ex[*]{
\gll  Die Wagen scheinen mir, dass ihm gefällt.\\
      the cars.3\pl{}  seem.3\pl{}     me.\dat{}   that him pleases.3\sg{}\\
}
\ex[*]{
\gll Die Wagen scheinen mir, dass ihm gefallen.\\
     the cars.3\pl{}  seem.3\pl{}     me.\dat{}   that him please.3\pl{}\\
}
\zl
This shows that scrambling/remote passive and extraction should not be dealt with by the same mechanism or if they
are dealt with by the same mechanism one has to make sure that there are specialized variants of the
mechanism that take the differences into account. 
I think what Groß and Osborne did is simply recode the attachment relations of phrase structure
grammars. \emph{die Idee} `the idea' has some relation to \emph{wird jeder verstehen} `will
everybody understand' in Figure~\ref{fig-die-idee-wird-jeder-verstehen-dg-rising}, as it does in GB, LFG, GPSG, HPSG, and
other similar frameworks. In HPSG, \emph{die Idee} `the idea' is the filler in a filler-head configuration. The remote
passive and local reorderings of arguments of auxiliaries, modal verbs, and other verbs that behave similarly
are explained by verbal complex formation where all non-verbal arguments depend on the highest verb \citep{HN94a}.

Concluding this chapter, it can be said that local reorderings and long"=distance dependencies are two
different things that should be described with different tools (or there should be further
constraints that differ for the respective phenomena when the same tool is used). Similarly, movement"=based analyses
of the passive are problematic since passive does not necessarily imply reordering. 





%      <!-- Local IspellDict: en_US-w_accents -->
