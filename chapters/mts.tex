%% -*- coding:utf-8 -*-

\chapter{Generative"=enumerative vs.\ model-theoretic approaches}
\label{Abschnitt-Generativ-Modelltheoretisch}


Generative"=enumerative\is{model"=theoretic grammar|(} approaches assume that a grammar generates a set of sequences of symbols (strings of words).
This is where the term Generative Grammar\is{Generative Grammar} comes from. Thus, it is possible to use the grammar on
page~\pageref{bsp-grammatik-psg}, repeated here as (\mex{1}), to derive the string \emph{er das Buch dem Mann gibt}
`he the book the man gives'.
\ea
\label{bsp-grammatik-psg-zwei}
\begin{tabular}[t]{@{}l@{ }l}
{NP} & {$\to$ D, N}\\          
{S}  & {$\to$ NP, NP, NP, V}
\end{tabular}\hspace{2cm}%
\begin{tabular}[t]{@{}l@{ }l}
{NP} & {$\to$ er}\\
{D}  & {$\to$ das}\\
{D}  & {$\to$ dem}\\
\end{tabular}\hspace{8mm}
\begin{tabular}[t]{@{}l@{ }l}
{N} & {$\to$ Buch}\\
{N} & {$\to$ Mann}\\
{V} & {$\to$ gibt}\\
\end{tabular}
\z
Beginning with the start symbol (S), symbols are replaced until one reaches a sequence of symbols only containing words.
The set of all strings derived in this way is the language described by the grammar.

The following are classed as generative"=enumerative approaches:
\begin{itemize}
\item all phrase structure grammars\is{phrase structure grammar}
\item Transformational Grammars in almost all variants\is{Transformational Grammar}
\item \gpsg in the formalism of \citet*{GKPS85a}
\item many variants of Categorial Grammar\indexcg
\item many variants of TAG\indextag
\item Chomsky's Minimalist Grammars
\end{itemize}
\lfg was also originally designed to be a generative grammar.

The opposite of such theories of grammar are model-theoretic or constraint"=based approaches (MTA).
MTAs formulate well"=formedness conditions on the expressions that the grammar describes.
In Section~\ref{sec-modelle-theorien}, we already discussed a model"=theoretic approach for theories that use
feature structures to model phenomena. To illustrate this point, I will discuss another HPSG example:
(\mex{1}) shows the lexical item for \emph{kennst} `know'. %in second person singular.
\begin{figure}
\eas
Lexical item for \emph{kennst}:\\
\label{le-kennst-mts}%
\onems{
phon \phonliste{ kennst }\\
synsem \onems{ loc  \ms{ cat  & \ms{ head & \ms[verb]{ vform & fin\\
                                                     dsl   & none\\
                                              }\\
                                   subcat & \sliste{ \npnom\ind{1}\sub{[\type{second},\type{sg}]}, \npacc\ind{2} }\\
                     }\\
                  cont & \ms{
                         ind & \ibox{3}\\
                         rels & \liste{ \ms[kennen]{
                                         event       & \ibox{3}\\
                                         experiencer & \ibox{1}\\
                                         theme       & \ibox{2}\\
                                         }\\
                                      }\\
                        }\\
                }\\
              nonloc  \ldots
            }
}
\zs
\vspace{-\baselineskip}
\end{figure}
In the description of (\mex{0}), it is ensured that the \phonv of the relevant linguistic sign is
\phonliste{ kennst }, that is, this value of \phon is constrained. There are parallel restrictions for the features given in (\mex{0}): the \synsemv
is given. In \synsem, there are restrictions on the \textsc{loc} and \nonlocv. In \textsc{cat}, there are
individual restrictions for \head and \subcat. The value of \subcat is a list with descriptions of dependent
elements. The descriptions are given as abbreviations here, which actually stand for complex feature descriptions that also
consist of feature"=value pairs. For the first argument of \emph{kennst}, a \headv of type 
\type{noun} is required, the \textsc{per} value in the semantic index has to be \type{second} and the
 \textsc{num} value has to be \type{sg}. The structure sharings in (\mex{0}) are a special kind of constraint. Values that
 are not specified in the descriptions of lexical entries can vary in accordance with the feature geometry given by the type
 system. In (\mex{0}), neither the \slashv of the nominative NP nor the one of the accusative NP is fixed. This means that \slasch can
 either be an empty or non"=empty list.

The constraints in lexical items such as (\mex{0}) interact with further constraints that hold for the signs of type
\type{phrase}. For instance, in head"=argument structures, the non"=head daughter must correspond to an element from the \subcatl of
the head daughter.

\addlines[-1]
Generative"=enumerative and model"=theoretic approaches view the same problem from different sides: the generative side only
allows what can be generated by a given set of rules, whereas the model"=theoretic approach allows everything that is not ruled out
by constraints.\footnote{
Compare this to an old joke: in dictatorships, everything that is not allowed is banned, in democracies, everything
that is not banned is allowed and in France, everything that is banned is allowed. Generative"=enumerative approaches correspond
to the dictatorships, model"=theoretic approaches are the democracies and France is something that has no correlate in linguistics.
}

\citet[\page 19--20]{PS2001a} and \citet{Pullum2007a} list the following model"=theoretic approaches:\footnote{
  See \citew{Pullum2007a} for a historical overview of Model Theoretic Syntax (MTS) and for further references.
}
\begin{itemize}
\item the non"=procedural variant of Transformational Grammar\is{Transformational Grammar} of Lakoff\aimention{George Lakoff}, that formulates
constraints on potential tree sequences,
\item Johnson and Postal's formalization of Relational Grammar\is{Relational Grammar} \citeyearpar{JP80a-u}, 
\item GPSG\indexgpsg in the variants developed by \citet{GPCKHL88a}, \citet{BGM93a-u} and \citet{Rogers97a},
\item LFG\indexlfg in the formalization of \citet{Kaplan95a}\footnote{
  According to \citet[Section~3.2]{Pullum2013a}, there seems to be a problem for model"=theoretic formalizations of so"=called
  \emph{constraining equations}.
} and   
\item HPSG\indexhpsg in the formalization of \citet{King99a-u}.
\end{itemize}
Categorial Grammars\indexcg \citep{BvN94a-u}, TAG\indextag \citep{RVS94a-u} and
Minimalist\indexmp approaches \citep{Veenstra98a} can be formulated in model"=theoretic terms.

\citet{PS2001a} point out various differences between these points of view. In the following sections, I will focus on two of these differences.\footnote{
	The reader should take note here: there are differing views with regard to how generative"=enumerative and MTS models are best formalized and not
	all of the assumptions discussed here are compatible with every formalism. The following sections mirror the important points in the general discussion.%
} Section~\ref{Abschnitt-MTS-ten-Hacken} deals with ten Hacken's objection to the model"=theoretic view.

\section{Graded acceptability}

Generative"=enumerative\is{gradability|(} approaches differ from model"=theoretic approaches in how they deal with the varying degrees of acceptability
of utterances. In generative"=enumerative approaches, a particular string is either included in the set of well"=formed expressions or it is not.
This means that it is not straightforwardly possible to say something about the degree of deviance: the first sentence in (\mex{1}) is judged grammatical
and the following three are equally ungrammatical.
\eal
\ex[]{
\gll Du kennst diesen Aufsatz.\\
	 you know.\textsc{2sg} this.\acc{} essay\\
}
\ex[*]{
\gll Du kennen diesen Aufsatz.\\
	you know.\textsc{3pl} this.\acc{} essay\\
}
\ex[*]{
\gll Du kennen dieser Aufsatz.\\
you know.\textsc{3pl} this.\nom{} essay\\
}
\ex[*]{
\gll Du kennen Aufsatz dieser.\\
you know.\textsc{3pl} essay this.\nom{}\\
}
\zl
At this point, critics of this view raise the objection that it is in fact possible to determine degrees of acceptability
in (\mex{0}b--d): in (\mex{0}b), there is no agreement between the subject and the verb, in
(\mex{0}c),  \emph{dieser Aufsatz} `this essay'  has the wrong case in addition, and in (\mex{0}d),
\emph{Aufsatz} `essay' and \emph{dieser} `this' occur in the wrong order. Furthermore,
 the sentence in (\mex{1}) violates grammatical rules of German, but is nevertheless still interpretable.
\ea
\gll Studenten stürmen mit Flugblättern und Megafon die Mensa und rufen alle auf zur Vollversammlung in der Glashalle \emph{zum} \emph{kommen}. \emph{Vielen} bleibt das Essen im Mund stecken und \emph{kommen} \emph{sofort} \emph{mit}.\footnotemark\\
students storm with flyers and megaphone the canteen and call all up to plenary.meeting in the glass.hall to.the come many.\dat{} stays the food in.the mouth stick and come immediately with\\
\footnotetext{
  Streikzeitung der Universität Bremen, 04.12.2003, p.\,2. The emphasis is mine.
}
\glt `Students stormed into the university canteen with flyers and a megaphone calling for everyone to come to a plenary meeting
in the glass hall. For many, the food stuck in their throats and they immediately joined them.'
\z
Chomsky (\citeyear[Chapter~5]{Chomsky75a}; \citeyear{Chomsky64a}) tried to use a string distance function to determine the relative
acceptability of utterances. This function compares the string of an ungrammatical expression with that of a grammatical expression
and assigns an ungrammaticality score of 1, 2 or 3 according to certain criteria. This treatment is not adequate, however, as there
are much more fine"=grained differences in acceptability and the string distance function also makes incorrect predictions.
For examples of this and technical problems with calculating the function, see \citew[\page 29]{PS2001a}.

In model"=theoretic approaches, grammar is understood as a system of well"=formedness conditions. An expression becomes worse, the
more well"=formedness conditions it violates \citep[\page 26--27]{PS2001a}. In (\mex{-1}b), the person and number requirements of
the lexical item for the verb \emph{kennst} are violated. In addition, the case requirements for the object have not been fulfilled in
(\mex{-1}c). There is a further violation of a linearization rule for the noun phrase in (\mex{-1}d).

Well"=formedness conditions can be weighted in such a way as to explain why certain violations lead to more severe deviations than
others. Furthermore, performance factors also play a role when judging sentences (for more on the distinction between performance
and competence, see Chapter~\ref{Abschnitt-Diskussion-Performanz}). As we will see in Chapter~\ref{Abschnitt-Diskussion-Performanz},
constraint"=based approaches work very well as performance"=compatible grammar models. If we combine the relevant grammatical theory
with performance models, we will arrive at explanations for graded acceptability differences owing to performance factors.
\is{gradability|)}

\section{Utterance fragments}

\mbox{}\citet[Section~3.2]{PS2001a} point out that generative"=enumerative theories do not assign structure to fragments.
For instance, neither the string \emph{and of the} nor the string \emph{the of and} would receive a
structure since none of these sequences is well-formed as an utterance and they are therefore not elements of the set of sequences generated by the grammar. However, \emph{and of the}
can occur as part of the coordination of PPs in sentences such as (\mex{1}) and would therefore have some structure in these cases,
for example the one given in Figure~\vref{fig-and-of-the}.%
\ea
That cat is afraid of the dog and of the parrot.
\z
\begin{figure}
\centering
\begin{tikzpicture}
\tikzset{level 1+/.style={level distance=3\baselineskip}}
%\tikzset{frontier/.style={distance from root=23\baselineskip}}
\Tree[.PP
        PP
        [.{PP[\textsc{coord} \emph{and} ]}
          [.Conj and ]
          [.{PP} 
            [.P of ]
            [.NP 
              [.Det the ]
              {\nbar}  ] ] ] 
]
\end{tikzpicture}
\caption{\label{fig-and-of-the}Structure of the fragment \emph{and of the} following
  \citew[\page 32]{PS2001a}}
\end{figure}%
As a result of the interaction of various constraints in a constraint"=based grammar, it emerges that \emph{the}
is part of an NP and this NP is an argument of \emph{of} and furthermore \emph{and} is combined with the relevant
\emph{of}-PP. In symmetric coordination, the first conjunct has the same syntactic properties as the second, which
is why the partial structure of \emph{and of the} allows one to draw conclusions about the category of the conjunct
despite it not being part of the string.

Ewan Klein\aimention{Ewan Klein} noted that Categorial Grammar\indexcg and Minimalist Grammars, which build up more complex
expressions from simpler ones, can sometimes create this kind of fragments \citep[\page 507]{Pullum2013a}.
This is certainly the case for Categorial Grammars with composition rules, which allow one to
combine any sequence of words to form a constituent. If one views derivations as logical proofs, as
is common in some variants of Categorial Grammar, then the actual derivation is irrelevant. What matters is whether a
proof can be found or not. However, if one is interested in the derived structures, then the argument brought forward by Pullum and Scholz is still
valid. For some variants of Categorial Grammar that motivate the combination of constituents based on their prosodic\is{prosody}
and information"=structural properties \citep[Section~3]{Steedman91a}, the problem persists since fragments have
a structure independent of the structure of the entire utterance and independent of their
information"=structural properties within this complete structure. This structure of the fragment
can be such that it is not possible to analyze it with type"=raising rules and composition
rules.

In any case, this argument holds for Minimalist\is{Minimalist Program (MP)} theories since it is not possible to have a combination
of \emph{the} with a nominal constituent if this constituent was not already built up from lexical
material by Merge.

\section{A problem for model"=theoretic approaches?}
\label{Abschnitt-MTS-ten-Hacken}

\mbox{}\Citet[\page 237--238]{TenHacken2007a} discusses the formal assumptions of HPSG\indexhpsg. In HPSG, feature descriptions are used to describe feature
structures. Feature structures must contain all the features belonging to a structure of a certain
type. Additionally, the features have to have a maximally"=specific value (see
Section~\ref{sec-modelle-theorien}). Ten Hacken discusses gender properties\is{gender|(} of the
English noun \emph{cousin}. In English, gender is important in order to ensure the correct binding
of pronouns (see page~\pageref{le-buch} for German): 
\eal
\ex The man$_i$ sleeps. He$_i$ snores.
\ex The woman$_i$ sleeps. He$_{*i}$ snores.
\zl
While \emph{he} in (\mex{0}a) can refer to \emph{man}, \emph{woman} is not a possible antecedent. Ten Hacken's problem is that \emph{cousin}
is not marked with respect to gender. Thus, it is possible to use it to refer to both male and female relatives.
As was explained in the discussion of the case value of \emph{Frau} `woman' in Section~\ref{sec-modelle-theorien}, it is possible for a
value in a description to remain unspecified. Thus, in the relevant feature structures, any
appropriate and maximally specific value is possible. The case of \emph{Frau} can therefore be nominative, genitive, dative or accusative in an actual feature structure.
Similarly, there are two possible genders for \emph{cousin} corresponding to the usages in (\mex{1}).
\eal
\ex I have a cousin$_i$. He$_i$ is very smart.
\ex I have a cousin$_i$. She$_i$ is very smart.
\zl

\noindent
Ten Hacken refers to examples such as (\mex{1}) and claims that these are problematic:
\eal
\ex Niels has two cousins.
\ex How many cousins does Niels have?
\zl
In plural usage, it is not possible to assume that \emph{cousins} is feminine or masculine since the set of relatives can contain
either women or men. It is interesting to note that (\mex{1}a) is possible in English, whereas German is forced to use (\mex{1}b)
to express the same meaning.
\eal
\ex Niels and Odette are cousins.
\ex 
\gll Niels und Odette sind Cousin und Cousine.\\
	 Niels and Odette are cousin.\mas{} and cousin.\fem\\
\zl
Ten Hacken concludes that the gender value has to remain unspecified and this shows, in his opinion, that model"=theoretic analyses
are unsuited to describing language.

If we consider what exactly ten Hacken noticed, then it becomes apparent how one can account for this in a model"=theoretic approach:
Ten Hacken claims that it does not make sense to specify a gender value for the plural form of \emph{cousin}. In a model"=theoretic approach, this can be captured
 in two ways. One can either assume that there are no gender features for referential indices in the
plural, or that one can add a gender value that plural nouns can have.

The first approach is supported by the fact that there are no inflectional differences between the
plural forms of pronouns with regard to gender. There is therefore no reason to distinguish genders
in the plural.
\eal
\ex Niels and Odette are cousins. They are very smart.
\ex The cousins/brothers/sisters are standing over there. They are very smart.
\zl
No distinctions are found in plural when it comes to nominal inflection (\emph{brothers},
\emph{sisters}, \emph{books}). In German, this is different. There are differences with both nominal
inflection and the reference of (some) noun phrases 
with regard to the sexus of the referent.
Examples of this are the previously mentioned examples \emph{Cousin} `male cousin' and
\emph{Cousine} `female cousin' as well as forms with the suffix \suffix{in} as in \emph{Kindergärtnerin} `female nursery teacher'.
However, gender is normally a grammatical notion that has nothing to do with sexus\is{sexus}.
An example is the neuter noun \emph{Mitglied} `member', which can refer to both female and male persons.

The question that one has to ask when discussing Ten Hacken's problem is the following: does gender play a role for pronominal binding
in German? If this is not the case, then the gender feature is only relevant within the morphology
component, and here the gender value is determined for each noun in the lexicon. For the binding of personal pronouns, there is no gender difference in German. 
\ea
\gll Die Schwestern / Brüder / Vereinsmitglieder / Geschwister stehen dort. Sie lächeln.\\
     the sisters.\fem{} {} brothers.\mas{} {} club.members.\neu{} {} siblings stand there they smile.\\
\glt `The sisters/brothers/club members/siblings are standing there. They are smiling.'
\z
Nevertheless, there are adverbials in German that agree in gender with the noun to which they refer \citep[Chapter~6]{Hoehle83}:
\eal
\label{Beispiel-einer-nach-dem-anderen}
\ex
\gll Die Fenster wurden eins nach dem anderen geschlossen.\\
	 the windows.\neu{} were one.\neu{} after the other closed\\
\glt `The windows were closed one after the other.'
\ex 
\gll Die Türen wurden eine nach der anderen geschlossen.\\
	the doors.\fem{} were one.\fem{} after the other closed\\
\glt `The doors were closed one after the other.'
\ex 
\gll Die Riegel wurden einer nach dem anderen zugeschoben.\\
	 the bolts.\mas{} were one.\mas{} after the other closed\\
	 \glt `The bolts were closed one after the other.'
\zl
For animate nouns, it is possible to diverge from the gender of the noun in question and use a form of
the adverbial that corresponds to the biological sex:
\eal
\ex 
\gll Die Mitglieder des Politbüros wurden eines / einer nach dem anderen aus dem Saal getragen.\\
	 the members.\neu{} of.the politburo were one.\neu{} {} one.\mas{} after the other out.of the hall carried\\
\glt `The members of the politburo were carried out of the hall one after the other.'
\ex 
\gll Die Mitglieder des Frauentanzklubs verließen eines / eine nach dem / der anderen im Schutze der Dunkelheit den
Keller.\\
the members.\neu{} of.the women's.dance.club left one.\neu{} {} one.\fem{} after the.\neu{} {} the.\fem{} other in.the protection of.the dark the
basement\\
\glt `The members of the women's dance club left the basement one after the other under cover of darkness.'
\zl
This deviation from gender in favor of sexus can also be seen with binding of personal and relative pronouns with nouns such as
\emph{Weib} `woman' (pej.) and \emph{Mädchen} `girl':
\eal
\ex 
\gll "`Farbe bringt die meiste Knete!"' verriet ein 14jähriges türkisches {\em Mädchen\/}, {\em die\/} die Mauerstückchen am
      Nachmittag am Checkpoint Charlie an Japaner und US-Bürger verkauft.\footnotemark\\
\hspaceThis{"`}color brings the most money revealed a 14-year.old Turkish girl.\neu{} who.\fem{} the wall.pieces
in.the afternoon at Checkpoint Charlie at Japanese and US-citizens sells\\  
\footnotetext{
        taz, 14.06.1990, p.\,6.
      }
\glt `\,``Color gets the most money'' said a 14-year old Turkish girl who sells pieces of the wall to Japanese and American citizens
at Checkpoint Charlie.'
\ex 
\gll Es ist ein junges {\em Mädchen\/}, {\em die\/} auf der Suche nach CDs bei Bolzes reinschaut.\footnotemark\\
	 it is a young girl.\neu{} who.\fem{} on the search for CDs at Bolzes stops.by\\
\footnotetext{
        taz, 13.03.1996, p.\,11.
      }
\glt `It is a young girl looking for CDs that stops by Bolzes.' 
\zl
For examples from Goethe, Kafka and Thomas Mann, see \citew[\page 417--418]{Mueller99a}. 

%\addlines
For inanimate nouns such as those in (\mex{-2}), agreement is obligatory. For the analysis of
German, one therefore does in fact require a gender feature in the plural. In English, this is not
the case since there are no parallel examples with pronouns inflecting for gender. One can therefore
either assume that plural indices do not have a gender feature or that the gender value is
\emph{none}. In the latter case, the feature would have a value and hence fulfill the formal requirements.
(\mex{1}) shows the first solution: plural indices are modeled by feature structures of
type \type{pl-ind} and the \textsc{gender} feature is just not appropriate for such objects.

\ea
\begin{tabular}[t]{@{}l@{~~}l@{\hspace{2cm}}l@{~~}l}
a.& singular index: &
b.& plural index:\\
  &\ms[sg-ind]{
    per & per\\
    num & sg\\
    gen & gender\\
    }
&&\ms[pl-ind]{
    per & per\\
    num & pl\\
    }\vspace{\baselineskip}~
\end{tabular}
\z
The second solution requires the type hierarchy in Figure~\vref{fig-typehierarchy-gender-ten-Hacken} for the subtypes of \type{gender}.
% moved above the figure
With such a type hierarchy \type{none} is a possible value of the \textsc{gen} feature and no
problem will arise.%
\is{gender|)}
\begin{figure}
\begin{forest}
typehierarchy
[ gender
   [fem] [mas] [neu] [none] ]
\end{forest}
\caption{\label{fig-typehierarchy-gender-ten-Hacken}Type hierarchy for one of the solutions of ten Hacken's problem}
\end{figure}%

In general, it is clear that cases such as the one constructed by ten Hacken will never be a problem since there are either
values that make sense, or there are contexts for which there is no value that makes sense and one therefore does not require
the features.

So, while ten Hacken's problem is a non-issue, there are certain problems of a more technical
nature. I have pointed out one such technical problem in \citew[Section~14.4]{Mueller99a}. I show
that spurious ambiguities arise for a particular analysis of verbal complexes in German when one
resolves the values of a binary feature (\textsc{flip}).  I also show how this problem can be
avoided by the complicated stipulation of a value in certain contexts.%
\is{model"=theoretic grammar|)}


%      <!-- Local IspellDict: en_US-w_accents -->
