%% -*- coding:utf-8 -*-
\title{Grammatical theory}
\subtitle{From transformational grammar to constraint-based approaches}
\author{Stefan Müller}
\typesetter{Stefan Müller}
\translator{Andrew Murphy, Stefan Müller}
\proofreader{Viola Auermann, Paul Kay}

\BackTitle{Grammatical theory}
\BackBody{This book introduces formal grammar theories that play a role in current linguistics or
  contributed tools that are relevant for current linguistic theorizing (Phrase Structure Grammar,
  Transformational Grammar/""Government \& Binding, Generalized Phrase Structure Grammar, Lexical
  Functional Grammar, Categorial Grammar, Head-​Driven Phrase Structure Grammar, Construction
  Grammar, Tree Adjoining Grammar). The key assumptions are explained and it is shown how the
  respective theory treats arguments and adjuncts, the active/passive alternation, local
  reorderings, verb placement, and fronting of constituents over long distances. The analyses are
  explained with German as the object language. 

In the second part of the book the approaches are compared with respect to their predictions regarding language
acquisition and psycholinguistic plausibility. The nativism hypothesis that assumes that humans
posses genetically determined innate language-specific knowledge is examined critically and
alternative models of language acquisition are discussed. In addition the second part addresses issues
that are discussed controversially in current theory building as for instance the question whether
flat or binary branching structures are more appropriate, the question whether constructions should
be treated on the phrasal or the lexical level, and the question whether abstract, non-visible
entities should play a role in syntactic analyses. It is shown that the analyses that are suggested
in the respective frameworks are often translatable into each other. The book closes with a chapter
that shows how properties that are common to all languages or to certain language classes can be
captured.

\vfill

``With this critical yet fair reflection on various grammatical theories, Müller fills what was a major gap in the literature.'' \href{http://dx.doi.org/10.1515/zrs-2012-0040}{Karen Lehmann, Zeitschrift für Rezensionen zur germanistischen Sprachwissenschaft, 2012}

``Stefan Müller’s recent introductory textbook, ``Grammatiktheorie'', is an astonishingly
comprehensive and insightful survey for beginning students of the present state of syntactic
theory.'' \href{http://dx.doi.org/10.1515/zfs-2012-0010}{Wolfgang Sternefeld und Frank Richter, Zeitschrift für Sprachwissenschaft, 2012}


``This is the kind of work that has been sought after for a while. [\dots] The impartial and objective discussion offered by the author is particularly refreshing.'' \href{http://dx.doi.org/10.1515/germ-2011-537}{Werner Abraham, Germanistik, 2012}

}
\dedication{For Max}
\renewcommand{\lsISBNdigital}{978-3-944675-21-3}
\renewcommand{\lsISBNhardcover}{978-3-946234-29-6}
\renewcommand{\lsISBNsoftcover}{978-3-946234-30-2}
\renewcommand{\lsSeries}{tbls} % use lowercase acronym, e.g. sidl, eotms, tgdi
\renewcommand{\lsSeriesNumber}{1} %will be assigned when the book enters the proofreading stage
\renewcommand{\lsURL}{http://langsci-press.org/catalog/book/25} % contact the coordinator for the right number
